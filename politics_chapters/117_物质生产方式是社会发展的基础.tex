\question 生产方式集中体现了人类社会的物质性。生产方式中的生产力体现着人们改造自然的现实的物质力量,生产关系是人们在物质生产中发生的``物质的社会关系'',生产
力和生产关系的统一所构成的生产方式使自然界的一部分转化为社会物质生活条件,
使生物的人上升为社会的人。生产方式是社会历史发展的决定力量。这种决定力量体现为()
\par\fourch{\textcolor{red}{物质生产方式是人类其他一切活动的首要前提}}{\textcolor{red}{物质生产方式决定着社会的结构和性质,制约着全部社会生活}}{\textcolor{red}{物质生产方式决定着社会形态从低级到高级的发展}}{物质生产方式是劳动者和劳动资料结合的特殊方式}
\begin{solution}本题考查物质生产方式是社会发展的基础。生产方式是社会历史发展的决定力量。首先,物质生产活动以及生产方式是人类社会赖以存在和发展的基础,是人
类其他一切活动的首要前提。其次,物质生产活动及生产方式决定着社会的结构、
性质和面貌,制约着人们的经济生活、政治生活和精神生活等全部社会生活。最后,
物质生产活动及生产方式的变化发展决定着整个社会历史的变化发展,决定着社会
形态从低级向高级的更替和发展。故选项ABC正确。选项D不是物质生产方式决定
作用的表现,不符合题意,故不选。
\end{solution}
