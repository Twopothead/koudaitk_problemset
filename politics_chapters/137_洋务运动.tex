\question 洋务派兴办民用企业的主要方式有( ~)
\par\twoch{\textcolor{red}{官办}}{\textcolor{red}{官督商办}}{\textcolor{red}{官商合办}}{商办}
\begin{solution}洋务派举办的民用企业的资金全部或大部由政府筹集,也吸收一部分商股,主要由政府派官员管理,有官办、官督商办、官商合办几种形式。其中,多数都采取官督商办的方式。因其资金主要来源于政府,因此不是商办的形式。
\end{solution}
\question 李鸿章说:``溯自各国通商以来,进口洋货日增月盛\ldots{}\ldots{}出口土货,年减一年,往往不能相敌。推原其故,由于各国制造均用机器。\ldots{}\ldots{}臣拟遴派绅商,在上海购买机器,设局仿造布匹,冀稍分洋商之利。''李鸿章的这段言论表明洋务派开始(
~)
\par\twoch{\textcolor{red}{兴办民用企业}}{\textcolor{red}{采用机器制造布匹,降低生产成本}}{发展民族资本主义}{\textcolor{red}{稍分洋商之利}}
\begin{solution}注意洋务运动是为了维护封建统治而发动的一场爱国救亡运动,从来没有主张变革社会制度或者发展民族资本主义。
\end{solution}
