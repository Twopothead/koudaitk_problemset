\question 恩格斯把费尔巴哈等旧唯物主义者称为半截子的唯物主义,并指出真正的唯物
主义者在理解现实世界(自然界和历史)时是``按照它本身在每一个不以先入为主的唯心主义
怪想来对待它的人面前所呈现的那样来理解\ldots{}\ldots{}除此以外,唯物主义并没有别的意义。''这
里的``半截子''主要指的是( ~)
\par\fourch{在坚持唯物论的同时,没有把唯物论和辩证法相结合}{在承认物质决定意识的同时,否认物质与意识的同一性}{\textcolor{red}{在自然观上是唯物主义的,历史观上则陷入唯心主义}}{把客观事物看作是既成的事实,但不承认事物的变化发展}
\begin{solution}题干是说旧唯物主义的缺陷,A是形而上学的唯物主义,B是不可知论,D是形而上学。
\end{solution}
\question 设想没有运动的物质的观点是( )
\par\twoch{主观唯心主义}{客观唯心主义}{庸俗唯物主义}{\textcolor{red}{形而上学唯物主义}}
\begin{solution}没有运动的物质即形而上学唯物主义。A主观唯心主义认为世界的``内心反省''的结果,B客观唯心主义是``绝对精神''的产物,C庸俗唯物主义是辩证唯物主义之前的唯物主义。
\end{solution}
\question 哲学是系统化、理论化的世界观和方法论。哲学研究的问题很多。其中被称为``哲学基本问题''的是(
)
\par\twoch{思维和存在何者为第一性问题}{\textcolor{red}{思维和存在的关系问题}}{思维和存在的同一性的问题}{思维能否正确反映存在的问题}
\begin{solution}本题考查的是对哲学基本问题的记忆。哲学的基本问题是思维和存在或者叫做物质和意识的关系问题,所以B为正确。A、C项是哲学基本问题的两个方面的具体内容,不是哲学基本问题本身。详而言之,B项与A、C项之间,是整体和局部的关系。D项是C项的另一种表述方式。
\end{solution}
