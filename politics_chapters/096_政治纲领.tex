\question 新民主主义社会存在五种经济成分,这就是国营经济、合作社经济、个体经济、私人资本主义经济和国家资本主义经济。主要的经济成分有三种:社会主义经济、个体经济和资本主义经济。其中,在国民经济中占绝对优势是
\par\fourch{半社会主义性质的合作社经济}{国家同私人资本合作的国家资本主义国营经济}{通过没收官僚资本而形成的社会主义国营经济}{\textcolor{red}{以农业和手工业为主体的个个体经济}}
\begin{solution}新民主主义社会存在着五种经济成分:社会主义性质的国营经济、半社会主义性质的合作社经济、农民和手工业者的个体经济、私人资本主义经济和国家资本主义经济。其中半社会主义性质的合作社经济是个体经济向社会主义集体经济过渡的形式,国家资本主义经济是私人资本主义经济向社会主义国营经济过渡的形式。所以,主要的经济成分是三种:社会主义经济、个体经济和资本主义经济。在这些经济成分中,通过没收官僚资本而形成的社会主义国营经济,掌握了主要经济命脉,居于领导地位。而以农业和手工业为主体的个体经济,则在国民经济中占绝对优势。
\end{solution}
