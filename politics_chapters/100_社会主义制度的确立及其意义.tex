\question 社会主义基本制度在我国初步确立。社会阶级关系发生变化的同时,我国社会的主要矛盾也发生了变化,社会主要矛盾是(
)
\par\twoch{无产阶级同资产阶级的矛盾}{\textcolor{red}{经济文化的发展不能满足人民需要的矛盾}}{走社会主义道路还是走资本主义道路的矛盾}{人民群众同党内腐败分子的矛盾}
\begin{solution}本题考查的知识点是社会主义基本制度的初步确立。1956年年底我国对农业、手工业和资本主义工商业的社会主义改造基本完成,社会主义基本制度在我国初步确立。我国的社会主义初级阶段从这时开始。人民对于经文化迅速发展的需要同经济文化不能满足人民需要的矛盾成为我国社会的主要矛盾。A,C是过渡时期存在的主要矛盾,已经基本解决。D是当前社会存在的矛盾,但不是主要矛盾,主要矛盾仍然是B项的内容。
\end{solution}
\question 1956年我国对农业、手工业和资本主义工商业的社会主义改造基本完成,这标志着(
)
\par\twoch{\textcolor{red}{阶级剥削制度彻底结束}}{\textcolor{red}{社会主义制度逐步确立,我国进入社会主义初级阶段}}{\textcolor{red}{确立了中国共产党领导的人民民主专政}}{\textcolor{red}{社会主义公有制成为我们社会的经济基础}}
\begin{solution}社会主义改造的基本完成,标志着社会主义制度在中国的确立,实现了中国历史上最深刻、最伟大的社会变革,为中国的社会主义现代化建设奠定了基础。
(1)社会主义改造的胜利,在一个几亿人口的大国中,能够比较顺利地实现消灭私有制这样复杂、困难和深刻的社会变革,不但没有造成生产力的破坏,反而促进了工农业和整个国民经济的发展,并且得到人民群众的普遍拥护而没有引起巨大的社会动荡,这的确是伟大的历史性胜利。
(2)社会主义改造的基本完成,我国社会的经济结构发生了根本变化,几千年来以生产资料私有制为基础的阶级剥削制度基本上被消灭,社会主义经济成了国民经济中的主导成分,社会主义经济制度在中国基本确立。它与1954年召开的第一届全国人民代表大会确立的社会主义政治体制一起,完成了历史上最深刻、最伟大的社会变革,中国从新民主主义社会进入社会主义初级阶段。
(3)中国共产党在实践中把马列主义的基本原理同中国社会主义革命的具体实际相结合,创造性地开辟了一条适合中国特点的社会主义改造道路,以新的经验和思想丰富了马克思主义的科学社会主义理论。
(4)社会主义改造的胜利,大大解放了我国的社会生产力,促进了生产力的发展,为社会主义建设的发展,人民生活水平的提高开辟了广阔的前景。
总之,中华人民共和国的成立和社会主义制度的建立,是20世纪中国历史上的第二次历史性巨变。这是中国从古未有的人民革命的大胜利,为中国的社会主义现代化建设创造了前提,奠定了基础。
在社会主义改造之前我们的政权是无产阶级领导的革命阶级的联合专政,从此之后就是无产阶级民主专政。
\end{solution}
