\question 在中国共产党的历史上,第一次鲜明地提出``马克思主义中国化''的命题和任务的会议是
\par\twoch{党的二大}{遵义会议}{\textcolor{red}{党的六届六中全会}}{党的七大}
\begin{solution}本题考查考生对六届六中全会内容的掌握。在六届六中全会上,毛泽东明确提出马克思主义中国化的口号,号召全党学习马、恩、列、斯的理论,并应用到中国的实际斗争中去。因此,备选项C符合题干要求,为本题答案。备选项A党的二大提出了民主革命纲领;B遵义会议确定了毛泽东在党内的实际地位并独立自主解决了军事问题和组织问题;D党的七大把毛泽东思想确定为党的指导思想。因此备选项ABD不符合题干要求,不是本题正确答案。
\end{solution}
\question 中国革命建设和改革的实践证明,要运用马克思主义指导实践,必须实现
马克思主义中国化,马克思之所以能够中国化的原因在于
\par\twoch{\textcolor{red}{马克思主义理论的内在要求}}{\textcolor{red}{马克思主义与中华民族优秀文化具有相融性}}{\textcolor{red}{中国革命建设和改革的实践需要马克思主义指导}}{马克思主义为中国革命建设和改革提供了现实发展模式}
\begin{solution}马克思之所以能够中国化的原因,首先在于马克思主义中国化是马克思主义理论的内在要求。恩格斯指出:``马克思的整个世界观不是教义,而是方法。它提供的不是现成的教条,而是进一步研究的出发点和供这种研究使用的方法。''各国的马克思主义者的任务就是结合各个国家不同时期的具体实际,将马克思主义进-步加以具体化。因此A选项正确。
马克思主义与中华民族优秀文化具有相融性。马克思主义中国化就是把马克思主义植根于中华民族优秀的思想文化之中,实现马克思主义和民族的特点相结合,并经过一定的民族形式表现出来。因此B选项正确。
马克思之所以能够中国化的原因,还在于马克思主义中国化是解决中国问题的需要。马克思主义中国化就是运用马克思主义解决中国革命、建设和改革的实际问题。在中国这样一个半殖民地半封建的东方大国里进行革命,不能机械套用马克思主义一般原理和照搬外国经验。同样,在中国进行社会主义建设和改革,也不能把马克思主义当做教条,必须紧密结合中国国情和时代条件,使马克思主义在中国具体化。因此C选项正确。
马克思主义具有普遍指导意义,但并不能为中国革命建设和改革提供现成发展模式,因此D选项错误。
\end{solution}
