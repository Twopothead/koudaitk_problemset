\question 否定之否定规律揭示的是( )
\par\twoch{事物发展的动力和源泉}{事物发展的状态和形式}{\textcolor{red}{事物发展的趋势和道路}}{事物发展的过程和结果}
\begin{solution}否定之否定规律揭示了事物发展的趋势和道路。
\end{solution}
\question 否定之否定规律揭示了( )
\par\twoch{事物发展的动力和源泉}{事物发展变化的基本形式和状态}{\textcolor{red}{事物发展的方向和道路}}{事物发展的不同趋势或趋向}
\begin{solution}本题考查对辩证法各个规律特殊作用的理解。选项A是对立统一规律的作用,选项B是质量互变规律的作用,选项D是干扰项。C揭示的是否定之否定规律的作用,所以本题选C。
\end{solution}
\question 辩证唯物主义的辩证否定观认为,辩证的否定是( )
\par\twoch{\textcolor{red}{事物的自我否定}}{\textcolor{red}{联系的环节}}{\textcolor{red}{发展的环节}}{\textcolor{red}{扬弃}}
\begin{solution}辩证否定观的基本内容有:第一,否定是事物的自我否定,是事物内部矛盾运动的结果。第二,否定是发展的环节。它是旧事物向新事物的转变,是从旧质到新质的飞跃。只有经过否定,旧事物才能向新事物转变。第三,否定是联系的环节。新事物孕育产生于旧事物,新旧事物是通过否定环节联系起来的。第四,辩证否定的实质是``扬弃'',即新事物对旧事物既批判又继承,既克服其消极因素又保留其积极因素。
\end{solution}
