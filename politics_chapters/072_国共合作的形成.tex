\question 1924年1月,中国国民党第一次全国代表大会在广州召开,大会通过的宣言对三民主义作出了新的解释,新三民主义成为第一次国共合作的政治基础,究其原因,是由于新三民主义的政纲
\par\fourch{\textcolor{red}{同中国共产党在民主革命阶段的纲领基本一致}}{把斗争的矛头直接指向北洋军阀}{体现了联俄、联共、扶助农工三大革命政策}{把民主主义概括为平均地权}
\begin{solution}本题考查中国近现代史纲要第四章《开天辟地的大事变》中关于第一次国共合作的内容。继1923年中共三大召开,提出建立第一次国共合作统一战线以后,1924年1月,国民党一大在广州召开,大会通过的宣言对三民主义作出了新的解释,即新三民主义,这个新三民主义的政纲同中共在民主革命阶段的纲领基本一致,因而成为国共合作的政治基础。大会实际上确定了联俄、联共、扶助农工三大革命政策。这样,国民党一大的成功召开,就标志着第一次国共合作的正式形成。
\end{solution}
\question 1924年改组后的国民党成为几个阶级的革命联盟,这几个阶级有( )
\par\twoch{\textcolor{red}{工人阶级}}{\textcolor{red}{农民阶级}}{\textcolor{red}{小资产阶级}}{\textcolor{red}{民族资产阶级}}
\begin{solution}孙中山领导的国民党原来大体是代表民族资产阶级和城市小资产阶级的政党。改组后的国民党,从原来的代表资产阶级的政党改变为工人阶级、农民阶级、小资产阶级和民族资产阶级的革命联盟。
\end{solution}
\question 1924年1月,孙中山在中国国民党第一次全国代表大会上对三民主义做出了新的解释,形成了``新三民主义'',孙中山的``新三民主义''(
)
\par\fourch{\textcolor{red}{提出民主权利应“为一般平民所共有”}}{\textcolor{red}{在民族主义中增加了反帝的内容}}{\textcolor{red}{提出要改善工农的生活状况}}{实行民主主义的联合战线}
\begin{solution}大纲原话,正确答案是ABC。
\end{solution}
