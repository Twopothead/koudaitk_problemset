\question 在马克思之前,英国济学的代表人物亚当•斯密已经认识到了商品的二因素.提出了劳动创造价值的观点;大卫•李嘉图甚至已经认识到决定商品价值量的是社会必要劳动量,而不是生产商品实际耗费的劳动量。但是由于他们没有区分劳动二重性,所以不能回答什么劳动创造价值,在价值的源泉等重大理论问题的认识上出现了混乱和错误。马克思在继承英国古典政治经济劳动创造价值的理论的同时,创立了劳动二重性理论,第一次确定了什么样的劳动形成价值,为什么形成价值以及怎样形成价值,阐明了具体劳动和抽象劳动在商品价值形成中额不同作用,从而为揭示剩余价值的真正来源,创立剩余价值理论奠定了基础。一下关于源泉的说法正确的有
\par\fourch{\textcolor{red}{劳动力商品的使用价值是价值的源泉}}{\textcolor{red}{雇佣劳动者剩余劳动是剩余价值的源泉}}{劳动是财富的唯一源泉}{抽象劳动是具体劳动的源泉}
\begin{solution}【答案】AB
【简析】劳动力商品在使用价值上有一个很大的特点.就是它的使用价值是价值的源泉,它在消费过程中能够创造新价值.而且这个新的价值比劳动力本身的价值更大。A正确。雇佣劳动者的剩余劳动是剩余价值产生的唯一源泉,剩余价值既不是由全部资本创造的,也不是由不变资本创造的.而是由可变资本创造的,B正确,劳动是财离的源泉之一,但不是唯一源泉,C
错误。D属于无中生有的干扰项,不选。
\end{solution}
\question 人们往往将汉语中的``价''、``值''二字与金银财宝等联系起来,而这两字的偏旁却都是``人'',示意价值在``人''。马克思劳动价值论透过商品交换的物与物的关系,揭示了商品价值的科学内涵,其主要观点有(
)
\par\twoch{劳动是社会财富的唯一源泉}{具体劳动是商品价值的实体}{\textcolor{red}{价值是凝结在商品中的一般人类劳动}}{\textcolor{red}{价值在本质上体现了生产者之间的社会关系}}
\begin{solution}社会财富不等同于商品价值,文化科学也是社会财富,可是不是生产劳动而来;资源矿藏也是社会财富,但是也不是劳动产生,所以A选项错误。价值对应的是抽象劳动,所以B选项明显错误。CD选项都比较直接,不解释。
\end{solution}
\question 具体劳动是指人们在特定的具体形式下所进行的劳动。抽象劳动是指撇开各种具体形式的一般的、无差别的劳动。具体劳动和抽象劳动是生产商品的同一劳动过程的两个方面,二者的区别在于(
)
\par\fourch{\textcolor{red}{具体劳动是劳动的具体形式,抽象劳动是一般的人类劳动}}{\textcolor{red}{不同的具体劳动的质不同,抽象劳动没有质的差别}}{\textcolor{red}{具体劳动反映人与自然的关系,抽象劳动体现商品生产者之间的关系}}{具体劳动是使用价值的唯一源泉,抽象劳动是价值的唯一源泉}
\begin{solution}D选项错误,使用价值的来源不限于具体劳动,还有劳动对象本身。
\end{solution}
\question 人们往往将汉语中的``价''、``值''二字与金银财宝等联系起来,而这两字的偏旁却都是``人'',示意价值在``人''。马克思劳动价值论透过商品交换的物与物的关系,揭示了商品价值的科学内涵,其主要观点有
\par\twoch{劳动是社会财富的唯一源泉}{具体劳动是商品价值的实体}{\textcolor{red}{价值是凝结在商品中的一般人类劳动}}{\textcolor{red}{价值在本质上体现了生产者之间的社会关系}}
\begin{solution}本题考查的是商品价值的内涵。答案A是错误的,劳动是价值的唯一源泉,社会财富是属于使用价值的范畴,使用价值的源泉是原材料和人类劳动。答案B也是错误的,具体劳动是商品使用价值的实体,抽象劳动是商品价值的实体。答案C和D是正确的。价值是凝结在商品中无差别的人类劳动,是商品交换的基础,本质上体现了生产者之间的社会关系。
\end{solution}
