\question 近代以来中华民族面临的两大历史任务,就是争取民族独立、人民解放和实现国家富强、人民富裕。关于两大历史任务关系,下列说法错误的是
\par\fourch{两大历史任务领导阶级都一样}{前一个任务是从根本上推翻半殖民地半封建的统治秩序,改变落后的生产关系和上层建筑;后一个任务是要改变近代中国经济、文化落后的地位和状况,发展社会生产力,实现中国的现代化}{前一个任务为后一个任务扫除障碍,创造必要的前提;后一个任务是前一个任务的最终目的和必然要求}{\textcolor{red}{相互区别又紧密联系,两大历史任务的主题、内容与实现方式都一样}}
\begin{solution}【解析】两个历史任务都是由无产阶级领导。第一项任务是1949年新中国成立,标志着取得民族独立、人民解放的任务的完成。1949年以后至目前,我们正在进行的是第二项历史任务。两个历史任务领导阶级都是无产阶级------通过其先锋队中国共产党领导的。因此,A选项正确。近代以来中华民族面临的两大历史任务,就是争取民族独立、人民解放和实现国家富强、人民富裕。它们是相互区别又紧密联系的。两大历史任务的主题、内容与实现方式都不一样,因此D选项符合题意。前一个任务是从根本上推翻半殖民地半封建的统治秩序,改变落后的生产关系和上层建筑;后一个任务是要改变近代中国经济、文化落后的地位和状况,发展社会生产
力,实现中国的现代化。前一个任务为后一个任务扫除障碍,创造必要的前提;后一个任务是前一个任务的最终目的和必然要求。综上,ABC选项观点正确。
\end{solution}
