\question 党的十一届三中全会以后,邓小平在总结历史经验教训的基础上,对社会主义社会的基本矛盾,特别是社会主义初级阶段的主要矛盾进行了深人的思考,在新的实践中丰富和发展了这一理论。其主要内容有
\par\fourch{\textcolor{red}{判断一种生产关系和生产力是否相适应,主要希它是否适应当时当地生产力的耍求,能否推动生产力发展}}{\textcolor{red}{提出在社会主义社会依然有解放生产力的问题}}{\textcolor{red}{社会主义社会基本矛盾、主要矛盾和根本任务是统一的,它们要求必须把经济建设作为党和国家的工作重心,不断解放和发展生产力}}{\textcolor{red}{改革是社会主义制度下解放和发展生产力的必由之路}}
\begin{solution}【简析】A、C、D是明显的正确选项。过去只讲在社会主义条件下发展生产力,没有讲还要通过改革解放生产力,不完全,邓小平在社会主义本质理论中首先就提到解放生产力,B也正确。
\end{solution}
\question 中共十八届五中全会通过的《中共中央关于全面深化改革若干重大问题的决定》提出,全面深化改革的总目标是
\par\fourch{完善和发展社会主义市场经济体制	}{坚持和发展社会主义基本经济制度}{\textcolor{red}{推进国家治理体系和治理能力现代化}}{\textcolor{red}{完善和发展中国特色社会主义制度}}
\begin{solution}【简析】全面深化改革的总目标是完善和发展中国特色社会主义制度,推进国家治理体系和治理能力现代化。C、D正确。
\end{solution}
\question 习近平总书〖己在系列重要讲话中提出,全面深化改革要把握和处理好的一些重大关系,包括处理好解放思想和实事求是的关系、整体推进和重点突破的关系、全局和局部的关系、顶层设计和摸着石头过河的关系、胆子要大和步子耍稳的关系、改革发展和稳定的关系,等等。改革、发展、稳定的统一是我国社会主义现代化建设的三个重要支点。正确把握和处理好三者的关系,必须
\par\fourch{\textcolor{red}{把改革力度、发展速度和社会可承受程度统一起来}}{\textcolor{red}{把改善人民生活作为正确处理改革、发展、稳定关系的重要结合点}}{把保持社会稳定作为推进改革和发展的根本出发点和落脚点}{\textcolor{red}{在保持社会稳定中推进改革和发展,通过改革发展促进社会稳定}}
\begin{solution}【简析】改革、发展、稳定的统一是我国社会主义现代化建设的H个重要支点,必须正确处理好3者之间的关系。改革是经济社会发展的强大动力,发展是解决一切绍济社会问题的关键,稳定是改革发展的前提。十多年来,我国改革开放的实践充分证明,只有社会稳定,改革发展才能不断推进;只有改革、发展不断推进,社会稳足才能其有坚实基础。要坚持改革、发展、稳定的统一,把改革力度、发展速度和社会可承受程度统一起来,把改善人民生活作为正确处理改革、发展、稳定关系的重要结合点,在保持社会稳定中推进改革和发展,通过改革发展促进社会稳定。A、B、D正确,C错误。
\end{solution}
