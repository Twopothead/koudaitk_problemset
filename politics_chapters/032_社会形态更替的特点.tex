\question 生产资料所有制是( ~)
\par\fourch{衡量生产力水平的客观标志}{\textcolor{red}{生产关系结构中的决定因素}}{\textcolor{red}{区分社会经济制度的根本标志}}{衡量社会道德水平的客观标志}
\begin{solution}生产资料所有制关系是生产关系中的决定因素,它决定着生产中人与人的关系,决定着产品的分配,因而决定着生产关系的性质,从而也就决定着社会经济制度的性质,成为区分社会经济制度的根本标志。生产工具是衡量生产力水平和社会经济发展水平的客观标志;社会进步的程度是衡量社会道德水平的客观标志,因此AD不能选。
\end{solution}
\question 历史唯物主义认为,阶级斗争在阶级社会中的作用是( )
\par\twoch{\textcolor{red}{社会形态更替的杠杆}}{\textcolor{red}{迫使统治阶级作出某些让步的重要手段}}{社会发展的根本动力}{\textcolor{red}{社会发展的直接动力}}
\begin{solution}此题考查的知识点是社会发展的动力和一般规律问题。阶级斗争在阶级社会发展中的巨大作用,突出的表现在社会形态的更替过程中,起到杠杆的作用。所以A项是正确的。阶级斗争的作用,还表现在同一社会形态内部发展的量变过程中,不断地给统治阶级以这样那样的打击,使得统治阶级不得不对被统治阶级作出某些让步,所以B项也是正确的。社会发展的根本动力是社会基本矛盾,所以C项不符题意,是错误选项。阶级斗争是阶级社会发展的直接动力,所以D项是正确的。
\end{solution}
\question 哲学家孔德认为:``人们必须认识到,人类进步能够改变的只是其速度,而不会出现任何发展顺序的颠
倒或越过任何重要的阶段。''对他的这一看法,分析正确的有( )
\par\fourch{\textcolor{red}{他否认社会形态更替的统一性和多样性的辩证统一}}{\textcolor{red}{他否认社会形态更替的客观必然性与历史主体选择性的统一}}{他的这一观点具有辩证法的倾向}{\textcolor{red}{他没有认识到社会形态的更替是前进性和曲折性的统一}}
\begin{solution}此题考查的知识点是社会形态更替问题上的辩证法。孔德看到了历史发展有其规律,人类进步是可能的,但他把这种进步过程看成是严格按照固定顺序进行的,而没有看到历史主体应具有的能动性和创造性,及由此而来的社会形态更替中的统一性和多样性、前进性和曲折性、历史必然性与主体选择性的统一,具有典型的形而上学倾向。
\end{solution}
\question 列宁曾指出:``世界历史发展的一般规律,不仅丝毫不排斥个别发展阶段在发展的形式或顺序上表现出特殊性,反而是以此为前提的。''社会发展过程中的这种统一性基础上的多样性,充分显示出人类以及各个民族解决自身矛盾的能力及其创造性。社会发展的决定性和主体的选择性使社会发展过程呈现出统一性和多样性。主要表现有
\par\fourch{\textcolor{red}{从纵向看,表现为社会形态更替的统一性和多样性}}{\textcolor{red}{统一性是社会形态运动由低级到高级、由简单到复杂的过程,人类的总体历史过程表现为五种社会形态的依次更替}}{\textcolor{red}{多样性是指不同的民族可以超越一种或几种社会形态而跳跃式地向前发展}}{\textcolor{red}{从横向看,表现为同类社会形态既有共同的本质,又有各自的特点}}
\begin{solution}【解析】社会发展的决定性和主体的选择性使社会发展过程呈现出统一性和多样性。它表现在两个方面:从纵向看,表现为社会形态更替的统一性和多样性。统一性是社会形态运动由低级到高级、由简单到复杂的过程,人类的总体历史过程表现为五种社会形态的依次更替。多样性是指不同的民族可以超越一种或几种社会形态而跳跃式地向前发展。社会形态更替的多样性并不能否定人类总体历史过程。某些民族可以实现跨越,但其跨越的方向、跨越的限度是受总体历史进程制约的。也就是说,跨越的方向要同人类总体历史进程相一致;实际存在着的社会形态及其生产力规定着跨越的限度,现实存在的较先进的社会形态对跨越具有导向作用。从横向看,社会发展过程的统一性和多样性表现为同类社会形态既有共同的本质,又有各自的特点。在现实社会中,每一种社会形态在不同的民族那里都有自己的特殊表现形式。一般来说,不同民族总是自觉或不自觉地依据本民族的特点、历史传统以及国际环境,来选择、设计、创造自己的社会存在形式。中国越过资本主义社会形态,直接走向社会主义,既是历史的必然,又是中国人民的自觉选择。从当代中国实际和时代特征出发,建设中国特色社会主义,同样既是历史的必然,又是中国人民新的自觉选择和伟大创造。据此,本题选ABCD。
\end{solution}
\question 不同民族总是自觉或不自觉地依据本民族的特点、历史传统以及国际环境,来选择、设计、创造自己的社会存在形式。中国越过资本主义社会形态,直接走向社会主义,既是历史的必然,又是中国人民的自觉选择。从当代中国实际和时
代特征出发,建设中国特色社会主义,同样既是历史的必然,又是中国人民新的
自觉选择和伟大创造。这一自觉选择
\par\fourch{\textcolor{red}{要以社会发展的客观必然性造成了一定历史阶段社会发展的基本趋势为基础}}{\textcolor{red}{说明社会形态更替的规律也是人们自己的社会行动的规律}}{有利于避免社会前进过程中所出现的反复、停滞和倒退现象}{\textcolor{red}{归根到底是人民群众的选择性}}
\begin{solution}ABD不同民族总是自觉或不自觉地依据本民族的特点、历史传统以及国际环境,来选择、设计、创造自己的社会存在形式。中国越过资本主义社会形态,直接走向社会主义,既是历史的必然,又是中国人民的自觉选择。从当代中国实际和时代特征出发,建设中国特色社会主义,同样既是历史的必然,又是中国人民新的自觉选择和伟大创造。
这说明社会形态更替的规律也是人们自己的社会行动的规律。规律的客观性并不否定人们历史活动的能动性,并不排斥人们在遵循社会发展规律的基础上,对于某种社会形态的历史选择性。人们的历史选择性包含三层意思:第一,社会发展的客观必然性造成了一定历史阶段社会发展的基本趋势,为人们的历史选择提供了基础,范围和可能性空间。第二,社会形态更替的过程也是一个合目的性与合规律性相统一的过程。第三,人们的历史选择性,归根到底是人民群众的选择性。人们对于社会形态的历史选择,最终取决于人民群众的根本利益、根本意愿以及对社会发展规律的把握顺应程度。
社会发展过程的曲折性是指社会前进过程中所出现的反复、停滞和倒退现象。曲折前进是历史的普遍规律。列宁说:``设想世界历史会一帆风顺、按部就班地向前发展,不会有时出现大幅度的退,那是不辩证的,不科学的,在理论上是不正确的。''所以C不对。
\end{solution}
\question 马克思说``无论哪一种社会形态,在它所能容纳的全部生产力发挥出来之前,是绝不会灭亡的;而新的更高的生产关系,在它存在的物质条件在旧社会的胞胎力成熟以前,是绝不会出现的。''这段话说明(
~)
\par\fourch{\textcolor{red}{生产力的发展是促使社会形态更替的最终原因}}{\textcolor{red}{一种新的生产关系的产生需要客观的物质条件的成熟}}{\textcolor{red}{无论哪一种社会形态,当它还能促进生产力发展时,是不会灭亡的}}{\textcolor{red}{社会形态总是具体的、历史的}}
\begin{solution}两个绝不会体现生产力和生产关系的矛盾运动推动社会发展。
\end{solution}
\question 人们的历史选择性的含义有( )
\par\fourch{\textcolor{red}{人的历史选择以社会发展的客观必然性为基础}}{\textcolor{red}{人们的历史选择性归根结底是人民群众的选择性}}{\textcolor{red}{社会形态更替的过程是一个合目的性与合规律性相统一的过程;目的性必须符合规律性}}{人的历史选择性决定了社会发展的多样性}
\begin{solution}该题目是客观规律和人的主观能动性的关系在历史观中的体现。社会发展的多样性是由各国不同的国情决定的。所以D为错误。
\end{solution}
\question 1989年,时任美国国务院顾问的弗朗西斯●福山抛出了所谓的``历史终结论'',认为西方实行的自由民主制度是``人类社会形态进步的终点''和
``人类最后一种的统治形式''。然而,20年来的历史告诉我们,终结的不是历史,而是西方的优越感。就在柏林墙倒塌20年后的2009年11月9日,BBC
公布了一份对27国民众的调查。结果半数以上的受访者不满资本主义制度,此次调查的主办方之一的``全球扫描''公司主席米勒对媒体表示,这说明随着1989年柏林墙的倒塌资本主义并没有取得看上去的压倒性胜利,这一点在这次金融危机中表现的尤其明显,``历史终结论''的破产说明
\par\fourch{社会规律和自然规律一样都是作为一种盲目的无意识力量起作用}{\textcolor{red}{人类历史的发展的曲折性不会改变历史发展的前进性}}{\textcolor{red}{一些国家社会发展的特殊形式不能否定历史发展的普遍规律}}{\textcolor{red}{人们对社会发展某个阶段的认识不能代替社会发展的整个过程}}
\begin{solution}``历史终结论''的破产说明,人类历史的发展的曲折性不会改变历史发展的前进性,一些国家社会发展的特殊形式不能否定历史发展的普遍规律,人们对社会发展某个阶段的认识不能代替社会发展的整个过程。但是,社会规律和自然规律是有相异之处的,社会规律是人有意识的能动活动,自然规律是盲目的无意识的力量起作用,所以,正确答案是选项BCD。
\end{solution}
