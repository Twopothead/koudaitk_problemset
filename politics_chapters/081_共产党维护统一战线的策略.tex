\question 1941年1月,震惊中外的皖南事变爆发后,《新华日报》刊出周恩来的题词手迹:
``为江南死国难者致哀!''``千古奇冤,江南一叶;同室操戈,相煎何急?!''大敌当
前,中国共产党以民族利益为重,坚持正确的方针和原则,避免了抗日民族统一战
线的破裂。这些方针和原则包括( )
\par\twoch{\textcolor{red}{又联合又斗争}}{\textcolor{red}{有理、有利、有节}}{针锋相对,寸土必争}{\textcolor{red}{发展进步势力,争取中间势力,孤立顽固势力}}
\begin{solution}为了抗日民族统一战线的坚持、扩大和巩固,中国共产党制定了``发展进步
势力,争取中间势力,孤立顽固势力''的策略总方针。以蒋介石集团为代表的国民
党亲英美派采取两面政策,既主张抗日,又防共、限共、溶共、反共。为此,共产
党必须以革命的两面政策来应对,即贯彻又联合又斗争的政策,同顽固派作斗争,
坚持有理、有利、有节的原则。因此,正确答案为ABD。
\end{solution}
\question 抗日战争进入相持阶段以后,团结抗战的局面逐步发生严重危机,1939年1月,国民党五届五中全会决定成立``防共委员会'',确定了``防共、限共、溶共、反共''的方针。为避免抗日民族统一战线内部分裂,中共中央提出的应对危机的口号是(
)
\par\twoch{\textcolor{red}{力求全国进步,反对向后倒退}}{发展进步势力,孤立顽固势力}{\textcolor{red}{巩固国内团结,反对内部分裂}}{\textcolor{red}{坚持抗战到底,反对中途妥协}}
\begin{solution}针对国民党的消极抗日,中共提出``力求全国进步,反对向后倒退'',``巩固国内团结,反对内部分裂''``坚持抗战到底,反对中途妥协''。属于大纲识记内容。
\end{solution}
\question 毛泽东曾指出:``在中国,这种中间势力有很大的力量,往往可以成为我们同顽固派斗争时决定胜负的因素,因此,必须对他们采取十分慎重的态度。''在实际工作中,毛泽东始终将争取中间势力作为中国共产党在抗日民族统一战线中的一项极其严重的任务,他认为争取中间势力需要一定的条件,包括(
)
\par\twoch{\textcolor{red}{共产党要有足够的力量}}{\textcolor{red}{尊重他们的利益}}{\textcolor{red}{要同顽固派作坚决的斗争,并能一步一步地取得胜利}}{采取“有理、有利、有节”策略}
\begin{solution}中共针对中间势力是争取中间势力,对顽固势力才是``有理、有利、有节''的策略。
\end{solution}
\question 1941年1月,震惊中外的皖南事变爆发后,《新华日报》刊出周恩来的题词手迹:``为江南死国难者致哀。''``千古奇冤,江南一叶,同室操戈,相煎何急?''大敌当前,中国共产党以民族利益为重,坚持正确的方针和原则,避免了抗日民族统一战线的破裂,这些方针和原则有
\par\twoch{\textcolor{red}{既联合又斗争}}{\textcolor{red}{有理,有利,有节}}{针锋相对,寸土必争}{\textcolor{red}{发展进步势力,中间势力,孤立顽固势力}}
\begin{solution}本题考查中国共产党巩固抗日民族统一战线的方针和原则。抗日战争时期,中国共产党处理民族矛盾和阶级矛盾的原则是:阶级矛盾服从于民族矛盾。因此,皖南事变后,中国共产党为了巩固统一战线,争取更多的人参加抗日,采取的方针和原则有:又联合又斗争,发展进步势力,争取中间势力,孤立顽固势力的策略总方针,以及同顽固派的斗争,坚持有理、有利、有节的策略原则。因此,选项ABD正确。选项C属于当时国内处理阶级斗争的对策。
\end{solution}
