\question 人民法院判处犯罪分子和犯罪的单位,向国家缴纳一定金钱的刑罚方法,称为(
)
\par\twoch{管制}{\textcolor{red}{罚金}}{罚款}{没收财产}
\begin{solution}罚款和罚金都是国家机关强制违法行为者在一定期限内向国家缴纳一定数量现金的处罚方法,但两者存在区别:罚金是由人民法院判处犯罪分子或犯罪单位向国家缴纳一定数额金钱的刑罚方法。罚款分为两种:一种是由人民法院依据民法或诉讼法作出的,罚款的对象是有妨碍民事诉讼行为的人;另一种是由公安机关或者其他行政机关依照行政法规作出的,罚款的对象是违反治安管理或者是违反海关、工商、税收等这样一些行政法规的人。所以,罚款的适用对象比罚金要宽得多。没收财产是将犯罪分子个人所有财产的一部分或者全部强制无偿地收归国有的刑罚方法。因此,B正确。
\end{solution}
