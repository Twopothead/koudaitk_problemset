\question 新民主主义社会属于社会主义体系,是因为社会主义因素在政治和经济上都居于领导地位,这种领导地位主要体现在(
)
\par\twoch{\textcolor{red}{无产阶级领导的联合专政}}{农业和手工业在国民经济中占绝对优势}{\textcolor{red}{社会主义国营经济掌握了主要经济命脉}}{土地改革基本完成}
\begin{solution}这个里面强调社会因素的重要性表现在什么地方,农业和手工业不属于社会主义经济属于个体经济,体现不出社会因素的领导地位。土地改革是民主革命的任务,也体现不出社会主义因素的重要性。
\end{solution}
