\question 马克思指出:``资本主义积累不断地并且同它的能力和规模化成比例地生产出相对的,即超过资本增殖的平均需要的,因而是过剩的或追加的工人人口。''``过剩的工人人口是积累或资本主义基础上的财富发展的必然产物,但是这种过剩人口反过来又成为资本主义积累的杠杆,甚至成为资本主义生产方式存在的一个条件。''上述论断表明
\par\fourch{\textcolor{red}{资本主义生产周期性特征需要有相对过剩的人口规律与之相适应}}{\textcolor{red}{资本主义社会过剩人口之所以是相对的,是因为它不为资本价值增殖所需要}}{\textcolor{red}{资本主义积累必然导致工人人口的供给相对于资本的需要而过剩}}{资本主义积累使得资本主义社会的人口失业规模呈现越来越大的趋势}
\begin{solution}资本积累会导致资本有机构成提高,产生相对过剩人口,即失业人口,并形成与资本主义生产周期性特征相适应的相对过剩人口规律。相对过剩人口即劳动力供给超过了资本对它的需求形成的过剩人口。ABC选项正确。
\end{solution}
\question 马克思指出:``资本主义社会的经济结构是从封建社会的经济结构中产生的,后者的解体使前者的要素得到解放。''这说明
\par\fourch{封建社会向资本主义社会过渡是人类社会的最终阶段}{资本主义社会产生与发展的根本原因是商品经济打破自然经济}{共产主义社会是历史发展的必然趋势}{\textcolor{red}{新的更高的生产关系,在它的物质存在条件在旧社会的胎胞里成熟以前,是绝不会出现的}}
\begin{solution}本题马克思语录意在指出,新事物的产生是必须是在旧事物的胎胞里成熟
以前才能出现的。体现新事物产生的原因之一,故选D。
\end{solution}
\question 从历史发展的角度看,资本主义生产资料所有制是不断演进和变化的。当今资本主义社会,居主导地位的资本所有制形式是(
)。
\par\fourch{私人资本所有制}{\textcolor{red}{法人资本所有制}}{私人股份资本所有制}{垄断资本私人所有制}
\begin{solution}【解析】第二次世界大战后,资本主义所有制发生了新的变化:①国家资本所有制形式形成并发挥重要作用;②法人资本所有制崛起并成为居主导地位的资本所有制形式。法人资本所有制是法人股东化的产物,其基本特点是:各类法人取代个人或家族股东成为企业的主要出资人,企业的股票高度集中于少数法人股东手中,法人股东直接参与公司治理,监督和制约管理阶层的经营行为,使公司资本的所有权与控制权重新趋于合一。
\end{solution}
\question 使生产者与生产资料相分离,将货币资本迅速集中于少数人手中的历史过程就是(
)
\par\twoch{资本积累}{资本剥削}{\textcolor{red}{资本原始积累}}{资本集中}
\begin{solution}本题是考查对基本概念的准确记忆。资本积累、资本集中、资本积聚都发生在资本主义社会当中。只有资本的原始积累发生在资本主义社会之前。资本原始积累就是使生产者与生产资料相分离,将货币资本迅速集中于少数人手中的历史过程。这个过程一方面使社会的生活资料和生产资料转化为资本,另一方面使直接生产者转化为雇佣工人。资本的原始积累为资本主义的产生创造了条件。
\end{solution}
