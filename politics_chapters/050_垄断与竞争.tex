\question 垄断是从自由竞争中形成的,是作为自由竞争的对立面产生的,但是,垄断并不能消除竞争,而是凌驾于竞争之上,与之并存。垄断资本主义阶段存在竞争的主要原因是
\par\fourch{\textcolor{red}{垄断没有消除产生竞争的经济条件}}{垄断没有消除产生竞争的政治条件}{\textcolor{red}{垄断必须通过竞争来维持}}{\textcolor{red}{不存在由一个垄断组织囊括一切部门、一切社会生产的绝对垄断}}
\begin{solution}【答案】ACD
【解析】垄断资本主义阶段存在竞争的主要原因:一是垄断没有消除产生竞争的经济条件。竞争是商品经济的一般规律,垄断产生后,没有消除以资本主义私有制为基础的商品经济。二
是垄断必须通过竞争来维持。各个垄断组织通过竞争发展起来,还需要不断增强自己的竞争
实力,巩固自己的垄断地位。三是不存在由一个垄断组织囊括一切部门、一切社会生产的绝
对垄断。在垄断条件下,在垄断组织内部、垄断组织之间、垄断组织同非垄断组织之间以及非垄断的中小企业之间存在着广泛而激烈的竞争。据此,本题选ACD。
\end{solution}
\question 垄断价格包括垄断高价和垄断低价两种形式。垄断高价是指垄断组织出售商品时规定的高于生产价格的价格;垄断低价是指垄断组织在购买非垄断企业所生产的原材料等生产资料时规定的低于生产价格的价格。垄断高价和垄断低价并不否定价值规律,垄断价格的形成只是使价值规律改变了表现形式。因为(
)
\par\fourch{\textcolor{red}{垄断价格不能完全脱离商品的价值}}{按垄断低价的买卖行为,仍然是等价交换}{\textcolor{red}{从整个社会看,商品价格总额和商品价值总额是一致的}}{\textcolor{red}{垄断高价是把其他商品生产者的一部分利润转移到垄断高价的商品上}}
\begin{solution}B垄断高价和垄断低价都不是等价交换。垄断高价是指垄断组织出售商品时规定的高于生产价格的价格;垄断低价是指垄断组织在购买垄断企业所生产的原材料时规定的地域生产价格的价格。
\end{solution}
\question 垄断资本主义的基本经济特征包括
\par\fourch{\textcolor{red}{垄断组织在经济生活中起决定作用}}{\textcolor{red}{资本输出有了特别重要的意义}}{\textcolor{red}{在金融资本的基础上形成金融寡头的统}}{垄断使竞争趋于缓和}
\begin{solution}本题考查``垄断资本主义生产关系的特征''这一知识点的内容。属于基本知识考查。列宁在帝国主义论中把垄断资本主义的基本经济特征概括为五个方面。垄断资本主义的基本经济特征是政治经济学的一个重要理论点,也是考研复习的一个相对比较重要的知识点,但是把垄断资本主义的基本经济特征作为多选题这一考查方式在近10年来的考研当中并不多见,考的相对比较多的是垄断资本主义基本经济特征二战后的新发展和变化,仅从这一角度讲,本题相对比较偏和冷。D选项是错误观点,不选;其他三项A、B、C是正确观点,正确选项。本题考查记忆。
\end{solution}
