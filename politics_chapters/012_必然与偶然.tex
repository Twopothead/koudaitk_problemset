\question 马克思指出``如果`偶然性'不起任何作用的话,那么世界历史就会带有非常神秘的性质。这些偶然性本身纳入总的发展过程,\ldots{}\ldots{}其中包括一开始就站在运动最前面的那些人物的性格这样一种`偶然情况'。''上述论断中指出
\par\fourch{历史是偶然性向必然性转化的过程}{历史的发展纯粹是偶然的}{\textcolor{red}{历史的必然性通过偶然性表现出来}}{\textcolor{red}{历史人物的性格这种偶然因素对历史发展有一定影响}}
\begin{solution}这段论述指出,偶然性其中包括历史人物的性格这种偶然因素,对历史发展有作用,并且历史发展的必然性是通过偶然性表现出来的。但这段论述没有涉及偶然性向必然性的转化,因此A不能选。而B是错误的观点,也不能选。
\end{solution}
\question 偶然性与必然性的关系是( )
\par\twoch{\textcolor{red}{偶然性中包含着必然性}}{\textcolor{red}{必然性制约着偶然性}}{\textcolor{red}{偶然性表现必然性,是必然性的补充}}{\textcolor{red}{必然性通过偶然性为自己发展开辟道路}}
\begin{solution}必然性和偶然性是辩证统一的关系。同一性表现在:其一,二者相互包含。必然性存在于偶然性之中,并通过大量的偶然性表现出来,偶然性为必然性开辟道路;偶然性背后隐藏着必然性,偶然性受必然性支配,是必然性的表现形式和补充。其二,二者在一定条件下可以互相转化。
矛盾性表现在:其一,产生和形成的原因不同。必然性产生于事物内部的根本矛盾,偶然性产生于非根本矛盾和外部条件;其二,表现形式不同。必然性在事物发展过程中比较稳定、时空上比较确定,是同类事物普遍具有的发展趋势;而偶然性则是不稳定的、暂时的、不确定的,是事物发展中的个别表现。其三,它们在事物发展中的地位和作用不同。必然性在事物发展中居于支配地位,决定着事物发展的方向;偶然性居于从属地位,对发展的必然过程起促进或延缓作用,使发展的确定趋势带有一定的特点和偏差。
\end{solution}
