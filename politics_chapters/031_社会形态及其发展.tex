\question ``无论历史的结局如何,人们总是通过每一个人追求他自己的、自觉预期的目的来创造他们的历史,而这许多按不同方向活动的愿望及其对外部世界的各种各样作用的合力,就是历史。''这段话说明(
~)
\par\fourch{社会历史发展无规律可循}{无数个人意志创造社会历史}{\textcolor{red}{社会历史发展具有客观必然性}}{\textcolor{red}{人们自己创造自己的历史}}
\begin{solution}此题考查的知识点是现实的人及其活动与社会历史。这段话的中心词是``合力'',社会历史的发展方向和趋势是由``合力''规定的。``合力''本身对于每一个具体主体而言,是不以其意志为转移的客观力量,``合力''就是客观规律,所以社会历史发展具有客观必然性和规律性(C)。``合力''形成并实现于人的有目的性的实践活动之中,所以人才是社会历史活动的主体(D)。选项B属于社会意识决定社会存在的唯心史观,不符合题干原文的意思。所以正确答案是CD。
\end{solution}
\question 马克思、恩格斯在《共产党宣言》中预言:``资产阶级的灭亡和无产阶级的胜利是同样不可避免的。''这是马克思、恩格斯所揭示的人类社会发展的客观规律。对此,有人提出:``如果资本主义的灭亡是有保证了的、是必然的,为什么还要费那么大的气力去为它安排葬礼呢?''这种观点的错误在于(
~)
\par\fourch{\textcolor{red}{抹杀了社会规律实现的特点}}{\textcolor{red}{否认了革命在社会质变中的作用}}{否认了历史观的决定论原则}{否定了科学是推动历史前进的革命力量}
\begin{solution}C观点错误,D与题干无关。
\end{solution}
\question 正如达尔文发现了物种的起源与进化的规律,马克思发现了人类社会发展的规律,创立了唯物史观,科学地解决了社会存在和社会意识的关系。历史唯物主义的基本原理是
\par\fourch{\textcolor{red}{人类社会是一个自然的、历史发展的过程}}{\textcolor{red}{社会基本矛盾是社会发展的根本动力}}{\textcolor{red}{社会存在决定社会意识,社会意识反作用于社会存在}}{人民群众和英雄是历史的创造者}
\begin{solution}人类社会的发展是一个自然历史过程是唯物史观的一个基本原理,故选项A正确;唯物史观认为社会基本矛盾是社会发展的根本动力,故选项B正确;社会存在和社会意识的关系问题,是社会历史观的基本问题。正确认识这一问题是解决其他社会历史观的基础和前提,所以选项C正确;至于D选项有一半是错误的,人民群众是历史的创造者,这个是历史唯物主义观,但是英雄是历史的创造者是属于唯心主义英雄史观,故D选项错误。
\end{solution}
