\question 1956年9月15日至27日,中国共产党第八次全国代表大会在北京举行。大会正确分析了社会主义改造完成后中国社会的主要矛盾和主要任务,制定了经济建设、政治建设、执政党建设的方针政策,指出
\par\fourch{\textcolor{red}{国内主要矛盾是人民对于经济文化迅速发展的需要同当前经济文化不能满足人民需要的状况之间的矛盾}}{\textcolor{red}{根本任务是在新的生产关系下保护和发展生产力}}{经济建设的方针是“三个主体、三个补充”}{\textcolor{red}{执政党建设上强调健全党内民主集中制,坚持集体领导制度,反对个人崇拜}}
\begin{solution}【简析】中共八大提出:社会主义制度在我国己经基本上建立起来;我们还必须为解放台湾、为彻底完成社会主义改造、最后消灭剥削制度和继续肃清反革命残余势力而斗争,但是国内主要矛盾己经不再是工人阶级和资严阶级的矛盾,而是人民对于经济文化迅速发展的需要同当前经济文化不能满足人民需要的状况之间的矛盾;全国人民的主要任务是集中力量发展社会生产力,实现国家工业化,逐步满足人民日益增长的物质和文化需要,虽然还有阶级斗争,还要加强人民民主专政,但其根本任务己经是在新的生产关系下保护和发展生产力。在执政党建设上,强调要提高全党的马克思列宁主义思想水平,健全党内民主集中制,坚持集体领导制度,反对个人崇拜,发展党内民主和人民民主,加强党和群众的
联系。A、B、D正确。在经济建设上,人会坚持既反保守又反冒进即在综合平衡中稳步前进的方针。陈云在人会发言中,提出``三个主体、三个补充''的思想,即:国家经营和集体经营是主体,一定数量的个体经营为补充;计划生产是主体,一定范围的自由生产为补充;国家市场是主体,一定范围的自由市场为补充。这个思想为大会所采纳,并写入决议,成为突破传统观念、探索适合中国特点的经
济体制的重要步骤。C 不符合题意。
\end{solution}
\question 中共八大指出,党和全国人民当前的主要任务是( ~)
\par\fourch{正确处理人民内部矛盾}{实现社会主义四个现代化}{把我国推进到社会主义社会}{\textcolor{red}{把我国从落后的农业国变为先进的工业国}}
\begin{solution}正确处理人民内部矛盾是中共八大以后,毛泽东在《关于正确处理人民内部矛盾的问题》中提出的;实现社会主义四个现代化是在十三届人大上提出的;三大改造完成以后,我国已经进入社会主义社会。
\end{solution}
\question 1956年11月召开的中共八届二中全会上正式提出整风,其要反对的错误倾向有(
~)
\par\twoch{\textcolor{red}{主观主义}}{\textcolor{red}{宗派主义}}{教条主义}{\textcolor{red}{官僚主义}}
\begin{solution}1957年4月27日,中共中央下发《关于整风运动的指示》,指出:由于党在全国范围内处于执政地位,有必要在全党进行一次反对官僚主义、宗派主义和主观主义的整风运动。教条主义是延安整风运动时主观主义的表现之一。
\end{solution}
