\question 我国新民主主义革命时期,一块块革命根据地的建立,相对于全国处于半封建半殖民地社会而言,这是极有重大意义的事情。这句话表明(
)
\par\twoch{\textcolor{red}{量变中渗透着质变}}{质变中渗透着量变}{\textcolor{red}{量变过程中部分质变的存在}}{质变过程中量的扩展}
\begin{solution}本题考查的知识点:量变和质变的辩证关系
唯物辩证法认为,量变指的是事物数量的变化,体现了事物渐进过程中的连续性;质变是事物根本性质的变化,是事物由一种质态向另一种质态的飞跃,体现了事物渐进过程和连续性的中断。量变引起质变有两种情形:(1)事物在数量上的增减,如在大小、速度、程度和规模等方面的变化引起质变。比如,水滴石穿,冰冻三尺非一日之寒,勿以恶小而为之、勿以善小而不为,千里之堤、溃于蚁穴。(2)事物在总体数量不变的情况下,由于构成事物的成分在结构和排列次序上发生变化而起质变。由此可以判定题干这句话在总的过程上说的是量变,当时全国性质在总的方面还是半封建半殖民地社会。同时,量变和质变是相互渗透的,在总的量变过程中有阶段性和局部性的部分质变。而革命根据地的建立就是量变过程中的局部的部分质变,是总的量变中渗透着的质变。所以,本题的正确答案是选项AC。
选项BD错误。当时全国性质在总的方面还是半封建半殖民地社会,没有实现质变。
\end{solution}
\question 同是一块石头,一半做成了佛像,一半做成了台阶。一天,台阶不服气地问佛像:``我们本是一块石头,凭什么人们都踩着我,而去朝拜你呢?
''佛说:``因为你只挨了一刀,而我经历了千刀万割。''下列与``一刀是阶,千刀成佛''体现一致原理的名言是(
)
\par\fourch{\textcolor{red}{大厦之成,非一木之材也;大海之阔,非一流之归也}}{\textcolor{red}{不矜细行,终累大德}}{\textcolor{red}{凿井者,起于三寸之坎,以就万忉之深}}{祸兮福之所倚,福兮祸之所伏}
\begin{solution}【解析】做这种题目最主要的是搞清楚古语的含义。A选项古语典出冯梦龙《东周列国志》,原文:
``臣闻大厦之成,非一木之材也;大海之阔,非一流之归也。''意思是:高大的房屋建筑的建成,不是靠
一棵树的木材原料就能做到的;大海之所以辽阔,不是靠一条河流的水注人进来就能形成辽阔态势
的。``不矜细行,终累大德'',语出《尚书•旅獒》,意思是:不顾惜小节方面的修养,到头来会伤害大节,酿成终生的遗憾。``凿井者,起于三寸之坎,以就万仞之深''。意思是:凿井的人,从挖很深的土坑开始,慢慢才能形成极深的井。ABC选项中的名言都体现了说明万事起于忽微,量变引起质变,
符合题意。D选项体现福祸转化的矛盾对立统一,不符合题意。
\end{solution}
\question 关于唯物辩证法的``度''的概念的正确的论断是( )
\par\twoch{度就是事物变化的关节点}{\textcolor{red}{度是事物保持自己质的数量界限}}{\textcolor{red}{质变就是对度的规定性的突破}}{\textcolor{red}{度是区分事物量变和质变的标准}}
\begin{solution}本题考查对度的概念的全面理解。度是保持事物质的稳定性的量的规定性,因此度不是关节点,关节点是度的两端。
\end{solution}
\question 量变和质变是辩证统一关系,其同一性表现在( )
\par\twoch{量变必然引起质变}{\textcolor{red}{量变是质变的必要准备}}{\textcolor{red}{质变是量变的必然结果}}{\textcolor{red}{量变和质变相互渗透}}
\begin{solution}量变和质变是事物的联系和发展所采取的两种状态和形式,二者之间是辩证同一关系。其同一性表现在:第一,量变是质变的必要准备。任何事物的变化都有一个量变的积累过程,没有量变的积累,质变就不会发生。第二,质变是量变的必然结果。量变达到一定程度、突破了度,必然引起质变。第三,量变和质变是相互渗透的。一方面,在总的量变过程中有阶段性和局部性的部分质变;另一方面,在质变过程中也有旧质在量上的收缩和新质在量上的扩张。量变必须累积到一定程度,突破了``度''的界限,才能引起质变。所以A项错误。
\end{solution}
\question 有一则箴言:``在溪水和岩石的斗争中,胜利的总是溪水,不是因为力量,而是因为坚持。''``坚持就是胜利''的哲理在于
\par\twoch{必然性通过偶然性开辟道路}{肯定中包含着否定的因素}{\textcolor{red}{量变必然引起质变}}{有其因必有其果}
\begin{solution}坚持就是胜利,体现了事物量变发展到一定阶段必然会引起质变,达到事物根本性质的变化,所以,本题体现的是量变必然引起质变,正确答案是选项C。
\end{solution}
