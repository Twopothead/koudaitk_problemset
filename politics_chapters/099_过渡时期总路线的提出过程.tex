\question 1953年12月,由毛泽东审阅通过的中共中央宣传部编写的《为动员一切力量把我国建设成为一个伟大的社会主义国家而斗争一一关于党在过渡时期总路线的学习和宣传提纲》中指出:我国由新民主主义社会逐步过渡到社会主义社会这一过渡时期之所以必要,并且需要一个相当长的时间,是由于
\par\fourch{新民主主义社会是一个独立的社会形态,要求一个相当长的时期逐步过渡到社会主义社会}{\textcolor{red}{我国经济和文化的落后,要求一个相当长的时期来创造为保证社会主义完全胜利所必要的经济上和文化上的前提}}{\textcolor{red}{我国有极其广大的个体农业和手工业及在国民经济中占很大一部分比重的资本主义工商业,要求一个相当长的时期来改造他们}}{在我国新民主主义社会中,非社会主义的因素不论在经济上还是在政治上都还居于领导地位,要求一个相当长的时期来改造他们}
\begin{solution}BC
1953年12月,由毛泽东审阅通过的中共中央宣传部编写的《为动员一切力量把我国建设成为一个伟大的社会主义国家而斗争一一关于党在过渡时期总路线的学习和宣传提纲》中指出:我国由新民主主义社会逐步过渡到社会主义社会这一过渡时期之所以必要,并且需要一个相当长的时间,是由于:``一、我国经济和文化的落后,要求一个相当长的时期来创造为保证社会主义完全胜利所必要的经济上和文化上的前提;二、我国有极其广大的个体农业和手工业及在国民经济中占很大一部分比重的资本主义工商业,要求一个相当长的时期来改造他们。''在我国新民主主义社会中,社会主义的因素不论在经济上还是在政治上都已经居于领导地位,但非社会主义因素仍有很大的比重。由于社会主义因
素居于领导地位,加上当时有利于发展社会主义的国际条件,决定了社会主义因
素将不断增长并获得最终胜利,非社会主义因素将不断受到限制和改造。为了促进社会生产力的进一步发展,为了实现国家富强、民族复兴、人民幸福,我国新民主主义社会必须适时地逐步过渡到社会主义社会。新民主主义社会是属于社会主义体系的,是逐步过渡到社会主义社会的过渡性质的社会。AD是干扰项。
\end{solution}
