\question 香港、澳门问题是历史上殖民主义侵略遗留下来的问题。香港是被英国殖民主义者通过向中国发动侵略战争,强迫清政府先后签订哪些不平等条约强占的
\par\twoch{\textcolor{red}{《南京条约》}}{《虎门条约》}{\textcolor{red}{《北京条约》}}{\textcolor{red}{《展拓香港界址专条》}}
\begin{solution}【答案】ACD
【解析】港澳台关系今年是个热点问题,《史纲》可能从历史角度去命题。同学们需要了解一下相关的背景知识。香港是被英国殖民主义者通过向中国发动侵略战争,强迫清政府先后签订
《南京条约》《北京条约》《展拓香港界址专条》等不平等条约而强占的。新中国成立后,中国共产党和中国政府根据当时的国际国内形势,对香港、澳门这两个地区采取了如下立场:香
港、澳门地区是中国的领土,不承认外国强加在中国人民头上的不平等条约;对于这一历史遗
留下来的问题,将在适当时机通过谈判予以解决;未解决之前暂时维持现状。1997年,香港回归祖国的怀抱。
\end{solution}
\question 帝国主义侵略中国的最终目的,是要瓜分中国、灭亡中国。1895年中国在甲午战争中战败后,列强掀起了瓜分中国的狂潮,这集中表现在(
)。
\par\fourch{从侵占中国周边邻国发展到蚕食中国边疆地区}{设立完全由外国人直接控制和统治的租界}{外国资本在中国近代工业中争夺垄断地位}{\textcolor{red}{竞相租借港湾和划分势力范围}}
\begin{solution}帝国主义列强对中国的争夺和瓜分的图谋,在1894年中日甲午战争爆发后达到高潮。《中日马关条约》的签订,更大大刺激了帝国主义列强瓜分中国领土的野心,并激化了列强争夺中国的矛盾。俄国法国德国三国干涉还辽,要求租借中国港湾作为报酬。由此,德、俄、英、法、日等国于1898年至1899年竞相租借港湾和划分势力范围,掀起了瓜分中国才狂潮。
\end{solution}
\question 标志着中国完全沦为半殖民地半封建社会的不平等条约是( ~)
\par\twoch{《南京条约》}{《北京条约》}{《马关条约》}{\textcolor{red}{《辛丑条约》}}
\begin{solution}通过《辛丑条约》,帝国主义列强强迫清政府做出永远禁止中国人成立或加入任何反对它们的组织的承诺,并规定清政府各级官员如对人民反抗斗争``弹压惩办''不力,``即行革职,永不叙用''。中外反动势力彻底勾结在一起,中国完全沦为半殖民地半封建社会。
\end{solution}
