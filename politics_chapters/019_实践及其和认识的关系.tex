\question 宋代诗人陆游在一首诗中说:``纸上得来终觉浅,绝知此事要躬行。''这是在强调
\par\twoch{实践是认识发展的动力}{实践是认识的最终目的和归宿}{\textcolor{red}{实践是认识的来源}}{学习获得的间接经验并不重要}
\begin{solution}此题考查的知识点是实践是认识的来源。题干的意思是,从书本上得到的知识毕竟比较肤浅,要透彻地认识事物还必须亲自实践。对于认识来源于实践,不能作狭溢的理解。首先,认识来源于实践并不否认人的大脑和感官在生理素质上的差异对认识的影响。其次,认识来源于实践并不否认学习间接经验的必要性和重要性。所以本题选C。
\end{solution}
\question 辩证唯物主义认为实践是认识发展的动力,这是因为( )
\par\fourch{\textcolor{red}{实践的发展不断提出认识的新课题}}{实践是检验认识真理性的唯一标准}{\textcolor{red}{实践锻炼和提高了认识主体的认识能力}}{\textcolor{red}{实践为认识的发展提供了必要条件}}
\begin{solution}此题考查的知识点是实践是认识发展的动力。辩证唯物主义认为实践是认识发展的动力,这是因为:首先,实践的发展不断提出认识的新课题,推动着认识向前发展。其次,实践为认识发展提供必要条件。一方面,实践的发展不断揭示客观世界越来越多的特性,为解决认识上的新课题积累越来越丰富的经验材料;另一方面,实践又提供日益完备的物质手段,不断强化主体的认识能力。最后,实践锻炼和提高了主体的认识能力。恩格斯说:``人的智力是按照人如何学会改变自然界而发展的。''B是检验真理的标准,跟认识发展的动力无关。
\end{solution}
\question 1930年5月2日至6月5日,毛泽东在寻乌作了
20多天的社会调查,开了10多天的调查会,写下了《反对本本主义》和《寻乌调查》两篇论著。首次提出了``没有调查,没有发言权''的科学论
断,还提出了``到群众中作实际调查去''``中国革命斗争的胜利要靠中国同志了解中国情况''等思想路线,初步形成了毛泽东思想活的灵魂的三个基本点,即实事求是、群众路线和独立自主的思想,寻乌由此成为党的实事求是思想路线的发祥地。毛泽东在《反对本本主义》一文中强调调查就像`十月怀胎',解决问题就像`一朝分娩''',其中蕴含的哲理主要是(
)
\par\twoch{实践是认识的来源}{\textcolor{red}{量变是质变的必要准备,质变是量变的必然结果}}{可能与现实}{本质与现象}
\begin{solution}【解析】本题比较简单。``调查就像`十月怀胎'
'',体现了量变是质变的必要准备,``解决问题就
像`一朝分娩''',体现了质变是量变的必然结果。据此选B。题干设问部分,没有强调``实践是
认识的来源'',故排除A。
\end{solution}
\question 牛顿的一句名言:``假若我能比别人瞭望得略为远些,那是因为我站在巨人们的肩膀上''这句话所体现的意义是
\par\fourch{认识既来源于直接经验也来源于间接经验}{\textcolor{red}{主体可以通过读书或传授等方法来获取间接经验,这是发展人类认识的必要途径}}{理性认识不仅以感性认识为基础,而且要通过感性的认识来说明}{\textcolor{red}{只有把间接经验与直接经验结合起来,才能有比较完全的知识}}
\begin{solution}BD牛顿说:``假若我能比别人瞭望得略为远些,那是因为我站在巨人们的肩膀上''。是表明认学习间接经验的重要性。由于具体的主体的生命和能力是有限的,不可能事事亲身实践,而且理论或认识本身也具有历史的继承性,所以主体可以也应该通过读书或传授等方法来获取间接经验,这是发展人类认识的必要途径,但是间接经验归根到底也是来源于前人或他人的实践,所以认识来源于实践而且人们接受间接经验也要或多或少地以某种直接经验为基础,只有把间接经验与直接经验结合起来,才能有比较完全的知识。A项否认认识来源于实践,是错误的。C项材料未有体现。
\end{solution}
\question 马克思主义认识论认为,主体和客体的关系,是( ~)
\par\fourch{\textcolor{red}{改造和被改造的关系}}{\textcolor{red}{认识和被认识的关系}}{利用和被利用的关系}{\textcolor{red}{限定和超越的关系}}
\begin{solution}主体与客体的关系,不仅仅是认识和被认识的关系,而且也是改造和被改造的关系。在主体改造客体的实践过程中,主体反映了客体。在实践过程中,主体一方面受到客体的限定和制约,另一方面又能不断地发展自己的能力和需求,以自觉能动的活动不断打破客体的限定,超越现实客体,从而使主体和客体同时得到改造、发展与完善。这种限定和超越的关系,就是主体和客体相互作用的实质。
\end{solution}
\question 未来学家尼葛洛庞蒂说:``预测未来的最好办法就是把它创造出来。''从认识与实践的关系看,这句话对我们的启示是
\par\twoch{认识总是滞后于实践}{实践是认识的先导}{\textcolor{red}{实践高于认识}}{实践与认识是合一的}
\begin{solution}本题考点:认识与实践的关系。
实践具有直接现实性,这是实践高于认识的真正优点,理论不具有直接现实性,只是在思想上预测或者指导,只有实践才能够直接地作用于对象,有效地改造和创造物质对象。可见,这位未来学家真正的意思是要以预见为基础,通过实践真正把科学的预测和理论转化为现实。
\end{solution}
