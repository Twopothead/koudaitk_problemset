\question 近代中国人包括统治阶级中的爱国人物在反侵略斗争中表现出来的爱国主义精神,铸成了中华民族的民族魂,他们乃是中华民族的脊梁。在抗击外国侵略的战争中,许多爱国官兵英勇献身以身殉国,其代表人物有
\par\fourch{\textcolor{red}{鸦片战争期间,广东水师提督关天培、江南提督陈化成、副都统海龄}}{中法战争期间,督办台湾事务大臣刘铭传、老将冯子材}{\textcolor{red}{第二次鸦片战争中,提督史荣椿、乐善}}{\textcolor{red}{中日甲午战争时,致远舰管带(舰长)邓世昌、经远舰管带林永升}}
\begin{solution}【解析】1859年6月,英法联军大举进攻大沽炮台,守军沉着应战,击沉、击伤敌舰多艘。中法战争期间,1884年8月,法舰进犯台湾基隆,同年10月,又进犯淡水,都被督办台湾事务大臣刘铭传指挥守军击退。1885年初,法舰炮轰浙江镇海炮台,也被守军击退。3月,在中越边境镇南关(今友谊关),年近70的老将冯子材身先士卒,率部勇猛冲杀,大败法军,取得镇南关大捷。所以B项为干扰项。在抗击外国侵略的战争中,许多爱国官兵英勇献身。如:鸦片战争期间,广东水师提督关天培、江南提督陈化成、副都统海龄(满族);第二次鸦片战争中,提督史荣椿、乐善《蒙古族);中日甲午战争时,致远舰管带(舰长)邓世昌、经远舰管带林永升等,都以身殉国。
\end{solution}
\question 近代中国新产生的新的被压迫阶级是工人阶级,``中国工人阶级比中国资产阶级资格要老。因而它的社会力量和社会基础也更广大些'',这是因为(
~)
\par\fourch{它伴随着中国民族资产阶级的发生、发展而来}{\textcolor{red}{它伴随着外国资本在中国的直接经营的企业而来}}{它伴随着洋务派创办的洋务企业而来}{它伴随着买办、官僚地主投资兴办的企业而来}
\begin{solution}帝国主义通过与清政府签订条约获得了在中国投资办厂的权利,最早的工人就是给外企打工的员工。所以中国工人阶级比资产阶级先产生。
\end{solution}
\question 义和团运动的领导者宣称:``若辈洋人,借通商与传教以掠夺国人之土地、粮食与衣服,不仅污蔑我们的圣教,尚以鸦片毒害我们,以淫邪污辱我们。自道光以来,夺取我们的土地,骗取我们的金钱;蚕食我们的子女如食物,筑我们的债台如高山;焚烧我们的宫殿,消灭我们的属国;占据上海,蹂躏台湾,强迫开放胶州,而现在又想来瓜分中国。''上述材料说明(
~)
\par\fourch{义和团运动具有宗教战争的性质}{\textcolor{red}{义和团运动是民族意识觉醒的结果}}{义和团运动盲目排外}{\textcolor{red}{义和团运动具有反帝爱国性质}}
\begin{solution}义和团运动的性质首先是一场反帝爱国主义,反对西方侵略,又发生在甲午中日战争后,所以BD正确。外国资本主义入侵,同时宗教文化也进来,但是义和团没有以宗教的名义发动战争,不具有宗教战争的性质。材料中也没有提到义和团盲目排外。
\end{solution}
