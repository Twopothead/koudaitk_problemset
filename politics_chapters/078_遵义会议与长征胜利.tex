\question 中共中央政治局与1935年1月15日至17日在遵义召开了扩大会议,史称``遵义会议''。下列关于遵义会议表述正确的是
\par\fourch{遵义会议创造性地解决了在农村环境中、在党组织和军队以及农民为主要成分的环境下,如何从加强思想建设人手,保持党的无产阶级先锋队性质和建设党建设的新型人民军队的
	问题}{\textcolor{red}{遵义会议表明:作为一个严肃的、对人民负责的马克思主义政党,中国共产党敢于正视自己的错误,并注意从自己所犯的错误中学习并汲取教训的}}{\textcolor{red}{遵义会议开始确立以毛泽东为代表的马克思主义的正确路线在中共中央的领导地位}}{\textcolor{red}{遵义会议集中解决了当时具有决定意义的军事问题和组织问题}}
\begin{solution}A项是古田会议的内容。
\end{solution}
\question 遵义会议是党的历史上一个生死攸关的转折点,经过激烈的争论,多数人同意以毛泽东为代表的正确意见,批评了博古、李德在第五次反``围剿''中的错误。会议通过了一系列重大决策,这些决策
\par\fourch{\textcolor{red}{是在中国共产党同共产国际的联系中断的情况下独立自主地作出的}}{\textcolor{red}{集中解决了当时具有决定意义的军事问题和组织问题}}{教育了广大的共产党员和红军指战员,他们开始产生对错误领导的怀疑、不满}{着重阐述了党必须依靠农民和掌握枪杆子的思想}
\begin{solution}【解析】1934年10月中旬,中共中央机关和中央红军(又称红一方面军8.6万人撤离根据地,向西突围转移,开始长征。长征初期,中共中央领导人博古依靠与共产国际有关系的军事顾问、德国人李德,犯了退却中的逃跑主义错误。在强渡湘江之后,红军和中央机关人员锐减到3万多人。严酷的事实教育了广大的共产党员和红军指战员,他们开始产生对错误领导的怀疑、不满。一些支持过``左''倾错误的中央领导人如张闻天、王稼祥等,也改变态度,转而支持毛泽东的正确主张。这样,当中央红军根据毛泽东的提议,改向敌人力量薄弱的贵州挺进,并在占领黔北重镇遵义之后,中共中央政治局于1935年1月15日至17日在这里召开了扩大会议(史称``遵义会议'')。C项是遵义会议出台一系列政策的背景。
遵义会议的政策集中解决了当时具有决定意义的军事问题和组织问题。经过激烈的争论,多数人同意以毛泽东为代表的正确意见,批评了博古、李德在第五次反``围剿''中的错误。会议増选毛泽东为中央政治局常务委员并委托张闻天起草《中央关于反对敌人五次``围剿''的总结的决议》(即遵义会议决议)。会后不久,中共中央政治局常务委员分工,根据毛泽东的提议,决定由张闻天代替博古负总的责任;博古任红军总政治部代理主任;并成立了由周恩来、毛泽东、王稼祥组成的新的三人团,全权负责红军的军事行动。会议的一系列重大决策,是在中国共产党同共产国际的联系中断的情况下独立自主地作出的。
遵义会议开始确立以毛泽东为代表的马克思主义的正确路线在中共中央的领导地位,从而在极其危急的情况下挽救了中国共产党、挽救了中国工农红军、挽救了中国革命,遵义会议是党的历史上一个生死攸关的转折点,它标志着中国共产党在政治上开始走向成熟。
D项是毛泽东在八七会议上着重阐述的思想,强调党``以后要非常注意军事,须知政权是由枪杆子中取得的'',实际上提出了以军事斗争作为党的工作重心的问题。
\end{solution}
\question 中国工农红军的长征是一部伟大的革命英雄主义的史诗。它向全中国和全世界宣告,中国共产党及其领导的人民军队,是一支不可战胜的力量。红军长征,铸就了伟大的长征精神。长征精神,一个重要内容就是坚持独立自主、实事求是,一切从实际出发的精神。在长征路上中共中央同错误路线进行的斗争有
\par\twoch{反对罗明路线的斗争}{\textcolor{red}{同红四方面军领导人张国焘分裂中央、分裂红军的严重错误进行斗争}}{\textcolor{red}{与博古、李德在第五次反“围剿〃中的逃跑主义路线的斗争}}{与党内在统一战线问题上的〃左”倾关门主义的错误倾向的斗争}
\begin{solution}BC
1934年10月中旬,中共中央机关和中央红军(又称红一方面军)8.6万人撤离根据地,向西突围转移,开始长征。长征初期,中共中央领导人博古依靠与共产国际有关系的军事顾问、德国人李德,犯了退却中的逃跑主义错误。在强渡湘江之后,红军和中央机关人员锐减到3万多人。严酷的事实教育了广大的共产党员和红军指战员,他们开始产生对错误领导的怀疑、不满。一些支持过〃左''倾错误的中央领导人如张闻天、王稼祥等,也改变态度,转而支持毛泽东的正确主张。这样,当中央红军根据毛泽东的提议,改向敌人力量薄弱的贵州挺进,并在占领黔北重镇遵义之后,中共中央政治局于1935年1月15日至17日在这里召开了扩大会议(史称``遵义会议'')。遵义会议集中解决了当时具有决定意义的军事问题和组织问题。经过激烈的争论,多数人同意以毛泽东为代表的正确意见,批评了博古、李德在第五次反``围剿〃中的错误。遵义会议后,中共中央又同红四方面军领导人张国焘分裂中央、分裂红军的严重错误进行了坚决的斗争。
1935年12月25日,中共中央在陕北瓦窑堡召开政治局扩大会议,提出了建立抗日民族统一战线的方针,批评了党内长期存在的〃左''倾冒险主义、关门主义的错误倾向,制定了抗日民族统一战线的策略方针,为抗日战争的到来做了思想上和理论上的准备。中国共产党在新的历史时期即将到来时掌握了政治上的主动权。反对〃罗明路线〃的斗争是1933年初,由于白区党的工作遭到严重破坏,临时中央政治局无法在上海立足,被迫迀到中央根据地。为了全面推行〃左''倾教条主义的方针、政策展开的一场反对毛泽东的正确主张的斗争。AD都是干扰项。
\end{solution}
