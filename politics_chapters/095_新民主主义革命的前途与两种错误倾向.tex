\question 关于新民主主义革命与旧民主主义革命相比,下列说法中正确的有
\par\fourch{\textcolor{red}{两者所面临的国情和社会主要矛盾相同}}{两者革命的对象和前途相同}{\textcolor{red}{区分二者的根本标志是革命领导权的不同}}{\textcolor{red}{两者的依靠力量和革命动力不同}}
\begin{solution}【解析】本题考查新旧民主主义对比。B选项说法错误,两者前途不同。新民主主义革命的前途是社会主义,不是资本主义。新民主主义的``新''表现为:一是领导阶级;二是前途;三是指导思想;四是发生的时代条件。其中领导阶级是区分新旧民主主义革命的根本标志和分水岭。
\end{solution}
\question 新民主主义革命时期,党内右倾提出的``二次革命论'',其错误在于( )
\par\fourch{混淆了新民主主义革命和资产阶级革命的界限}{割裂了新民主主义革命和资产阶级革命的联系}{混淆了新民主主义革命和社会主义革命的界限}{\textcolor{red}{割裂了新民主主义革命和社会主义革命的联系}}
\begin{solution}``二次革命论'',其错误在于割裂了新民主主义革命和社会主义革命的联系,注意我国的新民主主义革命是无产阶级领导的资产阶级革命。
\end{solution}
