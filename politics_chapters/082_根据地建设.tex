\question 抗日战争时期的``三三制''政权
\par\fourch
{\textcolor{red}{是指抗日民主政府在工作人员分配上实行“三三制”原则,即共产党员、非党的左派进步分子和不左不右的中间派各占1/3}}
{\textcolor{red}{是抗日民族统一战线性质的政权,有利于结成最广泛抗日民族统一战线}}
{\textcolor{red}{是一切赞成抗日又赞成民主的人们的政权}}
{\textcolor{red}{是敌后抗战的最好政权形式}}
\begin{solution}基本背诵内容
\end{solution}
\question 决定将中国共产党在抗日战争时期实行的减租减息政策改变为实现``耕者有其田''政策的是(
)
\par\twoch{《中国土地法大纲》}{\textcolor{red}{《关于清算、减租及土地问题的指示》}}{《兴国土地法》}{《井冈山土地法》}
\begin{solution}1946年5月4日,中共中央发出《关于清算、减租及土地问题的指示》(史称《五四指示》),决定将党在抗日战争时期实行的减租减息政策改变为实现``耕者有其田''的政策。《中国土地法大纲》,明确规定废除封建性及半封建性剥削的土地制度,实现耕者有其田的土地制度;《兴国土地法》、《井冈山土地法》制定于土地革命战争时期。故B正确。
\end{solution}
