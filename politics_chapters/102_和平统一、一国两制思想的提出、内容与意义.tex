\question 钓鱼岛及其附属岛屿是中国领土不可分割的一部分。中国最早发现、命名、利用和管辖钓鱼岛。1895年,清朝在甲午战争中战败,被迫
与日本签署不平等的《马关条约》,割让``台湾全岛及所有附属各岛屿''。钓鱼岛等作为台湾``附属岛屿''一并被割让给日本。1941年12月,中国政府正式对日宣战,宣布废除中日之间的一切条约。日本投降后,依据有关国际文件规定,钓鱼岛作为台湾的附属岛屿应与台湾一并归还中国。这些国际文件是
\par\twoch{\textcolor{red}{《波茨坦公告》}}{\textcolor{red}{《开罗宣言》}}{\textcolor{red}{《日本投降书》}}{《德黑兰宣言》}
\begin{solution}本题考查的是钓鱼岛作为台湾附属岛屿规划中国的国际文件。1941年12月,中国政府正式对日宣战,宣布废除中日之间的一切条约。1943年12月《开罗宣言》明文规定,``日本所窃取于中国之领土,例如东北四省、台湾、澎湖群岛等,归还中华民国。其他日本以武力或贪欲所攫取之土地,亦务将日本驱逐出境''。1945年7月《波茨坦公告》第八条规定:``《开罗宣言》之条件必将实施,而日本之主权必将限于本州、北海道、九州、四国及吾人所决定之其他小岛。''1945年9月2日,日本政府在《日本投降书》中明确接受《波茨坦公告》,并承诺忠诚履行《波茨坦公告》各项规定。上述事实表明,依据《开罗宣言》、《波茨坦公告》和《日本投降书》,钓鱼岛作为台湾的附属岛屿应与台湾一并归还中国。选项D是1943年苏、美、英三国首脑在德黑兰会议结束时发表的宣言,它规定盟国在西欧开辟第二战场,实行``霸王战役''计划,发动``诺曼底登陆''的时间,与台湾问题无关。因此,本题的正确答案是ABC。
\end{solution}
\question 第二次世界大战期间,明确规定将台湾、澎湖列岛归还中国的有关的是
\par\twoch{《德黑兰宣言》}{\textcolor{red}{《开罗宣言》}}{《雅尔塔协定》}{\textcolor{red}{《波茨坦公告》}}
\begin{solution}二战期间,涉及到台湾问题的国际条约有1943年的《开罗宣言》和1945年的《波茨坦公告》,它们都明确规定将澎湖列岛归还给中国。选项AC不符合题意,故不选。所以正确答案为BD。
\end{solution}
