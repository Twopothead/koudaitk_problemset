\question 马克斯、恩格斯在指导建立无产阶级政党的过程中,阐述了各国无产阶级政党相互关系的重要原则,主要有
\par\fourch{\textcolor{red}{坚持各国党的独立自主}}{\textcolor{red}{坚持无产阶级的国际联合}}{坚持合法斗争和暴力革命相结合}{\textcolor{red}{坚持各国党的完全平等}}
\begin{solution}马克思、恩格斯在指导建立无产阶级政党的过程中,阐述了各闰无产阶级政党相互关系的重要原则。一是坚持无产阶级的国际联合。在马克思、恩格斯的指导下,各国工人政党于1864年建立了工人运动的国际联合组织------``笫一国际''。
二是坚持各国党的独立自主和完全平等。强调无产阶级的国际联合,并不意味着各国无产阶级的斗争从属于另一国无产阶级的斗争。马克思层多次强调,第一国际只是各国工人运动联络和合作的中心,而不是指挥中心。A、B、D正确答案。C回答的不是题目所问的各国无产阶级政党相互关系的原则问题。
\end{solution}
\question 社会主义从理论到实践的飞跃,是通过无产阶级革命实现的。无产阶级革命是迄今人类历史上最广泛、最彻底、最深刻的革命,是不同于以往一切革命的最新类型的革命。无产阶级革命的主要特点是(
)
\par\fourch{\textcolor{red}{无产阶级革命是不断前进的历史过程}}{\textcolor{red}{无产阶级革命是为绝大多数人谋利益的运动}}{\textcolor{red}{无产阶级革命是要彻底消灭一切阶级和阶级统治的革命}}{\textcolor{red}{无产阶级革命是彻底消灭一切私有制、代之以生产资料公有制的革命}}
\begin{solution}无产阶级革命的主要特点是无产阶级革命是不断前进的历史过程,无产阶级革命是为绝大多数人谋利益的运动,无产阶级革命是要彻底消灭一切阶级和阶级统治的革命,无产阶级革命是彻底消灭一切私有制、代之以生产资料公有制的革命。
\end{solution}
