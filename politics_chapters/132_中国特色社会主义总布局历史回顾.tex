\question 按劳分配是社会主义的分配原则,也是出于主体地位的分配原则,之所以实行按劳分配,其前提条件是(
)
\par\twoch{\textcolor{red}{公有制}}{社会生产力的发展水平}{社会主义初级阶段的基本经济制度}{人们的劳动积极性}
\begin{solution}之所以在我国实行按劳分配是由公有制为主体和社会生产力发展水平决定的,其中,公有制为主体是实行按劳分配的前提条件和所有制基础,生产力发展水平是实行按劳分配的物质基础。
\end{solution}
\question 劳动、资本、技术、管理等生产要素是社会生产不可或缺的因素。在我国社会主义初级阶段,实行按生产要素分配的必要性和根据是
\par\twoch{生产要素可以转化为生产力}{\textcolor{red}{我国社会存在着生产要素的多种所有制}}{按生产要素分配是按劳分配的补充}{生产要素是价值的源泉}
\begin{solution}在我国社会主义初级阶段,实行按生产要素分配的必要性和根据是我国社会存在着生产要素的多种所有制。因此,本题正确答案是选项B。
\end{solution}
\question ``股份制是现代企业的一种资本组织形式,不能笼统地说股份制是公有还是私有'',这一观点表明(
)
\par\fourch{由法人股东而不是个人股东构成的股份制是公有制}{\textcolor{red}{公有制与私有制都可以通过股份制这一形式来实现}}{有公有制经济参股的就是公有制}{\textcolor{red}{股份制本身不具有公有还是私有的性质}}
\begin{solution}本题考查的是公有制经济的实现形式和公有制经济控股的重要意义两方面的内容。党的十五大指出:``股份制是现代企业的一种资本组织形式,有的所有权和经营权分离,有的提高企业和资本的动作效率,资本主义可以用,社会主义也可以用(B项正确)。不能笼统地说股份制是公有还是私有(D项正确),关键看控股权掌握在谁手中。国家和集体控股,具有明显的公有性(E项正确),有利于扩大公有资本的支配范围,增强公有制的主体作用。股份制不是公有制(不管是法人股东构成的股份制、还是有公有制参股的股份制都是如此),股份制中的国有成分、集体成分才是公有制。''答案为BDE三项。股份制不是所有制,不存在公有、私有之分,只有股份制经济中的国有和集体成分才是公有制。AC两项的说法错误。
\end{solution}
\question 人民政协以政治协商和民主监督作为自己的主要职能。民主监督主要是指( )
\par\fourch{\textcolor{red}{对国家的宪法和法律法规的实施情况进行监督}}{对国家工作人员提出质询、弹劾}{\textcolor{red}{对重大方针政策的贯彻执行情况进行监督}}{对人民代表大会进行工作监督}
\begin{solution}人民政协作为共产党领导的多党合作和政治协商的政治组织,始终以政治协商和民主监督为自己的主要职能。民主监督的主要内容包括:对国家的宪法和法律实施情况进行监督;对重大方针、政策的贯彻执行情况进行监督;对国家机关及其工作人员履行职责的情况进行监督。A、C、E项正确。
B、D两项错误,人民政协不同于人民代表大会,它不具有国家权力机关那种监督、检察、质询、弹劾等权力,也不能对国家权力机关人民代表大会进行工作监督。
\end{solution}
\question 在社会主义条件下,中国共产党与各民主党派长期共存,是因为( )
\par\fourch{无产阶级政党可以同资产阶级结成统一战线}{\textcolor{red}{双方有长期团结合作的历史}}{\textcolor{red}{各民主党派已经成为致力于社会主义事业的党派}}{\textcolor{red}{各民主党派在政治上接受了共产党领导}}
\begin{solution}本题考查的是中国共产党与各民主党派``长期共存、互相监督''的方针。1957年2月他又在《关于正确处理人民内部矛盾的问题》一文中阐述了实行``长期共存、互相监督''方针的依据:``为什么要让资产阶级和小资产阶级的民主党派同工人阶级政党长期共存呢?这是因为凡属一切确实致力于团结人民从事社会主义事业的、得到人民信任的党派,我们没有理由不对它们采取长期共存的方针(C项正确)。''各民主党派早在建国前夕就正式宣布接受中国共产党的领导(D项正确),中国共产党的领导是我国的多党合作制度的政治基础;新民主主义革命时期,各民主党派同中国共产党亲密合作,为民族的独立、人民解放做出了重大贡献,双方有长期团结合作的历史(B项正确);各民主党派可以发挥对共产党的监督作用,发挥民主党派的民主监督作用,能对加强和改善党的领导(E项正确)。答案为BCDE四项。社会主义改造完成后,资产阶级已经消亡,各民主党派不是资产阶级政党。A项的说法错误。
\end{solution}
\question 深化文化体制改革,要坚持的目标是( )
\par\twoch{以发展为目标}{以体制机制创新为目标}{以增强全民族文化创造活力为目标}{\textcolor{red}{以创造生产更多更好适应人民群众需求的精神文化产品为目标}}
\begin{solution}深化文化体制改革,要坚持以发展为主题,以改革为动力,以体制机制创新为重点,以创造生产更多更好适应人民群众需求的精神文化产品为目标,促进文化事业全面繁荣和文化产业快速发展。D项正确。
A项错误,发展是文化体制改革的主题;B项错误,体制机制创新是文化体制改革的重点;C项错误,增强全民族文化创造活力是建设社会主义文化强国的关键。
\end{solution}
\question 党的十六届六中全会提出的``建设社会主义核心价值体系''与``文化多样性''``坚持先进文化的前进方向''的内在联系是(
)
\par\fourch{\textcolor{red}{社会主义核心价值体系与文化多样性统一于社会主义文化建设中}}{\textcolor{red}{建设社会主义核心价值体系有利于坚持先进文化的前进方向}}{\textcolor{red}{尊重文化多样性不能违背社会主义核心价值体系}}{把握先进文化的前进方向关键在于尊重文化的多样性}
\begin{solution}本题以建设社会主义核心价值体系作为切入点,考查文化的多样性和文化前进方向。把握先进文化的前进方向关键在于坚持社会主义思想道德建设,D说法错误。
\end{solution}
\question 依法治国,建设社会主义法治国家是建设中国特色社会主义的重要目标。依法治国是( )
\par\fourch{\textcolor{red}{社会文明进步的显著标志,是国家长治久安的重要保障}}{发展社会主义民主,实现人民当家作主的根本保证}{\textcolor{red}{有利于社会主义市场经济体制的完善和发展,为扩大对外开放保驾护航}}{决定当代中国命运的关键抉择}
\begin{solution}此题考查的知识点是依法治国的地位和作用,属识记与记忆相结合的试题,难度适中。AC是正确选项。B项错误,因为中国共产党的领导是实现人民当家做主的根本保证。D项错误在于改革开放是决定当代中国命运的关键抉择。
\end{solution}
\question 在社会主义初级阶段,多种分配方式并存是收人分配制度的一大特点。按劳分配以
外的多种分配方式,其实质就是按对生产要素的占有状况进行分配。生产要素归纳
起来可以分为两类,一是物质生产条件,二是人的劳动,包括人们在生产过程中提
供的活劳动、技术、信息等。按生产要素分配有多种不同的分配形式,就其内容不
同可以分为( )
\par\fourch{\textcolor{red}{专利等管理和知识产权类的生产要素参与分配}}{\textcolor{red}{以劳动作为生产要素参与分配}}{以生产资料所有者的身份参与分配}{\textcolor{red}{以利息、租金等劳动以外的生产要素所有者参与分配}}
\begin{solution}本题考查多种分配方式并存。选项ABD均正确。选项C错误。按生产要素
分配不包括公有制中的按劳分配,因为按劳分配得到收人的劳动者不是凭借作为独
立的生产要素的所有者参与分配,而是以生产资料所有者的身份凭自己提供的劳动
来参与分配的。
\end{solution}
\question 李某是国内某家国有企业的技术员工,他的收入包括三部分。第一部分是企业每个月发给他的工资,第二部分是他的技术供企业采用所得的收人。同时,他家的土地被一家工厂占用,也有一部分收入。李某的收人参与的分配方式有
\par\twoch{\textcolor{red}{按劳分配}}{\textcolor{red}{劳动以外的生产要素参与分配}}{\textcolor{red}{管理和知识产权类的生产要素参与分配}}{以劳动作为生产要素参与分配}
\begin{solution}李某的第一部分收入属于按劳分配,第二部分收入属于管理和知识产权类的生产要素参与分配,第三部分属于劳动以外的生产要素参与分配。
\end{solution}
\question ``罗马城不是一天建起来的''。公平正义是一个逐步实现的过程,将随着社会经济发展不断螺旋式上升。这表明
\par\fourch{\textcolor{red}{维护和促进社会公正是一个渐进的过程,而不可能一蹴而就}}{\textcolor{red}{在新的起点上推进中国特色社会主义,就应把公平正义放到更加突出的位置,使追求公平正义体现到社会生活的方方面面,更好地促进社会和谐稳定}}{\textcolor{red}{要加紧建设对社会公平正义具有重大作用的制度,逐步建立以权利公平、机会公平、规则公平为主要内容的社会公平保障体系}}{公平正义是中国特色社会主义的本质属性}
\begin{solution}D项错误,中国特色社会主义的本质属性是社会和谐。
\end{solution}
\question 判断一个国家的政党制度究竟好不好,要从它的基本国情出发来认识,要从它的实践效果来分析:实践证明,中国共产党领导的多党合作和政治协商制度能够在中国特色社会主义共同目标下,把中国共产党领导和多党派合作有机结合起来,实现
\par\twoch{选举民主和协商民主的统一}{\textcolor{red}{广泛参与和集中领导的统一}}{\textcolor{red}{社会进步和国家稳定的统一}}{\textcolor{red}{充满活力和富有效率的统一}}
\begin{solution}新中国成立以来,中国共产党领导的多党合作与政治协商制度的重要性不断增强。实践证明,这一制度能够在中国特色社会主义共同目标下把中国共产党领导和多党派合作有机结合起来,实现广泛参与与集中领导的统一、社会进步与国家稳定、充满活力与富有效率的统一。B、C、D正确。中国共产党领导的多党合作和政治协商制度体现的是协商民主而不是选举民主,人民政协是协商民主的重要渠道和专门协商机构,A错误。
\end{solution}
\question 中国处理民族问题的根本原则也是中国民族政策的核心内容是
\par\twoch{维护祖国统一,反对民族分裂}{坚持民族平等}{\textcolor{red}{坚持民族团结}}{坚持各民族共同繁荣}
\begin{solution}【解析】社会主义时期处理民族问题的基本原则是:维护祖国统一,反对民族分裂坚持民族平等、民族团结、各民族共同繁荣。其中民族平等是民族团结、各民族共同繁荣的政治前提和基础是中国民族政策的基石。民族团结是维护国家统一、实现各民族共同发展的根本保证,是中国处理民族问题的根本原则,也是中国民族政策的核心内容。各民族的共同繁荣是解决民族问题的根本出发点和归宿。根据题意应选C。
\end{solution}
\question 我国社会主义民主政治的特有形式和独特优势,党的群众路线在政治领域的重要体现是
\par\twoch{民主集中制}{\textcolor{red}{协商民主}}{人民代表大会制度}{基层民主制度}
\begin{solution}【解析】党的十八届三中全会指出:要推进协商民主广泛多层制度化发展。协商民主是我国社会主义民主政治的特有形式和独特优势,是党的群众路线在政治领域的重要体现。在党的领导下,以经济社会发展重大问题和涉及群众切身利益的实际问题为内容,在全社会开展广泛协商,坚持协商于决策之前和决策实施之中。
\end{solution}
\question 实行民族区域自治,是党根据我国的历史发展、文化特点、民族关系和民族分布等具体情况作出的制度安排,符合各民族人民的共同利益和发展要求。具体而言(
)
\par\fourch{\textcolor{red}{统一的多民族国家的长期存在和发展,是我国实行民族区域自治的历史依据}}{中华民族多元一体、万流归宗的文化传统和文化结构,是我国实行民族区域自治的思想前提}{\textcolor{red}{近代以来在反抗外来侵略斗争中形成的爱国主义精神,是实行民族区域自治的政治基础}}{\textcolor{red}{各民族大杂居、小聚居的人口分布格局,各地区资源条件和发展的差异,是实行民族区域自治的现实条件}}
\begin{solution}ACD 为正确选项。其中的每一句,均可单独命制单选题。B是杜撰的干扰项。
\end{solution}
\question 加大个人收入分配调节力度,合理调整收入分配格局。为了实现这一目标,除了要着力提高低收入者收入、努力扩大中等收入者比重、切实对过高收入进行有效调节之外,还需要(
)
\par\twoch{鼓励自主创业、自谋职业}{\textcolor{red}{取缔非法收入}}{规范和协调劳动关系}{\textcolor{red}{规范垄断行业的收入}}
\begin{solution}加大个人收入分配调节力度,合理调整收入分配格局。一要着力提高低收入者收入。要强化支农惠农政策,促进农民持续增收,建立企业职工工资正常增长机制和支付保障机制,逐步提高扶贫标准和最低工资标准,使城乡居民特别是低收入者收入随着经济发展逐步较多地増加。二要努力扩大中等收入者比重。要通过采取多种措施,创造条件让更多群众拥有财产性收入,使更多低收入者进入中等收入者行列。三要切实对过高收入进行有效调节。要正确运用税收手段,使过高收入者的一部分收入通过税收等形式由国家集中用于再分配。四要取缔非法收入。要严格执法,对偷税漏税、侵吞公有财产、权钱交易等各种非法收入依法取缔。五要规范垄断行业的收入。AC是扩大就业的举措。
\end{solution}
