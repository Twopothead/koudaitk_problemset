\question 全面依法治国是``四个全面''战略布局的重要一环。全面建成小康社会'实现``两个一百年''的奋斗目标和中华民族伟大复兴的中国梦,全面深化改革、完善和发展中国特色社会主义制度,全面从严治党、保持长治久安,都必须在全面依法治国上作出总体部署。全面依法治国
\par\fourch{\textcolor{red}{为全面深化改革提供制度规则}}{\textcolor{red}{既是全面建成小康社会的内在要求,又是基本保障}}{\textcolor{red}{与全面从严治党本质一致、辩证统一}}{是实现中华民族伟大复兴中国梦的“关键一步”}
\begin{solution}【简析】全面依法治国是``四个全面''战略布局的重要一环。全面建成小康社会、实现``两个一百年''的奋斗目标和中华民族伟大复兴的中国梦,全面深化改革、完善和发展中il特色社会主义制度,全面从严治党、保持长治久安,都必须在全面依法治国上作出总体部著。第一,全面依法治国为全面深化改革提供制度规则。重大改革需要于法有据。第二,全面依法治国既是全面建成小康社会的内在要求,又是基本保障。全面深化改革、全面依法治国如鸟之两翼、车之双轮,推动全面建成小康社会的目标如期实现。第三,全面依法治国与全面从严治党本质一致、辩证统一。要跳出历史周期律、走好中国道路,必须落实好全面依法治国的战略部署。第四,全社会都能够尊法学法守法用法,法治才能成为中华民族伟大复兴中国梦的坚强保障。A、B、C正确。全面建成小康社会是阶段目标,是实现中华民族伟大复兴中国梦的``关键一步''。D不符合题意。
\end{solution}
