\question 以下选项中正确表达辩证唯物主义时空观内容的有,时间和空间( )
\par\twoch{是感性直观的先天形式}{\textcolor{red}{是物质运动的存在形式}}{\textcolor{red}{随物质运动速度的变化而变化}}{\textcolor{red}{是不可分割的}}
\begin{solution}本题考查的知识点:时间和空间
时空作为物质的存在形式,是有限性和无限性的统一。时间的无限性是指物质世界的存在和发展的持续性是无限的,无始无终。空间的无限性是指物质世界的广延性是无限的,无边无际。时空的有限性是指任何具体事物,其存在的时间、占有的空间都是有限的。时空既是绝对的,又是相对的,是绝对和相对的统一。时空的绝对性是指时空作为运动着的物质的存在形式,它的客观实在性是不变的、无条件的,因而是绝对的。时空的相对性是指时空特性的具体性、可变性。时空的具体特性随物质运动特性的变化而变化,人们关于时空的观念也是可变的、发展的。爱因斯坦的狭义相对论揭示了时空特性随物质运动速度的变化而变化,时空特性随物质形态的不同而不同。选项A把时空看成先天的形式,否定了时空的客观性,属于唯心主义的时空观。时间和空间是物质运动的存在形式,是不可分割的,并且随着物质运动速度的变化而变化。所以正确答案是选项BCD。
\end{solution}
\question 长江的年龄有多大?这里说的长江``年龄'',是指从青藏高原奔流而下注入东海的``贯通东流''水系的形成年代。如果说上游的沉积物从青藏高原、四川盆地顺延而下能到达下游,这就表明长江贯通了,这就是物源示踪。我国科学家采用这一方法,研究长江中下游盆地沉积物的来源,从而判别长江上游的物质何时到达下游,间接指示了长江贯通东流的时限。他们经过10多年的研究,提出长江贯通东流的时间距今约2300多万年。这一研究成果从一个侧面显示出
\par\fourch{时间和空间是有限的,物质运动是永恒的}{\textcolor{red}{时间和空间是通过物质运动的变化表现出来的}}{时间和空间是标示物质运动的观念形式}{\textcolor{red}{时间和空间是物质运动的存在形式}}
\begin{solution}此题考查的是时间和空间的特点。时空是客观的是物质运动的存在形式,具有有限性和无限性,绝对性和相对性的特点。A选项表述错误,时空是有限的也是无限的,是和物质的运动紧密结合的。C本身表述错误,时空是客观的,不是观念形式。所以正确答案是BD。
\end{solution}
