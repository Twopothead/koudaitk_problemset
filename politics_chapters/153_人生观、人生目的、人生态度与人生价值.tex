\question 促进个人与他人的和谐要坚持四个原则,其中作为保证的是
\par\twoch{平等原则}{宽容原则}{\textcolor{red}{诚信原则}}{互助原则}
\begin{solution}诚信是促进个人与他人和谐的保证。诚信包含着诚实和守信两方面的意思,诚是信的内在思想基础,信是诚的外在表现。诚信历来被视为处理个人与他人关系的基本准则。C正确。平等待人是促进个人与他人和谐的前提;宽容是促进个人与他人和谐必不可少的条件;互助是促进个人与他人和谐的必然要求,A、B、D不符合题意。
\end{solution}
\question 在人类历史长河中涌现过形形色色的人生观,只有以为人民服务为核心内容的人生观,才是科学高尚的人生观,才值得终生尊奉和践行。树立为人民服务的人生观,要坚决抵制各种错误思想的影响。由于受国内外各种错误思潮、腐朽观念等各种因素的影响,现实中还存在拜金主义、享乐主义和极端个人主义等对人生目的的错误看法。这些错误的思想观念容易侵蚀
大学生的纯洁心灵,不利于大学生树立科学高尚的人生观和价值观。上述种种错误的思想和观念,尽管在形式上五花八门,内容不尽一致,但它们却有着共同的特征,它们(
~)
\par\fourch{\textcolor{red}{都表达了剥削阶级的人生观对人生目的的主张}}{\textcolor{red}{都没有把握个人与社会的正确关系}}{都是生产资料私有制的产物,是资产阶级世界观的核心}{\textcolor{red}{对人的需要的理解是片面的,夸大了人生的某方面需要,而无视人的全面性和人生的整体需要}}
\begin{solution}【解析】拜金主义、享乐主义和极端个人主义等错误的人生观,尽管在形式上五花八门,内容不
尽一致,但它们却有着共同的特征。其一,它们都表达了剥削阶级的人生观对人生目的的主
张,反映的都是剥削阶级的腐朽观念,不可能具有劳动人民的宽广胸怀和远大志向,更不能代
表人民群众的利益。其二,它们都没有把握个人与社会的正确关系,忽视或否认社会性是人
的存在和活动的本质属性,它们讨论人生问题的出发点和落脚点都是褊狭的一己之私利。其
三,它们对人的需要的理解是片面的,夸大了人生的某方面需要,而无视人的全面性和人生的
整体需要。据此,本题选ABD。C选项不能人选。个人主义是生产资料私有制的产物,是资
产阶级世界观的核心,不能笼统说三者都是。
\end{solution}
\question 人生观的核心是( ~)
\par\twoch{人生价值}{\textcolor{red}{人生目的}}{人生态度}{人生信仰}
\begin{solution}人生目的决定着人生价值的大小、类型,是人生观的核心。教材仅论及``人生目的、人生态度、人生价值''三个概念。因此,B正确。
\end{solution}
\question 成就何种人生,是事业有成,还是庸碌无为;是崇高善良,还是卑鄙邪恶;是彪炳史册,还是遗臭万年,除了客观历史条件和机遇等因素的影响外,在很大程度上,将取决于人们有什么样的人生观、价值观。作为人生观核心的是(
~)
\par\twoch{\textcolor{red}{人生目的}}{人生态度}{人生价值}{人生意义}
\begin{solution}概念记忆题。
\end{solution}
\question 世界观是指人们对世界的总的根本的看法。人生观是指对人生的看法,也就是对于人类生存的目的、价值和意义的看法。世界观和人生观是紧密联系在一起的。这主要表现在(
~)
\par\fourch{\textcolor{red}{世界观决定人生观}}{人生观决定世界观}{\textcolor{red}{人生观对世界观的巩固、发展和变化起重要作用}}{世界观对人生观的巩固、发展和变化起重要作用}
\begin{solution}本题难度很小,书面知识记忆。
\end{solution}
