\question 在中国共产党的历史上,对毛泽东思想做出系统概括和阐述的党的文献有
\par\fourch{《关于若干历史问题的决议》}{\textcolor{red}{刘少奇在七大上所作的《关于修改党的章程的报告》}}{邓小平在八大上所作的《关于修改党的章程》的报告}{\textcolor{red}{《关于建国以来党的若干历史问题的决议》}}
\begin{solution}本题考关于毛泽东思想的科学含义问题,在中国共产党的历史上对毛泽东思想有两次概括:第一次是1945年中共七大的决议和刘少奇作的《关于修改党的章程》的报告,第二次是1981年十一届六中全会通过的《关于建国以来党的若干历史问题的决议》。两者既有一脉相承的继承关系,又有在新时期根据新情况做出的新认识,新判断。至于A、C是干扰项。A项是1945年中共六届七中全会通过的决议,总结了党自1921年产生以来领导中国革命的经验,尤其是1931年党的六届四中全会以来的路线是非,肯定了毛泽东同志是马克思列宁主义的普遍真理和中国革命的具体实践相结合的代表,为中共七大奠定了思想基础和组织基础。1956年召开的中共八大,邓小平在八大所作的《关于修改党的章程的报告》中,未提``毛泽东思想''这个概念,故AC应排除。本题历史性较强,考生需了解有关历史背景知识。
\end{solution}
