\question 培养法治思维,必须抛弃人治思维。法治思维与人治思维的区别,集中体现
\par\fourch{\textcolor{red}{在依据上,法治思维认为国家的法律是治国理政的基本依据;人治思维强调的是依靠个人的能力和德行治国理政}}{\textcolor{red}{在方式上,法治思维以一般性、普遍性的平等对待方式调节社会关系,解决矛盾纠纷,具有稳定性和一贯性;人治按照个人意志和感情进行治理,具有极大的任意性和非理性}}{\textcolor{red}{在价值上,法治思维是“多数人之治”的民主思维;人治思维是少数个人的集权专断}}{在标准上,法治思维尊良法、尊善法;人治思维不奉法、奉劣法}
\begin{solution}【解析】培养法治思维,必须抛弃人治思维。法治思维与人治思维的区别,集中体现在四个方面:一是在依据上,法治思维认为国家的法律是治国理政的基本依据,也是行为的根本指南。处理法律问题要以事实为根据、以法律为准绳:而人治思维则主张凭借个人尤其是掌权者、领导人的个人魅力、德性和才智来治国平天下。如古希腊柏拉图提出的``哲学王''之治,我国古代推崇``圣君''、``贤人''之治以及后世的``英雄''、``强人``、``能人''之治等,主要强调的都是依靠个人的能力和德行治国理政。二是在方式上,法治思维以一般性、普遍性的平等对待方式调节社会关系,解决矛盾纠纷,坚持法律面前人人平等原则,反对因人而异和亲疏远近,具有稳定性和一贯性;而人治漠视规则的普遍适用性,按照个人意志和感情进行治理,治人者以言代法,言出法随,朝令夕改,具有极大的任意性和非理性。三是在价值上,法治思维强调集中社会大焱的意志来进行决策和判断,是一种``多数人之治''的民主思维,而且这种民主是建立在法律的基础上的,避免陷入无政府主义或以民主之名搞乱社会。
而人治思维是少数人说了算的专断思维,虽然有时也强调集思广益,如通过开会讨论、搞群众运动的形式进行治理或作出决定,但主要表现为少数个人的集权专断。四是在标准上,法治思维与人治思维的分水岭不在于有没有法律或者法律的多寡与好坏,而在于最高的权威究竟是法律还是个人。显然D项为唯一干扰项。法治思维以法律为最高权威,强调``必须使民主制度化、法律化,使这种制度和法律不因领导人的改变而改变,不因领导人的看法和注意力的改变而改变。''人治思维则奉领导者个人的意志为最高权威,当法律的权威与个人的权威发生矛盾时,强调服从个人而非服从法律的权威。
\end{solution}
