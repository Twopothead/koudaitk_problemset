\question 据《北京青年报》报道,清华大学课程《毛泽东思想槪论》于2015年9月中旬受邀登录世界三大慕课平台之一的edX,面向全球开课。
---周之内已有来自130个国家和地区的近3000人选修。这也是国内首门登录国际英文慕课平台的思想政治理论课。开课第一天,来自美国、新西兰、英国等国的同学就在课程论坛上发帖.围绕``马克思主义中国化''等问越展开讨论。毛泽东思想是
\par\fourch{\textcolor{red}{马克思主义中国化第一次历史性飞跃的理论成果}}{中国革命建设和改革经验的科学总结}{\textcolor{red}{中国共产党和中国人民宝贵的精神财富}}{\textcolor{red}{中国特色社会主义理论体系的重要思想渊源}}
\begin{solution}【简析】A、C、D正确,毛泽东思想是在我国新民主主义革命、社会主义革命和社会主义建设的实践中,在总结我国革命和建设正反两方面历史经验的基础上,逐步形成和发展起来的。改革开放是在毛泽东逝世后的1978年以后开始的,B错误。
\end{solution}
\question 毛泽东在《中国革命和中国共产党》中论述了民主革命和社会主义革命的关系。他指出:``民主革命是社会主义革命的必要准备,社会主义革命是民主革命的必然趋势。''这两个革命阶段能够有机连接的原因是
\par\twoch{资本主义道路在中国走不通}{俄国十月革命为中国提供了经验}{\textcolor{red}{民主革命包含了社会主义因素}}{中国国情决定中国革命必须分两步走}
\begin{solution}本题考查民主革命和社会主义革命的关系。
中国共产党领导的革命分为两部分:新民主主义革命和社会主义革命。前者是后者的前提和必要准备,后者是前者发展的必然趋势,两者紧密结合在一起,主要原因是这两者都是由无产阶级领导的。这就决定了中国的民主革命中包含社会主义的因素,并且社会主义因素在其中占主导地位。所以C项正确。
A项错误,资本主义道路在中国走不通只是说明了中国选择社会主义道路的原因,没有解释民主革命与社会主义革命为什么能够结合在一起;B项错误,俄国十月社会主义革命是中国选择社会主义革命的外部条件;D项错误,中国特殊的国情决定了中国革命分两步走,但是这本身并不能解释为什么这两个阶段能够结合在一起。
故正确答案选C。
\end{solution}
\question 在马克思主义中国化第一个重大理论成果------毛泽东思想的指引下,中国共产党领导全国各族人民所取得的伟大成就有(
)
\par\fourch{\textcolor{red}{取得了新民主主义革命的胜利,建立了人民民主专政的中华人民共和国}}{\textcolor{red}{顺利地进行了社会主义改造,确立了社会主义基本制度}}{\textcolor{red}{发展了社会主义的经济、政治和文化,初步探索了社会主义建设道路}}{开辟了建设中国特色社会主义的正确道路}
\begin{solution}毛泽东思想的指引下,中国共产党带领全国人民取得了新民主主义革命的胜利,确立了社会主义基本制度以及初步探索了社会主义建设道路。选项D中国特色社会主义道路是马克思主义中国化的第二次飞跃的内容。
\end{solution}
\question 毛泽东思想是一个完整的科学体系,它的组成部分包括( )
\par\twoch{马克思主义的基本原理}{\textcolor{red}{丰富和发展了马克思列宁主义的许多独创性理论}}{\textcolor{red}{贯穿于其中的一以贯之的立场、观点、方法}}{毛泽东的晚年错误}
\begin{solution}A马克思主义基本原理不属于毛泽东思想体系的内容;D毛泽东思想不包括毛泽东晚年错误。
\end{solution}
