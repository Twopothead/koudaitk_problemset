\question 中共十八届三中全会通过的《中共中央关于全面深化改革若干重大问题的决定》提出,坚持和

完善公有制为主体、多种所有制经济共同发展的基本经济制度。以下选项内容正确的是
\par\fourch{\textcolor{red}{公有制经济和非公有制经济都是我国经济社会发展的重要基础}}{公有制经济和非公有制经济都是我国基本经济制度的重要实现形式}{\textcolor{red}{必须不断增强国有经济活力、控制力、影响力}}{\textcolor{red}{必须激发非公有制经济活力和创造力}}
\begin{solution}【简析】公有制为主体、多种所有制经济共同发展的基本经济制度,是中国特色社会主义制度的重要支柱,也是社会主义市场经济体制的根基。公有制经济和非公有制经济都是社会主义市场经济的重要组成部分,都是我国经济社会发展的重要基础。必须毫不动摇巩固和发展公宥制经济,坚持公宥制主体地位,发挥国有经济主导作用,不断增强国有经济活力、控制力、影响力;必须毫不动摇鼓励、支持、引导非公有制经济发展,激发非公有制经济活力和创造力,A、C、D正确。公有制经济和非公有制经济是我国社会主义初级阶段基本经济制度的组成部分,但不是基本经济制度的实现形式。因为所有制和所有制的实现形式是两个不同层次的问题。公有制经济和非公有制经济讲的是所有权的归厲,而所有制的实现形式要解决的是发展生产力的组织形式和经营方式问题,两者不能混淆,B错误。
\end{solution}
