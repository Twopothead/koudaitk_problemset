\question 华罗庚生前曾说:``我们最好把自己的生命看作是前人生命的延续,是现在人类共同的生命的一部分,同时也是后人生命的开端。如此延续下去,科学就会一天比一天更灿烂,社会就会一天比一天更美好。''这段话对我们如何实现人的个人价值的教益是(
~)
\par\fourch{\textcolor{red}{个人价值的实现与社会价值的实现是统一的}}{\textcolor{red}{个人价值的实现是一个历史过程}}{个人价值的实现是社会价值的实现的归宿}{个人价值的实现和个人生命的长短相一致}
\begin{solution}本题干扰项也是较为容易排除的。C选项,明显表述``反了'',应该是社会价值的实现是个人价值的实现的归宿。D选型,明显荒谬,如果D选项是对的,那么雷锋同志的人生价值就很少了。所以选出AB选项。
\end{solution}
\question 人生价值具有的特点有( ~)
\par\twoch{\textcolor{red}{客观性}}{\textcolor{red}{社会性}}{\textcolor{red}{差异性}}{\textcolor{red}{创造性}}
\begin{solution}人生价值是一种特殊的价值,它是人的生活实践对于社会和个人所具有的作用和意义。它是在社会的创造性的实践中实现的,它必须符合社会的客观规律,而且每个人由于具有个体的差异性,他们的人生价值也有所不同。C最易被漏选。
\end{solution}
\question 所谓人生的社会价值,就是个体的人生对于社会和他人的意义。人生社会价值的基本标志有(
~)
\par\twoch{\textcolor{red}{劳动}}{\textcolor{red}{贡献}}{索取}{实践}
\begin{solution}人生价值评价的根本尺度,是看一个人的人生活动是否符合社会发展的客观规律,是否通过实践促进了历史的进步。劳动以及通过劳动对社会和他人做出的贡献,是社会评价一个人的人生价值的普遍标准。
\end{solution}
\question 马克思指出:``人是最名副其实的政治动物,不仅是一种合群的动物,而且是只有在社会中才能独立的动物。''这表明(
~)
\par\fourch{\textcolor{red}{社会是个人生存和发展的基础}}{\textcolor{red}{个人是构成社会的前提}}{\textcolor{red}{个人与社会不可分离}}{\textcolor{red}{个人与社会是对立统一的}}
\begin{solution}题干中讲人与社会的对立统一的关系。社会是个人生存发展的基础,社会是由个人构成的。
\end{solution}
