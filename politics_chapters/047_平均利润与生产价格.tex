\question 商业资本作为一种独立的职能资本,也获得平均利润,其直接原因是( )
\par\fourch{\textcolor{red}{商业部门和产业部门之间的竞争和资本转移}}{产业资本家为销售商品将部分利润让渡给商业资本家}{商业资本家加强对商业雇员的剥削}{产业部门将工人创造的一部分剩余价值分割给商业部门}
\begin{solution}平均利润是按照平均利润率所取得的利润;平均利润率是社会剩余价值总量与社会总资本的比率。平均利润是部门之间竞争的结果,或者说,是资本在不同部门之间流动的结果。这里一定要明确:只有是一个部门(而不是部门内部的企业),才有获得平均利润的资格。B、D项是讲商业利润的实现方式;C项是讲商业利润的来源。
\end{solution}
\question 商业资本作为一种独立的职能资本,也获得平均利润,其直接原因是
\par\fourch{\textcolor{red}{商业部门和产业部门之间的竞争和资本转移}}{产业资本家为销售商品将部分利润让渡给商业资本家}{商业资本家加强对商业雇员的剥削}{产业部门将工人创造的一部分剩余价值分割给商业部门}
\begin{solution}本题考查的是``商业资本和商业利润''这一知识点的内容。商业资本和商业利润是剩余价值分配理论中的重要理论点和知识点,但并不是政治经济学这一学科在考研当中的常考点,本题是围绕商业资本和商业利润这两个基本概念进行考查。商业资本并不创造剩余价值,但商业资本作为一种独立的职能资本,也要获得平均利润。这是通过商业部门和产业部门之间的竞争,以及它们之间的资本转移实现的。因此,答案是A。本题的考查目的在于引导考生理解市场机制如何配置资源,如何实现按生产要素进行分配以及资本家的剥削实质。本题考查方式还在于引导考生重在理解,不要死记硬背。对于政治经济学这一学科的复习,考生应始终明白一条:死记硬背是一种事倍功半的复习方法。
\end{solution}
