\question 1931年初,中国共产党形成的土地革命的阶级路线的内容有( ~)
\par\fourch{\textcolor{red}{依靠贫雇农,联合中农}}{\textcolor{red}{保护工商业者,消灭地主阶级}}{\textcolor{red}{限制富农}}{在原耕地的基础上,实行抽多补少、抽肥补瘦}
\begin{solution}毛泽东和邓子恢等其他同志一起规定的土地革命中的阶级路线是:坚定地依靠贫农、雇农,联合中农,限制富农,保护中小工商业者,消灭地主阶级;其分配方法是:以乡为单位,按人口平分土地,在原耕地的基础上,实行抽多补少、抽肥补瘦。D为分配方法而非阶级路线的内容。
\end{solution}
\question 1928年,毛泽东按照``没收一切土地归苏维埃政府所有''的原则,主持制定了我国历史上第一部彻底消灭封建土地所有制的土地法,这就是(
)
\par\twoch{\textcolor{red}{《井冈山土地法》}}{《兴国土地法》}{《苏维埃土地法》}{《中国土地法大纲》}
\begin{solution}中国第一部土地法就是第一个农村革命根据地井冈山时期颁布的《井冈山土地法》。
\end{solution}
\question 毛泽东指出:``如果不帮助农民推翻封建地主阶级,就不能组成中国革命最强大的队伍而推翻帝国主义的统治。''其实质含义是(
)
\par\fourch{农民阶级是中国革命的领导阶级}{\textcolor{red}{农民阶级是中国革命最可靠的同盟军}}{\textcolor{red}{没有农民阶级参加,中国革命就不能取胜}}{农民阶级反帝反封建态度最坚决,是新的社会生产力的代表者}
\begin{solution}农民阶级不能是革命的领导阶级,革命领导阶级是工人阶级。农民阶级不是先进生产力的代表者,所以AD选项错。
\end{solution}
\question 1948年中国共产党制定了土地改革总路线。下列选项中对这一总路线所含内容理解正确的有
\par\fourch{\textcolor{red}{按照平分土地的原则,满足贫雇农的要求}}{\textcolor{red}{团结中农,允许中农保有比他人略多的土地}}{没收地主土地,不再对地主分配土地}{\textcolor{red}{实行耕者有其田,将土地的所有权分配给农民}}
\begin{solution}本题考查新民主主义的经济纲领中没收地主土地归农民所有的内容。1948年毛泽东《在晋绥干部会议上的讲话》一文中明确提出了土地改革的总路线,即``依靠贫民、团结中农,有步骤地、有分别地消灭封建剥削制度,发展农业生产。''其中土地改革的主要的和直接的任务,是满足贫雇农群众的要求,赞成平分土地的要求,是为了便于发动广大农民群众迅速消灭封建地主阶级的土地所有制,并非提倡绝对平均主义。土地改革的另一个任务,是满足某些中农的要求,必须容许一部分中农保有比一般贫农所得土地的平均水平为高的土地量。民主革命时期,实行耕者有其田,即指没收地主土地归农民私有,而非归国家所有,故A、B、D为正确选项。C是错误的选项,因为土地改革的目的是消灭封建剥削制度,即消灭封建地主之为阶级,而不是地主个人,因此,对地主必须分给和农民同样的土地财产,把它们改造成为自食其力的劳动者。本题有一定难度,专业性很强。
\end{solution}
