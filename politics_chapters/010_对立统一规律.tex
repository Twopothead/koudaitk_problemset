\question 对立统一规律揭示了( )
\par\twoch{事物发展的方向和道路}{\textcolor{red}{事物发展的源泉和动力}}{事物发展的状态和形式}{\textcolor{red}{事物普遍联系的本质内容}}
\begin{solution}本题考查的知识点:对立统一规律是唯物辩证的实质与核心
唯物辩证法包括对立统一规律、质量互变规律和否定之否定规律的三大规律,其中对立统一规律是唯物辩证法体系的实质和核心,因为对立统一规律揭示了事物普遍联系的根本内容和永恒发展的内在动力,从根本上回答了事物为什么会发展的问题;对立统一规律是贯穿质量互变规律、否定之否定规律以及唯物辩证法基本范畴的中心线索,也是理解这些规律和范畴的``钥匙'';对立统一规律提供了人们认识世界和改造世界的根本方法------矛盾分析法,它是对事物辩证认识的实质;是否承认对立统一学说是唯物辩证法和形而上学对立的实质。所以,本题的正确答案是选项BD。
选项A错误。否定之否定规律揭示了事物发展的方向和道路。
选项C错误。质量互变规律揭示了事物发展的状态和形式。
\end{solution}
\question 唯物辩证法的实质和核心是( )
\par\twoch{\textcolor{red}{对立统一规律}}{普遍联系规律}{质量互变规律}{否定之否定规律}
\begin{solution}对立统一规居于唯物辩证法的实质和核心地位。
\end{solution}
\question 矛盾的基本属性是( )
\par\twoch{普遍性和特殊性}{\textcolor{red}{斗争性和同一性}}{绝对性和相对性}{变动性和稳定性}
\begin{solution}A是矛盾的精髓,C运动的基本属性,D表述不正确。
\end{solution}
\question 矛盾同一性在事物发展中的作用表现为( )
\par\fourch{\textcolor{red}{矛盾双方在相互依存中得到发展}}{\textcolor{red}{矛盾双方相互吸取有利于自身发展的因素}}{调和矛盾双方的对立}{\textcolor{red}{规定事物发展的基本趋势}}
\begin{solution}矛盾统一性在事物发展中的作用表现在三个方面。
\end{solution}
\question 构成矛盾的两种基本属性是( )
\par\twoch{普遍性和特殊性}{\textcolor{red}{同一性和斗争性}}{绝对性和相对性}{对抗性和非对抗性}
\begin{solution}本题是对对立统一规律和矛盾概念最基本含义的理解。矛盾就是对立统一,就是斗争性和同一性的关系,所以矛盾的两种基本属性就是同一性和斗争性。
\end{solution}
