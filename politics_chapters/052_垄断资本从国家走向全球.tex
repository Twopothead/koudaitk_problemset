\question 国家垄断资本主义是国家政权和私人垄断资本融合在一起的垄断资本主义。第二次世界大战结束以来,在国家垄断资本主义获得充分发展的同时,资本主义国家通过宏观调节和微观规制对生产、流通、分配和消费各个环节的干预也更加深入。其中,微观规制的类型主要有
\par\twoch{\textcolor{red}{社会经济规制}}{\textcolor{red}{公共亊业规制}}{公众生活规制}{\textcolor{red}{反托拉斯法}}
\begin{solution}微观规制主要是国家运用法律手段规范市场秩序,限制垄断,保护竞争,维护社会公众的合法权益。ABD选项正确。
\end{solution}
\question 垄断资本向世界范围的扩展,产生了一系列的经济社会后果:对于资本输出国来讲,资本输出为其带来了巨额利润,带动和扩大了商品输出,大大改善了国际收支状况,对发展中国家的经济命脉形成控制。对于资本输入国主要是发展中国家来讲,资本输人对其经济和社会发展产生了一定的积极作用,如吸收了资金,引进了较先进的机器设备和工艺技术,培训了技术和管理人员,利用外贸和技术办厂,促进经济发展,增加了就业,扩大了外贸等。垄断资本向世界范围扩展的经济动因是
\par\fourch{\textcolor{red}{将国内过剩的资本输出,以在别国谋求高额利润}}{\textcolor{red}{将部分非要害技术转移到国外,以取得在别国的垄断优势}}{\textcolor{red}{争夺商品销售市场}}{\textcolor{red}{确保原材料和能源的可靠来源}}
\begin{solution}【解析】垄断资本在国内建立了垄断统治后,必然要把其统治势力扩展到国外,建立国际垄断统治。垄断资本向世界范围扩展的经济动因:一是将国内过剩的资本输出,以在别国谋求高额利润;二是将部分非要害技术转移到国外,以取得在别国的垄断优势;三是争夺商品销售市场;四是确保原材料和能源的可靠来源。这些经济上的动因与垄断资本主义政治上、文化上、外交上的利益交织一起,共同促进了垄断资本主义向世界范围的扩展。垄断资本向世界范围扩展的基本形式有三种:一是借贷资本输出;二是生产资本输出;三是商品资本输出。从输出资本的来源看,主要有两类:一类是私人资本输出;另一类是国家资本输出。
\end{solution}
\question 早在150多年前,马克思与恩格斯就已指出``不断扩大产品销路的需要,驱使资产阶级奔走于全球各地'',``资产阶级由于开拓了世界市场,使一切国家的生产和消费都成为世界性的了。\ldots{}\ldots{}在过去那种地方的和民族的自给自足的闭关自守状态,被各民族的各方面的互相往来和各个方面的互相依赖所代替了''。这段话说明(
)
\par\twoch{\textcolor{red}{全球化趋势具有客观必然性}}{\textcolor{red}{全球化是生产社会化发展的结果}}{\textcolor{red}{全球化是由资本主义国家推动的}}{发展中国家只能被动地参与全球化}
\begin{solution}D经济全球化对发展中国家也具有积极的影响,可以引进先进技术和管理经验,增强经济竞争力;通过吸引外资,扩大就业;利用不断扩大的国际市场解决产品销售问题;可以借助自由化和比较优势组建大型跨国公司,积极参与全球化进程。
\end{solution}
\question 金融寡头在经济领域中的统治主要是通过( )
\par\twoch{股份制实现的}{控股制实现的}{\textcolor{red}{参与制实现的}}{联合制实现的}
\begin{solution}金融寡头在经济领域中的统治主要是通过``参与制''实现的。所谓参与制就是金融寡头通过掌握一定数量的股票来层层控制企业的制度。
\end{solution}
