\question 土地改革完成后,毛泽东分析我国农民的两大积极性是指
\par\fourch{\textcolor{red}{个体经济积极性}}{\textcolor{red}{互助合作积极性}}{集体经济积极性	}{个体劳动积极性}
\begin{solution}【简析】土地改革完成后,我国广大农民从封建剥削制度下解放出来,生产积极性大大提高。这种积极性表现在两个方面:一是个体经济的积极性,二是互助合作的积极性。A、B正确。
\end{solution}
\question 社会主义基本制度的确立是中国历史上最深刻最伟大的社会变革,为当代中国一切发展进步奠定了制度基础,也为中国特色社会主义制度的创新和发展提供了重要条件。社会主义基本制度确立的重大意义在于
\par\fourch{\textcolor{red}{极大地提髙了工人阶级和广大劳动人民的积极性和创造性,极大地促进了我国社会生产力的发展}}{形成了社会主义制度的中国模式,为其他国家确立社会主义提供了标准}{\textcolor{red}{使广大劳动人民真正成为国家的主人}}{\textcolor{red}{是世界社会主义运动史上又一个历史性的伟大胜利,进一步改变了世界政治经济格局}}
\begin{solution}【简析】社会主义没有统一的标准和模式,B错误。
\end{solution}
