\question 在当今信息社会,现代科技进步和社会经济发展对信息资源、信息技术和信息产业的依赖性越来越大。在信息社会,智能化的综合网络将遍布社会的各个角落。``无论何事、无论何时、无论何地''人们都可以获得文字、声音、图像信息。这说明(
)
\par\fourch{信息社会改变了生产力的内容和性质,从而改变了生产关系的性质}{由虚拟网络建立的人与人之间的关系将成为新型的社会基本关系}{在信息社会中,网络信息关系将成为社会的基本关系}{\textcolor{red}{网络信息关系影响并推动社会发展,但并不能成为新型的社会基本关系}}
\begin{solution}网络信息关系影响和推动社会发展,但不能改变社会基本关系。
\end{solution}
\question 马克思指出:``我们首先应当确定一切人类生存的第一个前提,也就是一切历史的第一个前提,这个前提是:人们为了能够`创作历史',必须能够生活。\ldots{}\ldots{}因此第一个历史活动就是生产满足这些需要的资料,即生产物质生活本身''这句话表明(
)
\par\fourch{\textcolor{red}{物质生产是人类社会赖以存在的物质基础}}{\textcolor{red}{物质生产的发展决定着整个社会的性质和面貌}}{\textcolor{red}{物质生产方式的变革是社会历史变革的根本原因}}{物质生产对社会发展起着重要作用,但不是决定作用}
\begin{solution}D选项错误,就是决定作用。
\end{solution}
\question 物质生活的生产方式是社会历史发展的决定力量,其表现有( )
\par\twoch{\textcolor{red}{它制约着全部社会生活}}{\textcolor{red}{它决定社会形态的更替和发展}}{\textcolor{red}{它决定社会的面貌}}{\textcolor{red}{它决定社会的性质}}
\begin{solution}本题考查生产方式在社会存在和发展中的作用。生产方式是生产力和生产关系的统一,是社会历史发展的决定力量。物质资料生产方式是人类社会赖以存在和发展的基础。生产方式决定着社会的结构、性质和面貌,制约着人们的经济生活、政治生活和精神生活等全部社会生活。生产方式的变化发展决定整个社会历史的变化发展,决定社会形态的更替和发展。依据以上关于生产方式的论述,本题全选。
\end{solution}
