\question 第二次世界大战结束以来,在国家垄断资本主义获得充分发展的同时,国家通过宏观调节和微观规制对生产、流通、分配、消费各个环节的干预也更加深人。其中,微观规制的类型主要有
\par\twoch{\textcolor{red}{社会经济规制}}{\textcolor{red}{公共事业规制}}{财政政策、货币政策等经济手段}{\textcolor{red}{反托拉斯法}}
\begin{solution}【简析】微观规制主要是国家运用法律手段规范市场秩序,限制垄断,保护竞争,维护社会公众的合法权益。微观规制主要有三种类型:其一,反托拉斯法;其二,公共事业规制;其三,社会经济规制。A、B、D正确。国家运用财政政策、货币政策等经济手段,对社会总供求进行调节,属于宏观调节的主要手段,不是微观规制的内容。C错误。
\end{solution}
\question 全球最大的手机芯片厂商之一美国髙通公司2015年2月10日宣布,将向中国官方支付60.88亿元人民币(约合9.75亿美元)的反垄断罚款,了结为期14个月的反垄断调査,这是中国反垄断开出的最大一张罚单。2015年11月中旬,高通又被韩国反垄断机构认定其专利授权模式违反该国反垄断法,导致该公司股票在11月18日暴跌10\%。这是高通在该月遭遇的第二次监管挫折。高通之前还披露,虽然已经在中国和解了反垄断诉讼,但该公司在与部分中国手机制造商洽谈新的授权协议时仍然面临一些问题。这两起事件共导致高通股票市值蒸发200亿美元。垄断资本向世界扩展的经济动因包括
\par\fourch{\textcolor{red}{争夺商品销售市场}}{\textcolor{red}{将国内过剩的资本输出,以在别国谋求髙额利润}}{将部分要害核心技术转移到国外,以取得在别国的垄断优势}{\textcolor{red}{确保原材料和能源的可靠来源}}
\begin{solution}【简析】垄断资本在1肉建立了垄断统治后,必然要把其统治势力扩展到国外,建立国际垄断统治。垄断资本向世界范围扩展的经济动因:一是将国内过剩的资本输出,以在别国谋求高额利润;二是将部分非要害技术转移到国外,以取得在别国的垄断优势;三是争夺商品销售市场;四是确保原材料和能源的可靠来源。这呰经济上的动因与垄断资本主义政治上、文化上、外交上的利益交织一起,共同促进了垄断资本主义向世界范闹的扩展。正确,C错误。注意:C错在``要害核心技术'',正确说法应该是``非要害技术''。
\end{solution}
\question 各资本主义国家的垄断组织,通过订立协议建立起国际垄断资本的联盟,即国际垄断同盟,以便在世界范闹形成垄断,并在经济上瓜分世界。以下说法正确的是
\par\fourch{\textcolor{red}{国际垄断同盟在经济上瓜分世界是通过垄断组织间的协议实现的}}{早期的国际垄断同盟主要是跨国公司}{当代国际垄断同盟的形式以国际卡特尔和国家垄断资本主义的国际联盟为主}{\textcolor{red}{国家垄断资本主义的国际联盟,是国际垄断同盟的高级形式}}
\begin{solution}【简析】A、D正确。早期的国际垄断同盟主要是国际卡特尔,即若干垄断资本主义国家的生产或经荐某种产品的垄断组织,通过订立国际卡特尔协议,垄断和瓜分这种产品的世界市场,规定垄断价格,谋求垄断利润。当代国际垄断同盟的形式以跨国公司和国家垄断资本主义的国际联盟为主D
B、C错误。
\end{solution}
\question 随着资本输出的不断增加和垄断资本势力范ffl的迅速扩大,各资本主义国家的垄断组织,通过订立协议违立起围际垄断资本的联盟,即诚际垄断同盟,同时,还建立起国际经济调节机制,以加强国际协调\ldots{}.第二次世界大战后,从事国际经济协调、维护国际经济秩序的闽际性协调组织主要有
\par\twoch{联合国}{\textcolor{red}{国际货币基金组织}}{\textcolor{red}{世界银行}}{\textcolor{red}{世界贸易组织}}
\begin{solution}【简析】随着资本输出的不断增加和垄断资本势力范围的迅速扩大,各国之间的经济联系曰益密切,同时,彼此间的竞争更为激烈,矛盾和冲突也更为突出。在这个背景下,各资本主义国家的垄断组织,通过订立协议建立起国际垄断资本的联盟,即国际垄断同盟,以便在世界范围形成垄断,并在经济上瓜分世界。国际垄断资本还建立起国际经济调节机制,以加强国际协调。第二次世界大战后,从事国际经济协调、维护国际经济秩序的国际性协调组织主要有三个:国际货币基金组织、世界银行和世界贸易组织B、C、D正确,A错误。
\end{solution}
\question 经济全球化是指在生产不断发展、科技加速进步、社会分工和国际分工不断深化、生产的社会化和了际化程度不断提高的情况下,世界各国、各地区的经济活动越来越超出一国和地区的范围而相互联系、相互依赖的一体化过程。经济全球化的表现包括
\par\fourch{国际垂直分工成为居主导地位的分工形式}{\textcolor{red}{贸易的全球化}}{\textcolor{red}{金融的全球化}}{\textcolor{red}{企业生产经营的全球化}}
\begin{solution}【简析】经济全球化的表现包括:一是国际分工进一步深化。在经济全球化过程中,国际水平分工逐渐取代国际垂直分工成为居主导地位的分工形式。A错误。二是贸易的全球化。贸易全球化是指商品和劳务在全球范闹内自由流动。三是金融的全球化。金融全球化是指世界各国、各地区在金融业务、金融政策等方面相互协调、相互渗透、相互竞争不断加强,使全球金融市场更加开放、金融体系更加融合、金融交易更加自由的过程。四是企业生产经营的全球化。企业生产经营全球化指跨国公司在全球范围内建立分支机构,借助母公司与分支机构之间各种形式的联系,实行跨国投资和生产的过程。B、C、D正确。
\end{solution}
\question 导致经济全球化迅猛发展的因素主要有
\par\fourch{\textcolor{red}{科学技术的进步和生产力的发展}}{\textcolor{red}{跨国公司的发展}}{\textcolor{red}{各国经济体制的变革}}{发展中国家的斗争}
\begin{solution}【简析】略
\end{solution}
\question 当代资本主义国家经济出现的新变化主要表现在
\par\fourch{\textcolor{red}{生产资料所有制的变化}}{\textcolor{red}{劳资关系和分配关系的变化}}{改良主义政党在政治舞台上的影响日益扩大}{\textcolor{red}{经济调节机制和经济危机形态的变化}}
\begin{solution}【简析】改良主义政党在政治舞合上的影响日益扩大属于政治制度的变化,与题干所问的经济方面的变化不符,不选。
\end{solution}
\question 对资本主义国家的国家资本所有制认识正确的是
\par\fourch{\textcolor{red}{国家作为出资人,拥有国有企业的所有权和控制权}}{\textcolor{red}{固有企业的重要职能是推行政府的社会政策和经济政策,为私人垄断资本的发展提供服务和保障}}{是一种基于资本雇佣劳动的垄断资本集体所有制}{\textcolor{red}{所有制的性质仍然是资本主义性质,体现着总资本家剥削雇佣劳动者的关系}}
\begin{solution}【简析】A、B是资本主义国家所有制的主要特点,D是其地位与性质,都是正确选项。法人资本所有制是一种基于资本雇佣劳动的垄断资本集体所有制;国家资本所有制是生产资料由国家占有并服务于垄断资本的所有制形式,体现着总资本家剥削劳动者的关系,C错误。
\end{solution}
\question 当代资本主义国家在劳资关系和分配关系上出现的新变化布
\par\twoch{\textcolor{red}{职工参与决策}}{\textcolor{red}{终身雇佣、职工持股}}{\textcolor{red}{建立并实施了普及化、全民化的社会福利制度}}{建立职工选举管理者制度}
\begin{solution}【简析】资本主义国家没有也不可能建立职工选举管理者制度,D错误。
\end{solution}
\question 在当代资本主义生产关系中,阶层、阶级结构发生了新的变化,主要有
\par\fourch{\textcolor{red}{髙级职业经理成为大公司经营活动的实际控制者}}{\textcolor{red}{知识型和服务型劳动者的数坫不断增加}}{中产阶级迅速崛起,占有社会财富的绝大部分}{\textcolor{red}{资本家的地位和作用已经发生很大变化}}
\begin{solution}【简析】在当代资本主义生产关系中,阶层、阶级结构发生了新的变化。一趄资本家的地位和作用发生了很大的变化。资本所有权和经营权发生分离,拥有所有权的资本家一般不再a接经营和管理企业,而是靠拥有的企业股票等有价诬券的利息收人为生。二是高级职业经理成为大公司经营活动的实际控制者。三是知识型和服务型劳动者的数®不断增加,劳动方式发生了新变化。A、B、D正确。资本主义国家的少数富人占有社会财富的大部分,C不符合事实,不选。
\end{solution}
\question 与第二次世界大战前的资本生义相比,当代资本主义在许多方而已经并正在发生变化。其中在政治制度上的变化有
\par\fourch{\textcolor{red}{政治制度出现多元化的趋势,公民权利有所扩大}}{民粹主义影响上升}{\textcolor{red}{法制建设得到重视和加强,以协调社会各阶级、阶层之间的利益}}{\textcolor{red}{改良主义政党在政治舞台上的影响日益扩大}}
\begin{solution}【简析】当代资本主义在政治制度上的变化有:首先,政治制度出现多元化的趋势,公民权利有所扩大。其次,法制建设得到重视和加强,以协调社会各阶级、阶层之间的利益。最后,改良主义政党在政治舞台上的影响日益扩大。A、C、D正确,B错误。
\end{solution}
\question 当代资本主义出现的新变化原因是多方面的,主要有
\par\fourch{\textcolor{red}{科学技术革命和生产力的发展,是资本主义变化的根本推动力量}}{\textcolor{red}{工人阶级争取自身权力斗争的作用,是推动资本主义变化的重要力量}}{\textcolor{red}{社会主义制度初步显示的优越性对资本主义产生了一定影响}}{主张改良主义的政党对资本主义制度的改革触动资本主义统治的根基}
\begin{solution}【简析】当代资本主义发生新变化的原因主要有:首先,科学技术革命和也产力的发展,是资本主义变化的根本推动力诳;其次,工人阶级争取自身权刺斗争的作用,是推动资本主义变化的重要力量;再次,社会主义制度初步显示的优越性对资本主义产生了一定影响;最后,主张改良主义的政党对资本主义制度的改难,也对资本主义的变化发挥了重耍作用。A、B、C正确。当代资本主义的新变化是深刻的,其意义也是深远的,但是,这些变化并没布触动资本主义统治的根基,并没有改变资本主义制度的性质。D错误。
\end{solution}
