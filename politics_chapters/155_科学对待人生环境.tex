\question 人们通过生活实践所形成的对人生问题的一种稳定的心理倾向和基本意图是( ~)
\par\twoch{人生观}{人生价值}{\textcolor{red}{人生态度}}{人生目的}
\begin{solution}本题考查人生观与人生价值、人生态度、人生目的之间的区别。人生观是世界观的重要组成部分,是人们在实践中形成的对于人生目的和意义的根本看法。人生观主要是通过人生目的、人生态度和人生价值三个方面体现出来的。人生目的,回答``人为什么活着'';人生态度,表明人``应当怎样对待生活,究竟应该怎样活着''也即人们通过生活实践形成的对人生问题的一种稳定的心理倾向和基本意愿,极易与``人生目的''混淆。人生价值,判别``什么样的人生才有意义,什么样的人生目的最值得追求''。因此,C正确。
\end{solution}
