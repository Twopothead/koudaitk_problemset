\question 发展的过程性原理要求人们( )
\par\twoch{\textcolor{red}{坚持阶段论,反对超阶段论}}{\textcolor{red}{反对形而上学的“不变论”与“激变论”}}{\textcolor{red}{反对把知识、真理绝对化的观点}}{坚持世界是既成事物的集合体的观点}
\begin{solution}本题考查的知识点:发展的过程性
事物的发展是一个过程。一切事物只有经过一定的过程才能实现自身的发展。所谓过程是指一切事物都有其产生、发展和转化为其他事物的历史,都有它的过去、现在和未来。自然界、人类社会和思维领域中的一切现象都是作为一个过程而存在、作为过程而发展的。坚持事物发展是过程的思想,就要用历史的眼光看问题,把一切事物如实地看作是变化、发展的过程,既要了解它们的过去、观察它们的现在,又要预见它们的未来。因此,要坚持阶段论,反对超阶段论;反对形而上学的``不变论''与``激变论'';反对把知识、真理绝对化的观点。在今天,科学地认识建设中国特色社会主义的历史必然性、历史过程、历史阶段、发展规律和发展趋势,对我们坚定信念、积极投身社会主义现代化建设的伟大实践具有重要的现实意义。所以,本题的正确答案为选项ABC。
选项D错误。恩格斯指出:``世界不是既成事物的集合体,而是过程的集合体。''这是唯物辩证法的``一个伟大的基本思想''。事物发展的过程,从形式上看,是事物在时间上的持续性和空间上的广延性的交替;从内容上看,是事物在运动形式、形态、结构、功能和关系上的更新。
\end{solution}
