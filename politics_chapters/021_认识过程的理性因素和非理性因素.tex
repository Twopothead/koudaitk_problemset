\question 心理学家将一条饥饿的鳄鱼和许多小鱼放在同一个大水箱里,中间用透明的玻璃隔开。最初,鳄鱼毫不犹豫地向小鱼发起进攻,但每次都碰壁了。多次进攻无果后,它放弃了努力。后来心理学家取走玻璃挡板,小鱼在鳄鱼身边游来游去,但鳄鱼始终无动于衷,最后饿死了。
``鳄鱼试验''进一步佐证了( )
\par\fourch{\textcolor{red}{动物心理没有能动性,不能透过现象把握事物的本质}}{动物心理没有适应性,不能根据环境的变化而变化}{\textcolor{red}{动物心理没有创造性,不能通过行动改变环境以满足生存的需要}}{动物心理没有主观性,不能创造现实世界所没有的幻想的世界}
\begin{solution}本题考查动物的本能活动与人的意识活动的本质区别。主观能动性是人与其他动物的本质区别,动物的心理活动是其本能的被动地适应环境的活动,动物没有意识,意识是人特有的活动,意识活动具有自觉能动性的特点,意识活动具有主动创造性。动物心理有适应性可以适应环境,故B选项错误。``鳄鱼试验''进一步说明,动物只是被动、机械地适应环境,写物心理没有自觉、主动的特点,故AC符合题意,D与题意无关。
\end{solution}
\question 感性认识和理性认识有着密不可分的辩证联系,表现在(  )
\par\fourch{\textcolor{red}{理性认识依赖于感性认识}}{\textcolor{red}{感性认识有待于发展和深化为理性认识}}{\textcolor{red}{感性认识和理性认识相互渗透、相互包含}}{\textcolor{red}{感性认识和理性认识在实践的基础上辩证统一}}
\begin{solution}【解析】感性认识和理性认识是统一的认识过程中的两个阶段,它们既有区别,又有联系。感性认识和理性认识的相互联系表现在:①感性认识和理性认识互相依存。理性认识依赖于感性认识,这是认识论的唯物论;感性认识有待于发展到理性认识,这是认识论的辩证法。②在实际的认识过程中,感性认识和理性认识又是互相交织、互相渗透的。一方面,感性中渗透着理性的因素;另一方面,理性中渗透着感性的因素。感性认识和理性认识是辩证统一的,两者统一的基础是实践。感性认识是在实践中产生的,由感性认识到理性认识的过渡,也是在实践的基础上实现的。
\end{solution}
