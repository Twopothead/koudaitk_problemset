\question 真理的绝对性是指( ~)
\par\fourch{\textcolor{red}{任何真理都包含不依赖于人的意识的客观内容,这是无条件的、绝对的}}{\textcolor{red}{人类完全可以认识无限发展的物质世界,这是无条件的,绝对的}}{\textcolor{red}{人的认识是无限发展的}}{真理在广度上是有待扩展的}
\begin{solution}本题考查考生对真理的绝对性含义的记忆和理解。真理的绝对性包括了三层含义,其基本内容,就是选项A、B、C所表达的意思。选项D是关于真理的相对性的含义,是明显的干扰项。
\end{solution}
\question 马克思主义认识论认为,认识的辩证过程是(  )
\par\fourch{\textcolor{red}{从相对真理到绝对真理的发展}}{从间接经验到直接经验的转化}{\textcolor{red}{实践——认识——实践的无限循环}}{从抽象到具体再到抽象的上升运动}
\begin{solution}【解析】认识的过程是从实践到认识再到实践的循环过程,也是感性认识上升到理性认识的过程,是相对真理向绝对真理的转变过程。
\end{solution}
\question ``真理和谬误在一定条件下能互相转化'',这说明(  )
\par\fourch{真理就是谬误,谬误就是真理,两者没有绝对的界限}{真理与谬误在同一范围内可以互相转化}{\textcolor{red}{真理超出自己适用的范围会转化为谬误}}{\textcolor{red}{谬误回归适合的范围会转化为真理}}
\begin{solution}【解析】真理与谬误既对立又统一:①真理与谬误是对立的。真理和谬误决定于认识的内容是否如实地反映了客观事物,因此真理和谬误是性质不同的两种认识。所以,就一定范围、一定客观对象来说,真理就是真理、谬误就是谬误,二者有本质的区别,不能混淆,也不可能转化。②真理与谬误又是相互联系的。真理是与谬误相比较而存在的,没有谬误也就无所谓真理。真理的发展也是通过与谬误的斗争来实现的。③真理与谬误在一定条件下相互转化。真理与谬误的区别和对立并不是绝对的,任何真理都是在一定范围内、一定条件下才能够成立。如果超出这个范围,失去特定条件,它就会变成谬误。
\end{solution}
