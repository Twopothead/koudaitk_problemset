\question 脱离物质的运动和脱离运动的物质都是不可想象的。因此,运动就是物质,物质就是运动。对这句话理解正确的是(
)
\par\twoch{正确理解了物质和运动的关系}{\textcolor{red}{是形而上学唯物主义的物质观}}{\textcolor{red}{混淆了物质的属性与物质本身}}{是正确的命题,体现了运动是物质的根本属性}
\begin{solution}本题考查的知识点:物质和运动的关系
分析题干:通过对题干及选项的分析,可知此题关键词是``物质''、``运动'',由此判断此题考点是物质与运动的关系。马克思主义哲学认为,作为哲学范畴的运动是指宇宙中发生的一切变化和过程,它是物质的根本属性和存在方式。物质和运动是不可分割的。一方面,物质是运动的物质,没有不运动的物质。运动是物质所固有的根本属性和一切物质形态的存在方式。设想有不运动的物质是形而上学唯物主义的特征。另一方面,运动是物质的运动,没有无物质的运动。物质是运动的承担者,是一切运动和发展的实在基础,运动的原因也在物质自身。设想有离开物质的运动是唯心主义的观点。
分析选项:题干充分反映了物质和运动不可分割的联系。但是因此而把物质和运动等同起来,认为物质就是运动,运动就是物质,则是不正确的。物质和运动不可分,但不是说两者没有区别,不能把物质的属性同物质本身等同起来。所以,符合题意要求的选项是BC。AD选项没有正确理解题意,故排除。
\end{solution}
\question 关于运动,正确的论断有( )
\par\twoch{\textcolor{red}{它是物质的存在方式}}{它是物质的本质规定性}{\textcolor{red}{它是物质的根本属性}}{它是物质的最高共性}
\begin{solution}运动是指宇宙中发生的一切变化和过程,是物质的存在方式和根本属性。BD项是关于物质之客观实在性的论断。
\end{solution}
\question 运动和物质不可分割。将二者分割开来所导致的错误有( )
\par\twoch{不可知论}{\textcolor{red}{形而上学}}{\textcolor{red}{唯心主义}}{相对主义}
\begin{solution}物质和运动是不可分割的,一方面,物质是运动着的物质,运动是物质的根本属性,脱离运动的物质是不存在的;设想不运动的物质,将导致形而上学。另一方面,运动是物质的运动,物质是运动的主体、实在基础和承担者;设想无物质的运动,将导致唯心主义。
\end{solution}
\question 长江的年龄有多大?这里说的长江``年龄'',是指从青藏高原奔流而下注入东海的``贯通东流''水系的形成年代。如果说上游的沉积物从青藏高原、四川盆地顺延而下能到达下游,这就表明长江贯通了,这就是物源示踪。我国科学家采用这一方法,研究长江中下游盆地沉积物的来源,从而判别长江上游的物质何时到达下游,间接指示了长江贯通东流的时限。他们经过10多年的研究,提出长江贯通东流的时间距今约2300多万年。这一研究成果从一个侧面显示出
\par\fourch{时间和空间是有限的,物质运动是永恒的}{\textcolor{red}{时间和空间是通过物质运动的变化表现出来的}}{时间和空间是标示物质运动的观念形式}{\textcolor{red}{时间和空间是物质运动的存在形式}}
\begin{solution}此题考查的是时间和空间的特点。时空是客观的是物质运动的存在形式,具有有限性和无限性,绝对性和相对性的特点。A选项表述错误,时空是有限的也是无限的,是和物质的运动紧密结合的。C本身表述错误,时空是客观的,不是观念形式。所以正确答案是BD。
\end{solution}
