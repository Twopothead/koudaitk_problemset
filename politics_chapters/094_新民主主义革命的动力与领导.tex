\question 民主革命时期,中国革命的基本问题是( )
\par\twoch{\textcolor{red}{农民问题}}{统一战线问题}{无产阶级领导权问题}{武装斗争}
\begin{solution}此题考查的知识点是新民主主义革命的动力中的中国革命的基本问题,是一道识记性试题,难度适中。农民问题是中国革命的基本问题,新民主主义革命实质上就是中国共产党领导下的农民革命,中国革命战争实质上就是党领导下的农民战争,A选项正确。B选项,统一战线是无产阶级政党策略思想的重要内容。C选项,无产阶级领导权问题是中国革命的中心问题,又是新民主主义革命理论的核心问题。D选项,武装斗争是中国革命的特点和优点之一。
\end{solution}
\question 新民主主义革命理论的核心问题是
\par\twoch{分清敌我}{农民问题}{\textcolor{red}{无产阶级领导权}}{土地问题}
\begin{solution}【答案】C
【解析】无产阶级的领导权是中闻革命的中心问題,也是新民主主义取命理论的核心问题。区别新旧两种不同范畴的民主主义革命。根本的标志是革命的领导权掌握在无产阶级手中还是掌握在资产阶级手中。C
正确分清敌我是中国革命的首要问题。农民问题是中国革命的基本问题。土地问题
(没收封建地主阶级的土地归农民所有)是新民主主义革命的主要内容,A、B、D错误。
\end{solution}
\question 中国革命的中心问题,也是新民主主义革命理论的核心问题是
\par\twoch{\textcolor{red}{无产阶级的领导权}}{分清敌我}{没收封建地主阶级的土地归农民所有}{认清中国的国情}
\begin{solution}【答案】A
【解析】B项,分清敌我,乃是革命的首要问题;C项,没收封建地主阶级的土地归农民所有是新民主主义革命的中心内容;D项,认清中国的国情是革命的基本依据。考生一定注意区分这些
说法。无产阶级的领导权是中国革命的中心问题,也是新民主主义革命理论的核心问题。区别新旧两种不同范畴的民主主义革命,根本的标志是革命的领导权掌握在无产阶级手中还是掌握在资产阶级手中。因此,A项符合题干要求。
\end{solution}
\question 中国民主革命的基本问题是( )
\par\twoch{武装斗争问题}{党的建设问题}{统一战线问题}{\textcolor{red}{农民问题}}
\begin{solution}中国民主革命的实质是农民革命,实际上是无产阶级领导下的农民革命。这是因为:农民是中国民主革命的主力军,是无产阶级最可靠的同盟军,反帝反封建的资产阶级民主革命任务要完成就必须发动和依靠农民,推翻封建制度。正是在此意义上说,中国民主革命的基本问题是农民问题。
\end{solution}
\question 小资产阶级是无产阶级的可靠的同盟军,是中国革命的动力之一。小资产阶级主要包括(
)
\par\twoch{\textcolor{red}{知识分子}}{\textcolor{red}{小商小贩}}{\textcolor{red}{手工业者}}{\textcolor{red}{自由职业者}}
\begin{solution}本题考查小资产阶级的定义和范畴。小资产阶级在马克思学说是指介乎资产阶级/资本家及无产阶级者。主要包括广大知识分子、小手工业者、小商人、自由职业者等。小资产阶级占有一小部分生产资料或少量财产,一般既不受剥削也不剥削别人,主要靠自己的劳动为生。但是,其中有一小部分有轻微的剥削。
\end{solution}
\question 中国无产阶级所具有的优点和特点( )
\par\twoch{\textcolor{red}{坚决的斗争性和彻底的革命性}}{\textcolor{red}{分布集中}}{\textcolor{red}{和农民有着天然的联系,便于和农民结成亲密的联盟}}{数量庞大,是中国革命最基本的动力}
\begin{solution}中国无产阶级是中国革命最基本的动力。但是并不是数量庞大,无产阶级的数量非常少。
\end{solution}
