\question 1914年至1918年的第一次世界大战,是一场空前残酷的大屠杀。它改变了世界政治的格局,也改变了各帝国主义国家在中国的利益格局,对中国产生了巨大的影响。大战使中国的先进分子
\par\twoch{对中国传统文化产生怀疑}{\textcolor{red}{对西方资产阶级民主主义产生怀疑}}{认识到工人阶级的重要作用}{认识到必须优先改造国民性}
\begin{solution}第一次世界大战中,西方资本主义国家为各自利益彼此厮杀,暴露了资产阶级民主的弱点,使中国的先进分子对西方资产阶级民主主义产生怀疑。B选项正确。
\end{solution}
\question 新文化运动从1915年9月陈独秀在上海创办《青年》杂志(后改名《新青年》)开始,一直持续到五四运动之后。新文化运动是中国历史上一次前所未有的启蒙运动,其具有重要意义。以下选项内容不正确的是
\par\fourch{新文化运动提出的基本口号是民主和科学}{新文化运动的主要阵地是《新青年》杂志和北京大学}{新文化运动为外国各种思想流派传入中国敞开了大门}{\textcolor{red}{新文化运动是新民主义的新文化反对封建主义的旧文化}}
\begin{solution}10.新文化运动是从1915年9月陈独秀在上海创办《青年》杂志(后改名《新青年》)开始的。1917年1月,蔡元培出任北京大学校长,聘陈独秀为北大文科学长。《新青年》编辑部也随之迁至北京。1918年1月,《新青年》由陈独秀个人主编的刊物改为同人刊物(口袋题库app题型大家、不是那种同人。由编辑部成员合作经营并共同主持编辑业务的报刊。)。《新青年》杂志和北京大学成了新文化运动的主要阵地。新文化运动提出的基本口号是民主和科学。新文化运动是中国历史上一次前所未有的启蒙运动和空前深刻的思想解放运动。它以勇往直前的大无畏精神和与传统观念彻底决裂的激烈姿态,对封建专制主义、封建伦理道德和封建迷信愚昧进行了无情的批判,在社会上掀起的一股生气勃勃的思想解放潮流,冲决了禁锢人们思想的闸门,从而为外国各种思想流派传入中国敞开了大门,激励着人们去探求救国救民的真理。A、B、C不符合题意。五四以前的新文化运动主要是资产阶级民主主义的新文化反对封建主义的旧文化;五四以后的新文化运动,马克思主义开始逐步地在思想文化领域中发挥指导作用,是新民主主义的文化运动,D符合题意。
\end{solution}
\question 五四以前新文化运动的主要内容有( )
\par\twoch{\textcolor{red}{提倡民主、科学}}{宣传社会主义、马克思主义}{\textcolor{red}{提倡白话文,反对文言文}}{\textcolor{red}{提倡新文学,反对旧文学,主张文学革命}}
\begin{solution}五四运动之前的新文化运动主要是宣传资产阶级思想,提倡民主、科学,提倡白话文,反对文言文,提倡新文学,反对旧文学,主张文学革命,五四运动后主要是宣传马克思主义。
\end{solution}
\question 20世纪早期,一群受过西方教育的先进知识分子认识到,要确实改造中国,必须进行一场思想启蒙运动,在他们的带领下,中国迎来了一场前所未有的启蒙运动和空前深刻的思想解放运动------``新文化运动''。``新文化运动''之所以会产生的原因是(
)
\par\fourch{\textcolor{red}{辛亥革命没有解决中国社会的基本矛盾}}{\textcolor{red}{民族资本主义经济的发展}}{中国出现宣传社会主义思想的知识分子群体}{\textcolor{red}{袁世凯掀起尊孔复古逆流}}
\begin{solution}新文化运动早期是为了宣传资产阶级思想,反对袁世凯尊孔复古,所以答案是ABD。出现宣传社会主义思想是在五四运动后。
\end{solution}
