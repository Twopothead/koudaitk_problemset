\question 在解放战争胜利发展的同时,解放区开展了轰轰烈烈的土地改革运动。土地制度的改革
\par\fourch{\textcolor{red}{是从根本上摧毁中国封建制度根基的社会大变革}}{\textcolor{red}{消灭了封建的生产关系}}{\textcolor{red}{使得农村生产力得到解放,工农联盟进一步巩固和加强}}{标志着社会主义改造在农村已经开始}
\begin{solution}D项错误,因为社会主义改造在农村开始是三大改造。
\end{solution}
\question 在中国共产党七届二中全会上,毛泽东提出了``两个务必''的思想,即务必使同志们继续地保持谦虚、谨慎、不骄、不躁的作风,务必使同志们继续保持艰苦奋斗的作风。其原因主要是
\par\fourch{全国大陆即将解放}{\textcolor{red}{中国共产党即将成为执政党}}{党的工作方式发生了变化}{中国将由新民主义社会转变为社会主义社会}
\begin{solution}更多免费押题卷,请下载口袋题库app
题干强调的``两个务必''是从党的作风的高度提出来的,B是最符合题意的正确选项。因为执政对党尤其是对党的作风是严峻的考验。A、D没有直接指明党即将成为执政党,因而与党的作风联系不直接;C仅仅从党的工作方式发生变化的角度看提出``两个务必''的原因也不准确。
\end{solution}
\question 明确规定``废除封建性及半封建性剥削的土地制度,实现耕者有其田的土地制度'',``乡村中一切地主的土地及公地,由乡村农会接收'',分配给无地或少地的农民的是(
~)
\par\fourch{\textcolor{red}{《中国土地法大纲》}}{《关于清算、减租及土地问题的指示》}{《兴国土地法》}{《井冈山土地法》}
\begin{solution}《中国土地法大纲》
~明确规定``废除封建性及半封建性剥削的土地制度,实现耕者有其田的土地制度'',``乡村中一切地主的土地及公地,由乡村农会接收'',分配给无地或少地的农民。注意与五四指示相区别。
\end{solution}
