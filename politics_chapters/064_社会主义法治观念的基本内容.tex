\question 中国特色社会主义法治道路,是社会主义法治建设成就和经验的集中体现,是建设社会主义法治国家的唯一正确道路。它的核心要义包括
\par\twoch{\textcolor{red}{坚持党的领导}}{\textcolor{red}{坚持中国特色社会主义制度}}{\textcolor{red}{贯彻中国特色社会主义法治理论}}{坚持依法治国和以德治国相结合}
\begin{solution}中国特色社会主义法治道路,是社会主义法治建设成就和经验的集中体现,是建设社会来义法治国家的唯一正确道路。它包括坚持党的领导,坚持中国特色社会主义制度,贯彻中国特色社会主义法治理论三个方面的核心要义。A、B、C正确。坚持依法治国和以德治国相结合属于树立社会主义法制观念的内容,D不符合题意。
\end{solution}
\question 执法为民是社会主义法治理念的五项基本内容之一。执法为民是社会主义法治的本质要求,是人民当家作主的社会主义国家性质在法治上的必然反映。执法为民的基本要求是(
)
\par\twoch{严格执法}{\textcolor{red}{以人为本}}{\textcolor{red}{尊重和保障人权}}{\textcolor{red}{文明执法}}
\begin{solution}执法为民,包括三项基本要求:一是以人为本,二是尊重和保障人权,三是文明执法。属于大纲的实际内容。
\end{solution}
