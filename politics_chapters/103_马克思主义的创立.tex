\question 标志着马克思主义基本形成的论著是
\par\twoch{\textcolor{red}{《关于费尔巴哈的提纲》}}{\textcolor{red}{《德意志意识形态》}}{《哲学的贫困》}{《共产党宣言》}
\begin{solution}本题是马克思主义基本原理概论第一章与科学社会主义理论的新增考点。《关于费尔巴哈的提纲》和《德意志意识形态》标志着马克思主义的基本形成。《哲学的贫困》和《共产党宣言》的发表标志着马克思主义的公开问世。故选AB。
\end{solution}
\question 马克思和恩格斯进一步发展和完善了英国古典经济学理论是( )
\par\twoch{辩证法}{历史观}{\textcolor{red}{劳动价值论}}{剩余价值论}
\begin{solution}劳动价值论是批判地继承和吸收英国古典政治经济学的成果。D是马克思主义的独创。A,B的理论来源是德国古典哲学。
\end{solution}
\question 马克思主义公开问世的标志是( )
\par\twoch{《德意志意识形态》的出版}{《资本论》的出版}{《反杜林论》的出版}{\textcolor{red}{《共产党宣言》的公开发表}}
\begin{solution}1848年2月,《共产党宣言》公开发表,标志着马克思主义的形成。
\end{solution}
\question 马克思恩格斯最重要的理论贡献是( )
\par\twoch{辩证法}{劳动价值论}{\textcolor{red}{唯物史观}}{\textcolor{red}{剩余价值学说}}
\begin{solution}马克思恩格斯最重要的理论贡献即马克思恩格斯所创立的理论是唯物史观和剩余价值学说。
\end{solution}
\question 马克思恩格斯之所以能够实现社会主义思想从空想到科学的飞跃,是因为他们独创了(
)
\par\twoch{劳动价值论}{\textcolor{red}{剩余价值学说}}{\textcolor{red}{历史唯物主义}}{辩证法}
\begin{solution}劳动价值论是古典政治经济学的贡献,马克思是做了继承。辩证法古已有之,黑格尔将其发展到比较成熟的水平。马克思的贡献是在劳动价值论的基础上创立了剩余价值学说,揭示了资本主义剥削的秘密;将辩证唯物主义运用到社会领域,创立了历史唯物主义,揭示了人类社会发展的一般规律。在唯物史观和剩余价值论两大发现的基础上,马克思阐明了由资本主义社会转变为社会主义、共产主义社会的客观规律,阐明了无产阶级获得彻底解放的历史条件和无产阶级的历史使命,从而使社会主义由空想成为科学。
\end{solution}
