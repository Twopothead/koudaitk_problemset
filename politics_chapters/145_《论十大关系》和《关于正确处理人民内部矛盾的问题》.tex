\question 1957年2月,毛泽东在扩大的最高国务会议上发表《关于正确处理人民内部矛盾的问题》的讲话,阐明社会主义社会的重大理论问题,其中为社会主义制度的完善和发展奠定了理论基石的是
\par\fourch{\textcolor{red}{社会主义社会基本矛盾是非对抗性的}}{区分社会主义社会两类不同性质的社会矛盾}{团结全国各族人民进行一场新的战争一向自然界开战}{在社会主义制度下,人民的根本利益是一致的}
\begin{solution}1957年2月,毛泽东在扩大的最高国务会议上发表《关于正确处理人民内部矛盾的问题》的讲话,阐明社会主义社会的重大理论问题。主要内容有:第一,关于社会主义社会两类不同性质的社会矛盾。毛泽东指出:在社会主义制度下,人民的根本利益是一致的,但还存在着敌我矛盾和人民内部矛盾。必须区分社会主义社会两类不同性质的社会矛盾,把正确处理人民内部矛盾作为国家政治生活的主题。毛泽东提出正确处理人民内部矛盾的问题,有一个重要的指导思想,这就是:``团结全国各族人民进行一场新的战争一向自然界开战,发展我们的经济,发展我们的文化,使全体人民比较顺利地走过目前的过渡时期,巩固我们的新制度,建设我们的新国家''。第二,关于社会主义社会的基本矛盾。毛泽东首次提出社会主义社会基本矛盾的概念,并对社会主义社会的基本矛盾作了科学分析。他指出:矛盾是普遍存在的。社会主义社会充满着矛盾,正是这些矛盾推动着社会主义社会不断向前发展。在社会主义社会中,基本的矛盾仍然是生产关系和生产力之间的矛盾、上层建筑和经济基础之间的矛盾。社会主义社会基本矛盾是非对抗性的,可以通过社会主义制度本身的自我调整和自我完善不断地得到解决。这实际上为社会主义制度的完善和发展奠定了理论基石。故本
题选A。
\end{solution}
\question 以毛泽东为主要代表的中国共产党人开始探索中国自己的社会主义建设道路的标志是(
~)
\par\fourch{《关于正确处理人民内部矛盾的问题》}{\textcolor{red}{《论十大关系》}}{中共八大的召开}{“双百”方针的形成}
\begin{solution}《论十大关系》在新的历史条件下从经济方面(这是主要的)和政治方面提出了新的指导方针,为中共八大的召开作了理论准备。它是以毛泽东为主要代表的中国共产党人开始探索中国自己的社会主义建设道路的标志。《关于正确处理人民内部矛盾的问题》是一篇重要的马克思主义文献。它运用马克思主义对立统一规律,创造性地阐述了社会主义社会矛盾学说,是对科学社会主义理论的重要发展,进一步丰富和发展了中共八大路线,对中国社会主义事业具有长远的指导意义;中共八大正确地分析了国内的主要矛盾和主要任务,对于社会主义建设事业和党的事业的发展有着长远的指导意义;``双百''方针是在讨论十大关系时形成的。
\end{solution}
\question 1957年2月,毛泽东在扩大的最高国务会议上发表《关于正确处理人民内部矛盾的问题》的讲话,指出社会主义改造基本完成后,正确处理人民内部矛盾的方针有(
~)
\par\fourch{\textcolor{red}{“统筹兼顾,适当安排”}}{\textcolor{red}{“团结——批评——团结”}}{\textcolor{red}{“百花齐放,百家争鸣”}}{\textcolor{red}{“长期共存、互相监督”}}
\begin{solution}大纲解析原文表述,属于识记型题目。
\end{solution}
\question 毛泽东回顾说:``前八年照抄外国的经验。但从1956年提出十大关系起,开始找到自己的一条适合中国的路线。''毛泽东高度评价《论十大关系》是因为它(
~)
\par\fourch{正确分析了社会主义改造完成后中国社会的主要矛盾和主要任务}{\textcolor{red}{是中国共产党人开始探索中国自己的社会主义建设道路的标志}}{\textcolor{red}{总结经济建设的初步经验,借鉴苏联建设的经验教训,概括提出了十大关系}}{提出了“三个主体,三个补充”的思想}
\begin{solution}正确分析社会主义改造完成后中国社会主要矛盾和主要任务的是八大。``三个主体,三个补充''也是八大的内容。
\end{solution}
