\question 习近平总书记在上海考察调研时表示,``谁牵住了科技创新这个牛鼻子,谁走好了科技创新这步
先手棋,谁就能占领先机、贏得优势''。这里的``牛鼻子''指的是
\par\fourch{事物矛盾的特殊性}{\textcolor{red}{事物发展中的主要矛盾}}{认识指导下的实践}{实践基础上的理论}
\begin{solution}曾有俗语``牵牛要牵牛鼻子''就是强调善于抓住事物的主要矛盾。本题习近平指出``牵住科技创新这个牛鼻子''就是强调把科技创新作为经济发展的主要矛盾。故选B。
\end{solution}
\question 习近平总书记在湖南考察时强调,``我国经济发展要突破瓶颈、解决深层次矛盾和问题,根本出路在于创新,关键是要靠科技力量''。这说明依靠科技力量的创新,是解决我国经济发展瓶颈的
\par\twoch{\textcolor{red}{主要矛盾}}{矛盾的主要方面}{一点论}{主流}
\begin{solution}本题考查对马克思主义辩证法的理解应用。材料中的关键信息在于``关键要靠科技力量'',这里的``关键''就是抓主要矛盾。故选A。
\end{solution}
\question ``先试点后推广''是我国推进改革的一个成功做法。一项改革特别是重大改革,先在局部试点探索,取得经验、达成共识后,再把试点的经验和做法推广开来,这样的改革比较稳当。``先试点后推广''的辩证法依据是
\par\fourch{\textcolor{red}{矛盾的个性与共性在一定条件下能够相互转化,矛盾的共性寓于个性之中}}{必然性通过偶然性表现}{矛盾的个性表现共性并优于共性}{矛盾的个性在事物发展中起决定作用}
\begin{solution}【答案】A
【解析】本题考查矛盾的普遍性和特殊性的关系。C选项本身说法错误,矛盾的普遍性(共性)和特殊性(个性)各有特点,不能说谁优于谁,排除。D选项说法本身也错误,主要矛盾在事物发展中起决定作用而不是矛盾的个性,排除。B选项本身说法没有错,但是题干所给材料与``必然性和偶然性''没有关系,应该排除。题干材料``先试点后推广''体现了由``特殊到普遍再
到特殊''的工作方法,体现了矛盾的普遍性(共性)寓于特殊性(个性)之中,并通过特殊性表现
出来,矛盾的普遍性(共性)和特殊性(个性)在一定条件下能够相互转化,本题答案是A。
\end{solution}
