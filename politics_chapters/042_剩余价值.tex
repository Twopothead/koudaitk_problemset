\question 在瓜分剩余价值上,资本家之间存在竞争和矛盾,但在加强对工人阶级的剥削
以榨取更大量的剩余价值这一点上,资本家之间有着共同的阶级利益。不同部门
的资本家瓜分剩余价值的原则是
\par\twoch{等价交换}{资本周转速度的快慢}{\textcolor{red}{等量资本获得等量利润}}{资本积累规模的大小}
\begin{solution}资本主义生产是为了获得利润,因此,不同部门之间如果利润率不同,资本家之间就会展开激烈的竞争,使资本从利润率低的部门转向利润率高的部门,从而导致利润率趋于平均化,不同部门的资本家按照等量资本获得等量利润的原则来瓜分剩余价值。按照平均利润率来计算和获得的利润,叫做平均利润。随着利润转化为平均利润,商品价值就转化为生产价格,即商品的成本价格加平
均利润。在利润平均化规律作用下,产业资本家获得产业利润,商业资本获得商业利润,银行资本家获得银行利润,土地所有者获得地租。利润平均化规律,反映了在瓜分剩余价值上,资本家之间存在竞争和矛盾,但在加强对工人阶级的剥削以榨取更大量的剩余价值这一点上,资本家之间有着共同的阶级利益。
\end{solution}
\question 无论是绝对剩余价值还是相对剩余价值都是依靠( )
\par\twoch{延长工人工作时间而获得的}{提高劳动生产率而获得的}{\textcolor{red}{增加剩余劳动时间而获得的}}{降低工人的工资而获得的}
\begin{solution}在资本主义条件下,雇佣工人的劳动时间由必要劳动时间和剩余劳动时间构成。剩余劳动时间是生产剩余价值的时间。资本家为了获得更多的剩余价值,必须延长剩余劳动时间。其具体方法就是绝对剩余价值生产和相对剩余价值生产。绝对剩余价值生产是指在必要劳动时间不变的前提下,绝对延长总的工作日长度从而使剩余劳动时间延长来生产剩余价值的方法。用这种方法生产出来的剩余价值就是绝对剩余价值。相对剩余价值生产是在总的工作日长度不变的前提下,缩短必要劳动时间、从而相对延长剩余劳动时间来生产剩余价值的方法。所以,尽管所用的具体手段不同,但二者的共同点都是通过延长剩余劳动时间来获得更多的剩余价值。AB项是绝对剩余价值与相对剩余价值的差异;D项既不是绝对剩余价值也不是相对剩余价值产生的条件。
\end{solution}
\question 绝对剩余价值生产和相对剩余价值生产的共同点是( ~)
\par\fourch{\textcolor{red}{都延长了剩余劳动时间}}{\textcolor{red}{都体现着资本家对工人的剥削关系}}{\textcolor{red}{都增加了剩余价值量}}{\textcolor{red}{都提高了剩余价值率}}
\begin{solution}绝对剩余价值生产和相对剩余价值生产都是剩余价值生产,所以都体现着资本家对工人的剥削关系;二者都需要延长剩余劳动时间,区别是延长的方式不同:是通过加班还是通过提高劳动生产率;他们都导致剩余价值数量的增加,进而提高剩余价值率------因为剩余价值率是剩余价值和可变资本的比率。
\end{solution}
\question 在生产自动化条件下,资本家能够获得更多的剩余价值,其原因在于( )
\par\twoch{\textcolor{red}{生产工具更加先进}}{\textcolor{red}{工人的劳动更加复杂}}{\textcolor{red}{资本家获得了超额剩余价值}}{机器人、自动化生产线能够多创造价值}
\begin{solution}资本主义国家的生产自动化是人类社会科学技术进步的结晶,它的普遍采用会大幅度地提高劳动生产率,使资本家阶级获得比过去更多的剩余价值。资本主义条件下的生产自动化是资本家获取超额剩余价值的手段,而雇佣工人的剩余劳动仍然是这种剩余价值的唯一源泉。
\end{solution}
