\question ``艰苦奋斗始终是激励我们为实现国家富强、民族振兴和人民幸福而共同奋斗的强大精神力
量''。这说明``艰苦奋斗''是
\par\fourch{通往理想彼岸的桥梁}{\textcolor{red}{实现理想的重要条件}}{立党立国的根本指导思想}{大学生成长成才的必由之路}
\begin{solution}本题是思想道德修养与法律基础第一章的调整考点,``艰苦奋斗''是实现理想的重要条件。
\end{solution}
\question 邓小平说:``美好的前景如果没有切实的措施和工作去实现它,就有成为空话的危险。''这说明(
~)
\par\fourch{\textcolor{red}{社会实践是联系理想和现实的桥梁}}{\textcolor{red}{有了理想并不意味着成功,更不意味着已经成功}}{\textcolor{red}{把理想转变为现实需要艰苦奋斗、勇于实践}}{只要付诸行动,人们对于美好未来的向往和追求都能成为现实}
\begin{solution}邓小平的这句话主要在说明理想如何转化为现实的问题。``切实的措施和工作''就已经说明实践是联系理想和现实的桥梁,只有实践才可以把理想转化为现实。因此,ABC正确。
\end{solution}
\question 一个小孩正在作画,先用寥寥几笔起了个草稿,未成形且没颜色,正当小孩对画作继续加工时,围观的人七嘴八舌地嘲笑小孩,画的是鸡?是鸟?是怪物?但小孩并不气馁,最后大功告成,一只美丽的凤凰跃然纸上。小孩最终能够画出令人惊艳的美丽凤凰的原因在于(
~)
\par\twoch{\textcolor{red}{坚定的信念}}{精湛的画艺}{\textcolor{red}{持之以恒的精神}}{\textcolor{red}{不畏挫折的勇气}}
\begin{solution}题干中体现的是不畏挫折,坚持到底的精神。围观的人的嘲笑说明表明画艺并不精湛,所以本题排除B选项。
\end{solution}
