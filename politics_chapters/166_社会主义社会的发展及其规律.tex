\question 2016年是英国人托马斯莫尔发表《乌托邦》一书500周年。《乌托邦》的发表标志着空想社会主义的诞生。空想社会主义虽然不是科学的思想体系,但它对未来新制度的描绘,闪烁着诸多天才的火花。莫尔在《乌托邦》一书中描绘了一个美好的社会,在那个社会里
\par\fourch{\textcolor{red}{没有私有财产和剥削现象}}{\textcolor{red}{人们有计划地从事生产,不需要商品货币和市场}}{\textcolor{red}{城乡之间没有对立}}{实现按劳分配}
\begin{solution}【简析】莫尔1516年发表的《乌托邦》一书描绘了一个美好的社会:在那里,没有私有财产和剥削现象,人们有计划地从事生产,城乡之间没有对立,不需要商品货币和市场,实现按需分配。A、B、C正确,D错误。
\end{solution}
