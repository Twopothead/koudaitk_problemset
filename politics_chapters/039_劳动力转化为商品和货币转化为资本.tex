\question 劳动力是指人的劳动能力,是人的体力和脑力的总和。以下关于劳动力表述正确的是
\par\fourch{\textcolor{red}{劳动力商品的使用价值是价值的源泉}}{劳动力自身的价值在消费过程中能够转移到新产品中去并形成剩余价值}{\textcolor{red}{在资本主义条件下资本家购买的是雇佣工人的劳动力而不是劳动}}{\textcolor{red}{劳动力成为商品是简单商品生产发展到资本主义商品生产新阶段的标志.}}
\begin{solution}【简析】劳动力商品在使用价值上有一个很大的特点。就是它的使用价值是价值的源泉,它在消费过程中能够创造新价值,而且这个新的价值比劳动力本身的价值更大。A正确。在资本主义条件下资本家购买的是雇佣工人的劳动力而不是劳动。C正确。可变资本是用来购买劳动力的那部分资本,在生产过程中不是被转移到新产品中去,而是由工人的劳动再生产出来。B错误。劳动力成为商品,标志着简单商品生产发展到资本主义商品生产的新阶段。在这一阶段,资本家与工人的关系,形式上是``自由''、``平等''的关系,而实质上是资本主义雇佣劳动
关系。D正确。
\end{solution}
\question 在不同的国家或同一国家的不同时期,劳动者所必需的生活资料的数量和构成是有区别的,劳动力价值的最低界限,是由生活上不可缺少的生活资料的价值决定的。一旦劳动力价值降低到这个界限以下,劳动力就只能在萎缩的状态下维持。这表明
\par\fourch{劳动力商品的价值,是维持劳动力所必需的生活必需品的价值决定的}{\textcolor{red}{劳动力价值的构成包含一个历史的和道德的因素}}{劳动力商品的价值是由维持劳动者本人生存所必需的生活资料的价值决定}{劳动力商品的价值是使用价值的源泉}
\begin{solution}【解析】劳动力商品的价值,是由生产、发展、维持和延续劳动力所必需的生活必需品的价值决定的,它包括三个部分:①维持劳动者本人生存所必需的生活资料的价值;②维持劳动者家属的生存所必需的生活资料的价值;③劳动者接受教育和训练所支出的费用。A项与C
项表述不完整。劳动力价值的构成包含一个历史的和道德的因素,在不同的国家或同一国家的不同历史时期,劳动者所必需的生活资料的数量和构成也是有区别的,所以,劳动力价值的最低界限,是由生活上不可缺少的生活资料的价值决定的。一旦劳动力价值降低到这个界限以下,劳动力就只能在萎缩的状态下维持。B项正确。劳动力商品在使用价值上有一个很大的特点,就是它的使用价值是价值的源泉,它在消费过程中能够创造新价值,而且这个新的价值比劳动力本身的价值更大。C项表述错误。
\end{solution}
\question 形成商品价值的劳动是( ~)
\par\twoch{\textcolor{red}{抽象劳动}}{具体劳动}{脑力劳动}{体力劳动}
\begin{solution}本题是考查价值和抽象劳动两个概念之间的关系。价值是凝结在商品里的一般的无差别的人类劳动;这种撇开了劳动的具体形式的无差别的一般人类劳动就叫做抽象劳动。
\end{solution}
