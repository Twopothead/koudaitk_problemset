\question 价值规律作用的实现有赖于( )
\par\twoch{\textcolor{red}{劳动生产率提高}}{价格波动}{\textcolor{red}{资源的有效配置}}{\textcolor{red}{供求关系的变化}}
\begin{solution}此题考查的是价值规律作用的实现。在商品经济条件下,价值规律的表现形式是商品的价格围绕价值自发地波动。这种波动是由于供求关系变动的影响造成的。商品生产者之间展开激烈的市场竞争,在这种竞争中,必然产生价格的波动。正确选项为ACD。
\end{solution}
\question 资本主义意识形态的本质具体表现在( ~)
\par\fourch{\textcolor{red}{为资本主义经济基础服务}}{包括政治、经济、法律、哲学等内容}{\textcolor{red}{是资产阶级的阶级意识的集中体现}}{构成上层建筑的主要内容}
\begin{solution}资本主义国家意识形态具有鲜明的阶级性。这种阶级本质具体体现在:第一,资本主义的意识形态,是在资本主义社会条件下所形成的观念上层建筑,是对资本主义经济基础的反映,并为资本主义经济基础服务。资本主义意识形态的核心,是论证资本主义社会制度的合理性、资本主义民主的普遍性。第二,资本主义意识形态是资产阶级的阶级意识的集中体现。在资本主义条件下,资产阶级在进行阶级统治的实践中逐步形成了自己作为社会统治阶级的阶级意识。BD都是正确的论断,但与题干无关。B是意识形态的内容;D是意识形态的定位。
\end{solution}
