\question ``观察渗透理论''是美国科学哲学家汉森提出的著名命题。这个命题指出,我们的任何观察都不是纯粹客观的,具有不同知识背景的观察者观察同一事物,会得出不同的观察结果。``观察渗透理论''的命题指明的哲学原理是
\par\twoch{\textcolor{red}{感性中渗透着理性的因素}}{理性中渗透着感性的因素}{意识具有控制人的生理活动的作用}{实践是检验真理的唯一标准}
\begin{solution}【简析】在实际的认识过程中,感性认识和理性认识是互相交织、互相渗透的,
一方面,感性中渗透着理性的因素。人们在获得感性认识时,总是以原有的知识为背景,使用己有的概念和逻辑框架,在理性认识参与和指导下进行。同样接触客观事物,由于理论准备不同,感受就可能大不一样。所谓``只有理解了的东西才能更深刻地感觉它'',就是这个道理。现代科学哲学中所谓``观察渗透理论''的命题,也指明人总是以自己的历史文化为背景进行观察的。另一方面,理性中渗透着感性的因素。理性认识不仅以感性认识为基础,而且要通过感性的认识来说明。感性认识丰富的人与经验贫乏的人相比,对事物理解的深度是不一样的。
黑格尔说过,对于同一句格言,出自饱经风霜的老年人之口与出自缺乏阅历的青少年之口,其内涵是不同的。A正确。B错误。C、D本身说法正确但不符合题意,不选。
\end{solution}
\question 研究人员发现,把健康成年小鼠置于黑暗中一周后,它们辨别音高的能力也可显著提高。此前,科学界通常认为这种变化主要发生在未成年阶段,且需要更长的时间。这表明
\par\fourch{科学的价值在于造福社会}{\textcolor{red}{认识与实践的统一是具体的历史的}}{真理具有反复性和相对性}{\textcolor{red}{认识的真理性需要不断经受实践的检验}}
\begin{solution}【答案】BD
【解析】研究人员通过试验证明以前科学界通常认为的观点是错误的,说明认识与实践的统一是具体的历史的,同时说明认识的真理性需要不断经受实践的检验,据此BD选项。A选项观
点本身没有错误,但是与题意不符,题中没有涉及科学的价值在于造福社会,而是强调通过实践获得真理性的认识,故排除。C选项本身观点错误,认识具有反复性,真理不具有反复性,真理具有具体性,没有相对性,故排除。
\end{solution}
\question ``感觉到了的东西,我们不能立刻理解它。''``只有理解了的东西才能更深刻地感觉它。''这两句话表明感性认识和理性认识的关系是(  )
\par\fourch{\textcolor{red}{感性认识和理性认识在内容上有质的区别}}{\textcolor{red}{感性中渗透着理性的因素}}{\textcolor{red}{理性中渗透着感性的认识}}{\textcolor{red}{理性认识不仅以感性认识为基础,而且要用感性的认识来说明}}
\begin{solution}【解析】感性认识和理性认识是统一的认识过程中的两个阶段,它们既有区别,又有联系。感性认识和理性认识在内容和形式上都有质的区别。感性认识和理性认识又是有联系的。①感性认识和理性认识互相依存。理性认识依赖于感性认识,这是认识论的唯物论;感性认识有待于发展到理性认识,这是认识论的辩证法;②在实际的认识过程中,感性认识和理性认识又是互相交织、互相渗透的。感性中渗透着理性的因素;理性中渗透着感性的因素。
\end{solution}
\question 古希腊哲学家说:没有理性,眼睛是最坏的见证人。这一观点( )
\par\fourch{揭示了感性认识是整个认识过程的起点}{揭示了感性认识和理性认识的辩证统一}{认为理性认识可以脱离感性认识而存在,是错误的观点}{\textcolor{red}{强调理性认识的重要作用,完全否认了感性对认识的作用}}
\begin{solution}题干完全否认现象的作用即完全否认感性对认识的作用,只强调理性的重要作用。
\end{solution}
\question 列宁指出:``从生动的直观到抽象的思维,并从抽象的思维到实践,这就是认识真理、认识客观实在的辩证途径。''认识运动的辩证发展过程包括的两次飞跃是(
)
\par\twoch{\textcolor{red}{从感性认识到理性认识的飞跃}}{从理性认识到感性认识的飞跃}{从实践到理性认识的飞跃}{\textcolor{red}{从理性认识到实践的飞跃}}
\begin{solution}认识运动的两次飞跃是从感性认识到理性认识在到实践的飞跃。
\end{solution}
\question 理性认识向实践飞跃的重要意义在于,它使理性认识( )
\par\twoch{\textcolor{red}{变成改造世界的物质力量}}{\textcolor{red}{接受实践的检验并随实践的发展而发展}}{\textcolor{red}{发挥对实践的指导作用}}{起改变事物发展总趋势的作用}
\begin{solution}本题考查认识发展的过程,具体是考查理性认识到实践飞跃的意义。认识由理性认识向实践飞跃,这是认识过程中的第二次飞跃,是比由感性认识向理性认识飞跃更伟大的一次飞跃,而理性认识向实践的飞跃之所以更重要,就在于通过这次飞跃,使理性认识接受实践的检验,发挥其对实践的指导作用,使认识变成改造世界的物质力量,并使认识随实践的发展而不断发展,因此选项ABC均选。至于选项D不是理性认识向实践飞跃所能解决的,事物发展的总趋势是事物主客观综合作用的结果,有着更深刻的原因。
\end{solution}
\question 从感性认识向理性认识过渡,需要的条件有( )
\par\twoch{坚持理论与实践相结合的原则}{形成正确的实践观念}{\textcolor{red}{获取丰富和真实的感性材料}}{\textcolor{red}{运用辩证思维对感性材料进行加工}}
\begin{solution}从感性认识向理性认识的飞跃必须具备两个条件:一是勇于实践,深入调查,获取十分丰富和合乎实际的感性材料。这是正确实现由感性认识上升到理性认识的基础。二是必须经过理性思考的作用,将丰富的感性材料加工制作,去粗取精、去伪存真、由此及彼、由表及里,才能将感性认识上升为理性认识。而AB是实现从理性认识到实践的飞跃的条件。
\end{solution}
