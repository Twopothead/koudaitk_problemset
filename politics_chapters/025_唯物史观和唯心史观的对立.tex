\question 政治上层建筑是在意识形态指导下形成的,这种观点是( )
\par\twoch{历史唯心主义}{非马克思主义的}{\textcolor{red}{历史唯物主义的}}{非决定论的}
\begin{solution}历史唯物主义认为:上层建筑是指建立在一定经济基础上的意识形态以及相应的制度、组织和设施。上层建筑由意识形态和政治法律制度及设施、政治组织等两部分构成。在整个上层建筑中,政治上层建筑居主导地位,国家政权是它的核心。意识形态又称观念上层建筑,包括政治法律思想、道德、艺术、宗教、哲学等思想观点,政治法律制度及设施和政治组织又称政治上层建筑,包括:国家政治制度、立法司法制度和行政制度;国家政权机构、政党、军队、警察、法庭、监狱等政治组织形态和设施。观念上层建筑和政治上层建筑的关系是:政治上层建筑是在一定意识形态指导下建立起来的,是统治阶级意志的体现。C项正确。
A、B、D项错误,马克思创立的历史唯物主义认为政治上层建筑是在一定意识形态指导下建立起来的,是统治阶级意志的体现。由此可知,题干中的观点并非历史唯心主义和决定论的观点。
\end{solution}
\question ``时势造英雄''和``英雄造时势''( )
\par\twoch{\textcolor{red}{是两种根本对立的观点}}{这两种观点是互相补充的}{\textcolor{red}{前者是历史唯物主义,后者是历史唯心主义}}{\textcolor{red}{前者是科学历史观,后者是唯心史观}}
\begin{solution}时势造英雄是历史唯物主义观点,英雄造时势是英雄史观的观点。
\end{solution}
\question 马尔萨斯认为,资本主义的对外侵略、资本主义国家工人的悲惨生活,全然取决于``人口增长总要超过生活资料的增长''这一个``永恒的规律''的作用。马尔萨斯这一观点(
~)
\par\fourch{\textcolor{red}{把人口因素看成决定社会存在和发展的因素}}{\textcolor{red}{曲折地反映了资本主义制度下人口相对过剩的事实,却掩盖和歪曲了事实的本质}}{在重视人口作用的基础上,总结出人口发展的规律}{\textcolor{red}{注意到人口增长与生活资料增长之间的关系,具有积极意义}}
\begin{solution}C与题干无关。
\end{solution}
