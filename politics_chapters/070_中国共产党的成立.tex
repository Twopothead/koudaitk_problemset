\question 中国共产党成立后,``中国革命的面貌焕然一新'',其``新''主要表现在( )
\par\twoch{\textcolor{red}{以马克思主义为指导}}{以武装斗争为主要方法}{提出了彻底的反帝反封建的纲领}{\textcolor{red}{以社会主义、共产主义为远大目标}}
\begin{solution}本题考查的是中国共产党成立的伟大意义。第一,自从有了中国共产党,灾难深重的中国人民有了可以信赖的组织者和领导者,中国革命有了坚强的领导力量。第二,中国共产党的成立,使中国革命有了科学的指导思想。第三,党所提出的纲领和奋斗目标,代表着中国社会发展的正确方向,代表着中国无产阶级和其他广大劳动人民的根本利益。因此,中国共产生从诞生时起,就充满着生机和活力,预示着中国的光明和希望。第四,中国共产党的成立,也使中国革命有了新的革命方法,并沟通了中国革命和世界无产阶级革命之间的联系。中国革命的面目从此焕然一新。所以答案选AD。而C是中共二大提出的,B工农武装割据的思想是土地革命时期的思想。
\end{solution}
