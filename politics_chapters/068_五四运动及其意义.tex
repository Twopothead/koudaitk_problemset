\question 五四运动是中国近代史上一个划时代的事件。五四运动
\par\fourch{\textcolor{red}{形成了爱国、进步、民主、科学的五四精神,拉开了中国新民主主义革命的序幕}}{\textcolor{red}{促进了马克思主义在中国的传播,推进了中国共产党的建立}}{\textcolor{red}{表现了反帝反封建的彻底性,是一次真正的群众解放运动}}{是中国历史上一次前所未有的启蒙运动和空前深刻的思想解放运动}
\begin{solution}D项是新文化运动的意义。
\end{solution}
\question 中国工人阶级开始登上政治舞台、成为中国革命的领导力量是在( )
\par\twoch{\textcolor{red}{五四运动}}{京汉铁路罢工}{五卅运动}{省港大罢工}
\begin{solution}中国工人阶级在五四运动中开始登上政治舞台,在运动后期发挥了主力军作用。五四运动以后,无产阶级逐渐代替资产阶级成为近代中国民族民主革命的领导者。
\end{solution}
