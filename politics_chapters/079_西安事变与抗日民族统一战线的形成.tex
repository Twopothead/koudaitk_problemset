\question 毛泽东对孙中山晚年思想转变予以高度评价:"孙中山先生之所以伟大,不但因为他领导了伟大的辛亥革命(虽然是旧时期的民主革命),而且因为他能够`适乎世界之潮流,合乎人群之需要',提出了联俄、联共、扶助农工三大革命政策,对三民主义作了新的解释,新三民主义相对于旧三民主义的进步性体现在
\par\fourch{\textcolor{red}{突出了反帝的内容,强调对外实行中华民族的独立,同时主张国内各民族一律平等}}{强调政治革命应当与民族革命并行。民族革命是扫除"现在的恶劣政治",而政治革命则是扫除〃恶劣政治的根本〃}{\textcolor{red}{指出民主权利应〃为一般平民所共有〃,不应为〃少数人所得而私有〃}}{\textcolor{red}{把民生主义概括为〃平均地权〃和〃节制资本〃两大原则并提出要改善工农的生活状况}}
\begin{solution}ACD
1924年1月,国民党一大在广州召开,大会通过的宣言对三民主义作出了新的解释:在民族主义中突出了反帝的内容,强调对外实行中华民族的独立,同时主张国内各民族一律平等;在民权主义中强调了民主权利应``为一般平民所共有〃,不应为〃少数入所得而私有〃把民生主义概括为〃平均地权〃和〃节制资本〃两大原则(后来又提出了〃耕者有其田〃的主张),并提出要改善工农的生活状况。这个新三民主义的政纲同中共在民主革命阶段的纲领基本一致,因而成为国共合作的政治基础。旧三民主义已经指出政治革命的目的是建立民国。《军政府宣言》指出:''凡为国民皆平等以有参政权。大总统由国民公举。议会以国民公举之议员构成之。制定中华民国宪法,人人共守。敢有帝制自为者,天下共击之!〃政治革命应当与民族革命并行。民族革命是扫除``现在的恶劣政治'',而政治革命则是扫除``恶劣政治的根本'',从而把斗争矛头直接指向集国内民族压迫与封建专制统治于一身的清政府。故B项是干扰项。
\end{solution}
\question 抗日民族统一战线正式形成的标志是( )
\par\twoch{国民党五届三中全会的召开}{西安事变的和平解决}{\textcolor{red}{《中国共产党为公布国共合作宣言》的发表}}{\textcolor{red}{蒋介石在庐山发表谈话,承认中共的合法地位}}
\begin{solution}本题是一个干扰性极强的知识点题目,大多数同学本题失误都是误选了B选项,诚然,西安事变应该是这几个选项里面最出名的一个事情了,尤其是没怎么复习的同学,可能也就知道西安事变,然后就选了,就错了。西安事变的和平解决成为时局转换的枢纽,十年内战的局面由此结束,国内和平基本实现。而CD选项的表述,才是统一战线形成的标志。
\end{solution}
\question 以国共两党第二次合作为基础的抗日民族统一战线正式建立的标志有( ~)
\par\fourch{\textcolor{red}{国民党中央通讯社发表《中共中央为公布国共合作宣言》}}{\textcolor{red}{蒋介石在庐山发表实际上承认了中国共产党的合法地位的讲话}}{国民党五届三中全会议的召开}{中共洛川会议通过《抗日救国十大纲领》}
\begin{solution}记忆型题目。
\end{solution}
\question 以国共两党第二次合作为基础的抗日民族统一战线正式形成的标志是
\par\fourch{1936 年中共把“反蒋抗日”口号转变为“逼蒋抗日”。}{1936 年西安事变的和平解决}{\textcolor{red}{1937 年国民党中央通讯社发表《中国共产党为公布国共合作宣言》}}{\textcolor{red}{1937 年蒋介石发表谈话,实际承认共产党合法}}
\begin{solution}记忆类题目,B选项标志着国内和平基本实现。
\end{solution}
