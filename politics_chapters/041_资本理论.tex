\question 区分不变资本和可变资本的依据是( )
\par\twoch{资本各部分的流通形式不同}{\textcolor{red}{资本的不同部分在价值增殖过程中起不同的作用}}{资本各部分价值转移的方式不同}{资本各部分有不同的实物形式}
\begin{solution}按照资本在剩余价值生产(价值增殖)中所起的不同作用,生产资本可以划分为不变资本和可变资本。不变资本是以生产资料形式存在、其价值量在剩余价值生产过程中原封不动地转移到新产品中而没有发生变化的那部分资本;它是剩余价值生产的条件,用``C''表示。可变资本是以劳动力形式存在、其价值量在剩余价值生产过程中发生了变化、即发生了增殖的那部分资本;它是剩余价值产生的源泉,用``V''表示。C项是划分固定资本和流动资本的标准;AD是杜撰出来的干扰项目。
\end{solution}
\question 资本区分为不变资本和可变资本的意义在于( )
\par\twoch{\textcolor{red}{揭示了剩余价值的真正来源}}{揭示了货币转化为资本的关键}{\textcolor{red}{揭示了资本主义剥削的秘密}}{\textcolor{red}{为考察资本主义剥削程度提供了科学依据}}
\begin{solution}把资本区分为不变资本、可变资本的意义:①进一步明确剩余价值的来源,揭示资本主义剥削的实质;②为考查资本主义剥削的程度即剩余价值率提供科学依据;③为理解资本有机构成、平均利润理论奠定了基础。B项是劳动力成为商品的意义。
\end{solution}
