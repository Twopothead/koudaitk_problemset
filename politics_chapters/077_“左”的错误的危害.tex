\question 从1927年7月大革命失败到1935年1月遵义会议召开之前,左倾机会主义的错误先后三次在党中央的领导机关取得了统治地位,第一次是以瞿秋白为代表的``左,,倾盲动主义,第二次
是以李立三为代表的``左''倾冒险主义,第三次是以陈绍禹(王明)为代表的左倾机会主义。
其中以第三次最为严重。王明``左''倾机会主义的主要错误有
\par\fourch{在革命性质和统一战线问题上,混淆民主革命与社会主义革命的界限,只看到了两个革命
阶段的区别,没有看到两个革命阶段的联系}{放弃了对军队的领导权,坚持以城市为中心}{\textcolor{red}{在军事斗争问题上,实行进攻中的冒险主义、防御中的保守主义、退却中的逃跑主义}}{\textcolor{red}{在党内斗争和组织问题上,推行宗派主义和“残酷斗争,无情打击的方针}}
\begin{solution}因为王明的``左倾''只看到了两个革命阶段的联系,而没有看到两个革命阶段的区别。B项错误,因为放弃对军队领导权是陈独秀的右倾投降主义错误的表现。
\end{solution}
\question ``因为中国民族资产阶级根本上与剥削农民的豪绅地主相联结相吻合,中国革命要推翻豪绅地主阶级,便不能不同时推翻资产阶级'',这一观点的错误之处在于(
)
\par\twoch{忽视了反对帝国主义的必要性}{未能区分中国资产阶级的两部分}{\textcolor{red}{混淆了民主革命和社会主义革命的任务}}{不承认中国资产阶级与地主阶级的区别}
\begin{solution}民主革命的对象是封建主义;社会主义革命的对象是资产阶级,所以这个观点的错误在于混淆了两种革命之间区别。
\end{solution}
