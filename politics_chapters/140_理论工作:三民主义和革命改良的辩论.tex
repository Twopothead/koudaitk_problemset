\question 1905年11月,在同盟会机关报《民报》发刊词中,孙中山先生将同盟会的纲领概括为三大主义。后被称为三民主义。三民主义的核心是(
~)
\par\twoch{民族主义}{\textcolor{red}{民权主义}}{民生主义}{暴力革命}
\begin{solution}本题考查三民主义知识点。1905年,孙中山同黄兴、宋教仁等在日本东京组
建了中国同盟会。提出了纲领``驱除鞑虏,恢复中华,创立民国,平均地权''。后将这个纲领阐发为三民主义。民族主义是三民主义的前提,基本内容就是``驱除鞑虏,
恢复中华''。以``平均地权''为核心内容的民生主义,是孙中山三民主义中最具特色的部分。以``创立民国''为内涵的民权主义是三民主义的核心思想。
\end{solution}
\question 孙中山先生是伟大的民族英雄,伟大的爱国主义者,中国民主革命的伟大先躯,一生以革命为已任,立场救国救民,为中华民族作出了彪炳史册的贡献。孙中山先生的伟大表现在(
)。
\par\fourch{\textcolor{red}{坚定维护民主共和国制度和国家完整统一}}{发动了推翻北洋军阀统治为目标的北伐战争}{\textcolor{red}{重新解释三民主义并提出了联俄、联共、扶助农工三大政策}}{\textcolor{red}{领导了辛亥革命}}
\begin{solution}本题考查的是纪念孙中山诞辰150周年。习近平主席提到孙中山是伟大的爱国主义者,创立兴中会、同盟会,提出民族、民权、民生的三民主义,积极传播革命思想,广泛联合革命力量,连续发动武装起义,为推进民主革命四处奔走、大声疾呼。中国共产党成立后,孙中山先生同中国共产党人真诚合作,在中国共产党帮助下,把旧三民主义发展为新三民主义,实行联俄、联共、扶助农工三大政策。北伐战争是1926年-1927年,而孙中山先生1925年3月12日就逝世了,所以不能选B。
\end{solution}
\question 同盟会的政治纲领是``驱除鞑虏,恢复中华,创立民国,平均地权'',孙中山后来将其概括为三大主义,即民族主义、民权主义、民生主义,其中民族主义的含义是(
~)
\par\fourch{\textcolor{red}{反满,推翻清政府}}{排斥不同种族的人,建立汉族政权}{\textcolor{red}{建立民族独立的国家}}{\textcolor{red}{民族平等}}
\begin{solution}民族主义最大的缺陷就是没有反帝,并且反封主要指的是推翻满清贵族的统治。建立民族平等和独立的国家。
\end{solution}
\question 在孙中山的三民主义中,民生主义指的就是``平均地权'',也就是孙中山所说的社会革命,孙中山的民生主义(
~)
\par\fourch{\textcolor{red}{“平均地权”主张并非将土地所有权分给农民}}{\textcolor{red}{没有正面触及封建土地所有制}}{没有从正面鲜明的提出反对帝国主义的纲领}{\textcolor{red}{在革命中难以成为发动广大工农群众的理论武器}}
\begin{solution}民生即平均地权主要是指城市用地和城郊用地而不是耕地,所以没有主张讲土地所有权分给农民,没有触及封建土地所有制,也难以成为发动广大工农群众的理论武器。
\end{solution}
\question 1905年至1907年间,围绕中国究竟是采用革命手段还是改良方式这个问题,革命派与改良派展开了一场大论战。这场论战的主要内容包括(
~)
\par\fourch{\textcolor{red}{要不要以革命手段推翻清王朝}}{要不要实行政治革命}{\textcolor{red}{要不要推翻帝制,实行共和}}{\textcolor{red}{要不要社会革命}}
\begin{solution}考查的是辛亥革命中革命派与改良派的论战。这次论战的主要内容是:要不要以革命手段推翻清王朝,要不要推翻帝制,实行共和,要不要社会革命。
\end{solution}
