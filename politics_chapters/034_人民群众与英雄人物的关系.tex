\question 恩格斯指出:``历史是这样创造的:最终的结果总是从许多单个的意志的相互冲突中产生出来的,而其中每一个意志,又是由于许多特殊的生活条件,才成为它所成为的那样。这样就有无数互相交错的力量,有无数个力的平行四边形,因此就产生出一个合力,即历史结果\ldots{}\ldots{}''这句话揭示了历史(
)
\par\twoch{\textcolor{red}{是由千百万人共同创造的}}{\textcolor{red}{历史发展的趋势不依单个人的意志为转移}}{历史人物起推动社会前进的积极作用}{\textcolor{red}{离开了每个人的作用就不可能有群众的作用}}
\begin{solution}题干强调历史是人们合力作用的结果。C与题干无关。
\end{solution}
\question 党的群众观点和群众路线的理论基础有( )
\par\twoch{\textcolor{red}{人民群众是历史创造者的原理}}{\textcolor{red}{人民群众在历史发展中起决定作用的原理}}{社会主义必然战胜资本主义的原理}{上层建筑一定要适应经济基础状况的规律}
\begin{solution}本题考查党的群众观点和群众路线的理论基础。本题知识点非常清晰。唯物史观关于人民群众是历史创造者的原理,是无产阶级政党的群众观点和群众路线的理论基础。所以,答案是A、B。而选项C、D是历史唯物主义的基本原理,但是与试题所问不符。
\end{solution}
\question 人在社会发展中的作用表现在( )
\par\twoch{\textcolor{red}{社会历史是人们自己创造的}}{人们创造历史的活动可以摆脱既定历史条件的制约}{\textcolor{red}{人们可以认识、利用和驾驭社会规律,因此对社会发展的具体途径进行历史选择,通过不同的具体道路实现社会发展的客观规律}}{\textcolor{red}{社会规律存在和实现于人们的自觉活动之中,人们的自觉活动只有在认识和遵循社会规律的基础上才能得到顺利、有效的发挥}}
\begin{solution}本题考查人在历史中的作用。具有一定的综合性,要求加强理解。选项A、C、D都是正确的表达。选项B认为创造历史的活动可以摆脱历史条件的制约,这是错误的。人类的历史是人们自己创造的,但这些活动都是在一定的历史条件下并受历史条件制约而进行的,人们的活动不可能没有条件,所以选项B不选。
\end{solution}
\question 人民群众是历史的创造者,这表现在他们是( )
\par\twoch{\textcolor{red}{社会物质财富的创造者}}{\textcolor{red}{社会精神财富的创造者}}{\textcolor{red}{实现自身利益的根本力量}}{\textcolor{red}{社会历史变革的决定力量}}
\begin{solution}本题考查人民群众创造历史的观点。这属于历史唯物主义部分的基本知识点。人民群众是历史的创造者,表现在他们是物质财富的创造者、是社会精神财富的创造者、是社会变革的决定性力量;既是先进生产力和文化的创造主体,也是实现自身利益的根本力量。
\end{solution}
