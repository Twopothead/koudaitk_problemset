\question 人类进入了21世纪,与马克思所处的时代相比,社会经济条件发生了很大的变化,因此,必须深化对马克思劳动价值论的认识。以下选项中关于认识的深化描述正确的是
\par\fourch{\textcolor{red}{生产性劳动包括大部分非物质生产领域的服务性劳动}}{\textcolor{red}{科技劳动和管理劳动等脑力劳动作为更高层次的复杂劳动创造的价值要大大高于简单劳动}}{科学技术本身也能创造价值}{\textcolor{red}{在实际经济生活中,价值分配首先是由生产资料所有制关系决定的}}
\begin{solution}人类进人了21世纪,与马克思所处的时代相比,社会经济条件发生了很大的变化,因此,必须深化对马克思劳动价值论的认识。第一,深化对创造价值的劳动的认识,对生产性劳动做出新的界定。马克思在《资本论》中重点考察的是物质生产部门,认为物质生产领域的劳动才是生产性劳动并创造价值。在当今时代,随着第三产业的发展,服务性劳动的地位和作用越来越重要,生产性劳动应当包括大部分非物质生产领域的服务性劳动。第二,深化对科技人员、经营管理人员在社会生产和价值创造中所起的作用的认识。马克思在《资本论》中关于``总体工人''的论述中,对脑力劳动(包括科技和管理劳动)给予了肯定,认为这些劳动也是创造价值的劳动,但他重点研究的是物质生产领域的体力劳动。在当今社会,科技劳动和管理劳动等脑力劳动,不仅作为一般劳动在价值创造中起着重要作用,而且作为更高层次的复杂劳动创造的价值要大大高于简单劳动。科学技术本身并不能创造价值。但科学技术在生产中的应用有利于劳动生产率的提高;科学技术为人所掌捤,从而提高劳动效率,创造出更多的使用价值和价值。第三,深化对价值创造与价值分配关系的认识。价值创造与价值分配既有联系又有区别,价值创造属于生产领域的问题,而价值分配是属于分配领域的问题。价值创造是价值分配的前提和基础,没有价值创造也就没有价值分配;但价值分配又不仅仅取决于价值创造,在实际经济生活中,价值分配首先是由生产资料所有制关系决定的,有什么样的生产资料所有制关系.就有什么样的分配关系。A、B、D是正确答案。
\end{solution}
\question 人们往往将汉语中的``价''、``值''二字与金银财宝等联系起来,而这两字的偏旁却都是``人'',示意价值在``人''。马克思劳动价值论透过商品交换的物与物的关系,揭示了商品价值的科学内涵,其主要观点有(
)
\par\twoch{劳动是社会财富的唯一源泉}{具体劳动是商品价值的实体}{\textcolor{red}{价值是凝结在商品中的一般人类劳动}}{\textcolor{red}{价值在本质上体现了生产者之间的社会关系}}
\begin{solution}社会财富不等同于商品价值,文化科学也是社会财富,可是不是生产劳动而来;资源矿藏也是社会财富,但是也不是劳动产生,所以A选项错误。价值对应的是抽象劳动,所以B选项明显错误。CD选项都比较直接,不解释。
\end{solution}
\question 使用价值是指商品能够满足人们某种需要的属性,即商品的有用性。马克思主义政治经济学在研究商品时,之所以考察商品的使用价值,因为使用价值是(
)
\par\twoch{构成财富的物质内容}{人类生存、发展的物质条件}{满足人们需要的物质实体}{\textcolor{red}{商品交换价值和价值的物质承担者}}
\begin{solution}政治经济学中考研商品的使用价值是要体现是商品交换价值和价值的物质承担者。
\end{solution}
\question 商品具有使用价值和价值两个因素,是使用价值和价值的矛盾统一体。解决商品内在的使用价值和价值矛盾的关键是(
)
\par\twoch{生产商品的劳动生产率的提高}{\textcolor{red}{商品交换的实现}}{能充当交换媒介的货币的出现}{价值规律发挥作用}
\begin{solution}使用价值和价值的实现是通过交换,商品生产者占有其价值,消费者占有其使用价值,二者矛盾得到解决。
\end{solution}
\question 马克思指出,``处于流动状态的人类劳动力或人类劳动形成价值,但本身并不是价值。它在凝固的状态中,在物化的形式上才形成价值。这就是说,要把人类抽象劳动,凝结在一定的物体里面,即一定的对象里,它才形成价值。''商品的价值是指凝结在商品中的一般人类劳动,说明它是(
)
\par\twoch{\textcolor{red}{商品的本质属性}}{\textcolor{red}{由抽象劳动形成的}}{\textcolor{red}{体现商品生产者相互交换劳动的关系}}{\textcolor{red}{交换价值的内容和基础}}
\begin{solution}商品的价值是有抽象劳动形成的,是商品的本质属性,体现商品生产者相互交换劳动的关系,是交换价值的内容和基础。
\end{solution}
\question 商品的二因素是对立统一的,这对矛盾的解决有赖于( )
\par\twoch{劳动生产率的不断提高}{商品物质实体的消亡}{\textcolor{red}{商品交换的实现}}{货币的出现并充当交换媒介}
\begin{solution}价值和使用价值是辩证统一的关系,这一矛盾只有通过商品的买卖才能得到解决。通过商品交换,商品生产者达到了它的目的------得到价值;而消费者也实现了他自己的目的------获得了使用价值。
\end{solution}
\question 商品内在的使用价值与价值的矛盾,其完备的外在表现是( )
\par\twoch{商品与商品之间的对立}{\textcolor{red}{商品与货币之间的对立}}{私人劳动与社会劳动之间的对立}{资本与雇佣劳动之间的对立}
\begin{solution}货币的产生,使一切商品的价值有了一个固定的、相对同一的表现形式,也使商品内在的价值和使用价值之间的矛盾,外在地表现为货币(代表价值)和商品(代表使用价值)的矛盾。在货币形式下,整个商品世界分为两极:一极是各式各样的商品,它们以使用价值的形式存在,在交换中,它们要求转化为价值;而另一极则是货币。
\end{solution}
\question 以下关于价值、交换价值、价格相互关系的论述,正确的有( )
\par\twoch{\textcolor{red}{价值是交换价值的基础}}{\textcolor{red}{价值是价格的基础}}{\textcolor{red}{价格是交换价值的一种形式}}{\textcolor{red}{价值要借助交换价值和价格表现出来}}
\begin{solution}价值凝结在商品里面,它要表现出来就必须借助交换价值这一形式。交换价值是一种使用价值与另一种使用价值相交换的量的关系和比例。价值和交换价值是内容与形式的关系。
\end{solution}
\question 人们往往将汉语中的``价''、``值''二字与金银财宝等联系起来,而这两字的偏旁却都是``人'',示意价值在``人''。马克思劳动价值论透过商品交换的物与物的关系,揭示了商品价值的科学内涵,其主要观点有
\par\twoch{劳动是社会财富的唯一源泉}{具体劳动是商品价值的实体}{\textcolor{red}{价值是凝结在商品中的一般人类劳动}}{\textcolor{red}{价值在本质上体现了生产者之间的社会关系}}
\begin{solution}本题考查的是商品价值的内涵。答案A是错误的,劳动是价值的唯一源泉,社会财富是属于使用价值的范畴,使用价值的源泉是原材料和人类劳动。答案B也是错误的,具体劳动是商品使用价值的实体,抽象劳动是商品价值的实体。答案C和D是正确的。价值是凝结在商品中无差别的人类劳动,是商品交换的基础,本质上体现了生产者之间的社会关系。
\end{solution}
