\question 红色政权存在发展的根本原因是( )
\par\fourch{\textcolor{red}{中国是一个几个帝国主义国家间接统治的政治经济发展极端不平衡的半殖民地半封建的大国}}{有相当力量的正式红军的存在和共产党组织的坚强有力和各项政策的正确贯彻执行}{第一次国内革命战争的影响}{全国革命形势的向前发展}
\begin{solution}毛泽东在《中国的红色政权为什么能够存在?》、《井冈山的斗争》中,从理论上论述了中国红色政权发生、发展的原因和条件:中国是一个政治经济发展极不平衡的半殖民地半封建大国,这是红色政权存在发展的根本原因;国民革命政治影响的存在,是红色政权得以生存和发展的客观条件;全国革命形势继续向前发展,是红色政权存在和发展的又一客观条件;有相当力量的正式红军的存在,是红色政权能够存在和发展的必要的主观条件;共产党组织的坚强有力和各项政策的正确执行,是中国红色政权能够存在发展的前提和根本保证。
\end{solution}
\question ``农村包围城市、武装夺取政权''是马克思主义中国化的成功范例,集中反映这一理论的毛泽东著作有(
)
\par\twoch{\textcolor{red}{《井冈山的斗争》}}{\textcolor{red}{《中国的红色政权为什么能够存在?》}}{《中国革命战争的战略问题》}{\textcolor{red}{《星星之火,可以燎原》}}
\begin{solution}工农武装割据思想集中在《井冈山的斗争》《中国的红色政权为什么能够存在》和《星星之火,可以燎原》里面。
\end{solution}
\question 毛泽东曾说:``一国之内,在四围白色政权的包围中,有一小块或若干小块红色政权的区域长期地存在,这是世界各国从来没有的事。这种奇事的发生,有其独特的原因。而其存在和发展,亦必有相当的条件。''中国的红色政权能够存在和发展的条件有(
)
\par\twoch{\textcolor{red}{白区政权长期的分裂和战争}}{\textcolor{red}{工农兵群众的发展}}{\textcolor{red}{相当力量的正式红军的存在}}{\textcolor{red}{共产党组织的有力量和政策的不错误}}
\begin{solution}四个选项均是正确选项,大纲原话。
\end{solution}
