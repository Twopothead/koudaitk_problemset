\question 社会生产是连续不断进行的,这种连续不断重复的生产就是再生产。每次经济危机发生期间,总有许多企业或因为产品积压、或因订单缺乏等致使无法继续进行在生产而被迫倒闭。那些因产品积压而倒闭的企业主要是由于无法实现其生产过程中的
\par\twoch{劳动补偿}{\textcolor{red}{价值补偿}}{实物补偿}{增值补偿}
\begin{solution}此题考查的是社会再生产的核心问题及实现条件。社会再生产顺利进行,要求生产中所耗费的资本在价值上得到补偿。材料中所说产品积压,其实质就是产品无法顺利卖出,之前所付出的资本无法顺利得到价值补偿。因此,本题的正确答案是B。
\end{solution}
\question 任何社会再生产的内容都是( )
\par\fourch{劳动过程和价值增殖过程的统一}{劳动过程和价值形成过程的统一}{简单再生产和扩大再生产的统一}{\textcolor{red}{物质资料再生产和生产关系再生产的统一}}
\begin{solution}再生产是指不断重复的生产过程。任何社会的再生产就其内容而言,都是产品(物质资料)的再生产和生产关系再生产的统一。资本主义再生产是物质资料再生产和资本主义生产关系再生产的统一。A项是资本主义劳动过程的两重性;B项是商品生产的两重性;C项是再生产的类型。
\end{solution}
