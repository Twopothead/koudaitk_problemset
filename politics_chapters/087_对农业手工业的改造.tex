\question 中国共产党在推进手工业合作化的过程中,采取的方针是( ~)
\par\twoch{\textcolor{red}{积极引导}}{典型示范}{逐步推广}{\textcolor{red}{稳步前进}}
\begin{solution}在推进手工业合作化的过程中,中国共产党采取的是积极引导、稳步前进的方针。
\end{solution}
\question 所谓农业合作化,就是在中国共产党领导下,通过各种互助合作的形式,把以生产资料私有制为基础的个体农业经济,改造为以生产资料公有制为基础的农业合作经济的过程。1953年到1956年我国实行农业合作化的主要原因是(
)
\par\fourch{封建土地制度严重阻碍生产力的发展}{\textcolor{red}{小农经济难以满足国民经济发展的需要}}{按苏联模式建设社会主义}{一些领导人片面强调公有化的作用}
\begin{solution}社会主义改造属于变革生产方式,主要原因肯定是因为生产力的发展,小农经济难以满足国民经济发展的需要。
\end{solution}
\question 在推进手工业合作化的过程中,中国共产党采取的方针是( ~)
\par\twoch{自愿互利}{\textcolor{red}{积极引导}}{\textcolor{red}{稳步前进}}{国家帮助}
\begin{solution}自愿互利、国家帮助是农业合作化运动的基本原则。积极领导和稳步前进是手工业合作化过程中,中国共产党采取的方针。
\end{solution}
\question 中共中央在1953年12月通过的《关于发展农业生产合作社的决议》总结互助合作运动的经验,概括提出的过渡性经济组织形式主要是(
)
\par\twoch{\textcolor{red}{互助组}}{\textcolor{red}{初级农业生产合作社}}{\textcolor{red}{高级农业生产合作社}}{生产合作小组}
\begin{solution}农业生产合作社的三种过渡性经济组织形式是互助组、初级农业生产合作社、高级农业生产合作社。
\end{solution}
