\question 在我国,很多农民为了高产。采用的普遍做法是加大种植密度。这就造成通风透光差,田间小气候不好,作物很容易感染病虫害。同样的作物品种.国外的种植密度较国内要低很多。以澳大利亚为例,一亩地通常定植番茄600\textasciitilde{}800棵(我国则达到3300棵),植株在良好的环境下生长,不易生病.单株产量并不比国内低,同时质量高。这说明
\par\fourch{自然规律是主观与客观的具体的历史的统一}{\textcolor{red}{实践活动中要注意适度原则}}{\textcolor{red}{要尊重自然规律的前提下发挥主观能动性}}{回复原始生态市解决人与自然矛盾的最终途径}
\begin{solution}【答案】BC
【简析】规律是客观的。客现性是规律的根本特点,它的存在不依赖于人的意识。相反,人的意识及其指导下的实践却要受到规律的支配。不管人们是否认识到,承认不承认.规律都客现存在着,并以一定的方式起作用。A错误。实现人与自然的和谐处并不要求恢复原始生态错误B、C正确
\end{solution}
\question 将客观规律与主观能动性统一起来的基础是( )
\par\twoch{矛盾}{\textcolor{red}{实践}}{物质}{客观}
\begin{solution}实践具有直接现实性,即能够将主观(主观能动性)与客观(客观规律)联系起来,实现二者的统一。
\end{solution}
