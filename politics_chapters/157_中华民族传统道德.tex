\question 中华传统美德和中国革命道德是一脉相承的。其一脉相承性主要体现在
\par\fourch{\textcolor{red}{中华传统美德是中国革命道德的渊源之一}}{\textcolor{red}{它们形成的基础都是中华传统道德的精华}}{\textcolor{red}{它们都是对中国优良道德传统的延续和发展}}{中国革命道德超越了中华传统美德的时代局限}
\begin{solution}此题考查的知识点是对中国革命道德的理解,是一道理解性试题,难度较大。中国革命道德是指中国共产党人、人民军队、一切先进分子和人民群众在中国新民主主义革命和社会主义革命、建设与改革
中所形成的优良道德。它是马克思主义与中国革命、建设与改革的伟大实践相结合的产物,是对中国优良
道德传统的继承和发展,是中华传统美德新的升华和质的飞跃。中华传统美德是中国革命道德的渊源之
一,从一定意义上来说,没有中华传统美德的长期发展和丰厚积淀,就不可能有中国革命道德的形成和发
展。中国革命道德继承了中国传统道德的精华,摒弃了传统道德的糟粕,是中国优良传统道德的延续和发展,是超越了中华传统美德的时代局限而形成的一种崭新的道德。ABC选项正确。D选项不选的原因在于:它是讲中国革命道德与中华传统美德的不同。
\end{solution}
\question 下列名言与``诚者天之道也,思诚者人之道也。至诚而不动者,未之有也;不诚,未有能动者也''
意思表述一致的是
\par\fourch{守法和有良心的人,即使有迫切的需要也不会偷窃,可是,即使把百万金元给了盗贼,也没
法儿指望他从此不偷不盗}{没有伟大的品格,就没有伟大的人,甚至也没有伟大的艺术家,伟大的行动者}{不自见,故明;不自是,故彰;不自伐,故有功;不自矜,故长}{\textcolor{red}{小信成则大信立,故明主积于信。赏罚不信,则禁令不行}}
\begin{solution}本题考查思想道德对古文诗词的理解应用。抓住古诗文当中``诚''这一关
键字,对应D项中的``诚信''。
\end{solution}
\question 关于对待传统道德态度的表述,错误的有( ~)
\par\fourch{\textcolor{red}{坚持文化复古主义,中国的落后就是因为儒家文化的失落}}{古为今用,洋为中用}{\textcolor{red}{实行历史虚无主义,即中国要全盘西化}}{坚持以我为主,为我所用的基本原则}
\begin{solution}在对待传统道德的问题上,文化复古主义和历史虚无主义都是应该予以否定的错误思潮,我们应该坚持的是用马克思主义的立场、观点、看法,坚持以我为主、为我所用的原则,既反对全盘西化、机械照搬,又反对全盘否定,盲目排外,在批判的基础上加以借鉴、吸收,剔除其带有阶级和时代局限性的糟粕,吸收其带有普遍性和一般性,对今天有积极意义的精华。因此,AC正确。
\end{solution}
\question 中国古代思想家强调在``义''和``利''发生矛盾时,应当``义以为上''、``先义后利''、``见利思义'',主张``义然后取'',反对``重利轻义''和``见利忘义''。这种思想反映了中华民族优良道德传统的(
~)
\par\fourch{\textcolor{red}{注重整体利益、国家利益和民族利益}}{讲求谦敬礼让,强调克骄防矜}{倡导言行一致,强调恪守诚信}{\textcolor{red}{强调对社会、民族、国家的责任意识和奉献精神}}
\begin{solution}``义以为上,先义后利'',``重利轻义,见利忘义''强调的是注重整体利益,国家民族利益,强调对社会、国家民族的奉献精神。
\end{solution}
