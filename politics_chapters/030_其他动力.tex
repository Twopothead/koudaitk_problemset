\question ``随着新生产力的获得\ldots{}\ldots{}人们也就会改变自己的一切社会关系,手推磨产生的是封建主的社会,蒸汽磨产生的是工业资本家的社会。''这段话表明科学技术是(
)
\par\twoch{\textcolor{red}{历史上起推动作用的革命力量}}{历史变革中的唯一决定性力量}{\textcolor{red}{推动生产方式变革的重要力量}}{一切社会变革中的自主性力量}
\begin{solution}科学技术是对社会发展起到重要作用的,但是B选项说法错误,历史变革的决定力量是生产力的发展,D选项错误,如果科学技术是自主性力量,那么根本的推动力就不是生产力了。B、D选项都是夸大了科学技术的作用。
\end{solution}
\question ``随着新生产力的获得\ldots{}\ldots{}人们也就会改变自己的一切社会关系,手推磨产生的是封建主义的社会,蒸汽磨产生的是工业资本家的社会。''这段话表明科学技术是
\par\twoch{\textcolor{red}{历史上起推动作用的革命力量}}{历史变革中的唯一决定性力量}{\textcolor{red}{推动生产方式变革的重要力量}}{一切社会变革中的自主性力量}
\begin{solution}本题考查科学技术的作用,先进的生产工具是人们先进的科学技术的物化,它运用于生产过程就会加快生产发展的速度,提高生产的效率。科技革命推动生产方式的变革,进而推动社会制度的变革。在本题中,手推磨、蒸汽磨是先进的生产工具,是科学技术的物化,由于它们的广泛应用,推动了生产方式的变革,分别产生了封建主义的社会和工业资本家的社会,所以,AC正确,B选项明显错误,科学技术不是历史变革中的唯一决定性力量。科学技术是间接的生产力,不是自主性力量,自主性力量只能是运用科技的劳动者,D选项错误。因此,本题正确答案为AC选项。
\end{solution}
