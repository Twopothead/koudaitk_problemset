\question 根据马克思、恩格斯的论述及社会主义现实的启示所概括的科学社会主义的基本原则中,社会主义生产的根本目的是
\par\fourch{发展生产力,不断推动社会发展进步}{\textcolor{red}{满足全体社会成员的需要}}{全面建设社会主义,逐步进入共产主义}{实现现代化,建设社会主义强国}
\begin{solution}【简析】根据马克思、恩格斯的论述及社会主义现实的启示所概括的科学社会主义的基本原则中,社会主义生产的根本目的是:在生产资料公有制基础上组织生产,满足全体社会成员的需要。B正确,A、C、D不符合题意。
\end{solution}
\question 社会主义发展的历史进程不是一帆风顺的,高潮和低潮的相互交替构成波澜壮阔的社会主义发展史。社会主义发展史上的两次飞跃是(
)
\par\fourch{\textcolor{red}{19世纪中叶,社会主义从空想发展到科学}}{19世纪70年代,社会主义从理论发展到建立社会主义制度的实践}{\textcolor{red}{20世纪初,社会主义从理论发展到建立社会主义制度的实践}}{20世纪中叶,社会主义由一国实践到多国实践}
\begin{solution}社会主义发展史上的两次飞跃是19世纪中叶,社会主义从空想发展到科学,20世纪初,社会主义从理论发展到建立社会主义制度的实践。
\end{solution}
