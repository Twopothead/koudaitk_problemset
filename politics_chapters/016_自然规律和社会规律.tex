\question 社会规律是人们自己的``社会行动的规律'',这是因为( )
\par\twoch{\textcolor{red}{人是社会历史的主体}}{\textcolor{red}{人们自己创造自己的历史}}{历史发展方向是由人的思想和行动决定的}{\textcolor{red}{社会规律存在和实现于实践活动之中}}
\begin{solution}C观点错误。
\end{solution}
\question 自然规律和社会规律的不同点有( )
\par\fourch{自然规律是由客观物质力量决定的,社会规律是由人们的思想动机决定的}{自然规律没有阶级性,社会规律在阶级社会具有阶级性}{\textcolor{red}{自然规律可以重复出现,社会规律则是历史的,有不同的表现形式}}{\textcolor{red}{自然规律是作为盲目的无意识的力量起作用,社会规律要通过人的有意识的活动才能实现}}
\begin{solution}本题考查对社会历史规律的正确理解,着重是考查社会规律与自然规律的异同。A选项后半句是历史唯心主义命题,至于选项B,其实无论是自然规律还是社会规律,都是客观的,所谓客观的就是不依赖于人的意志而存在着,既然不依赖于人的意志,当然也就不依赖于阶级的意志了,所以自然规律和社会规律都没有阶级性,只不过在阶级社会先进的阶级容易认识并符合社会规律罢了,而且社会规律的实现有利于先进阶级的阶级利益,但不能由此得出社会规律在阶级社会具有阶级性的结论,这是两个概念。选项CD是符合历史唯物主义原理的关于社会规律的表述。
\end{solution}
\question 1989年,时任美国国务院顾问的弗朗西斯●福山抛出了所谓的``历史终结论'',认为西方实行的自由民主制度是``人类社会形态进步的终点''和
``人类最后一种的统治形式''。然而,20年来的历史告诉我们,终结的不是历史,而是西方的优越感。就在柏林墙倒塌20年后的2009年11月9日,BBC
公布了一份对27国民众的调查。结果半数以上的受访者不满资本主义制度,此次调查的主办方之一的``全球扫描''公司主席米勒对媒体表示,这说明随着1989年柏林墙的倒塌资本主义并没有取得看上去的压倒性胜利,这一点在这次金融危机中表现的尤其明显,``历史终结论''的破产说明
\par\fourch{社会规律和自然规律一样都是作为一种盲目的无意识力量起作用}{\textcolor{red}{人类历史的发展的曲折性不会改变历史发展的前进性}}{\textcolor{red}{一些国家社会发展的特殊形式不能否定历史发展的普遍规律}}{\textcolor{red}{人们对社会发展某个阶段的认识不能代替社会发展的整个过程}}
\begin{solution}``历史终结论''的破产说明,人类历史的发展的曲折性不会改变历史发展的前进性,一些国家社会发展的特殊形式不能否定历史发展的普遍规律,人们对社会发展某个阶段的认识不能代替社会发展的整个过程。但是,社会规律和自然规律是有相异之处的,社会规律是人有意识的能动活动,自然规律是盲目的无意识的力量起作用,所以,正确答案是选项BCD。
\end{solution}
\question 马克思指出:``一个社会即使探索到了本身运动的自然规律,\ldots{}\ldots{}它还是既不能跳过也不能用法令取消自然的发展阶段。但是它能缩短和减轻分娩的痛苦。''这表明
\par\fourch{\textcolor{red}{人类社会的发展是合规律性与合目的性的统一}}{社会发展过程与自然界演变过程一样都是自觉的}{\textcolor{red}{人的自觉选择在社会发展中具有重要作用}}{\textcolor{red}{人类总体历史进程是不可超越的}}
\begin{solution}本题考查社会规律及其特点的理解,属间接性试题。马克思在《资本论》第一卷序言中指出:``我的观点是把经济的社会形态的发展理解为一种自然史的过程。''``一个社会即使探索到了本身运动的自然规律,------本书的最终目的就是揭示现代社会的经济运动规律,------它还是既不能跳过也不能用法令取消自然的发展阶段。但是它能缩短和减轻分娩的痛苦。''马克思所说的社会运动的``自然规律'',是指社会也是一种自然历史过程即也是合规律的,因而,人类总体历史进程是客观的、不可超越。但人们可以探索到社会规律,并利用它来``缩短和减轻''新的社会形态产生的痛苦,这又说明人的自觉选择、社会发展的合目的性。所以,A、C、D项正确。B项错误,因为选项中包含的自然界演变过程是自觉的观点是错误的,正确的观点应是自发的。
\end{solution}
