\question 共产主义社会是人的自由而全面的发展的社会,这里的全面发展指的是
\par\twoch{\textcolor{red}{人的体力和智力得到发展}}{\textcolor{red}{人的才能和工作能力得到发展}}{\textcolor{red}{人的社会联系和社会交往得到发展}}{人无所不能}
\begin{solution}ABC
三项,共产主义社会人的发展是全面的发展,不仅体力和智力得到发展,各方面的才能和工作能力得到发展,而且人的社会联系和社会交往也得到发展。D
项,全面发展不是指人无所不能。
\end{solution}
\question 马克思指出:``在共产主义社会高级阶段,在迫使个人奴隶般地服从分工的情形已经消失,从而脑力劳动和体力劳动的对立也随之消失后;在劳动已经不仅仅是谋生的手段,而且本身成了生活的第一需要之后;随着个人的全面发展,它们的生产力也增长起来,而集体财富的一切源泉都充分涌流之后,------只有在那个时候,才能完全超出资产阶级权利的狭隘眼界,社会才能在自己的旗帜上写上:各尽所能,按需分配!''下列选项对``各尽所能,按需分配''的理解正确的是(
)
\par\fourch{各尽所能,按需分配第一次以人的劳动而不是特权或资本作为分配的标准}{各尽所能,按需分配所体现的平等权利还是被限制在资产阶级的框框里}{\textcolor{red}{各尽所能,按需分配最终实现人类分配上的真正平等}}{各尽所能,按需分配不可能真正实现}
\begin{solution}各尽所能,按需分配实现人类分配上的真正平等。
\end{solution}
