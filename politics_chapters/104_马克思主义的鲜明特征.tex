\question 马克思主义最鲜明的政治立场是致力于实现以劳动人民为主体的最广大人民的根本利益,是否始终站在最广大人民的立场上是(
)
\par\twoch{马克思主义的根本特性}{无产阶级的历史使命}{\textcolor{red}{唯物史观与唯心史观的分水岭}}{\textcolor{red}{判断马克思主义政党的试金石}}
\begin{solution}本题考查的知识点:马克思主义最鲜明的政治立场
马克思主义认为,人民群众是历史的创造者,人民群众的根本利益体现了社会发展的要求和方向。在社会历史发展过程中,人民群众起着决定性的作用。人民群众是历史的主体,是历史的创造者。首先,人民群众是物质财富的创造者。其次,人民群众是社会精神财富的创造者。再次,人民群众是社会变革的决定力量。最后,人民群众既是先进生产力和先进文化的创造主体,也是实现自身利益的根本力量。马克思主义政党的一切理论和奋斗,都应致力于实现最广大人民的根本利益。因此,是否始终站在最广大人民的立场上,是唯物史观与唯心史观的分水岭,也是判断马克思主义政党的试金石。所以,C、D两项是符合题意的正确选项。A、B两项不符合题意,不选。马克思主义的根本特性是鲜明的阶级性和实践性。无产阶级的历史使命是彻底解放全人类,实现每个人自由而全面发展的共产主义社会。
\end{solution}
\question 判断唯物史观与唯心史观的{分水岭}的是
\par\fourch{把实践当做物质性的活动}{社会存在和社会意识何者第一}{\textcolor{red}{是否始终站在最广大人民的立场上}}{从实践出发去理解社会生活的本质}
\begin{solution}【解析】C
项,是否始终站在最广大人民的立场上,是唯物史观与唯心史观的分水岭,也是判断马克思主义政党的试金石。
\end{solution}
\question 马克思主义哲学与唯心主义哲学、旧唯物主义哲学的根本区别在于
\par\twoch{坚持人的主体地位}{坚持用辩证发展的观点去认识世界}{坚持物质第一性、意识第二性}{\textcolor{red}{坚持从客观的物质实践活动去理解现实世界}}
\begin{solution}(1)本题考查马克思主义哲学是科学的世界观和方法论中马克思主义哲学的基本特征的理解。
(2)马克思主义哲学即辩证唯物主义和历史唯物主义的创立是哲学发展史上的伟大变革。马克思主义哲学的创始人马克思、恩格斯对实践概念的科学规定和实践观点的确立是实现哲学上伟大变革的关键。马克思指出:``从前的一切唯物主义(包括费尔巴哈的唯物主义)的主要缺点是:对象、现实、感性,只是从客体的或者直观的形式去理解,而不是把它们当作感性的人的活动,当作实践去理解,不是从主体方面去理解。''``和唯物主义相反,唯心主义却发展了能动的方面,但只是抽象地发展了,因为唯心主义当然是不知道现实的感性的活动本身的。''``哲学家们只是用不同的方式解释世界,问题在于改造世界。''实践的观点是马克思主义哲学的根本特征,是马克思主义哲学同以往哲学包括旧唯物主义和唯心主义哲学的根本区别,实践范畴是马克思主义哲学体系的中心范畴。
(3)D选项是符合试题要求的正确观点。A选项是人本主义的观点,B选项是唯物辩证法和唯心辩证法的共同观点,C选项是一切唯物主义的共同观点。故无论采用正选法还是排谬法,正确选项只能是D。
\end{solution}
