\question 资本主义制度下的社会财富表现为一种惊人的庞大的商品堆积,单个的商品表现为它的元素形式。
以下说法正确的是
\par\fourch{\textcolor{red}{不论财富的社会的形式如何,使用价值总是构成财富的物质的内容}}{劳动是使用价值的唯一源泉}{商品不一定是劳动产品}{具体劳动形成商品的价值实体}
\begin{solution}【简析】马克思指出:``不论财富的社会的形式如何,使用价值总是构成财富的物质的内容A正确。劳动是使用价值的源泉之一,不是唯一源泉,B错误。
商品一定是劳动产品,C错误。具体劳动形成商品的使用价值,抽象劳动形成商品的价值实体,D错误。
\end{solution}
\question 资本主义制度下的工资之所以掩盖了资本主义剥削关系,是因为
\par\twoch{\textcolor{red}{工资表现为“劳动的价格”}}{工资表现为劳动的价值}{工资模糊了个别劳动时间和社会必要劳动时间界限}{工资的价值只计算了转移到新产品中去得那一部分}
\begin{solution}A
(注:此题考的是为什么掩盖了,不是考工资本身是什么,正是因为工资表现为``劳动的价格''或工人全部劳动的报酬,模糊了工人必要劳动和剩余劳动的界限,才掩盖了资本主义剥削关系。)
【简析】在资本主义制度下,工人工资是劳动力的价值或价格,这是资本主义工资的本质。资本家购买工人的劳动力是以货币工资形式支付的,工资表现为``劳动的价格''或工人全部劳动的报酬,这就模糊了工人必要劳动和剩余劳动的界限,掩盖了资本主义剥削关系。A正确。
\end{solution}
