\question 中国工程院院士袁隆平曾结合自己的科研经历,语重心长地对年轻人说:``书本知识非常重要,电脑技术也很重要,但是书本电脑里面种不出水稻来,只有在田里才能种出水稻来。''这表明
\par\fourch{\textcolor{red}{实践是人类知识的基础和来源}}{实践水平的提高有赖于认识水平的提高}{理论对实践的指导作用没有正确与错误之分}{由实践到认识的第一次飞跃比认识到实践的第二次飞跃更重要}
\begin{solution}本题考查的是实践和认识的关系。实践是认识的基础,是认识的来源。袁隆平的话表明,只有通过实践才能获得认识,实践是认识的来源。A选项正确。实践水平的提高并不是依赖于认识水平的提高,B选项错误。正确的认识促进实践发展,错误的认识阻碍实践发展,C选项错误。两次飞跃中第二次飞跃更为重要,D选项错误。
\end{solution}
\question 社会存在决定社会意识,社会意识是社会存在的反映。社会意识具有相对独立性,即它在反映社会存在的同时,还有自己特有的发展形式和规律。社会意识相对独立性最突出的表现是
\par\fourch{社会意识与社会存在发展的不完全同步性}{社会意识内部各种形式之间的相互作用和影响}{社会意识各种形式各自有其历史继承性}{\textcolor{red}{社会意识对社会存在具有能动的反作用}}
\begin{solution}社会意识的相对独立性主要表现为:社会意识与社会存在发展的不平衡性;社会意识内部各种形式之间的相互影响及各自具有的历史继承性;社会意识对社会存在的能动的反作用。其中,社会意识对社会存在的能动的反作用是社会意识相对独立性的突出表现。D选项正确。
\end{solution}
\question 恩格斯指出``在历史上出现的一切社会关系和国家关系,一切宗教制度和法律制度,一切理论观念,只有理解了每一个与之相关的时代的物质生活条件,并从这些物质条件中被引申出来的时候,才能理解'',这表明(
)
\par\fourch{社会意识及其载体都是社会存在}{社会意识决定社会存在}{社会意识具有反作用}{\textcolor{red}{社会存在决定社会意识}}
\begin{solution}A,B观点错误,C与题干无关。
\end{solution}
\question 社会意识就是( )
\par\twoch{\textcolor{red}{社会生活的精神方面}}{经济上占统治地位的阶级的意识}{人民群众的意识}{政治上占统治地位的阶级的意识}
\begin{solution}本题考查社会意识的概念。答案可以直接看出是选项A。其他的选项都是本题的干扰项。
\end{solution}
\question 社会意识的相对独立性表现为( )
\par\fourch{\textcolor{red}{社会意识与社会存在发展的不平衡性、不同步性}}{\textcolor{red}{社会意识对社会存在的能动的反作用}}{\textcolor{red}{社会意识具有历史继承性}}{\textcolor{red}{各种社会意识形式之间的相互影响}}
\begin{solution}本题考查社会意识的相对独立性。这是基本的重要知识点。社会意识是对社会存在的反映,社会意识一旦产生之后,便具有自身的相对独立性,这种相对独立性表现为各个方面,如社会意识与社会存在的不平衡,社会意识自身的历史继承性,社会意识诸形式的相互影响,它对社会存在的能动反作用,所以本题四个选项都是社会意识相对独立性的体现。
\end{solution}
\question 关于社会意识对社会存在的能动的反作用,理解正确的有( ~)
\par\fourch{它具有历史继承性}{\textcolor{red}{先进的社会意识对社会发展起积极的促进作用}}{它与社会存在在发展上具有不平衡性}{\textcolor{red}{它的能动作用是通过指导人们的实践活动来实现的}}
\begin{solution}本题考查对社会意识的相对独立性和能动的反作用的理解掌握。社会意识与社会存在在发展上具有不平衡性,它内部各种形式之间具有相互影响和具有各自的历史继承性。但这些都只是社会意识的相对独立性的体现,而非社会意识的能动的反作用的体现。所以排除选项A、C。社会意识的能动反作用的相关体现,包括了选项B、D的内容,所以可选。
\end{solution}
