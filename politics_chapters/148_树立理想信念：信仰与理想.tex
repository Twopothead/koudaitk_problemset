\question 如果说社会是大海,人生是小舟,那么理想信念就是引航的灯塔和推进的风帆。没有科学的理想信念的人生,就像失去了方向和动力的小船,在生活的波浪中随处漂泊,甚至会沉没于急流之中。这说明理想信念能够(
~)
\par\fourch{保证人生的追求成功}{\textcolor{red}{指引人生的奋斗目标}}{\textcolor{red}{提供人生的前进动力}}{\textcolor{red}{提高人生的精神境界}}
\begin{solution}本题难度不大,理想信念的作用表现在三个方面,可以直接选出来BCD选项。
\end{solution}
\question 马克思曾经说过:``作为确定的人,现实的人,你就有规定,就有使命,就有任务,至于你是否意识到这一点,那都是无所谓的。这个任务是由于你的需要及其与现存世界的联系而产生的。''当代大学生承担的历史使命是(
~)
\par\fourch{实现社会主义的现代化}{\textcolor{red}{建设中国特色社会主义}}{\textcolor{red}{实现中华民族伟大复兴}}{实现全面建设小康社会}
\begin{solution}当代大学生承担的历史使命表述为建设中国特色社会主义和实现中华民族的伟大复兴。AD选项属于建设中国特色社会主义的内容。
\end{solution}
\question 文王拘而演《周易》;仲尼厄而作《春秋》;屈原放逐,乃赋《离骚》;左丘失明,厥有《国语》;孙子膑脚,兵法修列;不韦迁蜀,世传《吕览》;韩非囚秦,《说难》、《孤愤》;《诗》三百篇,大抵圣贤发愤之所为作也。这对我们正确对待实现理想过程中的逆境的启示是(
~)
\par\fourch{人们在逆境中更容易接近和实现目标}{\textcolor{red}{人们在逆境中可以磨练意志、陶冶品格}}{\textcolor{red}{逆境没有消解实现理想目标的可能性}}{\textcolor{red}{逆境增大了人们向理想目标前进的难度}}
\begin{solution}文王拘而演《周易》;仲尼厄而作《春秋》;屈原放逐,乃赋《离骚》;左丘失明,厥有《国语》;孙子膑脚,兵法修列;不韦迁蜀,世传《吕览》;韩非囚秦,《说难》、《孤愤》;《诗》三百篇,都是说明要实现理想新信念,要在逆境中奋斗,付出更大的努力和更多的艰辛才能成功。逆境只是增大了人们实现目标的难度,所以A排除。
\end{solution}
