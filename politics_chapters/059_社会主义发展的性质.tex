\question 社会主义在发展过程中出现挫折和反复,这表明( )
\par\fourch{\textcolor{red}{新生事物的成长不是一帆风顺的}}{\textcolor{red}{事物发展的道路是螺旋式的}}{社会发展的客观趋势不是不可以改变的}{\textcolor{red}{历史有时会向后作巨大的跳跃}}
\begin{solution}社会主义的发展是前进性与曲折性的统一。马克思主义认为,人类社会的发展从来不是一帆风顺,而是在曲折中前进的。社会主义作为崭新的社会形态,符合历史的发展趋势,具有强大的生命力,但社会主义的产生和成长意味着对资本主义旧社会的否定,必然遭到资本主义的拼死反抗,这就注定社会主义战胜资本主义是一个曲折的发展过程。社会主义在发展过程中,由于历史的和现实的、国际的和国内的各种因素的相互作用,社会主义的发展道路必然呈现出多样性的特点。这表明,坚持社会主义不等于坚持某种单一的社会主义模式,某种社会主义模式的失败不等于整个社会主义事业的失败。A、B、D、E项正确。
C项错误,人们不能改变社会发展的客观趋势。
\end{solution}
\question 恩格斯指出:``所谓`社会主义'不是一种一成不变的东西,而应当和任何其他社会制度一样,把它看成是经常变化和改革的社会。''社会主义改革的根源是(
)
\par\fourch{改革是社会主义社会发展的动力}{生产力发展水平不够高}{社会主义制度没有根本克服资本主义制度下生产力与生产关系的对抗性矛盾}{\textcolor{red}{社会主义社会的基本矛盾}}
\begin{solution}社会主义社会的基本矛盾生产力和生产关系之间的矛盾是社会改革的根源所在。
\end{solution}
\question 列宁指出:``设想世界历史会一帆风顺、按部就班地向前发展,不会有时出现大幅度的跃退,那是不辩证的,不科学的,在理论上是不正确的。''社会主义在曲折中发展的原因在于(
)
\par\fourch{\textcolor{red}{社会主义作为新生事物,其成长不会一帆风顺}}{\textcolor{red}{经济全球化对于社会主义的发展既有机遇又有挑战}}{\textcolor{red}{社会主义社会的基本矛盾推动社会发展是作为一个过程而展开的}}{\textcolor{red}{人们对社会主义社会的基本矛盾推动社会发展的认识有一个逐渐发展的过程}}
\begin{solution}社会主义作为新生事物,其成长不会一帆风顺,经济全球化对于社会主义的发展既有机遇又有挑战,社会主义社会的基本矛盾推动社会发展是作为一个过程而展开的,人们对社会主义社会的基本矛盾推动社会发展的认识有一个逐渐发展的过程。
\end{solution}
