\question 商品是( ~)
\par\fourch{\textcolor{red}{用来交换的劳动产品}}{\textcolor{red}{使用价值和价值的统一}}{满足生产者自己需要的劳动产品}{\textcolor{red}{一定生产关系的体现}}
\begin{solution}C选项明显错误,商品是要交换的,留着自己用的不叫商品。
\end{solution}
\question 商品经济的发展经历了简单商品经济和发达商品经济两个阶段。简单商品经济又称``小商品经济'',以生产资料个体所有制和个体劳动为基础的商品经济。简单商品经济包含着一系列内在矛盾,其中最基本的矛盾是(
~)
\par\fourch{\textcolor{red}{私人劳动和社会劳动的矛盾}}{具体劳动和抽象劳动的矛盾}{生产社会化和生产资料私人占有的矛盾}{个别劳动时间和社会必要劳动时间的矛盾}
\begin{solution}私人劳动和社会劳动的矛盾是商品经济的基本矛盾,最后演变为资本主义经济的基本矛盾。
\end{solution}
\question 马克思指出,在商品经济中,价值规律是``作为起调节作用的自然规律强制地为自己开辟道路,就像房屋倒在人的头上时重力定律强制为自己开辟道路一样''。这段话表明(
~)
\par\fourch{\textcolor{red}{价值规律是商品经济的一般规律或基本规律}}{\textcolor{red}{价值规律和自然规律一样具有客观性}}{\textcolor{red}{价值规律具有自发性}}{价值规律排斥人的主观能动性}
\begin{solution}D观点错误。
\end{solution}
\question 构成社会财富的物质内容是( ~)
\par\twoch{价值}{交换价值}{\textcolor{red}{使用价值}}{价格}
\begin{solution}使用价值是商品能够满足人们需要的属性,反映了物品的自然属性,它是一切物品包括劳动产品的共性,构成社会财富的物质内容。价值是商品特有的社会属性,交换价值和价格都是价值的表现形式。
\end{solution}
\question 简单商品经济的基本矛盾是私人劳动和社会劳动的矛盾。这是因为( )
\par\twoch{\textcolor{red}{它是商品各种内在矛盾的根源}}{\textcolor{red}{它决定着商品生产者的命运}}{它是决定和影响价格的重要因素}{\textcolor{red}{它贯穿私有制商品经济产生和发展的全过程}}
\begin{solution}私人劳动与社会劳动的矛盾是简单商品生产的基本矛盾,理由在于:其一,它是商品和商品生产一切矛盾如价值与使用价值,具体劳动和抽象劳动,个别劳动时间和社会必要劳动时间的根源;其二,它贯穿商品生产产生、发展的全过程;其三,它决定着以私有制为基础的商品生产者的命运。
\end{solution}
