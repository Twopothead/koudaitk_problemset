\question 马克思认为,``商品形式的奥秘不过在于:商品形式在人们面前把人们本身劳动的社
会性质反映成劳动产品本身的物的性质,反映成这些物的天然的社会属性,从而把
生产者同总劳动的社会关系反映成存在于生产者之外的物与物之间的社会关系。由
于这种转换,劳动产品成了商品,成了可感觉而又超感觉的物或社会的物。''这表明
( )
\par\fourch
{\textcolor{red}{商品本质上体现的是人与人之间的关系}}
{\textcolor{red}{商品把人与人之间的关系物化了}}
{\textcolor{red}{商品的”天然的社会属性”就在于人们本身劳动的社会性质}}
{商品之所以成为商品是因为它是劳动产品}
\begin{solution}本题考查考生对商品属性的理解。劳动产品在交换中取得了商品的形式,商品
交换的过程不但是人们交换使用价值的过程,也是人们相互比较自身劳动(商品价值)
的过程。这一过程表面上是物物交换(使用价值的交换),实际上则反映了商品生产者
之间的社会关系。因此,可以说商品把人与人之间的关系物化了,商品本质上体现的是
人与人之间的关系。普通商品具有自然的和社会的双重属性,自然属性在于其使用价
值,社会属性在于其所包含的无差别的一般人类劳动,也就是C项所说的人们本身劳
动的社会性质。选项D的说法是错误的,劳动产品天然并非是商品,只有用于交换的
劳动产品才是商品,因此商品之所以成为商品的根本原因在于:它是用于交换的劳动产
品,它反映了人们本身劳动的社会性质。因此,正确答案为ABC。
\end{solution}
\question 使用价值不同的商品之所以能按一定比例相交换,是因为它们都有价值,而价值可以互相比较是因为
\par\fourch{价值是一切劳动产品所共有的属性}{\textcolor{red}{价值在质的规定性上是相同的}}{价值是具体劳动创造的}{价值和使用价值具有同一性}
\begin{solution}使用价值反映的是人与自然之间的物质关系,是一切劳动产品所共有的属性,但价值并不是一切劳动产品所共有的属性,只有用来交换的劳动产品(即商品)才具有价值,A错误。使用价值不同的商品之所以能按一定比例相交换.就是因为它们都有价值。商品价值在质的规定性上是相同的,因而彼此可以比较。B正确。具体劳动形成商品的价值实体。C错误。D不符合题意。
\end{solution}
\question 商品的价值不仅有质的规定性,而且还有量的规定性。商品价值的计量尺度是
\par\twoch{\textcolor{red}{简单劳动}}{个别劳动}{劳动强度}{具体劳动}
\begin{solution}【答案】A
【简析】商品价值是以简单劳动为尺度度计量的.复杂劳动等于自乘的或多倍的简单劳动。A正确,
\end{solution}
\question 马克思指出:``我们在这里最初看到的利润,和剩余价值是一回事,不过它具有一个神秘的形式,而这个神秘化的形式必然会从资本主义生产方式中产生出来。''恩格斯也指出:``马克思一
有机会就提醒读者注意,绝不要把他所说的剩余价值同利润或资本盈利相混淆。''对这两段话理解错误的是(
)
\par\fourch{利润是剩余价值的一种具体形式}{剩余价值是利润的本质内容}{\textcolor{red}{剩余价值是资本的盈利}}{利润常常只是剩余价值的一部分}
\begin{solution}本题考查利润的本质。剩余价值转化为利润,是与生产成本这个概念紧密联系的。在资本主义的生产过程中,不仅耗费的资本在资本主义的生产过程中发挥着作用,预付资本中暂
时没有消耗掉、还没有转移到新产品的那部分不变资本也同样参与了商品的生产过程,同样是
剩余价值生产的不可缺少的物质要素,这样剩余价值就进一步表现为全部预付资本的增加额。
当不把剩余价值看做是雇佣工人剩余劳动的产物,而是把它看做是全部预付资本的产物或增
加额时,剩余价值就转化为利润Q可见,利润和剩余价值本是同一个东西,所不同的是,剩余价值是对可变资本而言的,而利润是对全部预付资本而言的。因此,剩余价值是利润的本质,利润是剩余价值的转化形式,即从现象的表现形式来看,利润常常只是剩余价值的一部分。因此
ABD选项观点都是正确的,不符合题意。C选项观点错误,符合题意。
\end{solution}
\question 关于经济学上的财富,有各种各样的名称,有人称之为物质财富,有人称之为自然财富,有人称之为人为财富,也有人称之为财富;明确地将经济学上的财富称为社会财富的是马克思,在马克思看来,在一切社会里,社会财富的物质内容都是由(
)
\par\twoch{金银构成的}{价值构成的}{货币构成的}{\textcolor{red}{使用价值构成的}}
\begin{solution}社会财富的物质内容即有使用价值构成的。
\end{solution}
\question 在人类历史上,自从出现商品交换以来,商品的价值形式已经历了四个发展阶段,有四种不同的表现形式,依次是(
~)
\par\fourch{\textcolor{red}{简单的或偶然的价值形式、总和的或扩大的价值形式、一般价值形式、货币形式}}{简单的或偶然的价值形式、一般价值形式、总和的或扩大的价值形式、货币形式}{简单的或偶然的价值形式、一般价值形式、货币形式、总和的或扩大的价值形式}{简单的或偶然的价值形式、货币形式、一般价值形式、总和的或扩大的价值形式}
\begin{solution}商品的价值形式有四种不同的表现形式,简单的或偶然的价值形式、总和的或扩大的价值形式、一般价值形式、货币形式。
\end{solution}
