\question 1925年5月,以五卅运动为起点,掀起了全国范围的大革命高潮。1926年7月,以推翻北洋军阀统治为目标的北伐战争开始。1927年3月国民革命军占领南京。1927年4月12日,蒋介石在上海发动反共政变,同年7月15日,汪精卫在武汉召开``分共''会议,并在其辖区内对
共产党员和革命群众实行搜捕和屠杀,国共合作全面破裂,大革命最终失败。所谓大革命的失败,主要是指
\par\fourch{北洋军阀的统治没有被推翻}{\textcolor{red}{反帝反封建的革命任务没有完成}}{国共合作全面破裂}{工人农民运动转入低潮}
\begin{solution}【简析】1925---1927年的大革命,虽然推翻了北洋军阀的统治,但国民党政府的统治依然是地主阶级和买办性的大资产阶级的统治,同北洋军阀的统治没有本质的区別。中国仍是一个处在帝国主义和封建主义统治之下的半殖民半封建社会,中国仍然迫切需要一个反帝反封违的资产阶级民主革命。所谓大革命的失败,主要是指反帝反封建的革命任务没有完成,B正确,
A错误。国共合作全面破裂是大革命失败的标志,工人农民运动转人低潮是大革命失败的后果,都不是大革命失败的含义,
C、D不符合题意。
\end{solution}
\question 第一次国共合作的成果主要有( ~)
\par\twoch{\textcolor{red}{建立黄埔军校}}{\textcolor{red}{北伐战争胜利发展}}{\textcolor{red}{工人和农民运动迅猛发展}}{推翻了帝国主义和封建军阀的统治}
\begin{solution}第一次国共合作基本上推翻了北洋军阀的统治,但中国仍然是半殖民地半封建社会,并未推翻帝国主义。D错误。
\end{solution}
