\question 以毛泽东为主要代表的中国共产党人在创建新中国和探索适合中国情况的社会主义建设道路过程中,创造了一系列重要的理论。在社会主义民主政治建设方面的成果主要有
\par\fourch{\textcolor{red}{要把“造成一个又有集中又有民主,又有纪律又有自由,又有统一意志、又有个人心情舒畅、生动活泼,那样一种政治局面”作为努力的目标}}{\textcolor{red}{把正确处理人民内部矛盾作为国家政治生活的主题,坚持人民民主,尽可能团结一切可以团结的力量}}{\textcolor{red}{处理好中国共产党同各民主党派的关系,坚持长期共存、互相监督的方针,巩固和扩大爱国统一战线}}{坚持民主集中制和集体领导原则,反对任何形式的个人崇拜,保证中国共产党决策的科学化、民主化}
\begin{solution}【解析】以毛泽东为主要代表的中国共产党人在创建新中国和探索适合中国情况的社会主义建设道路过程中,创造了一系列重要的理论。在社会主义民主政治建设方面的成果有要把``造成一个又有集中又有民主,又有纪律又有自由,又有统一意志、又有个人心情舒畅、生动活泼,那样一种政治局面''作为努力的目标;把正确处理人民内部矛盾作为国家政治生活的主题,坚持人民民主,尽可能团结一切可以团结的力量;处理好中国共产党同各民主党派的关系,坚持长期共存、互相监督的方针,巩固和扩大爱国统一战线;切实保障人民当家作主的各项权利,尤其是人民参与国家和社会事务管理的权利;社会主义法制要保护劳动人民利益,保护社会主义经济基础,保护社会生产力。D项``必须坚持民主集中制和集体领导原则,反对任何形式的个人崇拜,保证中国共产党决策的科学化、民主化''是文化大革命留给共产党的深刻教训之一。
\end{solution}
\question 毛泽东提出要造就``一个又有集中又有民主,又有纪律又有自由,又有统一意志、又有个人心情舒畅、生动活泼,那样一种政治局面''(通称``六又''政治局面)的思想是在(
~)
\par\fourch{《论十大关系》}{《关于正确处理人们内部矛盾的问题》}{中共八大}{\textcolor{red}{《一九五七年夏季的形势》}}
\begin{solution}``六又''的政治局面思想是在《一九五七年夏季的形势》中提出来的,大纲解析原文。
\end{solution}
