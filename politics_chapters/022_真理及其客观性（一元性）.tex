\question 实践是检验真理的标准。但实践标准有不确定性。其不确定性的含义包括( )
\par\twoch{\textcolor{red}{实践无法一下检验所有真理}}{\textcolor{red}{实践无法充分证明某一真理}}{实践能够检验一切真理}{\textcolor{red}{被实践检验过的真理需要继续接受检验}}
\begin{solution}实践标准的不确定性是指,由于一定历史阶段上的具体实践具有局限性,因此,其一,它无法一下子证明所有的真理;因为世界无限宽广。其二,它无法充分证明某一真理,因为事物复杂;其三,已被实践检验过的真理还要继续经受实践的检验,因为事物在不断发展变化。
\end{solution}
\question 逻辑证明在实践检验真理的过程中有重要作用,但不是检验真理的标准,这是因为(
)
\par\fourch{\textcolor{red}{逻辑证明只能证明前提与结论的一致性}}{\textcolor{red}{逻辑法则在实践中产生,且必须经过实践的检验}}{逻辑法则是不确定的}{逻辑证明是脱离实践的}
\begin{solution}本题考查逻辑证明在实践检验真理的过程中的重要作用。实践是检验真理的唯一标准,但并不排斥逻辑证明的作用。选项A、B反映了逻辑证明不能成为检验真理标准的原因。选项C、D本身就是错误的表达,是干扰项。
\end{solution}
