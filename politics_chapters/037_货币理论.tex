\question 1918年,马寅初在一次演讲时,有一位老农问他:``马教授,请问什么是经济学''马寅初笑着说:``我给这位朋友讲个故事吧,有个赶考的书生到旅店投宿,拿出十两银子,挑了该旅店标价十两银子的最好房间,店主立刻用它到隔壁的米店付了欠单,米店老板转身去屠夫处还了肉钱,屠夫马上去付清了赊帐的饲料款,饲料商赶紧到旅店还了房钱。就这样,十两银子又到了店主的手里。这时书生来说,房间不合适,要回银子就走了。你看,店主一文钱也没赚到,大家却把债务都还清了,所以,钱的流通越快越好,这就是经济学。''在这个故事中,货币所发挥的职能有
\par\twoch{\textcolor{red}{支付手段}}{\textcolor{red}{流通手段}}{\textcolor{red}{价值尺度}}{贮藏手段}
\begin{solution}本题正确答案为ABC。从题干中``挑了该旅店标价十两银子的最好房间''这句话中的``标价''两个字,可以判断,货币执行了价值尺度的功能,故C正确。支付手段,和流通手段两大职能的区别在于是否与实物交割同步,而本题``同步''与``不同步''两种情况都有出现,因此AB均正确。
\end{solution}
