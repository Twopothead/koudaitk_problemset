\question 在资产阶级民主革命思潮广泛传播、革命形势日益成熟的时候,康有为、梁启超等人坚持走改良道路,反对用革命手段推翻清朝统治。1905年至1907年间,围绕中国究竟是采用革命手段
还是改良方式这个问题,革命派与改良派分别以《民报》《新民丛报》为主要舆论阵地,展开了一场大论战。双方论战涉及的核心问题主要有三个,其中双方论战的焦点是(
)
\par\fourch{要不要推翻帝制,实行共和}{\textcolor{red}{要不要以革命手段推翻清王朝}}{要不要实行君主立宪}{要不要社会革命}
\begin{solution}【解析】要不要以革命手段推翻清王朝,这是双方论战的焦点。改良派说,革命会引起下层社会
暴乱,招致外国的干涉、瓜分,使中国``流血成河''``亡国灭种'',所以要爱国就不能革命,只能改良、立宪。革命派针锋相对地指出,清政府是帝国主义的``鹰犬'',因此爱国必须革命。只有通
过革命,才能``免瓜分之祸'',获得民族独立和社会进步。A和D选项内容也属于这次论战三个
内容的其他两个。C选项``要不要实行君主立宪''是维新派和保守派论战的内容之一。
\end{solution}
\question 中国同盟会成立的革命宗旨是( ~)
\par\fourch{反对君主立宪派}{实行民族主义、民权主义、民生主义}{驱除鞑虏,恢复中华,创立合众政府}{\textcolor{red}{驱除鞑虏,恢复中华,创立民国,平均地权}}
\begin{solution}考查同盟会纲领的表述。C选项是兴中会的纲领,注意区分兴中会与同盟会。B选项是三民主义的表述,是同盟会纲领的概括。
\end{solution}
\question 1905年8月20日,中国同盟会成立。它的纲领是( ~)
\par\fourch{驱除鞑虏,恢复中华}{\textcolor{red}{驱除鞑虏,恢复中华,创立民国,平均地权}}{驱除鞑虏,恢复中华,创立合众政府}{驱除鞑虏,恢复中华,创立民国}
\begin{solution}同盟会的纲领是民族、民权、民生三大主义。民族主义包括``驱除鞑虏,恢复中华''两项内容;民权主义的内容是``创立民国'',民生主义的内容为``平均地权''。
\end{solution}
