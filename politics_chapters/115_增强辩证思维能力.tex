\question 老子曾说:``天下皆知美之为美,斯恶已。皆知善之为善,斯不善已。故有无相生,难易
相成,长短相形,高下相倾,音声相和,前后相随。''这段话包含的哲学观点是()
\par\fourch{矛盾的对立面可以相互转化}{\textcolor{red}{矛盾的对立面之间相辅相成}}{矛盾的斗争性是绝对的}{\textcolor{red}{任何事物都有其对立的一面}}
\begin{solution}本题考查矛盾的观点。题干表明,事物都有其对立面,事物因其对立面而生
成,因此表现出一种相辅相成的态势,例如善与不善、有与无、长与短、高与低、
前与后等。这种相辅相成正是推动事物变化发展的力量所在。故选项BD正确。
\end{solution}
\question ``我们的事业越前进、越发展,新情况新问题就会越多,面临的风险和挑战就会越多,面对的不可预料的事情就会越多。我们必须增强忧患意识,做到居安思危。这要求我们
\par\fourch{\textcolor{red}{要善于看到矛盾的普遍性,勇于承认矛盾、揭露矛盾、分析矛盾、解决矛盾}}{\textcolor{red}{要认识矛盾的对立统一,全面的、一分为二的看问题}}{\textcolor{red}{要明白意识的能动性表现在具有髙度的创造性}}{要善于把握质量互变规律}
\begin{solution}本题是对马克思主义唯物论与辩证法的综合性考查。``问题越多、风险和挑战越多、不可预料的事情越多''体现了矛盾的普遍性存在,并要求全面地看问题。``忧患意识和居安思危''强调了人的意识的能动性。材料并没有体现出质量互变规律,故选ABC。
\end{solution}
