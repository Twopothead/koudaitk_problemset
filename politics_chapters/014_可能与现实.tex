\question 把握可能性这个范畴,要注意区分以下哪几种情况( )
\par\twoch{\textcolor{red}{可能性和不可能性}}{简单的可能性和复杂的可能性}{\textcolor{red}{好的可能性和坏的可能性}}{\textcolor{red}{现实的可能性和抽象的可能性}}
\begin{solution}本题考查的知识点:可能性与现实性
首先把题干审清楚,问怎么更好的把握可能性这个范畴。把握可能性这个范畴,需要注意区分以下几种不同的情况:(1)可能性和不可能性;可能性是指事物发展过程中潜在的东西,是包含在事物中并预示事物发展前途的种种趋势。不可能指违背事物发展的规律性必然性,在现实中没有任何客观根据和条件,因而,是永远不能实现的东西。(2)现实的可能性和抽象的可能性;现实的可能性是指在现实中有充分的根据和必要条件,因而在一定阶段上可以实现的可能性。抽象的可能性又叫非现实的可能性,是指在现实中虽有一定根据,但根据尚未展开,必要条件尚不具备,因而只在以后的发展阶段上才可以实现的可能性。(3)好的可能性和坏的可能性。故选项ACD正是我们需要区分的。选项B简单的可能性和复杂的可能性,是干扰选项,故不选。因此,本题正确答案为选项ACD。
\end{solution}
\question 关于事物的可能性,下述正确的观点有( )
\par\fourch{\textcolor{red}{可能性有好的可能性和坏的可能性}}{\textcolor{red}{可能性和现实性在一定条件下可以相互转化}}{抽象的可能性是主观意识范畴的可能性}{可能性是事物中有内在根据,合乎必然性的存在}
\begin{solution}本题考查的知识点:可能性与现实性
可能性指包含在现实事物之中的、预示着事物发展前途的种种趋势,是潜在的尚未实现的东西。把握可能性这个范畴,要对可能性的各种情况加以区分:可能性和不可能性;现实的可能性和抽象的可能性;两种相反的可能性即好的可能性和坏的可能性;可能性的程度大小。可能性和现实性是统一的。第一,可能性和现实性相互依存。第二,可能性和现实性在一定条件下可以相互转化。事物的发展,总是在现实性中产生出可能性,而可能性又不断变为现实性的转化过程。所以,选项A可能性有好的可能性和坏的可能性正确。选项B可能性和现实性在一定条件下可以相互转化正确,是符合题意的正确选项。抽象的可能性是非现实的可能性,也是一种可能性,不是主观意识。现实性是包含内在根据的、合乎必然性的存在。故选项CD观点错误。因此,本题正确答案是选项AB。
\end{solution}
