\question 1949年中华人民共和国的成立,标志着我国社会进入了由新民主主义到社会主义的过渡时期。新民主主义社会的基本特征是
\par\fourch{\textcolor{red}{政治制度是工人阶级领导的以工农联盟为基础的各革命阶级联合专政}}{经济制度是公有制为主体、多种所有制经济共同发展}{\textcolor{red}{文化上发展以马克思主义指导下的新民主主义文化}}{国内主要矛盾是工人阶级和资产阶级的矛盾}
\begin{solution}【答案】AC
【简析】新民主义社会在政治上实行工人阶级领导的,一工农联盟为基础的,各革命阶级联合专政的政治制度;在经济实行国营经济领导下地合作社经济、个体经济、私人资本主义经济和国家资本主义经济五种经济成分并存的经济制度;在文化上发展以马克思主义指导下地新民主义的文化,即名族的、科学的、大众的文化A、C正确,B错误。新民主主义社会国内主要矛盾有个变化过程,1952年底,随着土地改革的基本完成,工人阶级和资产阶级的矛盾才逐步成为国内的主要矛盾,D错误。
\end{solution}
