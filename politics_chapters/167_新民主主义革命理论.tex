\question 毛泽东第一次提出新民主主义革命的科学概念和总路线的著作是
\par\fourch{\textcolor{red}{《中国革命和中国共产党》}}{《新民主主义论》}{《在晋绥干部会议上的讲话》}{《论人民民主专政》}
\begin{solution}【简析】A正确。毛泽东在《中国革命和中国共产党》一文中,首次明确提出了``新民主主义革命''这个概念,并对新民主主义革命的总路线即对象、任务、动力、性质和前途等问题作了深刻的论述。
\end{solution}
\question 新民主主义革命的最基本的动力是
\par\twoch{贫农}{知识分子}{\textcolor{red}{无产阶级}}{工人和农民}
\begin{solution}【简析】C正确。无产阶级是新民主主义革命的最基本的动力,贫农是无产阶级最可靠的同盟军,知识分子是新民主主义革命的动力之一。
\end{solution}
\question 新民主主义革命的主要内容是
\par\fourch{没收封建地主阶级的土地归新民主主义国家所有}{没收官僚资产阶级的垄断资本归新民主主义国家所有}{\textcolor{red}{没收封建地主阶级的土地归农民所有}}{保护民族工商业}
\begin{solution}【简析】土地革命即没收封建地主阶级的土地归农民所有,是新民主主义革命的主要内容,也是新民主主义的经济纲领的内容之一。C是正确项。B、D是三大经济纲领中的两项内容,但不是新民主主义革命的主要内容。A项内容错误。新民主主义革命解决土地问题的目标,是消灭封建的土地制度,实现``耕者有其田''。这里要注意新民主主义革命的主要内容与中国革命的中心问题的区别,中心问题是领导权问题。
\end{solution}
\question 抗日战争时期,中国共产党在敌后抗日根据地实行的土地政策是
\par\fourch{没收地主土地分配给农民}{保持原有的土地状态}{没收一切土地平均分配}{\textcolor{red}{减租减息}}
\begin{solution}【简析】略
\end{solution}
\question 毛泽东系统阐述中国革命三大法宝的文章是
\par\fourch{\textcolor{red}{《〈共产党人〉发刊词》}}{《中国革命和中国共产党》}{《新民主主义论》}{《论联合政府》}
\begin{solution}【简析】略
\end{solution}
\question 毛泽东提出``须知政权是由枪杆子中取得的''著名论断的会议是
\par\twoch{中共四大}{\textcolor{red}{八七会议	}}{古田会议	}{遵义会议}
\begin{solution}【简析】略
\end{solution}
