\question 在当代资本主义条件下,科学技术的不断进步和生产社会化程度的不断提高,必然要求调整和变革那些不适应生产力发展的旧的生产关系。所以当代资本主义国家,在其根本制度不便的情况下,在生产资料所有制形式、劳资关系分配关系、阶级阶层结构等方面做出了很大的调整,但是对于资本主义社会根本不能变的是
\par\twoch{\textcolor{red}{生产资料私有制}}{\textcolor{red}{雇佣劳动制度}}{\textcolor{red}{追逐剩余价值的本性}}{\textcolor{red}{资本主义政治制度}}
\begin{solution}本题是对当代资本主义新变化的理解和记忆型考查。当代资本主义发生变化实质是在资本主义制度基本框架内的变化,并不意味着资本主义生产关系的根本性质发生了变化。故资本主义社会根本不能触动的是ABCD选项。
\end{solution}
\question 资本主义国家宏观调节的基本目标是实现
\par\twoch{\textcolor{red}{经济快速增长}}{预算与债务平衡}{\textcolor{red}{物价稳定}}{\textcolor{red}{充分就业}}
\begin{solution}答案】ACD
【简析】资本主义国家宏观调节主要是国家运用财政政策.货币政策等经济手段.对社会总供求进行调节.以实现经济快速增长、充分就业、物价稳定和国际收支平衡的基本目标。A、C、D正确
\end{solution}
