\question 从1979年11月起,在邓小平主持下,中共中央着手起草《关于建国以来党的若干历史问题的
决议》。在1981年6月27日到29日召开的中共十一届六中全会上,《决议》获得一致通过。
下列关于《决议》表述正确的是
\par\fourch{\textcolor{red}{《决议》科学地评价了毛泽东的历史地位,充分论述了作为党的指导思想的伟大意义}}{\textcolor{red}{《决议》肯定了中共十一届三中全会以来逐步确立的适合中国情况的建设社会主义现代化
强国的道路,进一步指明了中国社会主义事业和党的工作继续前进的方向}}{\textcolor{red}{历史决议的通过,标志着党和国家在指导思想上拨乱反正的胜利完成}}{历史决议的通过,标志着党和国家在指导思想上拨乱反正的开始}
\begin{solution}D项错误,因为标志着拨乱反正开始的是党的十一届三中全会。
\end{solution}
\question 关于真理标准问题的大讨论的重要意义有( ~)
\par\fourch{\textcolor{red}{是继延安整风之后又一场马克思主义思想解放运动}}{\textcolor{red}{成为拨乱反正和改革开放的思想先导}}{\textcolor{red}{为党重新确立实事求是的思想路线,实现历史性的转折作了思想理论准备}}{是新中国成立以来党的历史上具有深远意义的伟大转折}
\begin{solution}从1978年5月开始的关于真理标准问题的大讨论,强调实践是检验真理的唯一标准。这场讨论,是继延安整风之后又一场马克思主义思想解放运动,成为拨乱反正和改革开放的思想先导,为党重新确立实事求是的思想路线,纠正长期以来的``左''倾错误,实现历史性的转折作了思想理论准备。十一届三中全会是新中国成立以来中国共产党的历史上具有深远意义的伟大转折。
\end{solution}
