\documentclass[printbox]{BHCexam}
\biaoti{~$2018$~年全国硕士研究生招生考试}
\fubiaoti{考研政治试卷}
\usepackage{ctex}
\usepackage{palatino}
\usepackage{siunitx}%输入度数符号需要的单位宏包
\usepackage{tikz}
\usepackage{color, soul} %用color, 和 soul 包
\setulcolor{blue} %设置下划线的颜色为蓝
\setstcolor{yellow} %设置overstriking颜色为黄
\sethlcolor{green} %设置高亮显示为绿
\usepackage{ulem}

\usetikzlibrary{shapes.geometric, arrows}
\tikzstyle{startstop} = [rectangle, rounded corners, minimum width = 1cm, minimum height=0.5cm,text centered, draw = black]
\tikzstyle{io} = [trapezium, trapezium left angle=70, trapezium right angle=110, minimum width=0.5cm, minimum height=0.5cm, text centered, draw=black]
\tikzstyle{process} = [rectangle, minimum width=2cm, minimum height=0.5cm, text centered, draw=black]
\tikzstyle{decision} = [diamond, aspect = 3, text centered, draw=black]
% 箭头形式
\tikzstyle{arrow} = [->,>=latex]
\begin{document}
\maketitle %先注释掉
% %\mininotice
\notice %先注释掉
\printanswers % 我要打印答案

% \AddEnumerateCounter{\chinese}{\chinese}{}
% \maketitle  
% 暂时不搞标题节省时间
\tableofcontents
\clearpage

%选择题
% \xuanze
% \section{}
\subsection{001哲学基本问题及其内容(对哲学的划分)}
\begin{questions}

\question 
恩格斯把费尔巴哈等旧唯物主义者称为半截子的唯物主义,并指出真正的唯物 主义者在理解现实世界(自然界和历史)时是“按照它本身在每一个不以先入为主的唯心主义 怪想来对待它的人面前所呈现的那样来理解……除此以外,唯物主义并没有别的意义。”这 里的“半截子”主要指的是(  )
\twoch{恩格斯1}{恩格斯2}{恩格斯3}{恩格斯4}

\begin{solution}
    略
\end{solution}




% \question 
% 恩格斯把费尔巴哈等旧唯物主义者称为半截子的唯物主义,并指出真正的唯物 主义者在理解现实世界(自然界和历史)时是“按照它本身在每一个不以先入为主的唯心主义 怪想来对待它的人面前所呈现的那样来理解……除此以外,唯物主义并没有别的意义。”这 里的“半截子”主要指的是(  )
% \twoch{恩格斯1}{恩格斯2}{\textcolor{red}{恩格斯3}}{恩格斯4}
\input{get_problems_detail_tex/problems_detail/computer/003_线性表的存储结构/950}

\end{questions}

\end{document}
