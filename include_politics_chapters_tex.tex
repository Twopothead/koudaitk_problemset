% This tex file is generated by get_politics_chapters/index.sh automatically.
\section{[马原]马克思主义基本原理概论}

\subsection{001-哲学基本问题及其内容(对哲学的划分)}
\question 恩格斯把费尔巴哈等旧唯物主义者称为半截子的唯物主义,并指出真正的唯物
主义者在理解现实世界(自然界和历史)时是``按照它本身在每一个不以先入为主的唯心主义
怪想来对待它的人面前所呈现的那样来理解\ldots{}\ldots{}除此以外,唯物主义并没有别的意义。''这
里的``半截子''主要指的是( ~)
\par\fourch{在坚持唯物论的同时,没有把唯物论和辩证法相结合}{在承认物质决定意识的同时,否认物质与意识的同一性}{\textcolor{red}{在自然观上是唯物主义的,历史观上则陷入唯心主义}}{把客观事物看作是既成的事实,但不承认事物的变化发展}
\begin{solution}题干是说旧唯物主义的缺陷,A是形而上学的唯物主义,B是不可知论,D是形而上学。
\end{solution}
\question 设想没有运动的物质的观点是( )
\par\twoch{主观唯心主义}{客观唯心主义}{庸俗唯物主义}{\textcolor{red}{形而上学唯物主义}}
\begin{solution}没有运动的物质即形而上学唯物主义。A主观唯心主义认为世界的``内心反省''的结果,B客观唯心主义是``绝对精神''的产物,C庸俗唯物主义是辩证唯物主义之前的唯物主义。
\end{solution}
\question 哲学是系统化、理论化的世界观和方法论。哲学研究的问题很多。其中被称为``哲学基本问题''的是(
)
\par\twoch{思维和存在何者为第一性问题}{\textcolor{red}{思维和存在的关系问题}}{思维和存在的同一性的问题}{思维能否正确反映存在的问题}
\begin{solution}本题考查的是对哲学基本问题的记忆。哲学的基本问题是思维和存在或者叫做物质和意识的关系问题,所以B为正确。A、C项是哲学基本问题的两个方面的具体内容,不是哲学基本问题本身。详而言之,B项与A、C项之间,是整体和局部的关系。D项是C项的另一种表述方式。
\end{solution}

\subsection{002-哲学的其他划分}
\question 辩证法和形而上学的区别表现在( )
\par\twoch{\textcolor{red}{事物是否存在联系}}{\textcolor{red}{事物是否发展变化}}{事物是否能够被认识}{\textcolor{red}{事物变化、发展的根源何在}}
\begin{solution}根据对世界状态的不同回答,形成了辩证法和形而上学两种不同的观点。辩证法坚持用联系的、发展的观点看世界,认为发展的根本原因在于事物的内部矛盾。而形而上学则主张用孤立的、静止的观点看问题,否认事物内部矛盾的存在和作用。是否承认对立统一(矛盾)学说是唯物辩证法和形而上学对立的实质。C项是可知论和不可知论的区别。
\end{solution}

\subsection{003-物质与运动}
\question 脱离物质的运动和脱离运动的物质都是不可想象的。因此,运动就是物质,物质就是运动。对这句话理解正确的是(
)
\par\twoch{正确理解了物质和运动的关系}{\textcolor{red}{是形而上学唯物主义的物质观}}{\textcolor{red}{混淆了物质的属性与物质本身}}{是正确的命题,体现了运动是物质的根本属性}
\begin{solution}本题考查的知识点:物质和运动的关系
分析题干:通过对题干及选项的分析,可知此题关键词是``物质''、``运动'',由此判断此题考点是物质与运动的关系。马克思主义哲学认为,作为哲学范畴的运动是指宇宙中发生的一切变化和过程,它是物质的根本属性和存在方式。物质和运动是不可分割的。一方面,物质是运动的物质,没有不运动的物质。运动是物质所固有的根本属性和一切物质形态的存在方式。设想有不运动的物质是形而上学唯物主义的特征。另一方面,运动是物质的运动,没有无物质的运动。物质是运动的承担者,是一切运动和发展的实在基础,运动的原因也在物质自身。设想有离开物质的运动是唯心主义的观点。
分析选项:题干充分反映了物质和运动不可分割的联系。但是因此而把物质和运动等同起来,认为物质就是运动,运动就是物质,则是不正确的。物质和运动不可分,但不是说两者没有区别,不能把物质的属性同物质本身等同起来。所以,符合题意要求的选项是BC。AD选项没有正确理解题意,故排除。
\end{solution}
\question 关于运动,正确的论断有( )
\par\twoch{\textcolor{red}{它是物质的存在方式}}{它是物质的本质规定性}{\textcolor{red}{它是物质的根本属性}}{它是物质的最高共性}
\begin{solution}运动是指宇宙中发生的一切变化和过程,是物质的存在方式和根本属性。BD项是关于物质之客观实在性的论断。
\end{solution}
\question 运动和物质不可分割。将二者分割开来所导致的错误有( )
\par\twoch{不可知论}{\textcolor{red}{形而上学}}{\textcolor{red}{唯心主义}}{相对主义}
\begin{solution}物质和运动是不可分割的,一方面,物质是运动着的物质,运动是物质的根本属性,脱离运动的物质是不存在的;设想不运动的物质,将导致形而上学。另一方面,运动是物质的运动,物质是运动的主体、实在基础和承担者;设想无物质的运动,将导致唯心主义。
\end{solution}
\question 长江的年龄有多大?这里说的长江``年龄'',是指从青藏高原奔流而下注入东海的``贯通东流''水系的形成年代。如果说上游的沉积物从青藏高原、四川盆地顺延而下能到达下游,这就表明长江贯通了,这就是物源示踪。我国科学家采用这一方法,研究长江中下游盆地沉积物的来源,从而判别长江上游的物质何时到达下游,间接指示了长江贯通东流的时限。他们经过10多年的研究,提出长江贯通东流的时间距今约2300多万年。这一研究成果从一个侧面显示出
\par\fourch{时间和空间是有限的,物质运动是永恒的}{\textcolor{red}{时间和空间是通过物质运动的变化表现出来的}}{时间和空间是标示物质运动的观念形式}{\textcolor{red}{时间和空间是物质运动的存在形式}}
\begin{solution}此题考查的是时间和空间的特点。时空是客观的是物质运动的存在形式,具有有限性和无限性,绝对性和相对性的特点。A选项表述错误,时空是有限的也是无限的,是和物质的运动紧密结合的。C本身表述错误,时空是客观的,不是观念形式。所以正确答案是BD。
\end{solution}

\subsection{004-时间与空间}
\question 以下选项中正确表达辩证唯物主义时空观内容的有,时间和空间( )
\par\twoch{是感性直观的先天形式}{\textcolor{red}{是物质运动的存在形式}}{\textcolor{red}{随物质运动速度的变化而变化}}{\textcolor{red}{是不可分割的}}
\begin{solution}本题考查的知识点:时间和空间
时空作为物质的存在形式,是有限性和无限性的统一。时间的无限性是指物质世界的存在和发展的持续性是无限的,无始无终。空间的无限性是指物质世界的广延性是无限的,无边无际。时空的有限性是指任何具体事物,其存在的时间、占有的空间都是有限的。时空既是绝对的,又是相对的,是绝对和相对的统一。时空的绝对性是指时空作为运动着的物质的存在形式,它的客观实在性是不变的、无条件的,因而是绝对的。时空的相对性是指时空特性的具体性、可变性。时空的具体特性随物质运动特性的变化而变化,人们关于时空的观念也是可变的、发展的。爱因斯坦的狭义相对论揭示了时空特性随物质运动速度的变化而变化,时空特性随物质形态的不同而不同。选项A把时空看成先天的形式,否定了时空的客观性,属于唯心主义的时空观。时间和空间是物质运动的存在形式,是不可分割的,并且随着物质运动速度的变化而变化。所以正确答案是选项BCD。
\end{solution}
\question 长江的年龄有多大?这里说的长江``年龄'',是指从青藏高原奔流而下注入东海的``贯通东流''水系的形成年代。如果说上游的沉积物从青藏高原、四川盆地顺延而下能到达下游,这就表明长江贯通了,这就是物源示踪。我国科学家采用这一方法,研究长江中下游盆地沉积物的来源,从而判别长江上游的物质何时到达下游,间接指示了长江贯通东流的时限。他们经过10多年的研究,提出长江贯通东流的时间距今约2300多万年。这一研究成果从一个侧面显示出
\par\fourch{时间和空间是有限的,物质运动是永恒的}{\textcolor{red}{时间和空间是通过物质运动的变化表现出来的}}{时间和空间是标示物质运动的观念形式}{\textcolor{red}{时间和空间是物质运动的存在形式}}
\begin{solution}此题考查的是时间和空间的特点。时空是客观的是物质运动的存在形式,具有有限性和无限性,绝对性和相对性的特点。A选项表述错误,时空是有限的也是无限的,是和物质的运动紧密结合的。C本身表述错误,时空是客观的,不是观念形式。所以正确答案是BD。
\end{solution}

\subsection{005-意识的起源与本质}
\question 关于意识的本质的正确观点是 ( ~)
\par\fourch{\textcolor{red}{意识是人脑的机能}}{意识是人脑的分泌物}{意识是人脑主观自生的东西}{\textcolor{red}{意识是物质在人脑中的主观映象}}
\begin{solution}本题考查的知识点:意识的本质
选项BC说法都是错误的。这两种说法是典型的庸俗唯物主义者的观点。庸俗唯物论是十九世纪三十年代,新黑格尔派解体以后,出现的一个唯物主义哲学派别。它认为宇宙间一切都是物质的,精神也是物质的。这在当时,在反对认为一切都是精神的唯心主义观点上,起过积极的作用。不过,它认为精神这个物质是物质的人脑分泌出来的。说人脑分泌精神就如同肝脏分泌胆汁一样。这就把物质存在的形式庸俗化、简单化、绝对化了。庸俗唯物主义不属于唯物主义哲学派别范畴,即唯物主义哲学派别中不包括庸俗唯物主义。费尔巴哈在批判黑格尔的唯心主义错误时,把其辩证法的核心发展的观点也抽掉了,恩格斯批评费尔巴哈是把婴儿和洗澡水一起泼掉了。所以,我们在批判庸俗唯物论时,也要保护它认为意识是物质的这个正确的根本之点。
马克思主义哲学辩证唯物主义认为意识是人脑的机能,意识是物质在人脑中的主观映象。正确答案是选项AD。
\end{solution}
\question 关于龙的形象,自古以来就有``角似鹿、头似驼、眼似兔、项似蛇、腹似蜃、鳞似鱼、爪似鹰、掌似虎、耳似牛''的说法。这表明
\par\fourch{\textcolor{red}{观念的东西是移入人脑并在人脑中改造过的物质的东西}}{一切观念都是现实的模仿}{虚幻的观念也是对事物本质的反映}{\textcolor{red}{任何观念都可以从现实世界中找到其物质“原型”}}
\begin{solution}这道不定项选择题考查考生对意识本质的理解和把握。辩证唯物主义认为,意识是对客观存在的反映,是对客观存在的主观映象。对于这些基本观点考生都能把握。该题所给定的人们关于对龙的形象的各种说法,正好体现了意识的本质,即``任何观念都可以从现实世界中找到其物质`原型'''(D项);从内容上看,无论是正确的意识,还是错误的、虚幻的意识,归根到底都是对客观存在的反映,都是来源于客观外界,都能从客观存在中找到原型。但是虚幻的观念仅仅反映事物的现象,而不能反映事物的本质。因此,C选项是错误的;也就是列宁所概括的``观念的东西不外是移入人脑并在人脑中改造过的物质的东西而已''(A项)。但并非一切观念都是对现实的模仿(B项),人的主观性还可以根据现实进行想象或对现实进行虚幻的反映。B答案也是错误的,AD二项则是该题的正确选项。该考点既是考生要掌握的最基本的考点,也是老师辅导时指出的重中之重,而且题中给定的选项,也是在课堂上老师要求考生一一必须记在资料中对应知识点上的内容,没有任何难点,必得的2分。
\end{solution}

\subsection{006-意识的特征}
\question 关于龙的形象,自古以来就有``角似鹿、头似驼、眼似兔、项似蛇、腹似蜃、鳞似鱼、爪似鹰、掌似虎、耳似牛''的说法。这表明
\par\fourch{\textcolor{red}{观念的东西是移入人脑并在人脑中改造过的物质的东西}}{一切观念都是现实的模仿}{虚幻的观念也是对事物本质的反映}{\textcolor{red}{任何观念都可以从现实世界中找到其物质“原型”}}
\begin{solution}这道不定项选择题考查考生对意识本质的理解和把握。辩证唯物主义认为,意识是对客观存在的反映,是对客观存在的主观映象。对于这些基本观点考生都能把握。该题所给定的人们关于对龙的形象的各种说法,正好体现了意识的本质,即``任何观念都可以从现实世界中找到其物质`原型'''(D项);从内容上看,无论是正确的意识,还是错误的、虚幻的意识,归根到底都是对客观存在的反映,都是来源于客观外界,都能从客观存在中找到原型。但是虚幻的观念仅仅反映事物的现象,而不能反映事物的本质。因此,C选项是错误的;也就是列宁所概括的``观念的东西不外是移入人脑并在人脑中改造过的物质的东西而已''(A项)。但并非一切观念都是对现实的模仿(B项),人的主观性还可以根据现实进行想象或对现实进行虚幻的反映。B答案也是错误的,AD二项则是该题的正确选项。该考点既是考生要掌握的最基本的考点,也是老师辅导时指出的重中之重,而且题中给定的选项,也是在课堂上老师要求考生一一必须记在资料中对应知识点上的内容,没有任何难点,必得的2分。
\end{solution}

\subsection{007-人类社会与自然界}
\question ``坐地日行八万里,巡天遥看一千河'',这一诗句包含的哲理是( ~)
\par\fourch{物质运动的客观性和时空的主观性的统一}{物质运动无限性和有限性的统一}{时空的无限性和有限性的统一}{\textcolor{red}{运动的绝对性和静止的相对性的统一}}
\begin{solution}A,B,C在题干中没有体现。
\end{solution}

\subsection{008-发展的实质}
\question 发展的过程性原理要求人们( )
\par\twoch{\textcolor{red}{坚持阶段论,反对超阶段论}}{\textcolor{red}{反对形而上学的“不变论”与“激变论”}}{\textcolor{red}{反对把知识、真理绝对化的观点}}{坚持世界是既成事物的集合体的观点}
\begin{solution}本题考查的知识点:发展的过程性
事物的发展是一个过程。一切事物只有经过一定的过程才能实现自身的发展。所谓过程是指一切事物都有其产生、发展和转化为其他事物的历史,都有它的过去、现在和未来。自然界、人类社会和思维领域中的一切现象都是作为一个过程而存在、作为过程而发展的。坚持事物发展是过程的思想,就要用历史的眼光看问题,把一切事物如实地看作是变化、发展的过程,既要了解它们的过去、观察它们的现在,又要预见它们的未来。因此,要坚持阶段论,反对超阶段论;反对形而上学的``不变论''与``激变论'';反对把知识、真理绝对化的观点。在今天,科学地认识建设中国特色社会主义的历史必然性、历史过程、历史阶段、发展规律和发展趋势,对我们坚定信念、积极投身社会主义现代化建设的伟大实践具有重要的现实意义。所以,本题的正确答案为选项ABC。
选项D错误。恩格斯指出:``世界不是既成事物的集合体,而是过程的集合体。''这是唯物辩证法的``一个伟大的基本思想''。事物发展的过程,从形式上看,是事物在时间上的持续性和空间上的广延性的交替;从内容上看,是事物在运动形式、形态、结构、功能和关系上的更新。
\end{solution}

\subsection{009-量变与质变}
\question 我国新民主主义革命时期,一块块革命根据地的建立,相对于全国处于半封建半殖民地社会而言,这是极有重大意义的事情。这句话表明(
)
\par\twoch{\textcolor{red}{量变中渗透着质变}}{质变中渗透着量变}{\textcolor{red}{量变过程中部分质变的存在}}{质变过程中量的扩展}
\begin{solution}本题考查的知识点:量变和质变的辩证关系
唯物辩证法认为,量变指的是事物数量的变化,体现了事物渐进过程中的连续性;质变是事物根本性质的变化,是事物由一种质态向另一种质态的飞跃,体现了事物渐进过程和连续性的中断。量变引起质变有两种情形:(1)事物在数量上的增减,如在大小、速度、程度和规模等方面的变化引起质变。比如,水滴石穿,冰冻三尺非一日之寒,勿以恶小而为之、勿以善小而不为,千里之堤、溃于蚁穴。(2)事物在总体数量不变的情况下,由于构成事物的成分在结构和排列次序上发生变化而起质变。由此可以判定题干这句话在总的过程上说的是量变,当时全国性质在总的方面还是半封建半殖民地社会。同时,量变和质变是相互渗透的,在总的量变过程中有阶段性和局部性的部分质变。而革命根据地的建立就是量变过程中的局部的部分质变,是总的量变中渗透着的质变。所以,本题的正确答案是选项AC。
选项BD错误。当时全国性质在总的方面还是半封建半殖民地社会,没有实现质变。
\end{solution}
\question 同是一块石头,一半做成了佛像,一半做成了台阶。一天,台阶不服气地问佛像:``我们本是一块石头,凭什么人们都踩着我,而去朝拜你呢?
''佛说:``因为你只挨了一刀,而我经历了千刀万割。''下列与``一刀是阶,千刀成佛''体现一致原理的名言是(
)
\par\fourch{\textcolor{red}{大厦之成,非一木之材也;大海之阔,非一流之归也}}{\textcolor{red}{不矜细行,终累大德}}{\textcolor{red}{凿井者,起于三寸之坎,以就万忉之深}}{祸兮福之所倚,福兮祸之所伏}
\begin{solution}【解析】做这种题目最主要的是搞清楚古语的含义。A选项古语典出冯梦龙《东周列国志》,原文:
``臣闻大厦之成,非一木之材也;大海之阔,非一流之归也。''意思是:高大的房屋建筑的建成,不是靠
一棵树的木材原料就能做到的;大海之所以辽阔,不是靠一条河流的水注人进来就能形成辽阔态势
的。``不矜细行,终累大德'',语出《尚书•旅獒》,意思是:不顾惜小节方面的修养,到头来会伤害大节,酿成终生的遗憾。``凿井者,起于三寸之坎,以就万仞之深''。意思是:凿井的人,从挖很深的土坑开始,慢慢才能形成极深的井。ABC选项中的名言都体现了说明万事起于忽微,量变引起质变,
符合题意。D选项体现福祸转化的矛盾对立统一,不符合题意。
\end{solution}
\question 关于唯物辩证法的``度''的概念的正确的论断是( )
\par\twoch{度就是事物变化的关节点}{\textcolor{red}{度是事物保持自己质的数量界限}}{\textcolor{red}{质变就是对度的规定性的突破}}{\textcolor{red}{度是区分事物量变和质变的标准}}
\begin{solution}本题考查对度的概念的全面理解。度是保持事物质的稳定性的量的规定性,因此度不是关节点,关节点是度的两端。
\end{solution}
\question 量变和质变是辩证统一关系,其同一性表现在( )
\par\twoch{量变必然引起质变}{\textcolor{red}{量变是质变的必要准备}}{\textcolor{red}{质变是量变的必然结果}}{\textcolor{red}{量变和质变相互渗透}}
\begin{solution}量变和质变是事物的联系和发展所采取的两种状态和形式,二者之间是辩证同一关系。其同一性表现在:第一,量变是质变的必要准备。任何事物的变化都有一个量变的积累过程,没有量变的积累,质变就不会发生。第二,质变是量变的必然结果。量变达到一定程度、突破了度,必然引起质变。第三,量变和质变是相互渗透的。一方面,在总的量变过程中有阶段性和局部性的部分质变;另一方面,在质变过程中也有旧质在量上的收缩和新质在量上的扩张。量变必须累积到一定程度,突破了``度''的界限,才能引起质变。所以A项错误。
\end{solution}
\question 有一则箴言:``在溪水和岩石的斗争中,胜利的总是溪水,不是因为力量,而是因为坚持。''``坚持就是胜利''的哲理在于
\par\twoch{必然性通过偶然性开辟道路}{肯定中包含着否定的因素}{\textcolor{red}{量变必然引起质变}}{有其因必有其果}
\begin{solution}坚持就是胜利,体现了事物量变发展到一定阶段必然会引起质变,达到事物根本性质的变化,所以,本题体现的是量变必然引起质变,正确答案是选项C。
\end{solution}

\subsection{010-对立统一规律}
\question 对立统一规律揭示了( )
\par\twoch{事物发展的方向和道路}{\textcolor{red}{事物发展的源泉和动力}}{事物发展的状态和形式}{\textcolor{red}{事物普遍联系的本质内容}}
\begin{solution}本题考查的知识点:对立统一规律是唯物辩证的实质与核心
唯物辩证法包括对立统一规律、质量互变规律和否定之否定规律的三大规律,其中对立统一规律是唯物辩证法体系的实质和核心,因为对立统一规律揭示了事物普遍联系的根本内容和永恒发展的内在动力,从根本上回答了事物为什么会发展的问题;对立统一规律是贯穿质量互变规律、否定之否定规律以及唯物辩证法基本范畴的中心线索,也是理解这些规律和范畴的``钥匙'';对立统一规律提供了人们认识世界和改造世界的根本方法------矛盾分析法,它是对事物辩证认识的实质;是否承认对立统一学说是唯物辩证法和形而上学对立的实质。所以,本题的正确答案是选项BD。
选项A错误。否定之否定规律揭示了事物发展的方向和道路。
选项C错误。质量互变规律揭示了事物发展的状态和形式。
\end{solution}
\question 唯物辩证法的实质和核心是( )
\par\twoch{\textcolor{red}{对立统一规律}}{普遍联系规律}{质量互变规律}{否定之否定规律}
\begin{solution}对立统一规居于唯物辩证法的实质和核心地位。
\end{solution}
\question 矛盾的基本属性是( )
\par\twoch{普遍性和特殊性}{\textcolor{red}{斗争性和同一性}}{绝对性和相对性}{变动性和稳定性}
\begin{solution}A是矛盾的精髓,C运动的基本属性,D表述不正确。
\end{solution}
\question 矛盾同一性在事物发展中的作用表现为( )
\par\fourch{\textcolor{red}{矛盾双方在相互依存中得到发展}}{\textcolor{red}{矛盾双方相互吸取有利于自身发展的因素}}{调和矛盾双方的对立}{\textcolor{red}{规定事物发展的基本趋势}}
\begin{solution}矛盾统一性在事物发展中的作用表现在三个方面。
\end{solution}
\question 构成矛盾的两种基本属性是( )
\par\twoch{普遍性和特殊性}{\textcolor{red}{同一性和斗争性}}{绝对性和相对性}{对抗性和非对抗性}
\begin{solution}本题是对对立统一规律和矛盾概念最基本含义的理解。矛盾就是对立统一,就是斗争性和同一性的关系,所以矛盾的两种基本属性就是同一性和斗争性。
\end{solution}

\subsection{011-肯定与否定}
\question 否定之否定规律揭示的是( )
\par\twoch{事物发展的动力和源泉}{事物发展的状态和形式}{\textcolor{red}{事物发展的趋势和道路}}{事物发展的过程和结果}
\begin{solution}否定之否定规律揭示了事物发展的趋势和道路。
\end{solution}
\question 否定之否定规律揭示了( )
\par\twoch{事物发展的动力和源泉}{事物发展变化的基本形式和状态}{\textcolor{red}{事物发展的方向和道路}}{事物发展的不同趋势或趋向}
\begin{solution}本题考查对辩证法各个规律特殊作用的理解。选项A是对立统一规律的作用,选项B是质量互变规律的作用,选项D是干扰项。C揭示的是否定之否定规律的作用,所以本题选C。
\end{solution}
\question 辩证唯物主义的辩证否定观认为,辩证的否定是( )
\par\twoch{\textcolor{red}{事物的自我否定}}{\textcolor{red}{联系的环节}}{\textcolor{red}{发展的环节}}{\textcolor{red}{扬弃}}
\begin{solution}辩证否定观的基本内容有:第一,否定是事物的自我否定,是事物内部矛盾运动的结果。第二,否定是发展的环节。它是旧事物向新事物的转变,是从旧质到新质的飞跃。只有经过否定,旧事物才能向新事物转变。第三,否定是联系的环节。新事物孕育产生于旧事物,新旧事物是通过否定环节联系起来的。第四,辩证否定的实质是``扬弃'',即新事物对旧事物既批判又继承,既克服其消极因素又保留其积极因素。
\end{solution}

\subsection{012-必然与偶然}
\question 马克思指出``如果`偶然性'不起任何作用的话,那么世界历史就会带有非常神秘的性质。这些偶然性本身纳入总的发展过程,\ldots{}\ldots{}其中包括一开始就站在运动最前面的那些人物的性格这样一种`偶然情况'。''上述论断中指出
\par\fourch{历史是偶然性向必然性转化的过程}{历史的发展纯粹是偶然的}{\textcolor{red}{历史的必然性通过偶然性表现出来}}{\textcolor{red}{历史人物的性格这种偶然因素对历史发展有一定影响}}
\begin{solution}这段论述指出,偶然性其中包括历史人物的性格这种偶然因素,对历史发展有作用,并且历史发展的必然性是通过偶然性表现出来的。但这段论述没有涉及偶然性向必然性的转化,因此A不能选。而B是错误的观点,也不能选。
\end{solution}
\question 偶然性与必然性的关系是( )
\par\twoch{\textcolor{red}{偶然性中包含着必然性}}{\textcolor{red}{必然性制约着偶然性}}{\textcolor{red}{偶然性表现必然性,是必然性的补充}}{\textcolor{red}{必然性通过偶然性为自己发展开辟道路}}
\begin{solution}必然性和偶然性是辩证统一的关系。同一性表现在:其一,二者相互包含。必然性存在于偶然性之中,并通过大量的偶然性表现出来,偶然性为必然性开辟道路;偶然性背后隐藏着必然性,偶然性受必然性支配,是必然性的表现形式和补充。其二,二者在一定条件下可以互相转化。
矛盾性表现在:其一,产生和形成的原因不同。必然性产生于事物内部的根本矛盾,偶然性产生于非根本矛盾和外部条件;其二,表现形式不同。必然性在事物发展过程中比较稳定、时空上比较确定,是同类事物普遍具有的发展趋势;而偶然性则是不稳定的、暂时的、不确定的,是事物发展中的个别表现。其三,它们在事物发展中的地位和作用不同。必然性在事物发展中居于支配地位,决定着事物发展的方向;偶然性居于从属地位,对发展的必然过程起促进或延缓作用,使发展的确定趋势带有一定的特点和偏差。
\end{solution}

\subsection{013-现象与本质}
\question 下列关于现象和本质的关系表述错误的是( )
\par\fourch{\textcolor{red}{有些本质可以自己直接表现出来}}{任何本质都通过现象表现出来}{任何现象都表现本质}{任何假象都表现本质}
\begin{solution}本题考查的知识点:现象和本质
现象和本质是马克思主义哲学唯物辩证法的一对基本范畴,现象是事物的外部联系和表面特征,本质是事物的内部联系和根本性质,任何现象都是本质的表现,人们总是通过对于事物现象的去粗取精,去伪存真,由此及彼,由表及里的认识过程,才不断深化对于事物本质的认识。事物的本质往往通过表象反映出来。每一个客观事物,都是多种规定的复杂统一体,这些复杂的规定通过丰富多彩的现象表现出来。人们接触一个事物,总是先认识到它丰富多彩的现象,由感觉、知觉到表象,取得关于这个事物整体的、感性的认识。通过分析事物的现象,可以帮助我们认识事物的本质。本质总是通过现象表现出来,没有不表现本质的现象,因此选项A认为有些本质可以自己直接表现出来是错误的。现象分真象和假象,
即使是假象也表现本质。真象是直接地、正面地表现事物的本质的现象。假象是歪曲地表现事物本质的一种特殊现象。所以选项BCD都是正确的表述。因此,本题正确答案是选项A。
\end{solution}

\subsection{014-可能与现实}
\question 把握可能性这个范畴,要注意区分以下哪几种情况( )
\par\twoch{\textcolor{red}{可能性和不可能性}}{简单的可能性和复杂的可能性}{\textcolor{red}{好的可能性和坏的可能性}}{\textcolor{red}{现实的可能性和抽象的可能性}}
\begin{solution}本题考查的知识点:可能性与现实性
首先把题干审清楚,问怎么更好的把握可能性这个范畴。把握可能性这个范畴,需要注意区分以下几种不同的情况:(1)可能性和不可能性;可能性是指事物发展过程中潜在的东西,是包含在事物中并预示事物发展前途的种种趋势。不可能指违背事物发展的规律性必然性,在现实中没有任何客观根据和条件,因而,是永远不能实现的东西。(2)现实的可能性和抽象的可能性;现实的可能性是指在现实中有充分的根据和必要条件,因而在一定阶段上可以实现的可能性。抽象的可能性又叫非现实的可能性,是指在现实中虽有一定根据,但根据尚未展开,必要条件尚不具备,因而只在以后的发展阶段上才可以实现的可能性。(3)好的可能性和坏的可能性。故选项ACD正是我们需要区分的。选项B简单的可能性和复杂的可能性,是干扰选项,故不选。因此,本题正确答案为选项ACD。
\end{solution}
\question 关于事物的可能性,下述正确的观点有( )
\par\fourch{\textcolor{red}{可能性有好的可能性和坏的可能性}}{\textcolor{red}{可能性和现实性在一定条件下可以相互转化}}{抽象的可能性是主观意识范畴的可能性}{可能性是事物中有内在根据,合乎必然性的存在}
\begin{solution}本题考查的知识点:可能性与现实性
可能性指包含在现实事物之中的、预示着事物发展前途的种种趋势,是潜在的尚未实现的东西。把握可能性这个范畴,要对可能性的各种情况加以区分:可能性和不可能性;现实的可能性和抽象的可能性;两种相反的可能性即好的可能性和坏的可能性;可能性的程度大小。可能性和现实性是统一的。第一,可能性和现实性相互依存。第二,可能性和现实性在一定条件下可以相互转化。事物的发展,总是在现实性中产生出可能性,而可能性又不断变为现实性的转化过程。所以,选项A可能性有好的可能性和坏的可能性正确。选项B可能性和现实性在一定条件下可以相互转化正确,是符合题意的正确选项。抽象的可能性是非现实的可能性,也是一种可能性,不是主观意识。现实性是包含内在根据的、合乎必然性的存在。故选项CD观点错误。因此,本题正确答案是选项AB。
\end{solution}

\subsection{015-规律及其客观性}
\question 党的十六大指出,要不断深化对共产党执政规律、社会主义建设规律、人类社会发展规律的认识。这``三大规律''
\par\twoch{\textcolor{red}{是有层次的}}{\textcolor{red}{都是人的活动的规律}}{是人们在改造社会的实践活动中创造的规律}{\textcolor{red}{存在着个别、特殊和一般的关系}}
\begin{solution}本题考点:中国特色的社会主义发展规律。
这三大规律都属于社会规律,也是客观存在的,并不是人们能够主观创造的,因而不选C。
\end{solution}

\subsection{016-自然规律和社会规律}
\question 社会规律是人们自己的``社会行动的规律'',这是因为( )
\par\twoch{\textcolor{red}{人是社会历史的主体}}{\textcolor{red}{人们自己创造自己的历史}}{历史发展方向是由人的思想和行动决定的}{\textcolor{red}{社会规律存在和实现于实践活动之中}}
\begin{solution}C观点错误。
\end{solution}
\question 自然规律和社会规律的不同点有( )
\par\fourch{自然规律是由客观物质力量决定的,社会规律是由人们的思想动机决定的}{自然规律没有阶级性,社会规律在阶级社会具有阶级性}{\textcolor{red}{自然规律可以重复出现,社会规律则是历史的,有不同的表现形式}}{\textcolor{red}{自然规律是作为盲目的无意识的力量起作用,社会规律要通过人的有意识的活动才能实现}}
\begin{solution}本题考查对社会历史规律的正确理解,着重是考查社会规律与自然规律的异同。A选项后半句是历史唯心主义命题,至于选项B,其实无论是自然规律还是社会规律,都是客观的,所谓客观的就是不依赖于人的意志而存在着,既然不依赖于人的意志,当然也就不依赖于阶级的意志了,所以自然规律和社会规律都没有阶级性,只不过在阶级社会先进的阶级容易认识并符合社会规律罢了,而且社会规律的实现有利于先进阶级的阶级利益,但不能由此得出社会规律在阶级社会具有阶级性的结论,这是两个概念。选项CD是符合历史唯物主义原理的关于社会规律的表述。
\end{solution}
\question 1989年,时任美国国务院顾问的弗朗西斯●福山抛出了所谓的``历史终结论'',认为西方实行的自由民主制度是``人类社会形态进步的终点''和
``人类最后一种的统治形式''。然而,20年来的历史告诉我们,终结的不是历史,而是西方的优越感。就在柏林墙倒塌20年后的2009年11月9日,BBC
公布了一份对27国民众的调查。结果半数以上的受访者不满资本主义制度,此次调查的主办方之一的``全球扫描''公司主席米勒对媒体表示,这说明随着1989年柏林墙的倒塌资本主义并没有取得看上去的压倒性胜利,这一点在这次金融危机中表现的尤其明显,``历史终结论''的破产说明
\par\fourch{社会规律和自然规律一样都是作为一种盲目的无意识力量起作用}{\textcolor{red}{人类历史的发展的曲折性不会改变历史发展的前进性}}{\textcolor{red}{一些国家社会发展的特殊形式不能否定历史发展的普遍规律}}{\textcolor{red}{人们对社会发展某个阶段的认识不能代替社会发展的整个过程}}
\begin{solution}``历史终结论''的破产说明,人类历史的发展的曲折性不会改变历史发展的前进性,一些国家社会发展的特殊形式不能否定历史发展的普遍规律,人们对社会发展某个阶段的认识不能代替社会发展的整个过程。但是,社会规律和自然规律是有相异之处的,社会规律是人有意识的能动活动,自然规律是盲目的无意识的力量起作用,所以,正确答案是选项BCD。
\end{solution}
\question 马克思指出:``一个社会即使探索到了本身运动的自然规律,\ldots{}\ldots{}它还是既不能跳过也不能用法令取消自然的发展阶段。但是它能缩短和减轻分娩的痛苦。''这表明
\par\fourch{\textcolor{red}{人类社会的发展是合规律性与合目的性的统一}}{社会发展过程与自然界演变过程一样都是自觉的}{\textcolor{red}{人的自觉选择在社会发展中具有重要作用}}{\textcolor{red}{人类总体历史进程是不可超越的}}
\begin{solution}本题考查社会规律及其特点的理解,属间接性试题。马克思在《资本论》第一卷序言中指出:``我的观点是把经济的社会形态的发展理解为一种自然史的过程。''``一个社会即使探索到了本身运动的自然规律,------本书的最终目的就是揭示现代社会的经济运动规律,------它还是既不能跳过也不能用法令取消自然的发展阶段。但是它能缩短和减轻分娩的痛苦。''马克思所说的社会运动的``自然规律'',是指社会也是一种自然历史过程即也是合规律的,因而,人类总体历史进程是客观的、不可超越。但人们可以探索到社会规律,并利用它来``缩短和减轻''新的社会形态产生的痛苦,这又说明人的自觉选择、社会发展的合目的性。所以,A、C、D项正确。B项错误,因为选项中包含的自然界演变过程是自觉的观点是错误的,正确的观点应是自发的。
\end{solution}

\subsection{017-客观规律性和主观能动性}
\question 在我国,很多农民为了高产。采用的普遍做法是加大种植密度。这就造成通风透光差,田间小气候不好,作物很容易感染病虫害。同样的作物品种.国外的种植密度较国内要低很多。以澳大利亚为例,一亩地通常定植番茄600\textasciitilde{}800棵(我国则达到3300棵),植株在良好的环境下生长,不易生病.单株产量并不比国内低,同时质量高。这说明
\par\fourch{自然规律是主观与客观的具体的历史的统一}{\textcolor{red}{实践活动中要注意适度原则}}{\textcolor{red}{要尊重自然规律的前提下发挥主观能动性}}{回复原始生态市解决人与自然矛盾的最终途径}
\begin{solution}【答案】BC
【简析】规律是客观的。客现性是规律的根本特点,它的存在不依赖于人的意识。相反,人的意识及其指导下的实践却要受到规律的支配。不管人们是否认识到,承认不承认.规律都客现存在着,并以一定的方式起作用。A错误。实现人与自然的和谐处并不要求恢复原始生态错误B、C正确
\end{solution}
\question 将客观规律与主观能动性统一起来的基础是( )
\par\twoch{矛盾}{\textcolor{red}{实践}}{物质}{客观}
\begin{solution}实践具有直接现实性,即能够将主观(主观能动性)与客观(客观规律)联系起来,实现二者的统一。
\end{solution}

\subsection{018-可知论的两个对立}
\question 马克思主义认识论与唯心主义认识论的区别在于是否承认( )
\par\twoch{世界的可知性}{\textcolor{red}{客观事物是认识的对象}}{认识发展的辩证过程}{\textcolor{red}{社会实践是认识的基础}}
\begin{solution}A是可知论和不可知论的区别,C辩证唯物主义能动反映论和旧唯物主义直观反映论的区别。
\end{solution}

\subsection{019-实践及其和认识的关系}
\question 宋代诗人陆游在一首诗中说:``纸上得来终觉浅,绝知此事要躬行。''这是在强调
\par\twoch{实践是认识发展的动力}{实践是认识的最终目的和归宿}{\textcolor{red}{实践是认识的来源}}{学习获得的间接经验并不重要}
\begin{solution}此题考查的知识点是实践是认识的来源。题干的意思是,从书本上得到的知识毕竟比较肤浅,要透彻地认识事物还必须亲自实践。对于认识来源于实践,不能作狭溢的理解。首先,认识来源于实践并不否认人的大脑和感官在生理素质上的差异对认识的影响。其次,认识来源于实践并不否认学习间接经验的必要性和重要性。所以本题选C。
\end{solution}
\question 辩证唯物主义认为实践是认识发展的动力,这是因为( )
\par\fourch{\textcolor{red}{实践的发展不断提出认识的新课题}}{实践是检验认识真理性的唯一标准}{\textcolor{red}{实践锻炼和提高了认识主体的认识能力}}{\textcolor{red}{实践为认识的发展提供了必要条件}}
\begin{solution}此题考查的知识点是实践是认识发展的动力。辩证唯物主义认为实践是认识发展的动力,这是因为:首先,实践的发展不断提出认识的新课题,推动着认识向前发展。其次,实践为认识发展提供必要条件。一方面,实践的发展不断揭示客观世界越来越多的特性,为解决认识上的新课题积累越来越丰富的经验材料;另一方面,实践又提供日益完备的物质手段,不断强化主体的认识能力。最后,实践锻炼和提高了主体的认识能力。恩格斯说:``人的智力是按照人如何学会改变自然界而发展的。''B是检验真理的标准,跟认识发展的动力无关。
\end{solution}
\question 1930年5月2日至6月5日,毛泽东在寻乌作了
20多天的社会调查,开了10多天的调查会,写下了《反对本本主义》和《寻乌调查》两篇论著。首次提出了``没有调查,没有发言权''的科学论
断,还提出了``到群众中作实际调查去''``中国革命斗争的胜利要靠中国同志了解中国情况''等思想路线,初步形成了毛泽东思想活的灵魂的三个基本点,即实事求是、群众路线和独立自主的思想,寻乌由此成为党的实事求是思想路线的发祥地。毛泽东在《反对本本主义》一文中强调调查就像`十月怀胎',解决问题就像`一朝分娩''',其中蕴含的哲理主要是(
)
\par\twoch{实践是认识的来源}{\textcolor{red}{量变是质变的必要准备,质变是量变的必然结果}}{可能与现实}{本质与现象}
\begin{solution}【解析】本题比较简单。``调查就像`十月怀胎'
'',体现了量变是质变的必要准备,``解决问题就
像`一朝分娩''',体现了质变是量变的必然结果。据此选B。题干设问部分,没有强调``实践是
认识的来源'',故排除A。
\end{solution}
\question 牛顿的一句名言:``假若我能比别人瞭望得略为远些,那是因为我站在巨人们的肩膀上''这句话所体现的意义是
\par\fourch{认识既来源于直接经验也来源于间接经验}{\textcolor{red}{主体可以通过读书或传授等方法来获取间接经验,这是发展人类认识的必要途径}}{理性认识不仅以感性认识为基础,而且要通过感性的认识来说明}{\textcolor{red}{只有把间接经验与直接经验结合起来,才能有比较完全的知识}}
\begin{solution}BD牛顿说:``假若我能比别人瞭望得略为远些,那是因为我站在巨人们的肩膀上''。是表明认学习间接经验的重要性。由于具体的主体的生命和能力是有限的,不可能事事亲身实践,而且理论或认识本身也具有历史的继承性,所以主体可以也应该通过读书或传授等方法来获取间接经验,这是发展人类认识的必要途径,但是间接经验归根到底也是来源于前人或他人的实践,所以认识来源于实践而且人们接受间接经验也要或多或少地以某种直接经验为基础,只有把间接经验与直接经验结合起来,才能有比较完全的知识。A项否认认识来源于实践,是错误的。C项材料未有体现。
\end{solution}
\question 马克思主义认识论认为,主体和客体的关系,是( ~)
\par\fourch{\textcolor{red}{改造和被改造的关系}}{\textcolor{red}{认识和被认识的关系}}{利用和被利用的关系}{\textcolor{red}{限定和超越的关系}}
\begin{solution}主体与客体的关系,不仅仅是认识和被认识的关系,而且也是改造和被改造的关系。在主体改造客体的实践过程中,主体反映了客体。在实践过程中,主体一方面受到客体的限定和制约,另一方面又能不断地发展自己的能力和需求,以自觉能动的活动不断打破客体的限定,超越现实客体,从而使主体和客体同时得到改造、发展与完善。这种限定和超越的关系,就是主体和客体相互作用的实质。
\end{solution}
\question 未来学家尼葛洛庞蒂说:``预测未来的最好办法就是把它创造出来。''从认识与实践的关系看,这句话对我们的启示是
\par\twoch{认识总是滞后于实践}{实践是认识的先导}{\textcolor{red}{实践高于认识}}{实践与认识是合一的}
\begin{solution}本题考点:认识与实践的关系。
实践具有直接现实性,这是实践高于认识的真正优点,理论不具有直接现实性,只是在思想上预测或者指导,只有实践才能够直接地作用于对象,有效地改造和创造物质对象。可见,这位未来学家真正的意思是要以预见为基础,通过实践真正把科学的预测和理论转化为现实。
\end{solution}

\subsection{020-认识发展过程的两次飞跃}
\question ``观察渗透理论''是美国科学哲学家汉森提出的著名命题。这个命题指出,我们的任何观察都不是纯粹客观的,具有不同知识背景的观察者观察同一事物,会得出不同的观察结果。``观察渗透理论''的命题指明的哲学原理是
\par\twoch{\textcolor{red}{感性中渗透着理性的因素}}{理性中渗透着感性的因素}{意识具有控制人的生理活动的作用}{实践是检验真理的唯一标准}
\begin{solution}【简析】在实际的认识过程中,感性认识和理性认识是互相交织、互相渗透的,
一方面,感性中渗透着理性的因素。人们在获得感性认识时,总是以原有的知识为背景,使用己有的概念和逻辑框架,在理性认识参与和指导下进行。同样接触客观事物,由于理论准备不同,感受就可能大不一样。所谓``只有理解了的东西才能更深刻地感觉它'',就是这个道理。现代科学哲学中所谓``观察渗透理论''的命题,也指明人总是以自己的历史文化为背景进行观察的。另一方面,理性中渗透着感性的因素。理性认识不仅以感性认识为基础,而且要通过感性的认识来说明。感性认识丰富的人与经验贫乏的人相比,对事物理解的深度是不一样的。
黑格尔说过,对于同一句格言,出自饱经风霜的老年人之口与出自缺乏阅历的青少年之口,其内涵是不同的。A正确。B错误。C、D本身说法正确但不符合题意,不选。
\end{solution}
\question 研究人员发现,把健康成年小鼠置于黑暗中一周后,它们辨别音高的能力也可显著提高。此前,科学界通常认为这种变化主要发生在未成年阶段,且需要更长的时间。这表明
\par\fourch{科学的价值在于造福社会}{\textcolor{red}{认识与实践的统一是具体的历史的}}{真理具有反复性和相对性}{\textcolor{red}{认识的真理性需要不断经受实践的检验}}
\begin{solution}【答案】BD
【解析】研究人员通过试验证明以前科学界通常认为的观点是错误的,说明认识与实践的统一是具体的历史的,同时说明认识的真理性需要不断经受实践的检验,据此BD选项。A选项观
点本身没有错误,但是与题意不符,题中没有涉及科学的价值在于造福社会,而是强调通过实践获得真理性的认识,故排除。C选项本身观点错误,认识具有反复性,真理不具有反复性,真理具有具体性,没有相对性,故排除。
\end{solution}
\question ``感觉到了的东西,我们不能立刻理解它。''``只有理解了的东西才能更深刻地感觉它。''这两句话表明感性认识和理性认识的关系是(  )
\par\fourch{\textcolor{red}{感性认识和理性认识在内容上有质的区别}}{\textcolor{red}{感性中渗透着理性的因素}}{\textcolor{red}{理性中渗透着感性的认识}}{\textcolor{red}{理性认识不仅以感性认识为基础,而且要用感性的认识来说明}}
\begin{solution}【解析】感性认识和理性认识是统一的认识过程中的两个阶段,它们既有区别,又有联系。感性认识和理性认识在内容和形式上都有质的区别。感性认识和理性认识又是有联系的。①感性认识和理性认识互相依存。理性认识依赖于感性认识,这是认识论的唯物论;感性认识有待于发展到理性认识,这是认识论的辩证法;②在实际的认识过程中,感性认识和理性认识又是互相交织、互相渗透的。感性中渗透着理性的因素;理性中渗透着感性的因素。
\end{solution}
\question 古希腊哲学家说:没有理性,眼睛是最坏的见证人。这一观点( )
\par\fourch{揭示了感性认识是整个认识过程的起点}{揭示了感性认识和理性认识的辩证统一}{认为理性认识可以脱离感性认识而存在,是错误的观点}{\textcolor{red}{强调理性认识的重要作用,完全否认了感性对认识的作用}}
\begin{solution}题干完全否认现象的作用即完全否认感性对认识的作用,只强调理性的重要作用。
\end{solution}
\question 列宁指出:``从生动的直观到抽象的思维,并从抽象的思维到实践,这就是认识真理、认识客观实在的辩证途径。''认识运动的辩证发展过程包括的两次飞跃是(
)
\par\twoch{\textcolor{red}{从感性认识到理性认识的飞跃}}{从理性认识到感性认识的飞跃}{从实践到理性认识的飞跃}{\textcolor{red}{从理性认识到实践的飞跃}}
\begin{solution}认识运动的两次飞跃是从感性认识到理性认识在到实践的飞跃。
\end{solution}
\question 理性认识向实践飞跃的重要意义在于,它使理性认识( )
\par\twoch{\textcolor{red}{变成改造世界的物质力量}}{\textcolor{red}{接受实践的检验并随实践的发展而发展}}{\textcolor{red}{发挥对实践的指导作用}}{起改变事物发展总趋势的作用}
\begin{solution}本题考查认识发展的过程,具体是考查理性认识到实践飞跃的意义。认识由理性认识向实践飞跃,这是认识过程中的第二次飞跃,是比由感性认识向理性认识飞跃更伟大的一次飞跃,而理性认识向实践的飞跃之所以更重要,就在于通过这次飞跃,使理性认识接受实践的检验,发挥其对实践的指导作用,使认识变成改造世界的物质力量,并使认识随实践的发展而不断发展,因此选项ABC均选。至于选项D不是理性认识向实践飞跃所能解决的,事物发展的总趋势是事物主客观综合作用的结果,有着更深刻的原因。
\end{solution}
\question 从感性认识向理性认识过渡,需要的条件有( )
\par\twoch{坚持理论与实践相结合的原则}{形成正确的实践观念}{\textcolor{red}{获取丰富和真实的感性材料}}{\textcolor{red}{运用辩证思维对感性材料进行加工}}
\begin{solution}从感性认识向理性认识的飞跃必须具备两个条件:一是勇于实践,深入调查,获取十分丰富和合乎实际的感性材料。这是正确实现由感性认识上升到理性认识的基础。二是必须经过理性思考的作用,将丰富的感性材料加工制作,去粗取精、去伪存真、由此及彼、由表及里,才能将感性认识上升为理性认识。而AB是实现从理性认识到实践的飞跃的条件。
\end{solution}

\subsection{021-认识过程的理性因素和非理性因素}
\question 心理学家将一条饥饿的鳄鱼和许多小鱼放在同一个大水箱里,中间用透明的玻璃隔开。最初,鳄鱼毫不犹豫地向小鱼发起进攻,但每次都碰壁了。多次进攻无果后,它放弃了努力。后来心理学家取走玻璃挡板,小鱼在鳄鱼身边游来游去,但鳄鱼始终无动于衷,最后饿死了。
``鳄鱼试验''进一步佐证了( )
\par\fourch{\textcolor{red}{动物心理没有能动性,不能透过现象把握事物的本质}}{动物心理没有适应性,不能根据环境的变化而变化}{\textcolor{red}{动物心理没有创造性,不能通过行动改变环境以满足生存的需要}}{动物心理没有主观性,不能创造现实世界所没有的幻想的世界}
\begin{solution}本题考查动物的本能活动与人的意识活动的本质区别。主观能动性是人与其他动物的本质区别,动物的心理活动是其本能的被动地适应环境的活动,动物没有意识,意识是人特有的活动,意识活动具有自觉能动性的特点,意识活动具有主动创造性。动物心理有适应性可以适应环境,故B选项错误。``鳄鱼试验''进一步说明,动物只是被动、机械地适应环境,写物心理没有自觉、主动的特点,故AC符合题意,D与题意无关。
\end{solution}
\question 感性认识和理性认识有着密不可分的辩证联系,表现在(  )
\par\fourch{\textcolor{red}{理性认识依赖于感性认识}}{\textcolor{red}{感性认识有待于发展和深化为理性认识}}{\textcolor{red}{感性认识和理性认识相互渗透、相互包含}}{\textcolor{red}{感性认识和理性认识在实践的基础上辩证统一}}
\begin{solution}【解析】感性认识和理性认识是统一的认识过程中的两个阶段,它们既有区别,又有联系。感性认识和理性认识的相互联系表现在:①感性认识和理性认识互相依存。理性认识依赖于感性认识,这是认识论的唯物论;感性认识有待于发展到理性认识,这是认识论的辩证法。②在实际的认识过程中,感性认识和理性认识又是互相交织、互相渗透的。一方面,感性中渗透着理性的因素;另一方面,理性中渗透着感性的因素。感性认识和理性认识是辩证统一的,两者统一的基础是实践。感性认识是在实践中产生的,由感性认识到理性认识的过渡,也是在实践的基础上实现的。
\end{solution}

\subsection{022-真理及其客观性(一元性)}
\question 实践是检验真理的标准。但实践标准有不确定性。其不确定性的含义包括( )
\par\twoch{\textcolor{red}{实践无法一下检验所有真理}}{\textcolor{red}{实践无法充分证明某一真理}}{实践能够检验一切真理}{\textcolor{red}{被实践检验过的真理需要继续接受检验}}
\begin{solution}实践标准的不确定性是指,由于一定历史阶段上的具体实践具有局限性,因此,其一,它无法一下子证明所有的真理;因为世界无限宽广。其二,它无法充分证明某一真理,因为事物复杂;其三,已被实践检验过的真理还要继续经受实践的检验,因为事物在不断发展变化。
\end{solution}
\question 逻辑证明在实践检验真理的过程中有重要作用,但不是检验真理的标准,这是因为(
)
\par\fourch{\textcolor{red}{逻辑证明只能证明前提与结论的一致性}}{\textcolor{red}{逻辑法则在实践中产生,且必须经过实践的检验}}{逻辑法则是不确定的}{逻辑证明是脱离实践的}
\begin{solution}本题考查逻辑证明在实践检验真理的过程中的重要作用。实践是检验真理的唯一标准,但并不排斥逻辑证明的作用。选项A、B反映了逻辑证明不能成为检验真理标准的原因。选项C、D本身就是错误的表达,是干扰项。
\end{solution}

\subsection{023-真理的绝对性与相对性}
\question 真理的绝对性是指( ~)
\par\fourch{\textcolor{red}{任何真理都包含不依赖于人的意识的客观内容,这是无条件的、绝对的}}{\textcolor{red}{人类完全可以认识无限发展的物质世界,这是无条件的,绝对的}}{\textcolor{red}{人的认识是无限发展的}}{真理在广度上是有待扩展的}
\begin{solution}本题考查考生对真理的绝对性含义的记忆和理解。真理的绝对性包括了三层含义,其基本内容,就是选项A、B、C所表达的意思。选项D是关于真理的相对性的含义,是明显的干扰项。
\end{solution}
\question 马克思主义认识论认为,认识的辩证过程是(  )
\par\fourch{\textcolor{red}{从相对真理到绝对真理的发展}}{从间接经验到直接经验的转化}{\textcolor{red}{实践——认识——实践的无限循环}}{从抽象到具体再到抽象的上升运动}
\begin{solution}【解析】认识的过程是从实践到认识再到实践的循环过程,也是感性认识上升到理性认识的过程,是相对真理向绝对真理的转变过程。
\end{solution}
\question ``真理和谬误在一定条件下能互相转化'',这说明(  )
\par\fourch{真理就是谬误,谬误就是真理,两者没有绝对的界限}{真理与谬误在同一范围内可以互相转化}{\textcolor{red}{真理超出自己适用的范围会转化为谬误}}{\textcolor{red}{谬误回归适合的范围会转化为真理}}
\begin{solution}【解析】真理与谬误既对立又统一:①真理与谬误是对立的。真理和谬误决定于认识的内容是否如实地反映了客观事物,因此真理和谬误是性质不同的两种认识。所以,就一定范围、一定客观对象来说,真理就是真理、谬误就是谬误,二者有本质的区别,不能混淆,也不可能转化。②真理与谬误又是相互联系的。真理是与谬误相比较而存在的,没有谬误也就无所谓真理。真理的发展也是通过与谬误的斗争来实现的。③真理与谬误在一定条件下相互转化。真理与谬误的区别和对立并不是绝对的,任何真理都是在一定范围内、一定条件下才能够成立。如果超出这个范围,失去特定条件,它就会变成谬误。
\end{solution}

\subsection{024-真理与谬误}
\question 列宁说:``只要再多走一小步------仿佛是向同一方向迈的一小步------真理便会变成错误。''这句话符合
\par\twoch{二元论}{实践论}{\textcolor{red}{两点论}}{重点论}
\begin{solution}本题是对马克思主义哲学的综合性考查。二元论是对世界本原的认识,即认为世界有物质和意识两个本原。实践论也是实践的观点,是马克思主义认识论的首要的和基本的观点。重点论是在矛盾不平衡发展原理中强调的主要矛盾和矛盾的主要方面。两点论是涉及所有问题的一分为二的两个方面,包括利与弊、优与劣、成功与失败、真理与谬误等。故选C。
\end{solution}

\subsection{025-唯物史观和唯心史观的对立}
\question 政治上层建筑是在意识形态指导下形成的,这种观点是( )
\par\twoch{历史唯心主义}{非马克思主义的}{\textcolor{red}{历史唯物主义的}}{非决定论的}
\begin{solution}历史唯物主义认为:上层建筑是指建立在一定经济基础上的意识形态以及相应的制度、组织和设施。上层建筑由意识形态和政治法律制度及设施、政治组织等两部分构成。在整个上层建筑中,政治上层建筑居主导地位,国家政权是它的核心。意识形态又称观念上层建筑,包括政治法律思想、道德、艺术、宗教、哲学等思想观点,政治法律制度及设施和政治组织又称政治上层建筑,包括:国家政治制度、立法司法制度和行政制度;国家政权机构、政党、军队、警察、法庭、监狱等政治组织形态和设施。观念上层建筑和政治上层建筑的关系是:政治上层建筑是在一定意识形态指导下建立起来的,是统治阶级意志的体现。C项正确。
A、B、D项错误,马克思创立的历史唯物主义认为政治上层建筑是在一定意识形态指导下建立起来的,是统治阶级意志的体现。由此可知,题干中的观点并非历史唯心主义和决定论的观点。
\end{solution}
\question ``时势造英雄''和``英雄造时势''( )
\par\twoch{\textcolor{red}{是两种根本对立的观点}}{这两种观点是互相补充的}{\textcolor{red}{前者是历史唯物主义,后者是历史唯心主义}}{\textcolor{red}{前者是科学历史观,后者是唯心史观}}
\begin{solution}时势造英雄是历史唯物主义观点,英雄造时势是英雄史观的观点。
\end{solution}
\question 马尔萨斯认为,资本主义的对外侵略、资本主义国家工人的悲惨生活,全然取决于``人口增长总要超过生活资料的增长''这一个``永恒的规律''的作用。马尔萨斯这一观点(
~)
\par\fourch{\textcolor{red}{把人口因素看成决定社会存在和发展的因素}}{\textcolor{red}{曲折地反映了资本主义制度下人口相对过剩的事实,却掩盖和歪曲了事实的本质}}{在重视人口作用的基础上,总结出人口发展的规律}{\textcolor{red}{注意到人口增长与生活资料增长之间的关系,具有积极意义}}
\begin{solution}C与题干无关。
\end{solution}

\subsection{026-社会存在和社会意识及其关系}
\question 中国工程院院士袁隆平曾结合自己的科研经历,语重心长地对年轻人说:``书本知识非常重要,电脑技术也很重要,但是书本电脑里面种不出水稻来,只有在田里才能种出水稻来。''这表明
\par\fourch{\textcolor{red}{实践是人类知识的基础和来源}}{实践水平的提高有赖于认识水平的提高}{理论对实践的指导作用没有正确与错误之分}{由实践到认识的第一次飞跃比认识到实践的第二次飞跃更重要}
\begin{solution}本题考查的是实践和认识的关系。实践是认识的基础,是认识的来源。袁隆平的话表明,只有通过实践才能获得认识,实践是认识的来源。A选项正确。实践水平的提高并不是依赖于认识水平的提高,B选项错误。正确的认识促进实践发展,错误的认识阻碍实践发展,C选项错误。两次飞跃中第二次飞跃更为重要,D选项错误。
\end{solution}
\question 社会存在决定社会意识,社会意识是社会存在的反映。社会意识具有相对独立性,即它在反映社会存在的同时,还有自己特有的发展形式和规律。社会意识相对独立性最突出的表现是
\par\fourch{社会意识与社会存在发展的不完全同步性}{社会意识内部各种形式之间的相互作用和影响}{社会意识各种形式各自有其历史继承性}{\textcolor{red}{社会意识对社会存在具有能动的反作用}}
\begin{solution}社会意识的相对独立性主要表现为:社会意识与社会存在发展的不平衡性;社会意识内部各种形式之间的相互影响及各自具有的历史继承性;社会意识对社会存在的能动的反作用。其中,社会意识对社会存在的能动的反作用是社会意识相对独立性的突出表现。D选项正确。
\end{solution}
\question 恩格斯指出``在历史上出现的一切社会关系和国家关系,一切宗教制度和法律制度,一切理论观念,只有理解了每一个与之相关的时代的物质生活条件,并从这些物质条件中被引申出来的时候,才能理解'',这表明(
)
\par\fourch{社会意识及其载体都是社会存在}{社会意识决定社会存在}{社会意识具有反作用}{\textcolor{red}{社会存在决定社会意识}}
\begin{solution}A,B观点错误,C与题干无关。
\end{solution}
\question 社会意识就是( )
\par\twoch{\textcolor{red}{社会生活的精神方面}}{经济上占统治地位的阶级的意识}{人民群众的意识}{政治上占统治地位的阶级的意识}
\begin{solution}本题考查社会意识的概念。答案可以直接看出是选项A。其他的选项都是本题的干扰项。
\end{solution}
\question 社会意识的相对独立性表现为( )
\par\fourch{\textcolor{red}{社会意识与社会存在发展的不平衡性、不同步性}}{\textcolor{red}{社会意识对社会存在的能动的反作用}}{\textcolor{red}{社会意识具有历史继承性}}{\textcolor{red}{各种社会意识形式之间的相互影响}}
\begin{solution}本题考查社会意识的相对独立性。这是基本的重要知识点。社会意识是对社会存在的反映,社会意识一旦产生之后,便具有自身的相对独立性,这种相对独立性表现为各个方面,如社会意识与社会存在的不平衡,社会意识自身的历史继承性,社会意识诸形式的相互影响,它对社会存在的能动反作用,所以本题四个选项都是社会意识相对独立性的体现。
\end{solution}
\question 关于社会意识对社会存在的能动的反作用,理解正确的有( ~)
\par\fourch{它具有历史继承性}{\textcolor{red}{先进的社会意识对社会发展起积极的促进作用}}{它与社会存在在发展上具有不平衡性}{\textcolor{red}{它的能动作用是通过指导人们的实践活动来实现的}}
\begin{solution}本题考查对社会意识的相对独立性和能动的反作用的理解掌握。社会意识与社会存在在发展上具有不平衡性,它内部各种形式之间具有相互影响和具有各自的历史继承性。但这些都只是社会意识的相对独立性的体现,而非社会意识的能动的反作用的体现。所以排除选项A、C。社会意识的能动反作用的相关体现,包括了选项B、D的内容,所以可选。
\end{solution}

\subsection{027-生产方式}
\question 在当今信息社会,现代科技进步和社会经济发展对信息资源、信息技术和信息产业的依赖性越来越大。在信息社会,智能化的综合网络将遍布社会的各个角落。``无论何事、无论何时、无论何地''人们都可以获得文字、声音、图像信息。这说明(
)
\par\fourch{信息社会改变了生产力的内容和性质,从而改变了生产关系的性质}{由虚拟网络建立的人与人之间的关系将成为新型的社会基本关系}{在信息社会中,网络信息关系将成为社会的基本关系}{\textcolor{red}{网络信息关系影响并推动社会发展,但并不能成为新型的社会基本关系}}
\begin{solution}网络信息关系影响和推动社会发展,但不能改变社会基本关系。
\end{solution}
\question 马克思指出:``我们首先应当确定一切人类生存的第一个前提,也就是一切历史的第一个前提,这个前提是:人们为了能够`创作历史',必须能够生活。\ldots{}\ldots{}因此第一个历史活动就是生产满足这些需要的资料,即生产物质生活本身''这句话表明(
)
\par\fourch{\textcolor{red}{物质生产是人类社会赖以存在的物质基础}}{\textcolor{red}{物质生产的发展决定着整个社会的性质和面貌}}{\textcolor{red}{物质生产方式的变革是社会历史变革的根本原因}}{物质生产对社会发展起着重要作用,但不是决定作用}
\begin{solution}D选项错误,就是决定作用。
\end{solution}
\question 物质生活的生产方式是社会历史发展的决定力量,其表现有( )
\par\twoch{\textcolor{red}{它制约着全部社会生活}}{\textcolor{red}{它决定社会形态的更替和发展}}{\textcolor{red}{它决定社会的面貌}}{\textcolor{red}{它决定社会的性质}}
\begin{solution}本题考查生产方式在社会存在和发展中的作用。生产方式是生产力和生产关系的统一,是社会历史发展的决定力量。物质资料生产方式是人类社会赖以存在和发展的基础。生产方式决定着社会的结构、性质和面貌,制约着人们的经济生活、政治生活和精神生活等全部社会生活。生产方式的变化发展决定整个社会历史的变化发展,决定社会形态的更替和发展。依据以上关于生产方式的论述,本题全选。
\end{solution}

\subsection{028-生产力和生产关系}
\question 人类社会存在和发展的物质基础是( )
\par\twoch{社会生产}{自然环境的物质基础}{从事生产活动的人}{\textcolor{red}{物质资料的生产方式}}
\begin{solution}本题考查社会的物质基础。物质生产活动及生产方式是人类社会赖以存在和发展的基础,这是马克思主义历史唯物主义的基本观点。本题选D。
\end{solution}

\subsection{029-经济基础与上层建筑}
\question 一定社会形态的经济基础是( )
\par\twoch{生产力}{该社会的各种生产关系}{政治制度和法律制度}{\textcolor{red}{与一定生产力发展阶段相适应的生产关系的总和}}
\begin{solution}一定社会形态的经济基础是与一定生产力发展阶段相适应的生产关系的总和。
\end{solution}
\question 2011年4月,耶鲁大学出版了《马克思为什么是对的》一书,书中列举了当前西方社会10个典型的歪曲马克思主义的观点。其中一种观点认为:马克思主义将世间万物都归结于经济因素,艺术、宗教,政治、法律、道德等都被简单地视为经济的反映,对人类历史错综复杂的本质视而不见,而试图建立一种非黑即白的单一历史观,上述观点是对马克思主义关于经济基础和上层建筑辩证关系思想的严重歪曲,其表现为
\par\fourch{\textcolor{red}{把社会历史发展多重因素的综合作用歪曲为单一因素决定论}}{\textcolor{red}{把上层建筑与经济基础的相互作用歪曲为机械的单向作用}}{\textcolor{red}{把经济作为社会的“基础”所具有的归根到底的决定作用歪曲为唯一决定作用}}{把意识形态对社会历史始终具有的积极能动作用歪曲为消极被动作用}
\begin{solution}本题考查经济基础与上层建筑的关系。马克思主义认为,生产力是人类社会发展的最终决定力量。但并不否认艺术、宗教、政治、法律、道德等其他因素的作用。社会发展是``历史合力''的结果。经济基础决定上层建筑,上层建筑反作用于经济基础。经济基础与上层建筑的作用是双向互动的。D错在意识形态对社会发展既有积极作用又有消极作用。反映历史发展规律的先进的意识形态推动社会历史发展。反之,则阻碍社会历史发展。因此,答案是ABC。
\end{solution}

\subsection{030-其他动力}
\question ``随着新生产力的获得\ldots{}\ldots{}人们也就会改变自己的一切社会关系,手推磨产生的是封建主的社会,蒸汽磨产生的是工业资本家的社会。''这段话表明科学技术是(
)
\par\twoch{\textcolor{red}{历史上起推动作用的革命力量}}{历史变革中的唯一决定性力量}{\textcolor{red}{推动生产方式变革的重要力量}}{一切社会变革中的自主性力量}
\begin{solution}科学技术是对社会发展起到重要作用的,但是B选项说法错误,历史变革的决定力量是生产力的发展,D选项错误,如果科学技术是自主性力量,那么根本的推动力就不是生产力了。B、D选项都是夸大了科学技术的作用。
\end{solution}
\question ``随着新生产力的获得\ldots{}\ldots{}人们也就会改变自己的一切社会关系,手推磨产生的是封建主义的社会,蒸汽磨产生的是工业资本家的社会。''这段话表明科学技术是
\par\twoch{\textcolor{red}{历史上起推动作用的革命力量}}{历史变革中的唯一决定性力量}{\textcolor{red}{推动生产方式变革的重要力量}}{一切社会变革中的自主性力量}
\begin{solution}本题考查科学技术的作用,先进的生产工具是人们先进的科学技术的物化,它运用于生产过程就会加快生产发展的速度,提高生产的效率。科技革命推动生产方式的变革,进而推动社会制度的变革。在本题中,手推磨、蒸汽磨是先进的生产工具,是科学技术的物化,由于它们的广泛应用,推动了生产方式的变革,分别产生了封建主义的社会和工业资本家的社会,所以,AC正确,B选项明显错误,科学技术不是历史变革中的唯一决定性力量。科学技术是间接的生产力,不是自主性力量,自主性力量只能是运用科技的劳动者,D选项错误。因此,本题正确答案为AC选项。
\end{solution}

\subsection{031-社会形态及其发展}
\question ``无论历史的结局如何,人们总是通过每一个人追求他自己的、自觉预期的目的来创造他们的历史,而这许多按不同方向活动的愿望及其对外部世界的各种各样作用的合力,就是历史。''这段话说明(
~)
\par\fourch{社会历史发展无规律可循}{无数个人意志创造社会历史}{\textcolor{red}{社会历史发展具有客观必然性}}{\textcolor{red}{人们自己创造自己的历史}}
\begin{solution}此题考查的知识点是现实的人及其活动与社会历史。这段话的中心词是``合力'',社会历史的发展方向和趋势是由``合力''规定的。``合力''本身对于每一个具体主体而言,是不以其意志为转移的客观力量,``合力''就是客观规律,所以社会历史发展具有客观必然性和规律性(C)。``合力''形成并实现于人的有目的性的实践活动之中,所以人才是社会历史活动的主体(D)。选项B属于社会意识决定社会存在的唯心史观,不符合题干原文的意思。所以正确答案是CD。
\end{solution}
\question 马克思、恩格斯在《共产党宣言》中预言:``资产阶级的灭亡和无产阶级的胜利是同样不可避免的。''这是马克思、恩格斯所揭示的人类社会发展的客观规律。对此,有人提出:``如果资本主义的灭亡是有保证了的、是必然的,为什么还要费那么大的气力去为它安排葬礼呢?''这种观点的错误在于(
~)
\par\fourch{\textcolor{red}{抹杀了社会规律实现的特点}}{\textcolor{red}{否认了革命在社会质变中的作用}}{否认了历史观的决定论原则}{否定了科学是推动历史前进的革命力量}
\begin{solution}C观点错误,D与题干无关。
\end{solution}
\question 正如达尔文发现了物种的起源与进化的规律,马克思发现了人类社会发展的规律,创立了唯物史观,科学地解决了社会存在和社会意识的关系。历史唯物主义的基本原理是
\par\fourch{\textcolor{red}{人类社会是一个自然的、历史发展的过程}}{\textcolor{red}{社会基本矛盾是社会发展的根本动力}}{\textcolor{red}{社会存在决定社会意识,社会意识反作用于社会存在}}{人民群众和英雄是历史的创造者}
\begin{solution}人类社会的发展是一个自然历史过程是唯物史观的一个基本原理,故选项A正确;唯物史观认为社会基本矛盾是社会发展的根本动力,故选项B正确;社会存在和社会意识的关系问题,是社会历史观的基本问题。正确认识这一问题是解决其他社会历史观的基础和前提,所以选项C正确;至于D选项有一半是错误的,人民群众是历史的创造者,这个是历史唯物主义观,但是英雄是历史的创造者是属于唯心主义英雄史观,故D选项错误。
\end{solution}

\subsection{032-社会形态更替的特点}
\question 生产资料所有制是( ~)
\par\fourch{衡量生产力水平的客观标志}{\textcolor{red}{生产关系结构中的决定因素}}{\textcolor{red}{区分社会经济制度的根本标志}}{衡量社会道德水平的客观标志}
\begin{solution}生产资料所有制关系是生产关系中的决定因素,它决定着生产中人与人的关系,决定着产品的分配,因而决定着生产关系的性质,从而也就决定着社会经济制度的性质,成为区分社会经济制度的根本标志。生产工具是衡量生产力水平和社会经济发展水平的客观标志;社会进步的程度是衡量社会道德水平的客观标志,因此AD不能选。
\end{solution}
\question 历史唯物主义认为,阶级斗争在阶级社会中的作用是( )
\par\twoch{\textcolor{red}{社会形态更替的杠杆}}{\textcolor{red}{迫使统治阶级作出某些让步的重要手段}}{社会发展的根本动力}{\textcolor{red}{社会发展的直接动力}}
\begin{solution}此题考查的知识点是社会发展的动力和一般规律问题。阶级斗争在阶级社会发展中的巨大作用,突出的表现在社会形态的更替过程中,起到杠杆的作用。所以A项是正确的。阶级斗争的作用,还表现在同一社会形态内部发展的量变过程中,不断地给统治阶级以这样那样的打击,使得统治阶级不得不对被统治阶级作出某些让步,所以B项也是正确的。社会发展的根本动力是社会基本矛盾,所以C项不符题意,是错误选项。阶级斗争是阶级社会发展的直接动力,所以D项是正确的。
\end{solution}
\question 哲学家孔德认为:``人们必须认识到,人类进步能够改变的只是其速度,而不会出现任何发展顺序的颠
倒或越过任何重要的阶段。''对他的这一看法,分析正确的有( )
\par\fourch{\textcolor{red}{他否认社会形态更替的统一性和多样性的辩证统一}}{\textcolor{red}{他否认社会形态更替的客观必然性与历史主体选择性的统一}}{他的这一观点具有辩证法的倾向}{\textcolor{red}{他没有认识到社会形态的更替是前进性和曲折性的统一}}
\begin{solution}此题考查的知识点是社会形态更替问题上的辩证法。孔德看到了历史发展有其规律,人类进步是可能的,但他把这种进步过程看成是严格按照固定顺序进行的,而没有看到历史主体应具有的能动性和创造性,及由此而来的社会形态更替中的统一性和多样性、前进性和曲折性、历史必然性与主体选择性的统一,具有典型的形而上学倾向。
\end{solution}
\question 列宁曾指出:``世界历史发展的一般规律,不仅丝毫不排斥个别发展阶段在发展的形式或顺序上表现出特殊性,反而是以此为前提的。''社会发展过程中的这种统一性基础上的多样性,充分显示出人类以及各个民族解决自身矛盾的能力及其创造性。社会发展的决定性和主体的选择性使社会发展过程呈现出统一性和多样性。主要表现有
\par\fourch{\textcolor{red}{从纵向看,表现为社会形态更替的统一性和多样性}}{\textcolor{red}{统一性是社会形态运动由低级到高级、由简单到复杂的过程,人类的总体历史过程表现为五种社会形态的依次更替}}{\textcolor{red}{多样性是指不同的民族可以超越一种或几种社会形态而跳跃式地向前发展}}{\textcolor{red}{从横向看,表现为同类社会形态既有共同的本质,又有各自的特点}}
\begin{solution}【解析】社会发展的决定性和主体的选择性使社会发展过程呈现出统一性和多样性。它表现在两个方面:从纵向看,表现为社会形态更替的统一性和多样性。统一性是社会形态运动由低级到高级、由简单到复杂的过程,人类的总体历史过程表现为五种社会形态的依次更替。多样性是指不同的民族可以超越一种或几种社会形态而跳跃式地向前发展。社会形态更替的多样性并不能否定人类总体历史过程。某些民族可以实现跨越,但其跨越的方向、跨越的限度是受总体历史进程制约的。也就是说,跨越的方向要同人类总体历史进程相一致;实际存在着的社会形态及其生产力规定着跨越的限度,现实存在的较先进的社会形态对跨越具有导向作用。从横向看,社会发展过程的统一性和多样性表现为同类社会形态既有共同的本质,又有各自的特点。在现实社会中,每一种社会形态在不同的民族那里都有自己的特殊表现形式。一般来说,不同民族总是自觉或不自觉地依据本民族的特点、历史传统以及国际环境,来选择、设计、创造自己的社会存在形式。中国越过资本主义社会形态,直接走向社会主义,既是历史的必然,又是中国人民的自觉选择。从当代中国实际和时代特征出发,建设中国特色社会主义,同样既是历史的必然,又是中国人民新的自觉选择和伟大创造。据此,本题选ABCD。
\end{solution}
\question 不同民族总是自觉或不自觉地依据本民族的特点、历史传统以及国际环境,来选择、设计、创造自己的社会存在形式。中国越过资本主义社会形态,直接走向社会主义,既是历史的必然,又是中国人民的自觉选择。从当代中国实际和时
代特征出发,建设中国特色社会主义,同样既是历史的必然,又是中国人民新的
自觉选择和伟大创造。这一自觉选择
\par\fourch{\textcolor{red}{要以社会发展的客观必然性造成了一定历史阶段社会发展的基本趋势为基础}}{\textcolor{red}{说明社会形态更替的规律也是人们自己的社会行动的规律}}{有利于避免社会前进过程中所出现的反复、停滞和倒退现象}{\textcolor{red}{归根到底是人民群众的选择性}}
\begin{solution}ABD不同民族总是自觉或不自觉地依据本民族的特点、历史传统以及国际环境,来选择、设计、创造自己的社会存在形式。中国越过资本主义社会形态,直接走向社会主义,既是历史的必然,又是中国人民的自觉选择。从当代中国实际和时代特征出发,建设中国特色社会主义,同样既是历史的必然,又是中国人民新的自觉选择和伟大创造。
这说明社会形态更替的规律也是人们自己的社会行动的规律。规律的客观性并不否定人们历史活动的能动性,并不排斥人们在遵循社会发展规律的基础上,对于某种社会形态的历史选择性。人们的历史选择性包含三层意思:第一,社会发展的客观必然性造成了一定历史阶段社会发展的基本趋势,为人们的历史选择提供了基础,范围和可能性空间。第二,社会形态更替的过程也是一个合目的性与合规律性相统一的过程。第三,人们的历史选择性,归根到底是人民群众的选择性。人们对于社会形态的历史选择,最终取决于人民群众的根本利益、根本意愿以及对社会发展规律的把握顺应程度。
社会发展过程的曲折性是指社会前进过程中所出现的反复、停滞和倒退现象。曲折前进是历史的普遍规律。列宁说:``设想世界历史会一帆风顺、按部就班地向前发展,不会有时出现大幅度的退,那是不辩证的,不科学的,在理论上是不正确的。''所以C不对。
\end{solution}
\question 马克思说``无论哪一种社会形态,在它所能容纳的全部生产力发挥出来之前,是绝不会灭亡的;而新的更高的生产关系,在它存在的物质条件在旧社会的胞胎力成熟以前,是绝不会出现的。''这段话说明(
~)
\par\fourch{\textcolor{red}{生产力的发展是促使社会形态更替的最终原因}}{\textcolor{red}{一种新的生产关系的产生需要客观的物质条件的成熟}}{\textcolor{red}{无论哪一种社会形态,当它还能促进生产力发展时,是不会灭亡的}}{\textcolor{red}{社会形态总是具体的、历史的}}
\begin{solution}两个绝不会体现生产力和生产关系的矛盾运动推动社会发展。
\end{solution}
\question 人们的历史选择性的含义有( )
\par\fourch{\textcolor{red}{人的历史选择以社会发展的客观必然性为基础}}{\textcolor{red}{人们的历史选择性归根结底是人民群众的选择性}}{\textcolor{red}{社会形态更替的过程是一个合目的性与合规律性相统一的过程;目的性必须符合规律性}}{人的历史选择性决定了社会发展的多样性}
\begin{solution}该题目是客观规律和人的主观能动性的关系在历史观中的体现。社会发展的多样性是由各国不同的国情决定的。所以D为错误。
\end{solution}
\question 1989年,时任美国国务院顾问的弗朗西斯●福山抛出了所谓的``历史终结论'',认为西方实行的自由民主制度是``人类社会形态进步的终点''和
``人类最后一种的统治形式''。然而,20年来的历史告诉我们,终结的不是历史,而是西方的优越感。就在柏林墙倒塌20年后的2009年11月9日,BBC
公布了一份对27国民众的调查。结果半数以上的受访者不满资本主义制度,此次调查的主办方之一的``全球扫描''公司主席米勒对媒体表示,这说明随着1989年柏林墙的倒塌资本主义并没有取得看上去的压倒性胜利,这一点在这次金融危机中表现的尤其明显,``历史终结论''的破产说明
\par\fourch{社会规律和自然规律一样都是作为一种盲目的无意识力量起作用}{\textcolor{red}{人类历史的发展的曲折性不会改变历史发展的前进性}}{\textcolor{red}{一些国家社会发展的特殊形式不能否定历史发展的普遍规律}}{\textcolor{red}{人们对社会发展某个阶段的认识不能代替社会发展的整个过程}}
\begin{solution}``历史终结论''的破产说明,人类历史的发展的曲折性不会改变历史发展的前进性,一些国家社会发展的特殊形式不能否定历史发展的普遍规律,人们对社会发展某个阶段的认识不能代替社会发展的整个过程。但是,社会规律和自然规律是有相异之处的,社会规律是人有意识的能动活动,自然规律是盲目的无意识的力量起作用,所以,正确答案是选项BCD。
\end{solution}

\subsection{033-国家的起源和实质}
\question 我国社会主义初级阶段实行以公有制为主体、多种所有制共同发展的基本经济制度,促进了生产力的发展,说明实行这种制度遵循了(
~)
\par\fourch{\textcolor{red}{生产力决定生产关系的原理}}{经济基础决定上层建筑的原理}{生产力具有自我增殖能力的原理}{社会经济制度决定生产力状况的原理}
\begin{solution}我国社会主义初级阶段实行以公有制为主体,多种所有制共同发展的基本经济制度,是对我国社会主义生产关系某些方面和环节的调整,属于社会主义经济体制改革,这是由我国现阶段物质生产力状况相对落后而又多层次的现实状况决定的,因此,实行这一基本经济制度促进了生产力的发展。这体现了生产力决定生产关系的原理,A为正确选项。这种制度没有体现经济基础决定上层建筑的原理和生产力自我增殖的原理,因此,备选项BC应排除。备选项D颠倒了生产力和生产关系的决定、被决定的关系,是一个错误判断。
\end{solution}

\subsection{034-人民群众与英雄人物的关系}
\question 恩格斯指出:``历史是这样创造的:最终的结果总是从许多单个的意志的相互冲突中产生出来的,而其中每一个意志,又是由于许多特殊的生活条件,才成为它所成为的那样。这样就有无数互相交错的力量,有无数个力的平行四边形,因此就产生出一个合力,即历史结果\ldots{}\ldots{}''这句话揭示了历史(
)
\par\twoch{\textcolor{red}{是由千百万人共同创造的}}{\textcolor{red}{历史发展的趋势不依单个人的意志为转移}}{历史人物起推动社会前进的积极作用}{\textcolor{red}{离开了每个人的作用就不可能有群众的作用}}
\begin{solution}题干强调历史是人们合力作用的结果。C与题干无关。
\end{solution}
\question 党的群众观点和群众路线的理论基础有( )
\par\twoch{\textcolor{red}{人民群众是历史创造者的原理}}{\textcolor{red}{人民群众在历史发展中起决定作用的原理}}{社会主义必然战胜资本主义的原理}{上层建筑一定要适应经济基础状况的规律}
\begin{solution}本题考查党的群众观点和群众路线的理论基础。本题知识点非常清晰。唯物史观关于人民群众是历史创造者的原理,是无产阶级政党的群众观点和群众路线的理论基础。所以,答案是A、B。而选项C、D是历史唯物主义的基本原理,但是与试题所问不符。
\end{solution}
\question 人在社会发展中的作用表现在( )
\par\twoch{\textcolor{red}{社会历史是人们自己创造的}}{人们创造历史的活动可以摆脱既定历史条件的制约}{\textcolor{red}{人们可以认识、利用和驾驭社会规律,因此对社会发展的具体途径进行历史选择,通过不同的具体道路实现社会发展的客观规律}}{\textcolor{red}{社会规律存在和实现于人们的自觉活动之中,人们的自觉活动只有在认识和遵循社会规律的基础上才能得到顺利、有效的发挥}}
\begin{solution}本题考查人在历史中的作用。具有一定的综合性,要求加强理解。选项A、C、D都是正确的表达。选项B认为创造历史的活动可以摆脱历史条件的制约,这是错误的。人类的历史是人们自己创造的,但这些活动都是在一定的历史条件下并受历史条件制约而进行的,人们的活动不可能没有条件,所以选项B不选。
\end{solution}
\question 人民群众是历史的创造者,这表现在他们是( )
\par\twoch{\textcolor{red}{社会物质财富的创造者}}{\textcolor{red}{社会精神财富的创造者}}{\textcolor{red}{实现自身利益的根本力量}}{\textcolor{red}{社会历史变革的决定力量}}
\begin{solution}本题考查人民群众创造历史的观点。这属于历史唯物主义部分的基本知识点。人民群众是历史的创造者,表现在他们是物质财富的创造者、是社会精神财富的创造者、是社会变革的决定性力量;既是先进生产力和文化的创造主体,也是实现自身利益的根本力量。
\end{solution}

\subsection{035-劳动二重性}
\question 在马克思之前,英国济学的代表人物亚当•斯密已经认识到了商品的二因素.提出了劳动创造价值的观点;大卫•李嘉图甚至已经认识到决定商品价值量的是社会必要劳动量,而不是生产商品实际耗费的劳动量。但是由于他们没有区分劳动二重性,所以不能回答什么劳动创造价值,在价值的源泉等重大理论问题的认识上出现了混乱和错误。马克思在继承英国古典政治经济劳动创造价值的理论的同时,创立了劳动二重性理论,第一次确定了什么样的劳动形成价值,为什么形成价值以及怎样形成价值,阐明了具体劳动和抽象劳动在商品价值形成中额不同作用,从而为揭示剩余价值的真正来源,创立剩余价值理论奠定了基础。一下关于源泉的说法正确的有
\par\fourch{\textcolor{red}{劳动力商品的使用价值是价值的源泉}}{\textcolor{red}{雇佣劳动者剩余劳动是剩余价值的源泉}}{劳动是财富的唯一源泉}{抽象劳动是具体劳动的源泉}
\begin{solution}【答案】AB
【简析】劳动力商品在使用价值上有一个很大的特点.就是它的使用价值是价值的源泉,它在消费过程中能够创造新价值.而且这个新的价值比劳动力本身的价值更大。A正确。雇佣劳动者的剩余劳动是剩余价值产生的唯一源泉,剩余价值既不是由全部资本创造的,也不是由不变资本创造的.而是由可变资本创造的,B正确,劳动是财离的源泉之一,但不是唯一源泉,C
错误。D属于无中生有的干扰项,不选。
\end{solution}
\question 人们往往将汉语中的``价''、``值''二字与金银财宝等联系起来,而这两字的偏旁却都是``人'',示意价值在``人''。马克思劳动价值论透过商品交换的物与物的关系,揭示了商品价值的科学内涵,其主要观点有(
)
\par\twoch{劳动是社会财富的唯一源泉}{具体劳动是商品价值的实体}{\textcolor{red}{价值是凝结在商品中的一般人类劳动}}{\textcolor{red}{价值在本质上体现了生产者之间的社会关系}}
\begin{solution}社会财富不等同于商品价值,文化科学也是社会财富,可是不是生产劳动而来;资源矿藏也是社会财富,但是也不是劳动产生,所以A选项错误。价值对应的是抽象劳动,所以B选项明显错误。CD选项都比较直接,不解释。
\end{solution}
\question 具体劳动是指人们在特定的具体形式下所进行的劳动。抽象劳动是指撇开各种具体形式的一般的、无差别的劳动。具体劳动和抽象劳动是生产商品的同一劳动过程的两个方面,二者的区别在于(
)
\par\fourch{\textcolor{red}{具体劳动是劳动的具体形式,抽象劳动是一般的人类劳动}}{\textcolor{red}{不同的具体劳动的质不同,抽象劳动没有质的差别}}{\textcolor{red}{具体劳动反映人与自然的关系,抽象劳动体现商品生产者之间的关系}}{具体劳动是使用价值的唯一源泉,抽象劳动是价值的唯一源泉}
\begin{solution}D选项错误,使用价值的来源不限于具体劳动,还有劳动对象本身。
\end{solution}
\question 人们往往将汉语中的``价''、``值''二字与金银财宝等联系起来,而这两字的偏旁却都是``人'',示意价值在``人''。马克思劳动价值论透过商品交换的物与物的关系,揭示了商品价值的科学内涵,其主要观点有
\par\twoch{劳动是社会财富的唯一源泉}{具体劳动是商品价值的实体}{\textcolor{red}{价值是凝结在商品中的一般人类劳动}}{\textcolor{red}{价值在本质上体现了生产者之间的社会关系}}
\begin{solution}本题考查的是商品价值的内涵。答案A是错误的,劳动是价值的唯一源泉,社会财富是属于使用价值的范畴,使用价值的源泉是原材料和人类劳动。答案B也是错误的,具体劳动是商品使用价值的实体,抽象劳动是商品价值的实体。答案C和D是正确的。价值是凝结在商品中无差别的人类劳动,是商品交换的基础,本质上体现了生产者之间的社会关系。
\end{solution}

\subsection{036-商品二因素}
\question 人类进入了21世纪,与马克思所处的时代相比,社会经济条件发生了很大的变化,因此,必须深化对马克思劳动价值论的认识。以下选项中关于认识的深化描述正确的是
\par\fourch{\textcolor{red}{生产性劳动包括大部分非物质生产领域的服务性劳动}}{\textcolor{red}{科技劳动和管理劳动等脑力劳动作为更高层次的复杂劳动创造的价值要大大高于简单劳动}}{科学技术本身也能创造价值}{\textcolor{red}{在实际经济生活中,价值分配首先是由生产资料所有制关系决定的}}
\begin{solution}人类进人了21世纪,与马克思所处的时代相比,社会经济条件发生了很大的变化,因此,必须深化对马克思劳动价值论的认识。第一,深化对创造价值的劳动的认识,对生产性劳动做出新的界定。马克思在《资本论》中重点考察的是物质生产部门,认为物质生产领域的劳动才是生产性劳动并创造价值。在当今时代,随着第三产业的发展,服务性劳动的地位和作用越来越重要,生产性劳动应当包括大部分非物质生产领域的服务性劳动。第二,深化对科技人员、经营管理人员在社会生产和价值创造中所起的作用的认识。马克思在《资本论》中关于``总体工人''的论述中,对脑力劳动(包括科技和管理劳动)给予了肯定,认为这些劳动也是创造价值的劳动,但他重点研究的是物质生产领域的体力劳动。在当今社会,科技劳动和管理劳动等脑力劳动,不仅作为一般劳动在价值创造中起着重要作用,而且作为更高层次的复杂劳动创造的价值要大大高于简单劳动。科学技术本身并不能创造价值。但科学技术在生产中的应用有利于劳动生产率的提高;科学技术为人所掌捤,从而提高劳动效率,创造出更多的使用价值和价值。第三,深化对价值创造与价值分配关系的认识。价值创造与价值分配既有联系又有区别,价值创造属于生产领域的问题,而价值分配是属于分配领域的问题。价值创造是价值分配的前提和基础,没有价值创造也就没有价值分配;但价值分配又不仅仅取决于价值创造,在实际经济生活中,价值分配首先是由生产资料所有制关系决定的,有什么样的生产资料所有制关系.就有什么样的分配关系。A、B、D是正确答案。
\end{solution}
\question 人们往往将汉语中的``价''、``值''二字与金银财宝等联系起来,而这两字的偏旁却都是``人'',示意价值在``人''。马克思劳动价值论透过商品交换的物与物的关系,揭示了商品价值的科学内涵,其主要观点有(
)
\par\twoch{劳动是社会财富的唯一源泉}{具体劳动是商品价值的实体}{\textcolor{red}{价值是凝结在商品中的一般人类劳动}}{\textcolor{red}{价值在本质上体现了生产者之间的社会关系}}
\begin{solution}社会财富不等同于商品价值,文化科学也是社会财富,可是不是生产劳动而来;资源矿藏也是社会财富,但是也不是劳动产生,所以A选项错误。价值对应的是抽象劳动,所以B选项明显错误。CD选项都比较直接,不解释。
\end{solution}
\question 使用价值是指商品能够满足人们某种需要的属性,即商品的有用性。马克思主义政治经济学在研究商品时,之所以考察商品的使用价值,因为使用价值是(
)
\par\twoch{构成财富的物质内容}{人类生存、发展的物质条件}{满足人们需要的物质实体}{\textcolor{red}{商品交换价值和价值的物质承担者}}
\begin{solution}政治经济学中考研商品的使用价值是要体现是商品交换价值和价值的物质承担者。
\end{solution}
\question 商品具有使用价值和价值两个因素,是使用价值和价值的矛盾统一体。解决商品内在的使用价值和价值矛盾的关键是(
)
\par\twoch{生产商品的劳动生产率的提高}{\textcolor{red}{商品交换的实现}}{能充当交换媒介的货币的出现}{价值规律发挥作用}
\begin{solution}使用价值和价值的实现是通过交换,商品生产者占有其价值,消费者占有其使用价值,二者矛盾得到解决。
\end{solution}
\question 马克思指出,``处于流动状态的人类劳动力或人类劳动形成价值,但本身并不是价值。它在凝固的状态中,在物化的形式上才形成价值。这就是说,要把人类抽象劳动,凝结在一定的物体里面,即一定的对象里,它才形成价值。''商品的价值是指凝结在商品中的一般人类劳动,说明它是(
)
\par\twoch{\textcolor{red}{商品的本质属性}}{\textcolor{red}{由抽象劳动形成的}}{\textcolor{red}{体现商品生产者相互交换劳动的关系}}{\textcolor{red}{交换价值的内容和基础}}
\begin{solution}商品的价值是有抽象劳动形成的,是商品的本质属性,体现商品生产者相互交换劳动的关系,是交换价值的内容和基础。
\end{solution}
\question 商品的二因素是对立统一的,这对矛盾的解决有赖于( )
\par\twoch{劳动生产率的不断提高}{商品物质实体的消亡}{\textcolor{red}{商品交换的实现}}{货币的出现并充当交换媒介}
\begin{solution}价值和使用价值是辩证统一的关系,这一矛盾只有通过商品的买卖才能得到解决。通过商品交换,商品生产者达到了它的目的------得到价值;而消费者也实现了他自己的目的------获得了使用价值。
\end{solution}
\question 商品内在的使用价值与价值的矛盾,其完备的外在表现是( )
\par\twoch{商品与商品之间的对立}{\textcolor{red}{商品与货币之间的对立}}{私人劳动与社会劳动之间的对立}{资本与雇佣劳动之间的对立}
\begin{solution}货币的产生,使一切商品的价值有了一个固定的、相对同一的表现形式,也使商品内在的价值和使用价值之间的矛盾,外在地表现为货币(代表价值)和商品(代表使用价值)的矛盾。在货币形式下,整个商品世界分为两极:一极是各式各样的商品,它们以使用价值的形式存在,在交换中,它们要求转化为价值;而另一极则是货币。
\end{solution}
\question 以下关于价值、交换价值、价格相互关系的论述,正确的有( )
\par\twoch{\textcolor{red}{价值是交换价值的基础}}{\textcolor{red}{价值是价格的基础}}{\textcolor{red}{价格是交换价值的一种形式}}{\textcolor{red}{价值要借助交换价值和价格表现出来}}
\begin{solution}价值凝结在商品里面,它要表现出来就必须借助交换价值这一形式。交换价值是一种使用价值与另一种使用价值相交换的量的关系和比例。价值和交换价值是内容与形式的关系。
\end{solution}
\question 人们往往将汉语中的``价''、``值''二字与金银财宝等联系起来,而这两字的偏旁却都是``人'',示意价值在``人''。马克思劳动价值论透过商品交换的物与物的关系,揭示了商品价值的科学内涵,其主要观点有
\par\twoch{劳动是社会财富的唯一源泉}{具体劳动是商品价值的实体}{\textcolor{red}{价值是凝结在商品中的一般人类劳动}}{\textcolor{red}{价值在本质上体现了生产者之间的社会关系}}
\begin{solution}本题考查的是商品价值的内涵。答案A是错误的,劳动是价值的唯一源泉,社会财富是属于使用价值的范畴,使用价值的源泉是原材料和人类劳动。答案B也是错误的,具体劳动是商品使用价值的实体,抽象劳动是商品价值的实体。答案C和D是正确的。价值是凝结在商品中无差别的人类劳动,是商品交换的基础,本质上体现了生产者之间的社会关系。
\end{solution}

\subsection{037-货币理论}
\question 1918年,马寅初在一次演讲时,有一位老农问他:``马教授,请问什么是经济学''马寅初笑着说:``我给这位朋友讲个故事吧,有个赶考的书生到旅店投宿,拿出十两银子,挑了该旅店标价十两银子的最好房间,店主立刻用它到隔壁的米店付了欠单,米店老板转身去屠夫处还了肉钱,屠夫马上去付清了赊帐的饲料款,饲料商赶紧到旅店还了房钱。就这样,十两银子又到了店主的手里。这时书生来说,房间不合适,要回银子就走了。你看,店主一文钱也没赚到,大家却把债务都还清了,所以,钱的流通越快越好,这就是经济学。''在这个故事中,货币所发挥的职能有
\par\twoch{\textcolor{red}{支付手段}}{\textcolor{red}{流通手段}}{\textcolor{red}{价值尺度}}{贮藏手段}
\begin{solution}本题正确答案为ABC。从题干中``挑了该旅店标价十两银子的最好房间''这句话中的``标价''两个字,可以判断,货币执行了价值尺度的功能,故C正确。支付手段,和流通手段两大职能的区别在于是否与实物交割同步,而本题``同步''与``不同步''两种情况都有出现,因此AB均正确。
\end{solution}

\subsection{038-价值与价格关系}
\question 价值规律对生产、流通和消费的调节作用的实现形式是( )
\par\fourch{市场价格围绕生产成本上下波动}{使用价值与价值的矛盾运动}{\textcolor{red}{价格根据市场供求变化围绕价值波动}}{社会必要劳动时间与个别劳动时间的矛盾运动}
\begin{solution}价值规律的作用形式是商品的价格根据市场供求状况的变化围绕价值上下波动。这种变动会影响各个市场主体的利益,进而影响其行为,最终给整个经济生活带来影响。
\end{solution}

\subsection{039-劳动力转化为商品和货币转化为资本}
\question 劳动力是指人的劳动能力,是人的体力和脑力的总和。以下关于劳动力表述正确的是
\par\fourch{\textcolor{red}{劳动力商品的使用价值是价值的源泉}}{劳动力自身的价值在消费过程中能够转移到新产品中去并形成剩余价值}{\textcolor{red}{在资本主义条件下资本家购买的是雇佣工人的劳动力而不是劳动}}{\textcolor{red}{劳动力成为商品是简单商品生产发展到资本主义商品生产新阶段的标志.}}
\begin{solution}【简析】劳动力商品在使用价值上有一个很大的特点。就是它的使用价值是价值的源泉,它在消费过程中能够创造新价值,而且这个新的价值比劳动力本身的价值更大。A正确。在资本主义条件下资本家购买的是雇佣工人的劳动力而不是劳动。C正确。可变资本是用来购买劳动力的那部分资本,在生产过程中不是被转移到新产品中去,而是由工人的劳动再生产出来。B错误。劳动力成为商品,标志着简单商品生产发展到资本主义商品生产的新阶段。在这一阶段,资本家与工人的关系,形式上是``自由''、``平等''的关系,而实质上是资本主义雇佣劳动
关系。D正确。
\end{solution}
\question 在不同的国家或同一国家的不同时期,劳动者所必需的生活资料的数量和构成是有区别的,劳动力价值的最低界限,是由生活上不可缺少的生活资料的价值决定的。一旦劳动力价值降低到这个界限以下,劳动力就只能在萎缩的状态下维持。这表明
\par\fourch{劳动力商品的价值,是维持劳动力所必需的生活必需品的价值决定的}{\textcolor{red}{劳动力价值的构成包含一个历史的和道德的因素}}{劳动力商品的价值是由维持劳动者本人生存所必需的生活资料的价值决定}{劳动力商品的价值是使用价值的源泉}
\begin{solution}【解析】劳动力商品的价值,是由生产、发展、维持和延续劳动力所必需的生活必需品的价值决定的,它包括三个部分:①维持劳动者本人生存所必需的生活资料的价值;②维持劳动者家属的生存所必需的生活资料的价值;③劳动者接受教育和训练所支出的费用。A项与C
项表述不完整。劳动力价值的构成包含一个历史的和道德的因素,在不同的国家或同一国家的不同历史时期,劳动者所必需的生活资料的数量和构成也是有区别的,所以,劳动力价值的最低界限,是由生活上不可缺少的生活资料的价值决定的。一旦劳动力价值降低到这个界限以下,劳动力就只能在萎缩的状态下维持。B项正确。劳动力商品在使用价值上有一个很大的特点,就是它的使用价值是价值的源泉,它在消费过程中能够创造新价值,而且这个新的价值比劳动力本身的价值更大。C项表述错误。
\end{solution}
\question 形成商品价值的劳动是( ~)
\par\twoch{\textcolor{red}{抽象劳动}}{具体劳动}{脑力劳动}{体力劳动}
\begin{solution}本题是考查价值和抽象劳动两个概念之间的关系。价值是凝结在商品里的一般的无差别的人类劳动;这种撇开了劳动的具体形式的无差别的一般人类劳动就叫做抽象劳动。
\end{solution}

\subsection{040-资本主义所有制和工资的本质}
\question 在资本主义社会里,资本家雇佣工人进行劳动并支付相应的工资。资本主义工资本质是
\par\fourch{工人所获得的资本家的预付资本}{\textcolor{red}{工人劳动力的价值或价格}}{工人所创造的剩余价值的一部分}{工人全部劳动的报酬}
\begin{solution}本题选项A说法错误,因此不选;选项C说法错误,剩余价值被资本家无偿占有,因此不选;选项D是资本主义工资的表象,不是本质,因此选项D错误,本题的正确答案是选项B。
\end{solution}

\subsection{041-资本理论}
\question 区分不变资本和可变资本的依据是( )
\par\twoch{资本各部分的流通形式不同}{\textcolor{red}{资本的不同部分在价值增殖过程中起不同的作用}}{资本各部分价值转移的方式不同}{资本各部分有不同的实物形式}
\begin{solution}按照资本在剩余价值生产(价值增殖)中所起的不同作用,生产资本可以划分为不变资本和可变资本。不变资本是以生产资料形式存在、其价值量在剩余价值生产过程中原封不动地转移到新产品中而没有发生变化的那部分资本;它是剩余价值生产的条件,用``C''表示。可变资本是以劳动力形式存在、其价值量在剩余价值生产过程中发生了变化、即发生了增殖的那部分资本;它是剩余价值产生的源泉,用``V''表示。C项是划分固定资本和流动资本的标准;AD是杜撰出来的干扰项目。
\end{solution}
\question 资本区分为不变资本和可变资本的意义在于( )
\par\twoch{\textcolor{red}{揭示了剩余价值的真正来源}}{揭示了货币转化为资本的关键}{\textcolor{red}{揭示了资本主义剥削的秘密}}{\textcolor{red}{为考察资本主义剥削程度提供了科学依据}}
\begin{solution}把资本区分为不变资本、可变资本的意义:①进一步明确剩余价值的来源,揭示资本主义剥削的实质;②为考查资本主义剥削的程度即剩余价值率提供科学依据;③为理解资本有机构成、平均利润理论奠定了基础。B项是劳动力成为商品的意义。
\end{solution}

\subsection{042-剩余价值}
\question 在瓜分剩余价值上,资本家之间存在竞争和矛盾,但在加强对工人阶级的剥削
以榨取更大量的剩余价值这一点上,资本家之间有着共同的阶级利益。不同部门
的资本家瓜分剩余价值的原则是
\par\twoch{等价交换}{资本周转速度的快慢}{\textcolor{red}{等量资本获得等量利润}}{资本积累规模的大小}
\begin{solution}资本主义生产是为了获得利润,因此,不同部门之间如果利润率不同,资本家之间就会展开激烈的竞争,使资本从利润率低的部门转向利润率高的部门,从而导致利润率趋于平均化,不同部门的资本家按照等量资本获得等量利润的原则来瓜分剩余价值。按照平均利润率来计算和获得的利润,叫做平均利润。随着利润转化为平均利润,商品价值就转化为生产价格,即商品的成本价格加平
均利润。在利润平均化规律作用下,产业资本家获得产业利润,商业资本获得商业利润,银行资本家获得银行利润,土地所有者获得地租。利润平均化规律,反映了在瓜分剩余价值上,资本家之间存在竞争和矛盾,但在加强对工人阶级的剥削以榨取更大量的剩余价值这一点上,资本家之间有着共同的阶级利益。
\end{solution}
\question 无论是绝对剩余价值还是相对剩余价值都是依靠( )
\par\twoch{延长工人工作时间而获得的}{提高劳动生产率而获得的}{\textcolor{red}{增加剩余劳动时间而获得的}}{降低工人的工资而获得的}
\begin{solution}在资本主义条件下,雇佣工人的劳动时间由必要劳动时间和剩余劳动时间构成。剩余劳动时间是生产剩余价值的时间。资本家为了获得更多的剩余价值,必须延长剩余劳动时间。其具体方法就是绝对剩余价值生产和相对剩余价值生产。绝对剩余价值生产是指在必要劳动时间不变的前提下,绝对延长总的工作日长度从而使剩余劳动时间延长来生产剩余价值的方法。用这种方法生产出来的剩余价值就是绝对剩余价值。相对剩余价值生产是在总的工作日长度不变的前提下,缩短必要劳动时间、从而相对延长剩余劳动时间来生产剩余价值的方法。所以,尽管所用的具体手段不同,但二者的共同点都是通过延长剩余劳动时间来获得更多的剩余价值。AB项是绝对剩余价值与相对剩余价值的差异;D项既不是绝对剩余价值也不是相对剩余价值产生的条件。
\end{solution}
\question 绝对剩余价值生产和相对剩余价值生产的共同点是( ~)
\par\fourch{\textcolor{red}{都延长了剩余劳动时间}}{\textcolor{red}{都体现着资本家对工人的剥削关系}}{\textcolor{red}{都增加了剩余价值量}}{\textcolor{red}{都提高了剩余价值率}}
\begin{solution}绝对剩余价值生产和相对剩余价值生产都是剩余价值生产,所以都体现着资本家对工人的剥削关系;二者都需要延长剩余劳动时间,区别是延长的方式不同:是通过加班还是通过提高劳动生产率;他们都导致剩余价值数量的增加,进而提高剩余价值率------因为剩余价值率是剩余价值和可变资本的比率。
\end{solution}
\question 在生产自动化条件下,资本家能够获得更多的剩余价值,其原因在于( )
\par\twoch{\textcolor{red}{生产工具更加先进}}{\textcolor{red}{工人的劳动更加复杂}}{\textcolor{red}{资本家获得了超额剩余价值}}{机器人、自动化生产线能够多创造价值}
\begin{solution}资本主义国家的生产自动化是人类社会科学技术进步的结晶,它的普遍采用会大幅度地提高劳动生产率,使资本家阶级获得比过去更多的剩余价值。资本主义条件下的生产自动化是资本家获取超额剩余价值的手段,而雇佣工人的剩余劳动仍然是这种剩余价值的唯一源泉。
\end{solution}

\subsection{043-扩大再生产与资本积累}
\question 社会生产是连续不断进行的,这种连续不断重复的生产就是再生产。每次经济危机发生期间,总有许多企业或因为产品积压、或因订单缺乏等致使无法继续进行在生产而被迫倒闭。那些因产品积压而倒闭的企业主要是由于无法实现其生产过程中的
\par\twoch{劳动补偿}{\textcolor{red}{价值补偿}}{实物补偿}{增值补偿}
\begin{solution}此题考查的是社会再生产的核心问题及实现条件。社会再生产顺利进行,要求生产中所耗费的资本在价值上得到补偿。材料中所说产品积压,其实质就是产品无法顺利卖出,之前所付出的资本无法顺利得到价值补偿。因此,本题的正确答案是B。
\end{solution}
\question 任何社会再生产的内容都是( )
\par\fourch{劳动过程和价值增殖过程的统一}{劳动过程和价值形成过程的统一}{简单再生产和扩大再生产的统一}{\textcolor{red}{物质资料再生产和生产关系再生产的统一}}
\begin{solution}再生产是指不断重复的生产过程。任何社会的再生产就其内容而言,都是产品(物质资料)的再生产和生产关系再生产的统一。资本主义再生产是物质资料再生产和资本主义生产关系再生产的统一。A项是资本主义劳动过程的两重性;B项是商品生产的两重性;C项是再生产的类型。
\end{solution}

\subsection{044-资本的有机构成}
\question 某钢铁厂因铁矿石价格上涨,增加了该厂的预付资本数量,这使得该厂的资本构成产生了变化,所变化的资本构成是(
)
\par\twoch{资本技术构成}{\textcolor{red}{资本价值构成}}{资本物质构成}{资本有机构成}
\begin{solution}该工厂由于原料价格上涨,造成不变资本投入增加,那么价值构成=不变资本:可变资本。自然价值构成就发生了变化,由于该厂的技术水平没有变化,所以,技术构成和有机构成都不变。C选项的说法没有提及,不考虑。
\end{solution}
\question 某钢铁厂因铁矿石价格上涨,增加了该厂的预付资本数量,这使得该厂的资本构成发生了变化,所变化的资本构成是
\par\twoch{资本技术构成}{\textcolor{red}{资本价值构成}}{资本物质构成}{资本有机构成}
\begin{solution}本题考查的知识点是马克思主义政治经济学第三章中的资本构成。资本的构成可以从物质形式和价值形式两个方面考察。从物质形式上看,由生产技术水平决定的生产资料和劳动力之间的量的比例,叫做资本的技术构成。另一方面,从价值形式上看,不变资本和可变资本之间的比例,叫做资本的价值构成。马克思把由资本技术构成决定并且反映资本技术构成变化的资本的价值构成,叫做资本的有机构成,通常用c:v表示。题干中只涉及到预付资本的价值总量由于价格的涨落发生了变化,并未涉及到资本的技术构成发生变化,所以也不会影响到有机构成。因此,本题正确答案是B选项。
\end{solution}

\subsection{045-资本主义社会的基本矛盾与经济危机}
\question 资本主义经济危机
\par\fourch{其实质是社会生产能力超越了劳动人民的实际需求的相对过剩}{其可能性是由货币作为贮藏手段和支付手段引起的}{\textcolor{red}{是资本主义基本矛盾在资本主义范围内暂时的、强制性的解决形式}}{从根本上解决资本主义社会的内在矛盾,来维持资本主义制度的存在}
\begin{solution}【解析】生产资料资本主义私人占有和生产社会化之间的矛盾,是资本主义的基本矛盾。在资本主义条件下,随着科学技术的进步和社会生产力的不断发展,资本主义生产不断社会化。但是,在资本家私人占有生产资料和剥削雇佣劳动的生产关系中,社会化的生产力却变成资本的生产力,变成资本高效能地榨取剩余劳动、生产剩余价值、实现价值増殖的能力。这就形成了资本主义所特有的生产社会化和资本主义私人占有形式之间的矛盾。资本主义基本矛盾不断尖锐化就不可避免引发经济危机。
生产相对过剩是资本主义经济危机的本质特征。相对过剩是指相对于劳动人民有支付能力的需求来说社会生产的商品显得过剩,而不是与劳动人民的实际需求相比的绝对过剩。故A项不对。
经济危机的可能性是由货币作为支付手段和流通手段引起的。但是这仅仅是危机的形式上的可能性。故B项不对。资本主义经济危机爆发的根本原因是资本主义的基本矛盾,这种基本矛盾具体表现为两个方面:第一,表现为生产无限扩大的趋势与劳动人民有支付能力的需求相对缩小的矛盾。第二,表现为个别企业内部生产的有组织性和整个社会生产的无政府状态之间的矛盾。
资本主义经济危机具有周期性,这是由资本主义基本矛盾运动的阶段性决定的。当资本主义基本矛盾达到尖锐化程度时,社会生产结构严重失调,引发经济危机。而经济危机使企业倒闭、生产下降,供求矛盾得到缓解,随着资本主义经济的恢复和高涨,资本主义基本矛盾又重新激化,这必然再一次导致经济危机的爆发。
因此经济危机既是资本主义基本矛盾尖锐化的产物,同时又是这一矛盾在资本主义范围内暂时的、强制性的解决形式。危机的爆发缓解了生产和消费之间的对立,通过破坏生产力这种强制性方式实现了生产与消费之间的暂时平衡,使资本主义再生产得以继续。但是每一次经济危机都不可能从根本上解决资本主义社会的内在矛盾,反而使资本主义矛盾在更深层次和更大范围上发展。只要资本主义制度存在,经济危机就不可避免。故D项不对,应选C
项。
\end{solution}
\question 以下关于资本主义经济危机的论断哪些是正确的( )
\par\fourch{\textcolor{red}{经济危机的可能性在资本主义制度建立之前就存在}}{\textcolor{red}{资本主义经济危机是资本主义基本矛盾尖锐化的结果}}{\textcolor{red}{经济危机能够缓解生产和消费的关系}}{\textcolor{red}{只要资本主义制度存在,经济危机就不可避免}}
\begin{solution}经济危机的抽象的一般的可能性,存在于商品经济当中。首先,货币作为流通手段和支付手段的功能,就造成了买卖脱节的可能性。其次,伴随商品交换的发展所出现的赊购赊销的方式,则进一步加大了商品买卖脱节的风险。但是,所有这些仅仅是危机的形式上的可能性,或者说它们只能造成局部性的危机,无法形成为社会性的危机。只有资本主义基本矛盾,才使经济危机现实化和社会化。所以,资本主义经济危机的根源在于资本主义基本矛盾。这一基本矛盾来源于资本主义生产关系本身,来源于资本主义经济制度。当资本主义基本矛盾达到尖锐化程度时,社会生产结构严重失调,引发了经济危机。而经济危机的爆发,使企业纷纷倒闭,生产大大下降,从而使供求矛盾得到缓解,逐步渡过经济危机。但是,经济危机只能暂时缓解而不能根除资本主义基本矛盾。这样,随着资本主义经济的恢复和高涨,资本主义基本矛盾又重新激化,必然导致再一次经济危机的爆发。只要存在资本主义制度,经济危机就是不可避免的。
\end{solution}
\question ``信用制度加速了生产力的物质上的发展和世界市场的形成;使这二者作为新生产形式的物质基础发展到一定的高度,是资本主义生产方式的历史使命。同时信用加速了这种矛盾的暴力的爆发,即危机,因而加强了旧生产方式的解体的各种因素。''马克思的这一论述表明,资本主义信用制度
\par\fourch{已成为资本主义经济危机爆发的深层原因}{\textcolor{red}{促进了建立社会主义生产方式的物质基础的形成}}{\textcolor{red}{加速了资本主义生产方式内部矛盾发展和解体要素的形成}}{\textcolor{red}{推动商品经济的发展,又加深了商品经济运行中的矛盾}}
\begin{solution}信用制度加速了生产力的物质上的发展和世界市场的形成,由此推动了商品经济的发展并且为社会主义生产方式的建立奠定了物质基础,同时信用制度加速了经济危机的爆发,加强了旧生产方式解体的各种因素,但信用制度不是资本主义经济危机爆发的深层原因。因此,本题正确答案是BCD选项。
\end{solution}

\subsection{046-资本的循环与周转}
\question 马克思在分析剩余价值的生产、积累、流通以及分配过程,揭示资本主义经济特殊规律
的同时,也揭示了商品经济和社会化生产的一般规律。如果撇开资本主义制度因素,这些规律对发展社会主义市场经济也具有指导意义,具体包括
\par\twoch{\textcolor{red}{资本循环周转规律}}{\textcolor{red}{社会再生产规律}}{\textcolor{red}{资本积累规律}}{经济危机周期性规律}
\begin{solution}【解析】马克思在分析剩余价值的生产、积累、流通以及分配过程,揭示资本主义经济特殊规律的同时,也揭示了商品经济和社会化生产的一般规律。例如资本循环周转规律、社会再生产规律、积累规律等。如果撇开资本主义制度因素,这些规律对发展社会主义市场经济也具有指导意义。
资本主义经济危机具有周期性,这是由资本主义基本矛盾运动的阶段性决定的。当资本主义基本矛盾达到尖锐化程度时,社会生产结构严重失调,引发经济危机。而经济危机使企业倒闭、生产下降,供求矛盾得到缓解,随着资本主义经济的恢复和高涨,资本主义基本矛盾又重新激化,这必然再一次导致经济危机的爆发。危机的爆发缓解了生产和消费之间的对立,通过破坏生产力这种强制性方式实现了生产与消费之间的暂时平衡,使资本主义再生产得以继续。但是每一次经济危机都不可能从根本上解决资本主义社会的内在矛盾,反而使资本主义矛盾在更深层次和更大范围上发展。只要资本主义制度存在,经济危机就不可避免。这一规律并不适用于社会主义经济运行,D项为干扰项。
\end{solution}
\question 产业资本循环不断进行的两个基本条件是( )
\par\fourch{三种循环形式在时间上依次继起}{\textcolor{red}{三种职能形式在时间上依次继起}}{三种表现形式在空间上同时并存}{\textcolor{red}{三种职能形式在空间上同时并存}}
\begin{solution}产业资本循环顺利进行的必要条件有两个:①使产业资本的三种职能形式在空间上并存,即按照实际需要,将产业资本划分为三部分,分别分布在三个职能形式上;②使产业资本的三种职能形式在时间上继起,即每一种职能形式依次向下一种职能形式转化。资本主义制度使得资本循环顺利进行所需的上述条件不能稳定地保持,所以资本循环只能时断时续地进行。本题考查的是记忆的精确,AC都是杜撰出来的干扰项,但是非常逼真。
\end{solution}
\question 产业资本循环所经历的阶段有( )
\par\twoch{\textcolor{red}{购买阶段}}{流通阶段}{\textcolor{red}{生产阶段}}{\textcolor{red}{销售阶段}}
\begin{solution}产业资本循环经过的三个阶段是:购买阶段;生产阶段;销售阶段。购买阶段和销售阶段合称流通阶段。
\end{solution}

\subsection{047-平均利润与生产价格}
\question 商业资本作为一种独立的职能资本,也获得平均利润,其直接原因是( )
\par\fourch{\textcolor{red}{商业部门和产业部门之间的竞争和资本转移}}{产业资本家为销售商品将部分利润让渡给商业资本家}{商业资本家加强对商业雇员的剥削}{产业部门将工人创造的一部分剩余价值分割给商业部门}
\begin{solution}平均利润是按照平均利润率所取得的利润;平均利润率是社会剩余价值总量与社会总资本的比率。平均利润是部门之间竞争的结果,或者说,是资本在不同部门之间流动的结果。这里一定要明确:只有是一个部门(而不是部门内部的企业),才有获得平均利润的资格。B、D项是讲商业利润的实现方式;C项是讲商业利润的来源。
\end{solution}
\question 商业资本作为一种独立的职能资本,也获得平均利润,其直接原因是
\par\fourch{\textcolor{red}{商业部门和产业部门之间的竞争和资本转移}}{产业资本家为销售商品将部分利润让渡给商业资本家}{商业资本家加强对商业雇员的剥削}{产业部门将工人创造的一部分剩余价值分割给商业部门}
\begin{solution}本题考查的是``商业资本和商业利润''这一知识点的内容。商业资本和商业利润是剩余价值分配理论中的重要理论点和知识点,但并不是政治经济学这一学科在考研当中的常考点,本题是围绕商业资本和商业利润这两个基本概念进行考查。商业资本并不创造剩余价值,但商业资本作为一种独立的职能资本,也要获得平均利润。这是通过商业部门和产业部门之间的竞争,以及它们之间的资本转移实现的。因此,答案是A。本题的考查目的在于引导考生理解市场机制如何配置资源,如何实现按生产要素进行分配以及资本家的剥削实质。本题考查方式还在于引导考生重在理解,不要死记硬背。对于政治经济学这一学科的复习,考生应始终明白一条:死记硬背是一种事倍功半的复习方法。
\end{solution}

\subsection{048-资本主义产生的产生与原始积累}
\question 马克思指出:``资本主义积累不断地并且同它的能力和规模化成比例地生产出相对的,即超过资本增殖的平均需要的,因而是过剩的或追加的工人人口。''``过剩的工人人口是积累或资本主义基础上的财富发展的必然产物,但是这种过剩人口反过来又成为资本主义积累的杠杆,甚至成为资本主义生产方式存在的一个条件。''上述论断表明
\par\fourch{\textcolor{red}{资本主义生产周期性特征需要有相对过剩的人口规律与之相适应}}{\textcolor{red}{资本主义社会过剩人口之所以是相对的,是因为它不为资本价值增殖所需要}}{\textcolor{red}{资本主义积累必然导致工人人口的供给相对于资本的需要而过剩}}{资本主义积累使得资本主义社会的人口失业规模呈现越来越大的趋势}
\begin{solution}资本积累会导致资本有机构成提高,产生相对过剩人口,即失业人口,并形成与资本主义生产周期性特征相适应的相对过剩人口规律。相对过剩人口即劳动力供给超过了资本对它的需求形成的过剩人口。ABC选项正确。
\end{solution}
\question 马克思指出:``资本主义社会的经济结构是从封建社会的经济结构中产生的,后者的解体使前者的要素得到解放。''这说明
\par\fourch{封建社会向资本主义社会过渡是人类社会的最终阶段}{资本主义社会产生与发展的根本原因是商品经济打破自然经济}{共产主义社会是历史发展的必然趋势}{\textcolor{red}{新的更高的生产关系,在它的物质存在条件在旧社会的胎胞里成熟以前,是绝不会出现的}}
\begin{solution}本题马克思语录意在指出,新事物的产生是必须是在旧事物的胎胞里成熟
以前才能出现的。体现新事物产生的原因之一,故选D。
\end{solution}
\question 从历史发展的角度看,资本主义生产资料所有制是不断演进和变化的。当今资本主义社会,居主导地位的资本所有制形式是(
)。
\par\fourch{私人资本所有制}{\textcolor{red}{法人资本所有制}}{私人股份资本所有制}{垄断资本私人所有制}
\begin{solution}【解析】第二次世界大战后,资本主义所有制发生了新的变化:①国家资本所有制形式形成并发挥重要作用;②法人资本所有制崛起并成为居主导地位的资本所有制形式。法人资本所有制是法人股东化的产物,其基本特点是:各类法人取代个人或家族股东成为企业的主要出资人,企业的股票高度集中于少数法人股东手中,法人股东直接参与公司治理,监督和制约管理阶层的经营行为,使公司资本的所有权与控制权重新趋于合一。
\end{solution}
\question 使生产者与生产资料相分离,将货币资本迅速集中于少数人手中的历史过程就是(
)
\par\twoch{资本积累}{资本剥削}{\textcolor{red}{资本原始积累}}{资本集中}
\begin{solution}本题是考查对基本概念的准确记忆。资本积累、资本集中、资本积聚都发生在资本主义社会当中。只有资本的原始积累发生在资本主义社会之前。资本原始积累就是使生产者与生产资料相分离,将货币资本迅速集中于少数人手中的历史过程。这个过程一方面使社会的生活资料和生产资料转化为资本,另一方面使直接生产者转化为雇佣工人。资本的原始积累为资本主义的产生创造了条件。
\end{solution}

\subsection{049-资本主义的政治制度和意识形态}
\question 价值规律作用的实现有赖于( )
\par\twoch{\textcolor{red}{劳动生产率提高}}{价格波动}{\textcolor{red}{资源的有效配置}}{\textcolor{red}{供求关系的变化}}
\begin{solution}此题考查的是价值规律作用的实现。在商品经济条件下,价值规律的表现形式是商品的价格围绕价值自发地波动。这种波动是由于供求关系变动的影响造成的。商品生产者之间展开激烈的市场竞争,在这种竞争中,必然产生价格的波动。正确选项为ACD。
\end{solution}
\question 资本主义意识形态的本质具体表现在( ~)
\par\fourch{\textcolor{red}{为资本主义经济基础服务}}{包括政治、经济、法律、哲学等内容}{\textcolor{red}{是资产阶级的阶级意识的集中体现}}{构成上层建筑的主要内容}
\begin{solution}资本主义国家意识形态具有鲜明的阶级性。这种阶级本质具体体现在:第一,资本主义的意识形态,是在资本主义社会条件下所形成的观念上层建筑,是对资本主义经济基础的反映,并为资本主义经济基础服务。资本主义意识形态的核心,是论证资本主义社会制度的合理性、资本主义民主的普遍性。第二,资本主义意识形态是资产阶级的阶级意识的集中体现。在资本主义条件下,资产阶级在进行阶级统治的实践中逐步形成了自己作为社会统治阶级的阶级意识。BD都是正确的论断,但与题干无关。B是意识形态的内容;D是意识形态的定位。
\end{solution}

\subsection{050-垄断与竞争}
\question 垄断是从自由竞争中形成的,是作为自由竞争的对立面产生的,但是,垄断并不能消除竞争,而是凌驾于竞争之上,与之并存。垄断资本主义阶段存在竞争的主要原因是
\par\fourch{\textcolor{red}{垄断没有消除产生竞争的经济条件}}{垄断没有消除产生竞争的政治条件}{\textcolor{red}{垄断必须通过竞争来维持}}{\textcolor{red}{不存在由一个垄断组织囊括一切部门、一切社会生产的绝对垄断}}
\begin{solution}【答案】ACD
【解析】垄断资本主义阶段存在竞争的主要原因:一是垄断没有消除产生竞争的经济条件。竞争是商品经济的一般规律,垄断产生后,没有消除以资本主义私有制为基础的商品经济。二
是垄断必须通过竞争来维持。各个垄断组织通过竞争发展起来,还需要不断增强自己的竞争
实力,巩固自己的垄断地位。三是不存在由一个垄断组织囊括一切部门、一切社会生产的绝
对垄断。在垄断条件下,在垄断组织内部、垄断组织之间、垄断组织同非垄断组织之间以及非垄断的中小企业之间存在着广泛而激烈的竞争。据此,本题选ACD。
\end{solution}
\question 垄断价格包括垄断高价和垄断低价两种形式。垄断高价是指垄断组织出售商品时规定的高于生产价格的价格;垄断低价是指垄断组织在购买非垄断企业所生产的原材料等生产资料时规定的低于生产价格的价格。垄断高价和垄断低价并不否定价值规律,垄断价格的形成只是使价值规律改变了表现形式。因为(
)
\par\fourch{\textcolor{red}{垄断价格不能完全脱离商品的价值}}{按垄断低价的买卖行为,仍然是等价交换}{\textcolor{red}{从整个社会看,商品价格总额和商品价值总额是一致的}}{\textcolor{red}{垄断高价是把其他商品生产者的一部分利润转移到垄断高价的商品上}}
\begin{solution}B垄断高价和垄断低价都不是等价交换。垄断高价是指垄断组织出售商品时规定的高于生产价格的价格;垄断低价是指垄断组织在购买垄断企业所生产的原材料时规定的地域生产价格的价格。
\end{solution}
\question 垄断资本主义的基本经济特征包括
\par\fourch{\textcolor{red}{垄断组织在经济生活中起决定作用}}{\textcolor{red}{资本输出有了特别重要的意义}}{\textcolor{red}{在金融资本的基础上形成金融寡头的统}}{垄断使竞争趋于缓和}
\begin{solution}本题考查``垄断资本主义生产关系的特征''这一知识点的内容。属于基本知识考查。列宁在帝国主义论中把垄断资本主义的基本经济特征概括为五个方面。垄断资本主义的基本经济特征是政治经济学的一个重要理论点,也是考研复习的一个相对比较重要的知识点,但是把垄断资本主义的基本经济特征作为多选题这一考查方式在近10年来的考研当中并不多见,考的相对比较多的是垄断资本主义基本经济特征二战后的新发展和变化,仅从这一角度讲,本题相对比较偏和冷。D选项是错误观点,不选;其他三项A、B、C是正确观点,正确选项。本题考查记忆。
\end{solution}

\subsection{051-金融资本和垄断利润}
\question 20世纪七八十年代,西方各国政府放松对银行利率的管制,实行浮动汇率制度,取消外汇管制,金融市场互相开放。金融机构开始突破原有的专业分工界限,综合经营各种金融业务,金融工具不断创新,融资方式的证券化趋势迅猛发展。由此看来,金融垄断资本得以形成和壮大的重要制度条件是:
\par\twoch{金融寡头的出现}{\textcolor{red}{金融自由化}}{\textcolor{red}{金融创新}}{宏观调控政策}
\begin{solution}(注意书上答案多了一个D属于校对错误)本题是政治经济学的新增考点,金融垄断资本得以形成和壮大的重要制度条件是B金融自由化C金融创新。
\end{solution}

\subsection{052-垄断资本从国家走向全球}
\question 国家垄断资本主义是国家政权和私人垄断资本融合在一起的垄断资本主义。第二次世界大战结束以来,在国家垄断资本主义获得充分发展的同时,资本主义国家通过宏观调节和微观规制对生产、流通、分配和消费各个环节的干预也更加深入。其中,微观规制的类型主要有
\par\twoch{\textcolor{red}{社会经济规制}}{\textcolor{red}{公共亊业规制}}{公众生活规制}{\textcolor{red}{反托拉斯法}}
\begin{solution}微观规制主要是国家运用法律手段规范市场秩序,限制垄断,保护竞争,维护社会公众的合法权益。ABD选项正确。
\end{solution}
\question 垄断资本向世界范围的扩展,产生了一系列的经济社会后果:对于资本输出国来讲,资本输出为其带来了巨额利润,带动和扩大了商品输出,大大改善了国际收支状况,对发展中国家的经济命脉形成控制。对于资本输入国主要是发展中国家来讲,资本输人对其经济和社会发展产生了一定的积极作用,如吸收了资金,引进了较先进的机器设备和工艺技术,培训了技术和管理人员,利用外贸和技术办厂,促进经济发展,增加了就业,扩大了外贸等。垄断资本向世界范围扩展的经济动因是
\par\fourch{\textcolor{red}{将国内过剩的资本输出,以在别国谋求高额利润}}{\textcolor{red}{将部分非要害技术转移到国外,以取得在别国的垄断优势}}{\textcolor{red}{争夺商品销售市场}}{\textcolor{red}{确保原材料和能源的可靠来源}}
\begin{solution}【解析】垄断资本在国内建立了垄断统治后,必然要把其统治势力扩展到国外,建立国际垄断统治。垄断资本向世界范围扩展的经济动因:一是将国内过剩的资本输出,以在别国谋求高额利润;二是将部分非要害技术转移到国外,以取得在别国的垄断优势;三是争夺商品销售市场;四是确保原材料和能源的可靠来源。这些经济上的动因与垄断资本主义政治上、文化上、外交上的利益交织一起,共同促进了垄断资本主义向世界范围的扩展。垄断资本向世界范围扩展的基本形式有三种:一是借贷资本输出;二是生产资本输出;三是商品资本输出。从输出资本的来源看,主要有两类:一类是私人资本输出;另一类是国家资本输出。
\end{solution}
\question 早在150多年前,马克思与恩格斯就已指出``不断扩大产品销路的需要,驱使资产阶级奔走于全球各地'',``资产阶级由于开拓了世界市场,使一切国家的生产和消费都成为世界性的了。\ldots{}\ldots{}在过去那种地方的和民族的自给自足的闭关自守状态,被各民族的各方面的互相往来和各个方面的互相依赖所代替了''。这段话说明(
)
\par\twoch{\textcolor{red}{全球化趋势具有客观必然性}}{\textcolor{red}{全球化是生产社会化发展的结果}}{\textcolor{red}{全球化是由资本主义国家推动的}}{发展中国家只能被动地参与全球化}
\begin{solution}D经济全球化对发展中国家也具有积极的影响,可以引进先进技术和管理经验,增强经济竞争力;通过吸引外资,扩大就业;利用不断扩大的国际市场解决产品销售问题;可以借助自由化和比较优势组建大型跨国公司,积极参与全球化进程。
\end{solution}
\question 金融寡头在经济领域中的统治主要是通过( )
\par\twoch{股份制实现的}{控股制实现的}{\textcolor{red}{参与制实现的}}{联合制实现的}
\begin{solution}金融寡头在经济领域中的统治主要是通过``参与制''实现的。所谓参与制就是金融寡头通过掌握一定数量的股票来层层控制企业的制度。
\end{solution}

\subsection{053-当代资本主义变化的表现}
\question 伴随着生产力发展,科技进步及阶级关系调整,当代资本主义社会的劳资关系和分配关系发生了很大变化。其中资本家及其代理人为缓和劳资关系所采取的激励制度有
\par\twoch{\textcolor{red}{职工参与决策制度}}{\textcolor{red}{职工终身雇佣制度}}{职工选举管理制度}{\textcolor{red}{职工持股制度}}
\begin{solution}随着社会生产力的发展和工人阶级反抗力量的不断壮大,资本家及其代理人开始采取一些缓和劳资关系的激励制度,促使工人自觉地服从资本家的意志。这些制度主要有:职工参与决策,终身雇佣,职工持股。因此,正确答案为ABD。
\end{solution}
\question 在当今资本主义社会,资本家及其代理人采取了一些缓和劳资关系的举措和激励制度,这些制度有(
)
\par\twoch{\textcolor{red}{职工参与决策}}{福特制和泰罗制}{\textcolor{red}{终身雇佣}}{\textcolor{red}{职工持股}}
\begin{solution}随着社会生产力的发展对劳动者自觉性要求的提高以及工人阶级反抗力量的不断壮大,资本家及其代理人开始采取一些缓和劳资关系的激励制度,以便使工人自愿地服从资本家的意志。这些制度主要有:其一,职工参与决策。这一制度旨在协调劳资关系,缓和阶级矛盾。按照这种制度,有的国家在企业的监事会中,劳资双方各占一半席位,对企业重大问题共同进行决策。其二,终身雇佣。这是一种用工制度。按照该制度,工人一旦进入公司工作,只要不违反公司纪律,就会终身被雇佣。其三,职工持股。该制度旨在通过使职工持有一部分本公司的股份来调动工人的生产积极性,使工人产生归属感,在生产中努力提高劳动生产率,增加剩余价值生产。
\end{solution}

\subsection{054-当代资本主义变化的原因和实质}
\question 在当代资本主义条件下,科学技术的不断进步和生产社会化程度的不断提高,必然要求调整和变革那些不适应生产力发展的旧的生产关系。所以当代资本主义国家,在其根本制度不便的情况下,在生产资料所有制形式、劳资关系分配关系、阶级阶层结构等方面做出了很大的调整,但是对于资本主义社会根本不能变的是
\par\twoch{\textcolor{red}{生产资料私有制}}{\textcolor{red}{雇佣劳动制度}}{\textcolor{red}{追逐剩余价值的本性}}{\textcolor{red}{资本主义政治制度}}
\begin{solution}本题是对当代资本主义新变化的理解和记忆型考查。当代资本主义发生变化实质是在资本主义制度基本框架内的变化,并不意味着资本主义生产关系的根本性质发生了变化。故资本主义社会根本不能触动的是ABCD选项。
\end{solution}
\question 资本主义国家宏观调节的基本目标是实现
\par\twoch{\textcolor{red}{经济快速增长}}{预算与债务平衡}{\textcolor{red}{物价稳定}}{\textcolor{red}{充分就业}}
\begin{solution}答案】ACD
【简析】资本主义国家宏观调节主要是国家运用财政政策.货币政策等经济手段.对社会总供求进行调节.以实现经济快速增长、充分就业、物价稳定和国际收支平衡的基本目标。A、C、D正确
\end{solution}

\subsection{055-资本主义的历史地位}
\question 1989年,时任美国国务院顾问的弗朗西斯●福山抛出了所谓的``历史终结论'',认为西方实行的自由民主制度是``人类社会形态进步的终点''和
``人类最后一种的统治形式''。然而,20年来的历史告诉我们,终结的不是历史,而是西方的优越感。就在柏林墙倒塌20年后的2009年11月9日,BBC
公布了一份对27国民众的调查。结果半数以上的受访者不满资本主义制度,此次调查的主办方之一的``全球扫描''公司主席米勒对媒体表示,这说明随着1989年柏林墙的倒塌资本主义并没有取得看上去的压倒性胜利,这一点在这次金融危机中表现的尤其明显,``历史终结论''的破产说明
\par\fourch{社会规律和自然规律一样都是作为一种盲目的无意识力量起作用}{\textcolor{red}{人类历史的发展的曲折性不会改变历史发展的前进性}}{\textcolor{red}{一些国家社会发展的特殊形式不能否定历史发展的普遍规律}}{\textcolor{red}{人们对社会发展某个阶段的认识不能代替社会发展的整个过程}}
\begin{solution}``历史终结论''的破产说明,人类历史的发展的曲折性不会改变历史发展的前进性,一些国家社会发展的特殊形式不能否定历史发展的普遍规律,人们对社会发展某个阶段的认识不能代替社会发展的整个过程。但是,社会规律和自然规律是有相异之处的,社会规律是人有意识的能动活动,自然规律是盲目的无意识的力量起作用,所以,正确答案是选项BCD。
\end{solution}

\subsection{056-空想社会主义}
\question 社会主义发展的历史进程不是一帆风顺的,高潮和低潮的相互交替构成波澜壮阔的社会主义发展史。社会主义发展史上的两次飞跃是(
)
\par\fourch{\textcolor{red}{19世纪中叶,社会主义从空想发展到科学}}{19世纪70年代,社会主义从理论发展到建立社会主义制度的实践}{\textcolor{red}{20世纪初,社会主义从理论发展到建立社会主义制度的实践}}{20世纪中叶,社会主义由一国实践到多国实践}
\begin{solution}社会主义发展史上的两次飞跃是19世纪中叶,社会主义从空想发展到科学,20世纪初,社会主义从理论发展到建立社会主义制度的实践。
\end{solution}

\subsection{057-科学社会主义}
\question 根据马克思、恩格斯的论述及社会主义现实的启示所概括的科学社会主义的基本原则中,社会主义生产的根本目的是
\par\fourch{发展生产力,不断推动社会发展进步}{\textcolor{red}{满足全体社会成员的需要}}{全面建设社会主义,逐步进入共产主义}{实现现代化,建设社会主义强国}
\begin{solution}【简析】根据马克思、恩格斯的论述及社会主义现实的启示所概括的科学社会主义的基本原则中,社会主义生产的根本目的是:在生产资料公有制基础上组织生产,满足全体社会成员的需要。B正确,A、C、D不符合题意。
\end{solution}
\question 社会主义发展的历史进程不是一帆风顺的,高潮和低潮的相互交替构成波澜壮阔的社会主义发展史。社会主义发展史上的两次飞跃是(
)
\par\fourch{\textcolor{red}{19世纪中叶,社会主义从空想发展到科学}}{19世纪70年代,社会主义从理论发展到建立社会主义制度的实践}{\textcolor{red}{20世纪初,社会主义从理论发展到建立社会主义制度的实践}}{20世纪中叶,社会主义由一国实践到多国实践}
\begin{solution}社会主义发展史上的两次飞跃是19世纪中叶,社会主义从空想发展到科学,20世纪初,社会主义从理论发展到建立社会主义制度的实践。
\end{solution}

\subsection{058-十月革命与列宁时期的探索}
\question 社会主义发展的历史进程不是一帆风顺的,高潮和低潮的相互交替构成波澜壮阔的社会主义发展史。社会主义发展史上的两次飞跃是(
)
\par\fourch{\textcolor{red}{19世纪中叶,社会主义从空想发展到科学}}{19世纪70年代,社会主义从理论发展到建立社会主义制度的实践}{\textcolor{red}{20世纪初,社会主义从理论发展到建立社会主义制度的实践}}{20世纪中叶,社会主义由一国实践到多国实践}
\begin{solution}社会主义发展史上的两次飞跃是19世纪中叶,社会主义从空想发展到科学,20世纪初,社会主义从理论发展到建立社会主义制度的实践。
\end{solution}
\question 世界上第一个社会主义国家是( )
\par\twoch{巴黎公社}{\textcolor{red}{苏维埃俄国}}{中华人民共和国}{巴伐利亚苏维埃共和国}
\begin{solution}十月革命的胜利,开辟了人类历史的新纪元,苏维埃俄国成为世界上第一个社会主义国家。巴伐利亚苏维埃共和国是在十月革命的感召下由德国无产阶级在1918年十一月革命中所建立起来的政权。
\end{solution}

\subsection{059-社会主义发展的性质}
\question 社会主义在发展过程中出现挫折和反复,这表明( )
\par\fourch{\textcolor{red}{新生事物的成长不是一帆风顺的}}{\textcolor{red}{事物发展的道路是螺旋式的}}{社会发展的客观趋势不是不可以改变的}{\textcolor{red}{历史有时会向后作巨大的跳跃}}
\begin{solution}社会主义的发展是前进性与曲折性的统一。马克思主义认为,人类社会的发展从来不是一帆风顺,而是在曲折中前进的。社会主义作为崭新的社会形态,符合历史的发展趋势,具有强大的生命力,但社会主义的产生和成长意味着对资本主义旧社会的否定,必然遭到资本主义的拼死反抗,这就注定社会主义战胜资本主义是一个曲折的发展过程。社会主义在发展过程中,由于历史的和现实的、国际的和国内的各种因素的相互作用,社会主义的发展道路必然呈现出多样性的特点。这表明,坚持社会主义不等于坚持某种单一的社会主义模式,某种社会主义模式的失败不等于整个社会主义事业的失败。A、B、D、E项正确。
C项错误,人们不能改变社会发展的客观趋势。
\end{solution}
\question 恩格斯指出:``所谓`社会主义'不是一种一成不变的东西,而应当和任何其他社会制度一样,把它看成是经常变化和改革的社会。''社会主义改革的根源是(
)
\par\fourch{改革是社会主义社会发展的动力}{生产力发展水平不够高}{社会主义制度没有根本克服资本主义制度下生产力与生产关系的对抗性矛盾}{\textcolor{red}{社会主义社会的基本矛盾}}
\begin{solution}社会主义社会的基本矛盾生产力和生产关系之间的矛盾是社会改革的根源所在。
\end{solution}
\question 列宁指出:``设想世界历史会一帆风顺、按部就班地向前发展,不会有时出现大幅度的跃退,那是不辩证的,不科学的,在理论上是不正确的。''社会主义在曲折中发展的原因在于(
)
\par\fourch{\textcolor{red}{社会主义作为新生事物,其成长不会一帆风顺}}{\textcolor{red}{经济全球化对于社会主义的发展既有机遇又有挑战}}{\textcolor{red}{社会主义社会的基本矛盾推动社会发展是作为一个过程而展开的}}{\textcolor{red}{人们对社会主义社会的基本矛盾推动社会发展的认识有一个逐渐发展的过程}}
\begin{solution}社会主义作为新生事物,其成长不会一帆风顺,经济全球化对于社会主义的发展既有机遇又有挑战,社会主义社会的基本矛盾推动社会发展是作为一个过程而展开的,人们对社会主义社会的基本矛盾推动社会发展的认识有一个逐渐发展的过程。
\end{solution}

\subsection{060-无产阶级政党及其作用}
\question 马克斯、恩格斯在指导建立无产阶级政党的过程中,阐述了各国无产阶级政党相互关系的重要原则,主要有
\par\fourch{\textcolor{red}{坚持各国党的独立自主}}{\textcolor{red}{坚持无产阶级的国际联合}}{坚持合法斗争和暴力革命相结合}{\textcolor{red}{坚持各国党的完全平等}}
\begin{solution}马克思、恩格斯在指导建立无产阶级政党的过程中,阐述了各闰无产阶级政党相互关系的重要原则。一是坚持无产阶级的国际联合。在马克思、恩格斯的指导下,各国工人政党于1864年建立了工人运动的国际联合组织------``笫一国际''。
二是坚持各国党的独立自主和完全平等。强调无产阶级的国际联合,并不意味着各国无产阶级的斗争从属于另一国无产阶级的斗争。马克思层多次强调,第一国际只是各国工人运动联络和合作的中心,而不是指挥中心。A、B、D正确答案。C回答的不是题目所问的各国无产阶级政党相互关系的原则问题。
\end{solution}
\question 社会主义从理论到实践的飞跃,是通过无产阶级革命实现的。无产阶级革命是迄今人类历史上最广泛、最彻底、最深刻的革命,是不同于以往一切革命的最新类型的革命。无产阶级革命的主要特点是(
)
\par\fourch{\textcolor{red}{无产阶级革命是不断前进的历史过程}}{\textcolor{red}{无产阶级革命是为绝大多数人谋利益的运动}}{\textcolor{red}{无产阶级革命是要彻底消灭一切阶级和阶级统治的革命}}{\textcolor{red}{无产阶级革命是彻底消灭一切私有制、代之以生产资料公有制的革命}}
\begin{solution}无产阶级革命的主要特点是无产阶级革命是不断前进的历史过程,无产阶级革命是为绝大多数人谋利益的运动,无产阶级革命是要彻底消灭一切阶级和阶级统治的革命,无产阶级革命是彻底消灭一切私有制、代之以生产资料公有制的革命。
\end{solution}

\subsection{061-法律的运行}
\question 社会主义法律运行的相关内容,下列说法中错误的是( )
\par\fourch{法律制定是法律运行的起始性和关键性环节}{法律遵守是法律实施和实现的基本途径,守法既包含履行义务,又包含正确行使权利}{\textcolor{red}{国务院有权制定行政法及部门规章}}{行政执法是法律实施和实现的重要环节}
\begin{solution}【解析】本题考查我国社会主义法律的运行。法律的运行是一个从创制、实施到实现的过程。这个过程主要包括法律制定(立法)、法律遵守(守法)、法律执行(执法)、法律适用(司法)等环节。法律
制定就是有立法权的国家机关依照法定职权和程序制定规范性法律文件的活动,是法律运行的起
始性和关键性环节。因此,A选项观点正确。根据我国《宪法》《立法法》等的规定,全国人民代表大
会及其常务委员会行使国家立法权。国务院有权根据宪法和法律制定行政法规。国务院各部门可
以根据宪法、法律和行政法规,在本部门的权限范围内,制定部门规章。省、自治区、直辖市的人民
代表大会及其常委会根据本行政区域的具体情况和实际需要,在不同宪法、法律和行政法规相抵触
的前提下,可以制定地方性法规。法律遵守(守法),人们通常把守法仅仅理解为履行法律义务。其
实,守法意味着一切组织和个人严格依法办事的活动和状态。依法办事包括两层含义:一是依法享
有并行使权利;二是依法承担并履行义务。因此,不能将守法仅仅理解为履行义务,它还包含着正
确行使权利。因此,B选项观点正确。C选项中行政法是由全国人大制定,并非国务院,国务院只能
制定行政法规,同学们一定要注意区分``行政法''和``行政法规''。因此C选项是错误观点,符合题
意。在法律运行中,行政执法是最大量、最经常的工作,是实现国家职能和法律价值的重要环节。
因此,D选项观点正确。
\end{solution}

\subsection{062-我国的国家制度}
\question 在我国国家机构中,依法独立行使审判权的是( )
\par\twoch{国务院}{\textcolor{red}{人民法院}}{人民检察院}{人民代表大会}
\begin{solution}《宪法》规定:中华人民共和国国务院即中央人民政府,是最高国家权力机关的执行机关,是最高国家行政机关;中华人民共和国人民法院是国家的审判机关;中华人民共和国人民检察院是国家的法律监督机关;中华人民共和国全国人民代表大会是最高国家权力机关,常设机关是全国人民代表大会常务委员会。
\end{solution}

\subsection{063-公民的基本权利和义务}
\question 政治权利和自由是指公民作为国家政治生活主体依法享有的参加国家政治生活的权利和自由,是国家为公民直接参与政治活动提供的基本保障,这一基本权利具体包括
\par\twoch{人身自由权}{\textcolor{red}{选举权和被选举权}}{宗教信仰自由}{\textcolor{red}{政治自由}}
\begin{solution}本题考查我国公民基本权利中的政治权利和自由的基本内容。政治权利和自由具体包括两个方面:一是选举权和被选举权。二是政治自由,我国《宪法》规定:``中华人民共和国公民有言论、出版、集会、结社、游行、示威的自由。''因此选项BD为正确答案。选项A人身自由权与选项C宗教信仰自由也都是公民的基本权利,但与题意不符。
\end{solution}

\subsection{064-社会主义法治观念的基本内容}
\question 中国特色社会主义法治道路,是社会主义法治建设成就和经验的集中体现,是建设社会主义法治国家的唯一正确道路。它的核心要义包括
\par\twoch{\textcolor{red}{坚持党的领导}}{\textcolor{red}{坚持中国特色社会主义制度}}{\textcolor{red}{贯彻中国特色社会主义法治理论}}{坚持依法治国和以德治国相结合}
\begin{solution}中国特色社会主义法治道路,是社会主义法治建设成就和经验的集中体现,是建设社会来义法治国家的唯一正确道路。它包括坚持党的领导,坚持中国特色社会主义制度,贯彻中国特色社会主义法治理论三个方面的核心要义。A、B、C正确。坚持依法治国和以德治国相结合属于树立社会主义法制观念的内容,D不符合题意。
\end{solution}
\question 执法为民是社会主义法治理念的五项基本内容之一。执法为民是社会主义法治的本质要求,是人民当家作主的社会主义国家性质在法治上的必然反映。执法为民的基本要求是(
)
\par\twoch{严格执法}{\textcolor{red}{以人为本}}{\textcolor{red}{尊重和保障人权}}{\textcolor{red}{文明执法}}
\begin{solution}执法为民,包括三项基本要求:一是以人为本,二是尊重和保障人权,三是文明执法。属于大纲的实际内容。
\end{solution}

\subsection{065-法治思维的含义、特征}
\question 培养法治思维,必须抛弃人治思维。法治思维与人治思维的区别,集中体现
\par\fourch{\textcolor{red}{在依据上,法治思维认为国家的法律是治国理政的基本依据;人治思维强调的是依靠个人的能力和德行治国理政}}{\textcolor{red}{在方式上,法治思维以一般性、普遍性的平等对待方式调节社会关系,解决矛盾纠纷,具有稳定性和一贯性;人治按照个人意志和感情进行治理,具有极大的任意性和非理性}}{\textcolor{red}{在价值上,法治思维是“多数人之治”的民主思维;人治思维是少数个人的集权专断}}{在标准上,法治思维尊良法、尊善法;人治思维不奉法、奉劣法}
\begin{solution}【解析】培养法治思维,必须抛弃人治思维。法治思维与人治思维的区别,集中体现在四个方面:一是在依据上,法治思维认为国家的法律是治国理政的基本依据,也是行为的根本指南。处理法律问题要以事实为根据、以法律为准绳:而人治思维则主张凭借个人尤其是掌权者、领导人的个人魅力、德性和才智来治国平天下。如古希腊柏拉图提出的``哲学王''之治,我国古代推崇``圣君''、``贤人''之治以及后世的``英雄''、``强人``、``能人''之治等,主要强调的都是依靠个人的能力和德行治国理政。二是在方式上,法治思维以一般性、普遍性的平等对待方式调节社会关系,解决矛盾纠纷,坚持法律面前人人平等原则,反对因人而异和亲疏远近,具有稳定性和一贯性;而人治漠视规则的普遍适用性,按照个人意志和感情进行治理,治人者以言代法,言出法随,朝令夕改,具有极大的任意性和非理性。三是在价值上,法治思维强调集中社会大焱的意志来进行决策和判断,是一种``多数人之治''的民主思维,而且这种民主是建立在法律的基础上的,避免陷入无政府主义或以民主之名搞乱社会。
而人治思维是少数人说了算的专断思维,虽然有时也强调集思广益,如通过开会讨论、搞群众运动的形式进行治理或作出决定,但主要表现为少数个人的集权专断。四是在标准上,法治思维与人治思维的分水岭不在于有没有法律或者法律的多寡与好坏,而在于最高的权威究竟是法律还是个人。显然D项为唯一干扰项。法治思维以法律为最高权威,强调``必须使民主制度化、法律化,使这种制度和法律不因领导人的改变而改变,不因领导人的看法和注意力的改变而改变。''人治思维则奉领导者个人的意志为最高权威,当法律的权威与个人的权威发生矛盾时,强调服从个人而非服从法律的权威。
\end{solution}

\subsection{066-尊重法律权威的基本要求}
\question 讲证据是提高法律修养的重要方面,法律上的证据不同于一般的事实,证据具有(
)
\par\twoch{\textcolor{red}{合法性}}{\textcolor{red}{客观性}}{\textcolor{red}{关联性}}{说服力}
\begin{solution}法律上的证据具有三个特征:合法性、客观性、关联性。 因此,正确答案是ABC。
\end{solution}

\subsection{067-新文化运动的内容和意义}
\question 1914年至1918年的第一次世界大战,是一场空前残酷的大屠杀。它改变了世界政治的格局,也改变了各帝国主义国家在中国的利益格局,对中国产生了巨大的影响。大战使中国的先进分子
\par\twoch{对中国传统文化产生怀疑}{\textcolor{red}{对西方资产阶级民主主义产生怀疑}}{认识到工人阶级的重要作用}{认识到必须优先改造国民性}
\begin{solution}第一次世界大战中,西方资本主义国家为各自利益彼此厮杀,暴露了资产阶级民主的弱点,使中国的先进分子对西方资产阶级民主主义产生怀疑。B选项正确。
\end{solution}
\question 新文化运动从1915年9月陈独秀在上海创办《青年》杂志(后改名《新青年》)开始,一直持续到五四运动之后。新文化运动是中国历史上一次前所未有的启蒙运动,其具有重要意义。以下选项内容不正确的是
\par\fourch{新文化运动提出的基本口号是民主和科学}{新文化运动的主要阵地是《新青年》杂志和北京大学}{新文化运动为外国各种思想流派传入中国敞开了大门}{\textcolor{red}{新文化运动是新民主义的新文化反对封建主义的旧文化}}
\begin{solution}10.新文化运动是从1915年9月陈独秀在上海创办《青年》杂志(后改名《新青年》)开始的。1917年1月,蔡元培出任北京大学校长,聘陈独秀为北大文科学长。《新青年》编辑部也随之迁至北京。1918年1月,《新青年》由陈独秀个人主编的刊物改为同人刊物(口袋题库app题型大家、不是那种同人。由编辑部成员合作经营并共同主持编辑业务的报刊。)。《新青年》杂志和北京大学成了新文化运动的主要阵地。新文化运动提出的基本口号是民主和科学。新文化运动是中国历史上一次前所未有的启蒙运动和空前深刻的思想解放运动。它以勇往直前的大无畏精神和与传统观念彻底决裂的激烈姿态,对封建专制主义、封建伦理道德和封建迷信愚昧进行了无情的批判,在社会上掀起的一股生气勃勃的思想解放潮流,冲决了禁锢人们思想的闸门,从而为外国各种思想流派传入中国敞开了大门,激励着人们去探求救国救民的真理。A、B、C不符合题意。五四以前的新文化运动主要是资产阶级民主主义的新文化反对封建主义的旧文化;五四以后的新文化运动,马克思主义开始逐步地在思想文化领域中发挥指导作用,是新民主主义的文化运动,D符合题意。
\end{solution}
\question 五四以前新文化运动的主要内容有( )
\par\twoch{\textcolor{red}{提倡民主、科学}}{宣传社会主义、马克思主义}{\textcolor{red}{提倡白话文,反对文言文}}{\textcolor{red}{提倡新文学,反对旧文学,主张文学革命}}
\begin{solution}五四运动之前的新文化运动主要是宣传资产阶级思想,提倡民主、科学,提倡白话文,反对文言文,提倡新文学,反对旧文学,主张文学革命,五四运动后主要是宣传马克思主义。
\end{solution}
\question 20世纪早期,一群受过西方教育的先进知识分子认识到,要确实改造中国,必须进行一场思想启蒙运动,在他们的带领下,中国迎来了一场前所未有的启蒙运动和空前深刻的思想解放运动------``新文化运动''。``新文化运动''之所以会产生的原因是(
)
\par\fourch{\textcolor{red}{辛亥革命没有解决中国社会的基本矛盾}}{\textcolor{red}{民族资本主义经济的发展}}{中国出现宣传社会主义思想的知识分子群体}{\textcolor{red}{袁世凯掀起尊孔复古逆流}}
\begin{solution}新文化运动早期是为了宣传资产阶级思想,反对袁世凯尊孔复古,所以答案是ABD。出现宣传社会主义思想是在五四运动后。
\end{solution}

\subsection{068-五四运动及其意义}
\question 五四运动是中国近代史上一个划时代的事件。五四运动
\par\fourch{\textcolor{red}{形成了爱国、进步、民主、科学的五四精神,拉开了中国新民主主义革命的序幕}}{\textcolor{red}{促进了马克思主义在中国的传播,推进了中国共产党的建立}}{\textcolor{red}{表现了反帝反封建的彻底性,是一次真正的群众解放运动}}{是中国历史上一次前所未有的启蒙运动和空前深刻的思想解放运动}
\begin{solution}D项是新文化运动的意义。
\end{solution}
\question 中国工人阶级开始登上政治舞台、成为中国革命的领导力量是在( )
\par\twoch{\textcolor{red}{五四运动}}{京汉铁路罢工}{五卅运动}{省港大罢工}
\begin{solution}中国工人阶级在五四运动中开始登上政治舞台,在运动后期发挥了主力军作用。五四运动以后,无产阶级逐渐代替资产阶级成为近代中国民族民主革命的领导者。
\end{solution}

\subsection{069-十月革命的影响与新民主主义革命的开端}
\question 毛泽东在《论人民民主专政》一文中指出:``十月革命一声炮响,给中国送来了马克思主义。''这是说(
)
\par\fourch{十月革命以后马克思主义才在中国开始得到传播}{\textcolor{red}{十月革命给予中国人的一个启示是:经济文化落后的国家也可以用社会主义思想指引自己走向解放之道}}{\textcolor{red}{十月革命推动中国的先进分子把自己的目光从西方转向东方,从资产阶级民主主义转向社会主义}}{\textcolor{red}{资本主义制度并不是永恒的,无产阶级和其他劳动群众完全可以依靠自身的力量创造出维护绝大多数人利益的崭新的社会制度}}
\begin{solution}按照新民主主义革命论,正确答案是BCD.十月革命前中国也有马克思主义的传播只不过比较小众,十月革命后,马克思主义已经在中国传播,而不是五四运动开始。
\end{solution}

\subsection{070-中国共产党的成立}
\question 中国共产党成立后,``中国革命的面貌焕然一新'',其``新''主要表现在( )
\par\twoch{\textcolor{red}{以马克思主义为指导}}{以武装斗争为主要方法}{提出了彻底的反帝反封建的纲领}{\textcolor{red}{以社会主义、共产主义为远大目标}}
\begin{solution}本题考查的是中国共产党成立的伟大意义。第一,自从有了中国共产党,灾难深重的中国人民有了可以信赖的组织者和领导者,中国革命有了坚强的领导力量。第二,中国共产党的成立,使中国革命有了科学的指导思想。第三,党所提出的纲领和奋斗目标,代表着中国社会发展的正确方向,代表着中国无产阶级和其他广大劳动人民的根本利益。因此,中国共产生从诞生时起,就充满着生机和活力,预示着中国的光明和希望。第四,中国共产党的成立,也使中国革命有了新的革命方法,并沟通了中国革命和世界无产阶级革命之间的联系。中国革命的面目从此焕然一新。所以答案选AD。而C是中共二大提出的,B工农武装割据的思想是土地革命时期的思想。
\end{solution}

\subsection{071-中国共产党的早期运动}
\question 中国各地共产党早期组织成立后着重进行的工作包括( )
\par\twoch{\textcolor{red}{研究和宣传马克思主义}}{制定民主革命纲领}{\textcolor{red}{到工人中去开展宣传和组织工作}}{\textcolor{red}{开展关于建党问题的讨论和实际组织工作}}
\begin{solution}早起的共产党成立后主要是研究和宣传马克思主义、开展关于建党问题的讨论和实际组织工作、到工人中去宣传和组织工作,而B是中共二大的内容,中共二大的时候中国共产党才意识到最高纲领------实现共产主义在目前是不现实的,提出最低纲领即民主革命纲领。
\end{solution}

\subsection{072-国共合作的形成}
\question 1924年1月,中国国民党第一次全国代表大会在广州召开,大会通过的宣言对三民主义作出了新的解释,新三民主义成为第一次国共合作的政治基础,究其原因,是由于新三民主义的政纲
\par\fourch{\textcolor{red}{同中国共产党在民主革命阶段的纲领基本一致}}{把斗争的矛头直接指向北洋军阀}{体现了联俄、联共、扶助农工三大革命政策}{把民主主义概括为平均地权}
\begin{solution}本题考查中国近现代史纲要第四章《开天辟地的大事变》中关于第一次国共合作的内容。继1923年中共三大召开,提出建立第一次国共合作统一战线以后,1924年1月,国民党一大在广州召开,大会通过的宣言对三民主义作出了新的解释,即新三民主义,这个新三民主义的政纲同中共在民主革命阶段的纲领基本一致,因而成为国共合作的政治基础。大会实际上确定了联俄、联共、扶助农工三大革命政策。这样,国民党一大的成功召开,就标志着第一次国共合作的正式形成。
\end{solution}
\question 1924年改组后的国民党成为几个阶级的革命联盟,这几个阶级有( )
\par\twoch{\textcolor{red}{工人阶级}}{\textcolor{red}{农民阶级}}{\textcolor{red}{小资产阶级}}{\textcolor{red}{民族资产阶级}}
\begin{solution}孙中山领导的国民党原来大体是代表民族资产阶级和城市小资产阶级的政党。改组后的国民党,从原来的代表资产阶级的政党改变为工人阶级、农民阶级、小资产阶级和民族资产阶级的革命联盟。
\end{solution}
\question 1924年1月,孙中山在中国国民党第一次全国代表大会上对三民主义做出了新的解释,形成了``新三民主义'',孙中山的``新三民主义''(
)
\par\fourch{\textcolor{red}{提出民主权利应“为一般平民所共有”}}{\textcolor{red}{在民族主义中增加了反帝的内容}}{\textcolor{red}{提出要改善工农的生活状况}}{实行民主主义的联合战线}
\begin{solution}大纲原话,正确答案是ABC。
\end{solution}

\subsection{073-大革命的意义,失败原因和教训}
\question 1925年5月,以五卅运动为起点,掀起了全国范围的大革命高潮。1926年7月,以推翻北洋军阀统治为目标的北伐战争开始。1927年3月国民革命军占领南京。1927年4月12日,蒋介石在上海发动反共政变,同年7月15日,汪精卫在武汉召开``分共''会议,并在其辖区内对
共产党员和革命群众实行搜捕和屠杀,国共合作全面破裂,大革命最终失败。所谓大革命的失败,主要是指
\par\fourch{北洋军阀的统治没有被推翻}{\textcolor{red}{反帝反封建的革命任务没有完成}}{国共合作全面破裂}{工人农民运动转入低潮}
\begin{solution}【简析】1925---1927年的大革命,虽然推翻了北洋军阀的统治,但国民党政府的统治依然是地主阶级和买办性的大资产阶级的统治,同北洋军阀的统治没有本质的区別。中国仍是一个处在帝国主义和封建主义统治之下的半殖民半封建社会,中国仍然迫切需要一个反帝反封违的资产阶级民主革命。所谓大革命的失败,主要是指反帝反封建的革命任务没有完成,B正确,
A错误。国共合作全面破裂是大革命失败的标志,工人农民运动转人低潮是大革命失败的后果,都不是大革命失败的含义,
C、D不符合题意。
\end{solution}
\question 第一次国共合作的成果主要有( ~)
\par\twoch{\textcolor{red}{建立黄埔军校}}{\textcolor{red}{北伐战争胜利发展}}{\textcolor{red}{工人和农民运动迅猛发展}}{推翻了帝国主义和封建军阀的统治}
\begin{solution}第一次国共合作基本上推翻了北洋军阀的统治,但中国仍然是半殖民地半封建社会,并未推翻帝国主义。D错误。
\end{solution}

\subsection{074-国民党的统治和反抗国民党的斗争}
\question 红色政权存在发展的客观条件有( )
\par\fourch{\textcolor{red}{中国是一个几个帝国主义国家间接统治的政治经济发展极端不平衡的半殖民地半封建的大国}}{有相当力量的正式红军的存在}{\textcolor{red}{第一次国内革命战争的影响}}{\textcolor{red}{全国革命形势的向前发展}}
\begin{solution}B是红色政权能够存在和发展的必要的主观条件。
\end{solution}

\subsection{075-革命新道路的探索}
\question 红色政权存在发展的根本原因是( )
\par\fourch{\textcolor{red}{中国是一个几个帝国主义国家间接统治的政治经济发展极端不平衡的半殖民地半封建的大国}}{有相当力量的正式红军的存在和共产党组织的坚强有力和各项政策的正确贯彻执行}{第一次国内革命战争的影响}{全国革命形势的向前发展}
\begin{solution}毛泽东在《中国的红色政权为什么能够存在?》、《井冈山的斗争》中,从理论上论述了中国红色政权发生、发展的原因和条件:中国是一个政治经济发展极不平衡的半殖民地半封建大国,这是红色政权存在发展的根本原因;国民革命政治影响的存在,是红色政权得以生存和发展的客观条件;全国革命形势继续向前发展,是红色政权存在和发展的又一客观条件;有相当力量的正式红军的存在,是红色政权能够存在和发展的必要的主观条件;共产党组织的坚强有力和各项政策的正确执行,是中国红色政权能够存在发展的前提和根本保证。
\end{solution}
\question ``农村包围城市、武装夺取政权''是马克思主义中国化的成功范例,集中反映这一理论的毛泽东著作有(
)
\par\twoch{\textcolor{red}{《井冈山的斗争》}}{\textcolor{red}{《中国的红色政权为什么能够存在?》}}{《中国革命战争的战略问题》}{\textcolor{red}{《星星之火,可以燎原》}}
\begin{solution}工农武装割据思想集中在《井冈山的斗争》《中国的红色政权为什么能够存在》和《星星之火,可以燎原》里面。
\end{solution}
\question 毛泽东曾说:``一国之内,在四围白色政权的包围中,有一小块或若干小块红色政权的区域长期地存在,这是世界各国从来没有的事。这种奇事的发生,有其独特的原因。而其存在和发展,亦必有相当的条件。''中国的红色政权能够存在和发展的条件有(
)
\par\twoch{\textcolor{red}{白区政权长期的分裂和战争}}{\textcolor{red}{工农兵群众的发展}}{\textcolor{red}{相当力量的正式红军的存在}}{\textcolor{red}{共产党组织的有力量和政策的不错误}}
\begin{solution}四个选项均是正确选项,大纲原话。
\end{solution}

\subsection{076-土地革命}
\question 1931年初,中国共产党形成的土地革命的阶级路线的内容有( ~)
\par\fourch{\textcolor{red}{依靠贫雇农,联合中农}}{\textcolor{red}{保护工商业者,消灭地主阶级}}{\textcolor{red}{限制富农}}{在原耕地的基础上,实行抽多补少、抽肥补瘦}
\begin{solution}毛泽东和邓子恢等其他同志一起规定的土地革命中的阶级路线是:坚定地依靠贫农、雇农,联合中农,限制富农,保护中小工商业者,消灭地主阶级;其分配方法是:以乡为单位,按人口平分土地,在原耕地的基础上,实行抽多补少、抽肥补瘦。D为分配方法而非阶级路线的内容。
\end{solution}
\question 1928年,毛泽东按照``没收一切土地归苏维埃政府所有''的原则,主持制定了我国历史上第一部彻底消灭封建土地所有制的土地法,这就是(
)
\par\twoch{\textcolor{red}{《井冈山土地法》}}{《兴国土地法》}{《苏维埃土地法》}{《中国土地法大纲》}
\begin{solution}中国第一部土地法就是第一个农村革命根据地井冈山时期颁布的《井冈山土地法》。
\end{solution}
\question 毛泽东指出:``如果不帮助农民推翻封建地主阶级,就不能组成中国革命最强大的队伍而推翻帝国主义的统治。''其实质含义是(
)
\par\fourch{农民阶级是中国革命的领导阶级}{\textcolor{red}{农民阶级是中国革命最可靠的同盟军}}{\textcolor{red}{没有农民阶级参加,中国革命就不能取胜}}{农民阶级反帝反封建态度最坚决,是新的社会生产力的代表者}
\begin{solution}农民阶级不能是革命的领导阶级,革命领导阶级是工人阶级。农民阶级不是先进生产力的代表者,所以AD选项错。
\end{solution}
\question 1948年中国共产党制定了土地改革总路线。下列选项中对这一总路线所含内容理解正确的有
\par\fourch{\textcolor{red}{按照平分土地的原则,满足贫雇农的要求}}{\textcolor{red}{团结中农,允许中农保有比他人略多的土地}}{没收地主土地,不再对地主分配土地}{\textcolor{red}{实行耕者有其田,将土地的所有权分配给农民}}
\begin{solution}本题考查新民主主义的经济纲领中没收地主土地归农民所有的内容。1948年毛泽东《在晋绥干部会议上的讲话》一文中明确提出了土地改革的总路线,即``依靠贫民、团结中农,有步骤地、有分别地消灭封建剥削制度,发展农业生产。''其中土地改革的主要的和直接的任务,是满足贫雇农群众的要求,赞成平分土地的要求,是为了便于发动广大农民群众迅速消灭封建地主阶级的土地所有制,并非提倡绝对平均主义。土地改革的另一个任务,是满足某些中农的要求,必须容许一部分中农保有比一般贫农所得土地的平均水平为高的土地量。民主革命时期,实行耕者有其田,即指没收地主土地归农民私有,而非归国家所有,故A、B、D为正确选项。C是错误的选项,因为土地改革的目的是消灭封建剥削制度,即消灭封建地主之为阶级,而不是地主个人,因此,对地主必须分给和农民同样的土地财产,把它们改造成为自食其力的劳动者。本题有一定难度,专业性很强。
\end{solution}

\subsection{077-“左”的错误的危害}
\question 从1927年7月大革命失败到1935年1月遵义会议召开之前,左倾机会主义的错误先后三次在党中央的领导机关取得了统治地位,第一次是以瞿秋白为代表的``左,,倾盲动主义,第二次
是以李立三为代表的``左''倾冒险主义,第三次是以陈绍禹(王明)为代表的左倾机会主义。
其中以第三次最为严重。王明``左''倾机会主义的主要错误有
\par\fourch{在革命性质和统一战线问题上,混淆民主革命与社会主义革命的界限,只看到了两个革命
阶段的区别,没有看到两个革命阶段的联系}{放弃了对军队的领导权,坚持以城市为中心}{\textcolor{red}{在军事斗争问题上,实行进攻中的冒险主义、防御中的保守主义、退却中的逃跑主义}}{\textcolor{red}{在党内斗争和组织问题上,推行宗派主义和“残酷斗争,无情打击的方针}}
\begin{solution}因为王明的``左倾''只看到了两个革命阶段的联系,而没有看到两个革命阶段的区别。B项错误,因为放弃对军队领导权是陈独秀的右倾投降主义错误的表现。
\end{solution}
\question ``因为中国民族资产阶级根本上与剥削农民的豪绅地主相联结相吻合,中国革命要推翻豪绅地主阶级,便不能不同时推翻资产阶级'',这一观点的错误之处在于(
)
\par\twoch{忽视了反对帝国主义的必要性}{未能区分中国资产阶级的两部分}{\textcolor{red}{混淆了民主革命和社会主义革命的任务}}{不承认中国资产阶级与地主阶级的区别}
\begin{solution}民主革命的对象是封建主义;社会主义革命的对象是资产阶级,所以这个观点的错误在于混淆了两种革命之间区别。
\end{solution}

\subsection{078-遵义会议与长征胜利}
\question 中共中央政治局与1935年1月15日至17日在遵义召开了扩大会议,史称``遵义会议''。下列关于遵义会议表述正确的是
\par\fourch{遵义会议创造性地解决了在农村环境中、在党组织和军队以及农民为主要成分的环境下,如何从加强思想建设人手,保持党的无产阶级先锋队性质和建设党建设的新型人民军队的
	问题}{\textcolor{red}{遵义会议表明:作为一个严肃的、对人民负责的马克思主义政党,中国共产党敢于正视自己的错误,并注意从自己所犯的错误中学习并汲取教训的}}{\textcolor{red}{遵义会议开始确立以毛泽东为代表的马克思主义的正确路线在中共中央的领导地位}}{\textcolor{red}{遵义会议集中解决了当时具有决定意义的军事问题和组织问题}}
\begin{solution}A项是古田会议的内容。
\end{solution}
\question 遵义会议是党的历史上一个生死攸关的转折点,经过激烈的争论,多数人同意以毛泽东为代表的正确意见,批评了博古、李德在第五次反``围剿''中的错误。会议通过了一系列重大决策,这些决策
\par\fourch{\textcolor{red}{是在中国共产党同共产国际的联系中断的情况下独立自主地作出的}}{\textcolor{red}{集中解决了当时具有决定意义的军事问题和组织问题}}{教育了广大的共产党员和红军指战员,他们开始产生对错误领导的怀疑、不满}{着重阐述了党必须依靠农民和掌握枪杆子的思想}
\begin{solution}【解析】1934年10月中旬,中共中央机关和中央红军(又称红一方面军8.6万人撤离根据地,向西突围转移,开始长征。长征初期,中共中央领导人博古依靠与共产国际有关系的军事顾问、德国人李德,犯了退却中的逃跑主义错误。在强渡湘江之后,红军和中央机关人员锐减到3万多人。严酷的事实教育了广大的共产党员和红军指战员,他们开始产生对错误领导的怀疑、不满。一些支持过``左''倾错误的中央领导人如张闻天、王稼祥等,也改变态度,转而支持毛泽东的正确主张。这样,当中央红军根据毛泽东的提议,改向敌人力量薄弱的贵州挺进,并在占领黔北重镇遵义之后,中共中央政治局于1935年1月15日至17日在这里召开了扩大会议(史称``遵义会议'')。C项是遵义会议出台一系列政策的背景。
遵义会议的政策集中解决了当时具有决定意义的军事问题和组织问题。经过激烈的争论,多数人同意以毛泽东为代表的正确意见,批评了博古、李德在第五次反``围剿''中的错误。会议増选毛泽东为中央政治局常务委员并委托张闻天起草《中央关于反对敌人五次``围剿''的总结的决议》(即遵义会议决议)。会后不久,中共中央政治局常务委员分工,根据毛泽东的提议,决定由张闻天代替博古负总的责任;博古任红军总政治部代理主任;并成立了由周恩来、毛泽东、王稼祥组成的新的三人团,全权负责红军的军事行动。会议的一系列重大决策,是在中国共产党同共产国际的联系中断的情况下独立自主地作出的。
遵义会议开始确立以毛泽东为代表的马克思主义的正确路线在中共中央的领导地位,从而在极其危急的情况下挽救了中国共产党、挽救了中国工农红军、挽救了中国革命,遵义会议是党的历史上一个生死攸关的转折点,它标志着中国共产党在政治上开始走向成熟。
D项是毛泽东在八七会议上着重阐述的思想,强调党``以后要非常注意军事,须知政权是由枪杆子中取得的'',实际上提出了以军事斗争作为党的工作重心的问题。
\end{solution}
\question 中国工农红军的长征是一部伟大的革命英雄主义的史诗。它向全中国和全世界宣告,中国共产党及其领导的人民军队,是一支不可战胜的力量。红军长征,铸就了伟大的长征精神。长征精神,一个重要内容就是坚持独立自主、实事求是,一切从实际出发的精神。在长征路上中共中央同错误路线进行的斗争有
\par\twoch{反对罗明路线的斗争}{\textcolor{red}{同红四方面军领导人张国焘分裂中央、分裂红军的严重错误进行斗争}}{\textcolor{red}{与博古、李德在第五次反“围剿〃中的逃跑主义路线的斗争}}{与党内在统一战线问题上的〃左”倾关门主义的错误倾向的斗争}
\begin{solution}BC
1934年10月中旬,中共中央机关和中央红军(又称红一方面军)8.6万人撤离根据地,向西突围转移,开始长征。长征初期,中共中央领导人博古依靠与共产国际有关系的军事顾问、德国人李德,犯了退却中的逃跑主义错误。在强渡湘江之后,红军和中央机关人员锐减到3万多人。严酷的事实教育了广大的共产党员和红军指战员,他们开始产生对错误领导的怀疑、不满。一些支持过〃左''倾错误的中央领导人如张闻天、王稼祥等,也改变态度,转而支持毛泽东的正确主张。这样,当中央红军根据毛泽东的提议,改向敌人力量薄弱的贵州挺进,并在占领黔北重镇遵义之后,中共中央政治局于1935年1月15日至17日在这里召开了扩大会议(史称``遵义会议'')。遵义会议集中解决了当时具有决定意义的军事问题和组织问题。经过激烈的争论,多数人同意以毛泽东为代表的正确意见,批评了博古、李德在第五次反``围剿〃中的错误。遵义会议后,中共中央又同红四方面军领导人张国焘分裂中央、分裂红军的严重错误进行了坚决的斗争。
1935年12月25日,中共中央在陕北瓦窑堡召开政治局扩大会议,提出了建立抗日民族统一战线的方针,批评了党内长期存在的〃左''倾冒险主义、关门主义的错误倾向,制定了抗日民族统一战线的策略方针,为抗日战争的到来做了思想上和理论上的准备。中国共产党在新的历史时期即将到来时掌握了政治上的主动权。反对〃罗明路线〃的斗争是1933年初,由于白区党的工作遭到严重破坏,临时中央政治局无法在上海立足,被迫迀到中央根据地。为了全面推行〃左''倾教条主义的方针、政策展开的一场反对毛泽东的正确主张的斗争。AD都是干扰项。
\end{solution}

\subsection{079-西安事变与抗日民族统一战线的形成}
\question 毛泽东对孙中山晚年思想转变予以高度评价:"孙中山先生之所以伟大,不但因为他领导了伟大的辛亥革命(虽然是旧时期的民主革命),而且因为他能够`适乎世界之潮流,合乎人群之需要',提出了联俄、联共、扶助农工三大革命政策,对三民主义作了新的解释,新三民主义相对于旧三民主义的进步性体现在
\par\fourch{\textcolor{red}{突出了反帝的内容,强调对外实行中华民族的独立,同时主张国内各民族一律平等}}{强调政治革命应当与民族革命并行。民族革命是扫除"现在的恶劣政治",而政治革命则是扫除〃恶劣政治的根本〃}{\textcolor{red}{指出民主权利应〃为一般平民所共有〃,不应为〃少数人所得而私有〃}}{\textcolor{red}{把民生主义概括为〃平均地权〃和〃节制资本〃两大原则并提出要改善工农的生活状况}}
\begin{solution}ACD
1924年1月,国民党一大在广州召开,大会通过的宣言对三民主义作出了新的解释:在民族主义中突出了反帝的内容,强调对外实行中华民族的独立,同时主张国内各民族一律平等;在民权主义中强调了民主权利应``为一般平民所共有〃,不应为〃少数入所得而私有〃把民生主义概括为〃平均地权〃和〃节制资本〃两大原则(后来又提出了〃耕者有其田〃的主张),并提出要改善工农的生活状况。这个新三民主义的政纲同中共在民主革命阶段的纲领基本一致,因而成为国共合作的政治基础。旧三民主义已经指出政治革命的目的是建立民国。《军政府宣言》指出:''凡为国民皆平等以有参政权。大总统由国民公举。议会以国民公举之议员构成之。制定中华民国宪法,人人共守。敢有帝制自为者,天下共击之!〃政治革命应当与民族革命并行。民族革命是扫除``现在的恶劣政治'',而政治革命则是扫除``恶劣政治的根本'',从而把斗争矛头直接指向集国内民族压迫与封建专制统治于一身的清政府。故B项是干扰项。
\end{solution}
\question 抗日民族统一战线正式形成的标志是( )
\par\twoch{国民党五届三中全会的召开}{西安事变的和平解决}{\textcolor{red}{《中国共产党为公布国共合作宣言》的发表}}{\textcolor{red}{蒋介石在庐山发表谈话,承认中共的合法地位}}
\begin{solution}本题是一个干扰性极强的知识点题目,大多数同学本题失误都是误选了B选项,诚然,西安事变应该是这几个选项里面最出名的一个事情了,尤其是没怎么复习的同学,可能也就知道西安事变,然后就选了,就错了。西安事变的和平解决成为时局转换的枢纽,十年内战的局面由此结束,国内和平基本实现。而CD选项的表述,才是统一战线形成的标志。
\end{solution}
\question 以国共两党第二次合作为基础的抗日民族统一战线正式建立的标志有( ~)
\par\fourch{\textcolor{red}{国民党中央通讯社发表《中共中央为公布国共合作宣言》}}{\textcolor{red}{蒋介石在庐山发表实际上承认了中国共产党的合法地位的讲话}}{国民党五届三中全会议的召开}{中共洛川会议通过《抗日救国十大纲领》}
\begin{solution}记忆型题目。
\end{solution}
\question 以国共两党第二次合作为基础的抗日民族统一战线正式形成的标志是
\par\fourch{1936 年中共把“反蒋抗日”口号转变为“逼蒋抗日”。}{1936 年西安事变的和平解决}{\textcolor{red}{1937 年国民党中央通讯社发表《中国共产党为公布国共合作宣言》}}{\textcolor{red}{1937 年蒋介石发表谈话,实际承认共产党合法}}
\begin{solution}记忆类题目,B选项标志着国内和平基本实现。
\end{solution}

\subsection{080-正面战场与敌后战场}
\question 在抗日战争的战略相持阶段,主要的抗日作战方式是
\par\twoch{国民党的正面战场}{共产党的正面战场}{\textcolor{red}{敌后游击战争}}{敌后游击战争和国民党正面战场相结合}
\begin{solution}考察抗日战争三个阶段的特点,记忆类题目。
\end{solution}

\subsection{081-共产党维护统一战线的策略}
\question 1941年1月,震惊中外的皖南事变爆发后,《新华日报》刊出周恩来的题词手迹:
``为江南死国难者致哀!''``千古奇冤,江南一叶;同室操戈,相煎何急?!''大敌当
前,中国共产党以民族利益为重,坚持正确的方针和原则,避免了抗日民族统一战
线的破裂。这些方针和原则包括( )
\par\twoch{\textcolor{red}{又联合又斗争}}{\textcolor{red}{有理、有利、有节}}{针锋相对,寸土必争}{\textcolor{red}{发展进步势力,争取中间势力,孤立顽固势力}}
\begin{solution}为了抗日民族统一战线的坚持、扩大和巩固,中国共产党制定了``发展进步
势力,争取中间势力,孤立顽固势力''的策略总方针。以蒋介石集团为代表的国民
党亲英美派采取两面政策,既主张抗日,又防共、限共、溶共、反共。为此,共产
党必须以革命的两面政策来应对,即贯彻又联合又斗争的政策,同顽固派作斗争,
坚持有理、有利、有节的原则。因此,正确答案为ABD。
\end{solution}
\question 抗日战争进入相持阶段以后,团结抗战的局面逐步发生严重危机,1939年1月,国民党五届五中全会决定成立``防共委员会'',确定了``防共、限共、溶共、反共''的方针。为避免抗日民族统一战线内部分裂,中共中央提出的应对危机的口号是(
)
\par\twoch{\textcolor{red}{力求全国进步,反对向后倒退}}{发展进步势力,孤立顽固势力}{\textcolor{red}{巩固国内团结,反对内部分裂}}{\textcolor{red}{坚持抗战到底,反对中途妥协}}
\begin{solution}针对国民党的消极抗日,中共提出``力求全国进步,反对向后倒退'',``巩固国内团结,反对内部分裂''``坚持抗战到底,反对中途妥协''。属于大纲识记内容。
\end{solution}
\question 毛泽东曾指出:``在中国,这种中间势力有很大的力量,往往可以成为我们同顽固派斗争时决定胜负的因素,因此,必须对他们采取十分慎重的态度。''在实际工作中,毛泽东始终将争取中间势力作为中国共产党在抗日民族统一战线中的一项极其严重的任务,他认为争取中间势力需要一定的条件,包括(
)
\par\twoch{\textcolor{red}{共产党要有足够的力量}}{\textcolor{red}{尊重他们的利益}}{\textcolor{red}{要同顽固派作坚决的斗争,并能一步一步地取得胜利}}{采取“有理、有利、有节”策略}
\begin{solution}中共针对中间势力是争取中间势力,对顽固势力才是``有理、有利、有节''的策略。
\end{solution}
\question 1941年1月,震惊中外的皖南事变爆发后,《新华日报》刊出周恩来的题词手迹:``为江南死国难者致哀。''``千古奇冤,江南一叶,同室操戈,相煎何急?''大敌当前,中国共产党以民族利益为重,坚持正确的方针和原则,避免了抗日民族统一战线的破裂,这些方针和原则有
\par\twoch{\textcolor{red}{既联合又斗争}}{\textcolor{red}{有理,有利,有节}}{针锋相对,寸土必争}{\textcolor{red}{发展进步势力,中间势力,孤立顽固势力}}
\begin{solution}本题考查中国共产党巩固抗日民族统一战线的方针和原则。抗日战争时期,中国共产党处理民族矛盾和阶级矛盾的原则是:阶级矛盾服从于民族矛盾。因此,皖南事变后,中国共产党为了巩固统一战线,争取更多的人参加抗日,采取的方针和原则有:又联合又斗争,发展进步势力,争取中间势力,孤立顽固势力的策略总方针,以及同顽固派的斗争,坚持有理、有利、有节的策略原则。因此,选项ABD正确。选项C属于当时国内处理阶级斗争的对策。
\end{solution}

\subsection{082-根据地建设}
\question 抗日战争时期的``三三制''政权
\par\fourch
{\textcolor{red}{是指抗日民主政府在工作人员分配上实行“三三制”原则,即共产党员、非党的左派进步分子和不左不右的中间派各占1/3}}
{\textcolor{red}{是抗日民族统一战线性质的政权,有利于结成最广泛抗日民族统一战线}}
{\textcolor{red}{是一切赞成抗日又赞成民主的人们的政权}}
{\textcolor{red}{是敌后抗战的最好政权形式}}
\begin{solution}基本背诵内容
\end{solution}
\question 决定将中国共产党在抗日战争时期实行的减租减息政策改变为实现``耕者有其田''政策的是(
)
\par\twoch{《中国土地法大纲》}{\textcolor{red}{《关于清算、减租及土地问题的指示》}}{《兴国土地法》}{《井冈山土地法》}
\begin{solution}1946年5月4日,中共中央发出《关于清算、减租及土地问题的指示》(史称《五四指示》),决定将党在抗日战争时期实行的减租减息政策改变为实现``耕者有其田''的政策。《中国土地法大纲》,明确规定废除封建性及半封建性剥削的土地制度,实现耕者有其田的土地制度;《兴国土地法》、《井冈山土地法》制定于土地革命战争时期。故B正确。
\end{solution}

\subsection{083-党的建设与延安整风}
\question 中国共产党在20世纪40年代前期开展的整风运动,被认为是``深刻影响二十世纪中国历史进程的重大事件,这场整风运动(
)
\par\twoch{最主要的任务是反对宗派主义}{\textcolor{red}{是一场伟大的思想解放运动}}{确立了毛泽东思想为全党的指导思想}{\textcolor{red}{使马克思主义思想路线在全党范围内确立起来}}
\begin{solution}延安整风主要针对的是教条主义,而不是宗派主义。七大正式确立了毛泽东思想为全党的指导思想。BD是正确选项。
\end{solution}
\question 1942年,毛泽东在《整顿党的作风》中指出,我们要的是马克思列宁主义的学风。学风问题主要是指
\par\twoch{对待知识分子的态度问题}{\textcolor{red}{领导机关、全体干部、全体党员的思想方法问题}}{\textcolor{red}{我们对待马克思列宁主义的态度问题}}{\textcolor{red}{全党同志的工作态度问题}}
\begin{solution}本题考查考生对延安整风运动内容的掌握。毛泽东在《整顿党的作风》中指出:所谓学风,不但是学校的学风,而且是全党的学风。学风问题是领导机关、全体干部、全体党员的思想方法问题,是我们对待马克思列宁主义的态度问题,是全党同志的工作态度问题。因此,备选项BCD符合题干要求,为本题正确答案。
\end{solution}
\question 1941 年以后党内开展的整风运动的最主要任务是
\par\twoch{反对宗派主义}{反对官僚主义}{反对党八股}{\textcolor{red}{反对主观主义}}
\begin{solution}记忆类题目。
\end{solution}

\subsection{084-抗战的胜利与中国在二战中的地位}
\question 钓鱼岛及其附属岛屿是中国领土不可分割的一部分。中国最早发现、命名、利用和管辖钓鱼岛。1895年,清朝在甲午战争中战败,被迫
与日本签署不平等的《马关条约》,割让``台湾全岛及所有附属各岛屿''。钓鱼岛等作为台湾``附属岛屿''一并被割让给日本。1941年12月,中国政府正式对日宣战,宣布废除中日之间的一切条约。日本投降后,依据有关国际文件规定,钓鱼岛作为台湾的附属岛屿应与台湾一并归还中国。这些国际文件是
\par\twoch{\textcolor{red}{《波茨坦公告》}}{\textcolor{red}{《开罗宣言》}}{\textcolor{red}{《日本投降书》}}{《德黑兰宣言》}
\begin{solution}本题考查的是钓鱼岛作为台湾附属岛屿规划中国的国际文件。1941年12月,中国政府正式对日宣战,宣布废除中日之间的一切条约。1943年12月《开罗宣言》明文规定,``日本所窃取于中国之领土,例如东北四省、台湾、澎湖群岛等,归还中华民国。其他日本以武力或贪欲所攫取之土地,亦务将日本驱逐出境''。1945年7月《波茨坦公告》第八条规定:``《开罗宣言》之条件必将实施,而日本之主权必将限于本州、北海道、九州、四国及吾人所决定之其他小岛。''1945年9月2日,日本政府在《日本投降书》中明确接受《波茨坦公告》,并承诺忠诚履行《波茨坦公告》各项规定。上述事实表明,依据《开罗宣言》、《波茨坦公告》和《日本投降书》,钓鱼岛作为台湾的附属岛屿应与台湾一并归还中国。选项D是1943年苏、美、英三国首脑在德黑兰会议结束时发表的宣言,它规定盟国在西欧开辟第二战场,实行``霸王战役''计划,发动``诺曼底登陆''的时间,与台湾问题无关。因此,本题的正确答案是ABC。
\end{solution}
\question 第二次世界大战期间,明确规定将台湾、澎湖列岛归还中国的有关的是
\par\twoch{《德黑兰宣言》}{\textcolor{red}{《开罗宣言》}}{《雅尔塔协定》}{\textcolor{red}{《波茨坦公告》}}
\begin{solution}二战期间,涉及到台湾问题的国际条约有1943年的《开罗宣言》和1945年的《波茨坦公告》,它们都明确规定将澎湖列岛归还给中国。选项AC不符合题意,故不选。所以正确答案为BD。
\end{solution}

\subsection{085-抗日战争胜利的意义、原因和基本经验}
\question 2015年是中国人民抗日战争胜利70周年:抗日战争的伟大胜利,是近代以来中国反抗外敌入侵的第一次完全胜利,为中华民族由近代以来陷人深重危机走向伟大复兴确立了历史转折点。
这一伟大胜利的意义在于
\par\fourch{\textcolor{red}{彻底粉碎了日本军国主义殖民奴役中国的图谋}}{实现了民族独立和人民解放}{\textcolor{red}{重新确立了中囯在世界上的大国地位}}{\textcolor{red}{开辟了中华民族伟大复兴的光明前景}}
\begin{solution}经过长达14年的抗战,中国人民打败了日本侵略者,宣告了日本军国主义的彻底失败,宣告了中国人民抗日战争和世界反法西斯战争的最后胜利。中国人民抗日战争,是近代以来中国抗击外敌入侵的第一次完全胜利。这一伟大胜利,彻底粉碎了日本军国主义殖民奴役中国的图谋,洗刷了近代以来抗击外来侵略屡战屡败的民族耻辱。这一伟大胜利,重新确立了中国在世界上的大国地位,使中国人民赢得了世界爱好和平人民的尊敬。这一伟大胜利,开辟了中华民族伟大复兴的光明前景,开启了古老中国凤凰涅槃、浴火重生的新征程。A、C、D正确。但是,抗日战争的胜利并不意味着中国已经实现了民族独立和人民解放。抗日战争胜利后,中国广大人民热切希望实现和平、民主,为建立新中国而奋斗。而国民党统治集团坚持独裁统治,继续走半殖民地半封建社会的老路。中国人民在经过三年的解放战争后,才建立了新中国,真正实现了民族独立和人民解放。B错误。
\end{solution}

\subsection{086-新民主主义社会的性质}
\question 1949年中华人民共和国的成立,标志着我国社会进入了由新民主主义到社会主义的过渡时期。新民主主义社会的基本特征是
\par\fourch{\textcolor{red}{政治制度是工人阶级领导的以工农联盟为基础的各革命阶级联合专政}}{经济制度是公有制为主体、多种所有制经济共同发展}{\textcolor{red}{文化上发展以马克思主义指导下的新民主主义文化}}{国内主要矛盾是工人阶级和资产阶级的矛盾}
\begin{solution}【答案】AC
【简析】新民主义社会在政治上实行工人阶级领导的,一工农联盟为基础的,各革命阶级联合专政的政治制度;在经济实行国营经济领导下地合作社经济、个体经济、私人资本主义经济和国家资本主义经济五种经济成分并存的经济制度;在文化上发展以马克思主义指导下地新民主义的文化,即名族的、科学的、大众的文化A、C正确,B错误。新民主主义社会国内主要矛盾有个变化过程,1952年底,随着土地改革的基本完成,工人阶级和资产阶级的矛盾才逐步成为国内的主要矛盾,D错误。
\end{solution}

\subsection{087-对农业手工业的改造}
\question 中国共产党在推进手工业合作化的过程中,采取的方针是( ~)
\par\twoch{\textcolor{red}{积极引导}}{典型示范}{逐步推广}{\textcolor{red}{稳步前进}}
\begin{solution}在推进手工业合作化的过程中,中国共产党采取的是积极引导、稳步前进的方针。
\end{solution}
\question 所谓农业合作化,就是在中国共产党领导下,通过各种互助合作的形式,把以生产资料私有制为基础的个体农业经济,改造为以生产资料公有制为基础的农业合作经济的过程。1953年到1956年我国实行农业合作化的主要原因是(
)
\par\fourch{封建土地制度严重阻碍生产力的发展}{\textcolor{red}{小农经济难以满足国民经济发展的需要}}{按苏联模式建设社会主义}{一些领导人片面强调公有化的作用}
\begin{solution}社会主义改造属于变革生产方式,主要原因肯定是因为生产力的发展,小农经济难以满足国民经济发展的需要。
\end{solution}
\question 在推进手工业合作化的过程中,中国共产党采取的方针是( ~)
\par\twoch{自愿互利}{\textcolor{red}{积极引导}}{\textcolor{red}{稳步前进}}{国家帮助}
\begin{solution}自愿互利、国家帮助是农业合作化运动的基本原则。积极领导和稳步前进是手工业合作化过程中,中国共产党采取的方针。
\end{solution}
\question 中共中央在1953年12月通过的《关于发展农业生产合作社的决议》总结互助合作运动的经验,概括提出的过渡性经济组织形式主要是(
)
\par\twoch{\textcolor{red}{互助组}}{\textcolor{red}{初级农业生产合作社}}{\textcolor{red}{高级农业生产合作社}}{生产合作小组}
\begin{solution}农业生产合作社的三种过渡性经济组织形式是互助组、初级农业生产合作社、高级农业生产合作社。
\end{solution}

\subsection{088-社会主义制度的确立及其意义}
\question 社会主义革命的目的是( )
\par\twoch{发展生产力}{\textcolor{red}{解放生产力}}{实现共产主义}{建立社会主义制度}
\begin{solution}社会主义革命是一场生产关系的变革,希望通过生产关系的变革解放生产力。
\end{solution}
\question 社会主义改造是由私有制到公有制的一场伟大的变革,这场变革的实质在于( )
\par\twoch{对社会制度的变革}{对生产力的变革}{对经济成分的变革}{\textcolor{red}{对生产关系的变革}}
\begin{solution}所有制属于生产关系范畴,所以这场变革的实质在于生产关系的变革。
\end{solution}

\subsection{089-马克思主义中国化的提出与内涵}
\question 在中国共产党的历史上,第一次鲜明地提出``马克思主义中国化''的命题和任务的会议是
\par\twoch{党的二大}{遵义会议}{\textcolor{red}{党的六届六中全会}}{党的七大}
\begin{solution}本题考查考生对六届六中全会内容的掌握。在六届六中全会上,毛泽东明确提出马克思主义中国化的口号,号召全党学习马、恩、列、斯的理论,并应用到中国的实际斗争中去。因此,备选项C符合题干要求,为本题答案。备选项A党的二大提出了民主革命纲领;B遵义会议确定了毛泽东在党内的实际地位并独立自主解决了军事问题和组织问题;D党的七大把毛泽东思想确定为党的指导思想。因此备选项ABD不符合题干要求,不是本题正确答案。
\end{solution}
\question 中国革命建设和改革的实践证明,要运用马克思主义指导实践,必须实现
马克思主义中国化,马克思之所以能够中国化的原因在于
\par\twoch{\textcolor{red}{马克思主义理论的内在要求}}{\textcolor{red}{马克思主义与中华民族优秀文化具有相融性}}{\textcolor{red}{中国革命建设和改革的实践需要马克思主义指导}}{马克思主义为中国革命建设和改革提供了现实发展模式}
\begin{solution}马克思之所以能够中国化的原因,首先在于马克思主义中国化是马克思主义理论的内在要求。恩格斯指出:``马克思的整个世界观不是教义,而是方法。它提供的不是现成的教条,而是进一步研究的出发点和供这种研究使用的方法。''各国的马克思主义者的任务就是结合各个国家不同时期的具体实际,将马克思主义进-步加以具体化。因此A选项正确。
马克思主义与中华民族优秀文化具有相融性。马克思主义中国化就是把马克思主义植根于中华民族优秀的思想文化之中,实现马克思主义和民族的特点相结合,并经过一定的民族形式表现出来。因此B选项正确。
马克思之所以能够中国化的原因,还在于马克思主义中国化是解决中国问题的需要。马克思主义中国化就是运用马克思主义解决中国革命、建设和改革的实际问题。在中国这样一个半殖民地半封建的东方大国里进行革命,不能机械套用马克思主义一般原理和照搬外国经验。同样,在中国进行社会主义建设和改革,也不能把马克思主义当做教条,必须紧密结合中国国情和时代条件,使马克思主义在中国具体化。因此C选项正确。
马克思主义具有普遍指导意义,但并不能为中国革命建设和改革提供现成发展模式,因此D选项错误。
\end{solution}


\section{[毛中特]毛泽东思想和中国特色社会主义理论体系}

\subsection{090-毛泽东思想的形成过程}
\question 关于毛泽东思想,下列描述中正确的有
\par\fourch{\textcolor{red}{1945年中共七大,刘少奇在《关于修改党章的报告》中正式命名为毛泽东思想,写进党章}}{\textcolor{red}{毛泽东思想活的灵魂是实事求是、独立自主和群众路线,精髓是实事求是}}{\textcolor{red}{坚持党对军队的绝对领导,是建设新型人民军队的根本原则,也是毛泽东建军思想的核心}}{中国共产党领导的革命建设和改革的实践,是毛泽东思想形成的实践基础}
\begin{solution}【答案】ABC
【解析】本题考查关于毛泽东思想的相关内容。D选项观点错误。在毛泽东思想形成的实践基础中没有``改革'',改革实践是中国特色社会主义理论的实践基础。同学们复习时要注意区分。ABC选项观点正确。三个选项也可以单独命制单项选择题。
\end{solution}
\question 毛泽东思想形成的时代背景和面临的历史任务是( )
\par\twoch{中国沦为半殖民地半封建社会}{\textcolor{red}{民族独立和国家解放}}{\textcolor{red}{帝国主义战争与无产阶级革命的时代}}{建立社会主义制度}
\begin{solution}A选项是毛泽东思想形成时的国情,时代背景是帝国主义战争与无产阶级革命的时代,历史任务是民族独立和国家解放。而不是建立社会主义制度,这是下一阶段的任务。
\end{solution}

\subsection{091-毛泽东思想的内容与评价}
\question 据《北京青年报》报道,清华大学课程《毛泽东思想槪论》于2015年9月中旬受邀登录世界三大慕课平台之一的edX,面向全球开课。
---周之内已有来自130个国家和地区的近3000人选修。这也是国内首门登录国际英文慕课平台的思想政治理论课。开课第一天,来自美国、新西兰、英国等国的同学就在课程论坛上发帖.围绕``马克思主义中国化''等问越展开讨论。毛泽东思想是
\par\fourch{\textcolor{red}{马克思主义中国化第一次历史性飞跃的理论成果}}{中国革命建设和改革经验的科学总结}{\textcolor{red}{中国共产党和中国人民宝贵的精神财富}}{\textcolor{red}{中国特色社会主义理论体系的重要思想渊源}}
\begin{solution}【简析】A、C、D正确,毛泽东思想是在我国新民主主义革命、社会主义革命和社会主义建设的实践中,在总结我国革命和建设正反两方面历史经验的基础上,逐步形成和发展起来的。改革开放是在毛泽东逝世后的1978年以后开始的,B错误。
\end{solution}
\question 毛泽东在《中国革命和中国共产党》中论述了民主革命和社会主义革命的关系。他指出:``民主革命是社会主义革命的必要准备,社会主义革命是民主革命的必然趋势。''这两个革命阶段能够有机连接的原因是
\par\twoch{资本主义道路在中国走不通}{俄国十月革命为中国提供了经验}{\textcolor{red}{民主革命包含了社会主义因素}}{中国国情决定中国革命必须分两步走}
\begin{solution}本题考查民主革命和社会主义革命的关系。
中国共产党领导的革命分为两部分:新民主主义革命和社会主义革命。前者是后者的前提和必要准备,后者是前者发展的必然趋势,两者紧密结合在一起,主要原因是这两者都是由无产阶级领导的。这就决定了中国的民主革命中包含社会主义的因素,并且社会主义因素在其中占主导地位。所以C项正确。
A项错误,资本主义道路在中国走不通只是说明了中国选择社会主义道路的原因,没有解释民主革命与社会主义革命为什么能够结合在一起;B项错误,俄国十月社会主义革命是中国选择社会主义革命的外部条件;D项错误,中国特殊的国情决定了中国革命分两步走,但是这本身并不能解释为什么这两个阶段能够结合在一起。
故正确答案选C。
\end{solution}
\question 在马克思主义中国化第一个重大理论成果------毛泽东思想的指引下,中国共产党领导全国各族人民所取得的伟大成就有(
)
\par\fourch{\textcolor{red}{取得了新民主主义革命的胜利,建立了人民民主专政的中华人民共和国}}{\textcolor{red}{顺利地进行了社会主义改造,确立了社会主义基本制度}}{\textcolor{red}{发展了社会主义的经济、政治和文化,初步探索了社会主义建设道路}}{开辟了建设中国特色社会主义的正确道路}
\begin{solution}毛泽东思想的指引下,中国共产党带领全国人民取得了新民主主义革命的胜利,确立了社会主义基本制度以及初步探索了社会主义建设道路。选项D中国特色社会主义道路是马克思主义中国化的第二次飞跃的内容。
\end{solution}
\question 毛泽东思想是一个完整的科学体系,它的组成部分包括( )
\par\twoch{马克思主义的基本原理}{\textcolor{red}{丰富和发展了马克思列宁主义的许多独创性理论}}{\textcolor{red}{贯穿于其中的一以贯之的立场、观点、方法}}{毛泽东的晚年错误}
\begin{solution}A马克思主义基本原理不属于毛泽东思想体系的内容;D毛泽东思想不包括毛泽东晚年错误。
\end{solution}

\subsection{092-近代中国的基本国情与产生原因}
\question 近代中国半殖民地半封建的社会性质,决定了
\par\fourch{\textcolor{red}{中国革命的动力是工人、农民、小资产阶级和民族资产阶级}}{\textcolor{red}{中国革命的根本任务是推翻三座大山,实现民族独立和人民解放}}{中国革命是世界无产阶级社会主义革命的一部分}{中国革命的前途是经由新民主主义走向社会主义}
\begin{solution}社会性质决定社会主要矛盾,决定革命性质,所以可以决定革命动力和革命的任务。C项是由十月革命决定的,D项是由领导者决定的。
\end{solution}

\subsection{094-新民主主义革命的动力与领导}
\question 民主革命时期,中国革命的基本问题是( )
\par\twoch{\textcolor{red}{农民问题}}{统一战线问题}{无产阶级领导权问题}{武装斗争}
\begin{solution}此题考查的知识点是新民主主义革命的动力中的中国革命的基本问题,是一道识记性试题,难度适中。农民问题是中国革命的基本问题,新民主主义革命实质上就是中国共产党领导下的农民革命,中国革命战争实质上就是党领导下的农民战争,A选项正确。B选项,统一战线是无产阶级政党策略思想的重要内容。C选项,无产阶级领导权问题是中国革命的中心问题,又是新民主主义革命理论的核心问题。D选项,武装斗争是中国革命的特点和优点之一。
\end{solution}
\question 新民主主义革命理论的核心问题是
\par\twoch{分清敌我}{农民问题}{\textcolor{red}{无产阶级领导权}}{土地问题}
\begin{solution}【答案】C
【解析】无产阶级的领导权是中闻革命的中心问題,也是新民主主义取命理论的核心问题。区别新旧两种不同范畴的民主主义革命。根本的标志是革命的领导权掌握在无产阶级手中还是掌握在资产阶级手中。C
正确分清敌我是中国革命的首要问题。农民问题是中国革命的基本问题。土地问题
(没收封建地主阶级的土地归农民所有)是新民主主义革命的主要内容,A、B、D错误。
\end{solution}
\question 中国革命的中心问题,也是新民主主义革命理论的核心问题是
\par\twoch{\textcolor{red}{无产阶级的领导权}}{分清敌我}{没收封建地主阶级的土地归农民所有}{认清中国的国情}
\begin{solution}【答案】A
【解析】B项,分清敌我,乃是革命的首要问题;C项,没收封建地主阶级的土地归农民所有是新民主主义革命的中心内容;D项,认清中国的国情是革命的基本依据。考生一定注意区分这些
说法。无产阶级的领导权是中国革命的中心问题,也是新民主主义革命理论的核心问题。区别新旧两种不同范畴的民主主义革命,根本的标志是革命的领导权掌握在无产阶级手中还是掌握在资产阶级手中。因此,A项符合题干要求。
\end{solution}
\question 中国民主革命的基本问题是( )
\par\twoch{武装斗争问题}{党的建设问题}{统一战线问题}{\textcolor{red}{农民问题}}
\begin{solution}中国民主革命的实质是农民革命,实际上是无产阶级领导下的农民革命。这是因为:农民是中国民主革命的主力军,是无产阶级最可靠的同盟军,反帝反封建的资产阶级民主革命任务要完成就必须发动和依靠农民,推翻封建制度。正是在此意义上说,中国民主革命的基本问题是农民问题。
\end{solution}
\question 小资产阶级是无产阶级的可靠的同盟军,是中国革命的动力之一。小资产阶级主要包括(
)
\par\twoch{\textcolor{red}{知识分子}}{\textcolor{red}{小商小贩}}{\textcolor{red}{手工业者}}{\textcolor{red}{自由职业者}}
\begin{solution}本题考查小资产阶级的定义和范畴。小资产阶级在马克思学说是指介乎资产阶级/资本家及无产阶级者。主要包括广大知识分子、小手工业者、小商人、自由职业者等。小资产阶级占有一小部分生产资料或少量财产,一般既不受剥削也不剥削别人,主要靠自己的劳动为生。但是,其中有一小部分有轻微的剥削。
\end{solution}
\question 中国无产阶级所具有的优点和特点( )
\par\twoch{\textcolor{red}{坚决的斗争性和彻底的革命性}}{\textcolor{red}{分布集中}}{\textcolor{red}{和农民有着天然的联系,便于和农民结成亲密的联盟}}{数量庞大,是中国革命最基本的动力}
\begin{solution}中国无产阶级是中国革命最基本的动力。但是并不是数量庞大,无产阶级的数量非常少。
\end{solution}

\subsection{095-新民主主义革命的前途与两种错误倾向}
\question 关于新民主主义革命与旧民主主义革命相比,下列说法中正确的有
\par\fourch{\textcolor{red}{两者所面临的国情和社会主要矛盾相同}}{两者革命的对象和前途相同}{\textcolor{red}{区分二者的根本标志是革命领导权的不同}}{\textcolor{red}{两者的依靠力量和革命动力不同}}
\begin{solution}【解析】本题考查新旧民主主义对比。B选项说法错误,两者前途不同。新民主主义革命的前途是社会主义,不是资本主义。新民主主义的``新''表现为:一是领导阶级;二是前途;三是指导思想;四是发生的时代条件。其中领导阶级是区分新旧民主主义革命的根本标志和分水岭。
\end{solution}
\question 新民主主义革命时期,党内右倾提出的``二次革命论'',其错误在于( )
\par\fourch{混淆了新民主主义革命和资产阶级革命的界限}{割裂了新民主主义革命和资产阶级革命的联系}{混淆了新民主主义革命和社会主义革命的界限}{\textcolor{red}{割裂了新民主主义革命和社会主义革命的联系}}
\begin{solution}``二次革命论'',其错误在于割裂了新民主主义革命和社会主义革命的联系,注意我国的新民主主义革命是无产阶级领导的资产阶级革命。
\end{solution}

\subsection{096-政治纲领}
\question 新民主主义社会存在五种经济成分,这就是国营经济、合作社经济、个体经济、私人资本主义经济和国家资本主义经济。主要的经济成分有三种:社会主义经济、个体经济和资本主义经济。其中,在国民经济中占绝对优势是
\par\fourch{半社会主义性质的合作社经济}{国家同私人资本合作的国家资本主义国营经济}{通过没收官僚资本而形成的社会主义国营经济}{\textcolor{red}{以农业和手工业为主体的个个体经济}}
\begin{solution}新民主主义社会存在着五种经济成分:社会主义性质的国营经济、半社会主义性质的合作社经济、农民和手工业者的个体经济、私人资本主义经济和国家资本主义经济。其中半社会主义性质的合作社经济是个体经济向社会主义集体经济过渡的形式,国家资本主义经济是私人资本主义经济向社会主义国营经济过渡的形式。所以,主要的经济成分是三种:社会主义经济、个体经济和资本主义经济。在这些经济成分中,通过没收官僚资本而形成的社会主义国营经济,掌握了主要经济命脉,居于领导地位。而以农业和手工业为主体的个体经济,则在国民经济中占绝对优势。
\end{solution}

\subsection{097-文化纲领}
\question 新民主主义文化纲领提出,新民主主义文化是无产阶级领导的人民大众的反帝反封建的文化,即民族的科学的大众的文化。其中,``科学的''应作何理解(
)
\par\twoch{\textcolor{red}{反对封建思想和迷信思想}}{\textcolor{red}{尊重中国的历史,反对民族虚无主义}}{\textcolor{red}{主张实事求是、客观真理及理论和实践的一致性}}{\textcolor{red}{剔除封建时代文化的糟粕,吸收其民主性精华}}
\begin{solution}新民主主义文化是科学的,强调的是科学的内容。它反对一切封建思想和迷信思想,主张实事求是,主张客观真理,主张理论和实践的统一。同时,对于中国古代文化,要剔除其封建性的糟粕,吸取其民主性的精华,决不能无批判地兼收并蓄。
\end{solution}

\subsection{098-新民主主义社会的经济、政治和文化}
\question 新民主主义社会属于社会主义体系,是因为社会主义因素在政治和经济上都居于领导地位,这种领导地位主要体现在(
)
\par\twoch{\textcolor{red}{无产阶级领导的联合专政}}{农业和手工业在国民经济中占绝对优势}{\textcolor{red}{社会主义国营经济掌握了主要经济命脉}}{土地改革基本完成}
\begin{solution}这个里面强调社会因素的重要性表现在什么地方,农业和手工业不属于社会主义经济属于个体经济,体现不出社会因素的领导地位。土地改革是民主革命的任务,也体现不出社会主义因素的重要性。
\end{solution}

\subsection{099-过渡时期总路线的提出过程}
\question 1953年12月,由毛泽东审阅通过的中共中央宣传部编写的《为动员一切力量把我国建设成为一个伟大的社会主义国家而斗争一一关于党在过渡时期总路线的学习和宣传提纲》中指出:我国由新民主主义社会逐步过渡到社会主义社会这一过渡时期之所以必要,并且需要一个相当长的时间,是由于
\par\fourch{新民主主义社会是一个独立的社会形态,要求一个相当长的时期逐步过渡到社会主义社会}{\textcolor{red}{我国经济和文化的落后,要求一个相当长的时期来创造为保证社会主义完全胜利所必要的经济上和文化上的前提}}{\textcolor{red}{我国有极其广大的个体农业和手工业及在国民经济中占很大一部分比重的资本主义工商业,要求一个相当长的时期来改造他们}}{在我国新民主主义社会中,非社会主义的因素不论在经济上还是在政治上都还居于领导地位,要求一个相当长的时期来改造他们}
\begin{solution}BC
1953年12月,由毛泽东审阅通过的中共中央宣传部编写的《为动员一切力量把我国建设成为一个伟大的社会主义国家而斗争一一关于党在过渡时期总路线的学习和宣传提纲》中指出:我国由新民主主义社会逐步过渡到社会主义社会这一过渡时期之所以必要,并且需要一个相当长的时间,是由于:``一、我国经济和文化的落后,要求一个相当长的时期来创造为保证社会主义完全胜利所必要的经济上和文化上的前提;二、我国有极其广大的个体农业和手工业及在国民经济中占很大一部分比重的资本主义工商业,要求一个相当长的时期来改造他们。''在我国新民主主义社会中,社会主义的因素不论在经济上还是在政治上都已经居于领导地位,但非社会主义因素仍有很大的比重。由于社会主义因
素居于领导地位,加上当时有利于发展社会主义的国际条件,决定了社会主义因
素将不断增长并获得最终胜利,非社会主义因素将不断受到限制和改造。为了促进社会生产力的进一步发展,为了实现国家富强、民族复兴、人民幸福,我国新民主主义社会必须适时地逐步过渡到社会主义社会。新民主主义社会是属于社会主义体系的,是逐步过渡到社会主义社会的过渡性质的社会。AD是干扰项。
\end{solution}

\subsection{100-社会主义制度的确立及其意义}
\question 社会主义基本制度在我国初步确立。社会阶级关系发生变化的同时,我国社会的主要矛盾也发生了变化,社会主要矛盾是(
)
\par\twoch{无产阶级同资产阶级的矛盾}{\textcolor{red}{经济文化的发展不能满足人民需要的矛盾}}{走社会主义道路还是走资本主义道路的矛盾}{人民群众同党内腐败分子的矛盾}
\begin{solution}本题考查的知识点是社会主义基本制度的初步确立。1956年年底我国对农业、手工业和资本主义工商业的社会主义改造基本完成,社会主义基本制度在我国初步确立。我国的社会主义初级阶段从这时开始。人民对于经文化迅速发展的需要同经济文化不能满足人民需要的矛盾成为我国社会的主要矛盾。A,C是过渡时期存在的主要矛盾,已经基本解决。D是当前社会存在的矛盾,但不是主要矛盾,主要矛盾仍然是B项的内容。
\end{solution}
\question 1956年我国对农业、手工业和资本主义工商业的社会主义改造基本完成,这标志着(
)
\par\twoch{\textcolor{red}{阶级剥削制度彻底结束}}{\textcolor{red}{社会主义制度逐步确立,我国进入社会主义初级阶段}}{\textcolor{red}{确立了中国共产党领导的人民民主专政}}{\textcolor{red}{社会主义公有制成为我们社会的经济基础}}
\begin{solution}社会主义改造的基本完成,标志着社会主义制度在中国的确立,实现了中国历史上最深刻、最伟大的社会变革,为中国的社会主义现代化建设奠定了基础。
(1)社会主义改造的胜利,在一个几亿人口的大国中,能够比较顺利地实现消灭私有制这样复杂、困难和深刻的社会变革,不但没有造成生产力的破坏,反而促进了工农业和整个国民经济的发展,并且得到人民群众的普遍拥护而没有引起巨大的社会动荡,这的确是伟大的历史性胜利。
(2)社会主义改造的基本完成,我国社会的经济结构发生了根本变化,几千年来以生产资料私有制为基础的阶级剥削制度基本上被消灭,社会主义经济成了国民经济中的主导成分,社会主义经济制度在中国基本确立。它与1954年召开的第一届全国人民代表大会确立的社会主义政治体制一起,完成了历史上最深刻、最伟大的社会变革,中国从新民主主义社会进入社会主义初级阶段。
(3)中国共产党在实践中把马列主义的基本原理同中国社会主义革命的具体实际相结合,创造性地开辟了一条适合中国特点的社会主义改造道路,以新的经验和思想丰富了马克思主义的科学社会主义理论。
(4)社会主义改造的胜利,大大解放了我国的社会生产力,促进了生产力的发展,为社会主义建设的发展,人民生活水平的提高开辟了广阔的前景。
总之,中华人民共和国的成立和社会主义制度的建立,是20世纪中国历史上的第二次历史性巨变。这是中国从古未有的人民革命的大胜利,为中国的社会主义现代化建设创造了前提,奠定了基础。
在社会主义改造之前我们的政权是无产阶级领导的革命阶级的联合专政,从此之后就是无产阶级民主专政。
\end{solution}

\subsection{101-改革的性质}
\question 我国实行改革开放政策的理论基础是( )
\par\twoch{\textcolor{red}{社会主义社会基本矛盾学说}}{社会主义社会初级阶段论}{社会主义社会主要矛盾论}{我国经济发展所遇到的问题}
\begin{solution}本题考查的知识点是社会主义社会的基本矛盾。社会主义社会的基本矛盾(生产力和生产关系,经济基础和上层建筑的矛盾)是推动社会主义社会不断前进的根本动力。邓小平把社会基本矛盾、主要矛盾和根本任务统一起来,指出解决社会主义初级阶段主要矛盾的途径是改革。
\end{solution}

\subsection{102-和平统一、一国两制思想的提出、内容与意义}
\question 钓鱼岛及其附属岛屿是中国领土不可分割的一部分。中国最早发现、命名、利用和管辖钓鱼岛。1895年,清朝在甲午战争中战败,被迫
与日本签署不平等的《马关条约》,割让``台湾全岛及所有附属各岛屿''。钓鱼岛等作为台湾``附属岛屿''一并被割让给日本。1941年12月,中国政府正式对日宣战,宣布废除中日之间的一切条约。日本投降后,依据有关国际文件规定,钓鱼岛作为台湾的附属岛屿应与台湾一并归还中国。这些国际文件是
\par\twoch{\textcolor{red}{《波茨坦公告》}}{\textcolor{red}{《开罗宣言》}}{\textcolor{red}{《日本投降书》}}{《德黑兰宣言》}
\begin{solution}本题考查的是钓鱼岛作为台湾附属岛屿规划中国的国际文件。1941年12月,中国政府正式对日宣战,宣布废除中日之间的一切条约。1943年12月《开罗宣言》明文规定,``日本所窃取于中国之领土,例如东北四省、台湾、澎湖群岛等,归还中华民国。其他日本以武力或贪欲所攫取之土地,亦务将日本驱逐出境''。1945年7月《波茨坦公告》第八条规定:``《开罗宣言》之条件必将实施,而日本之主权必将限于本州、北海道、九州、四国及吾人所决定之其他小岛。''1945年9月2日,日本政府在《日本投降书》中明确接受《波茨坦公告》,并承诺忠诚履行《波茨坦公告》各项规定。上述事实表明,依据《开罗宣言》、《波茨坦公告》和《日本投降书》,钓鱼岛作为台湾的附属岛屿应与台湾一并归还中国。选项D是1943年苏、美、英三国首脑在德黑兰会议结束时发表的宣言,它规定盟国在西欧开辟第二战场,实行``霸王战役''计划,发动``诺曼底登陆''的时间,与台湾问题无关。因此,本题的正确答案是ABC。
\end{solution}
\question 第二次世界大战期间,明确规定将台湾、澎湖列岛归还中国的有关的是
\par\twoch{《德黑兰宣言》}{\textcolor{red}{《开罗宣言》}}{《雅尔塔协定》}{\textcolor{red}{《波茨坦公告》}}
\begin{solution}二战期间,涉及到台湾问题的国际条约有1943年的《开罗宣言》和1945年的《波茨坦公告》,它们都明确规定将澎湖列岛归还给中国。选项AC不符合题意,故不选。所以正确答案为BD。
\end{solution}

\subsection{103-马克思主义的创立}
\question 标志着马克思主义基本形成的论著是
\par\twoch{\textcolor{red}{《关于费尔巴哈的提纲》}}{\textcolor{red}{《德意志意识形态》}}{《哲学的贫困》}{《共产党宣言》}
\begin{solution}本题是马克思主义基本原理概论第一章与科学社会主义理论的新增考点。《关于费尔巴哈的提纲》和《德意志意识形态》标志着马克思主义的基本形成。《哲学的贫困》和《共产党宣言》的发表标志着马克思主义的公开问世。故选AB。
\end{solution}
\question 马克思和恩格斯进一步发展和完善了英国古典经济学理论是( )
\par\twoch{辩证法}{历史观}{\textcolor{red}{劳动价值论}}{剩余价值论}
\begin{solution}劳动价值论是批判地继承和吸收英国古典政治经济学的成果。D是马克思主义的独创。A,B的理论来源是德国古典哲学。
\end{solution}
\question 马克思主义公开问世的标志是( )
\par\twoch{《德意志意识形态》的出版}{《资本论》的出版}{《反杜林论》的出版}{\textcolor{red}{《共产党宣言》的公开发表}}
\begin{solution}1848年2月,《共产党宣言》公开发表,标志着马克思主义的形成。
\end{solution}
\question 马克思恩格斯最重要的理论贡献是( )
\par\twoch{辩证法}{劳动价值论}{\textcolor{red}{唯物史观}}{\textcolor{red}{剩余价值学说}}
\begin{solution}马克思恩格斯最重要的理论贡献即马克思恩格斯所创立的理论是唯物史观和剩余价值学说。
\end{solution}
\question 马克思恩格斯之所以能够实现社会主义思想从空想到科学的飞跃,是因为他们独创了(
)
\par\twoch{劳动价值论}{\textcolor{red}{剩余价值学说}}{\textcolor{red}{历史唯物主义}}{辩证法}
\begin{solution}劳动价值论是古典政治经济学的贡献,马克思是做了继承。辩证法古已有之,黑格尔将其发展到比较成熟的水平。马克思的贡献是在劳动价值论的基础上创立了剩余价值学说,揭示了资本主义剥削的秘密;将辩证唯物主义运用到社会领域,创立了历史唯物主义,揭示了人类社会发展的一般规律。在唯物史观和剩余价值论两大发现的基础上,马克思阐明了由资本主义社会转变为社会主义、共产主义社会的客观规律,阐明了无产阶级获得彻底解放的历史条件和无产阶级的历史使命,从而使社会主义由空想成为科学。
\end{solution}

\subsection{104-马克思主义的鲜明特征}
\question 马克思主义最鲜明的政治立场是致力于实现以劳动人民为主体的最广大人民的根本利益,是否始终站在最广大人民的立场上是(
)
\par\twoch{马克思主义的根本特性}{无产阶级的历史使命}{\textcolor{red}{唯物史观与唯心史观的分水岭}}{\textcolor{red}{判断马克思主义政党的试金石}}
\begin{solution}本题考查的知识点:马克思主义最鲜明的政治立场
马克思主义认为,人民群众是历史的创造者,人民群众的根本利益体现了社会发展的要求和方向。在社会历史发展过程中,人民群众起着决定性的作用。人民群众是历史的主体,是历史的创造者。首先,人民群众是物质财富的创造者。其次,人民群众是社会精神财富的创造者。再次,人民群众是社会变革的决定力量。最后,人民群众既是先进生产力和先进文化的创造主体,也是实现自身利益的根本力量。马克思主义政党的一切理论和奋斗,都应致力于实现最广大人民的根本利益。因此,是否始终站在最广大人民的立场上,是唯物史观与唯心史观的分水岭,也是判断马克思主义政党的试金石。所以,C、D两项是符合题意的正确选项。A、B两项不符合题意,不选。马克思主义的根本特性是鲜明的阶级性和实践性。无产阶级的历史使命是彻底解放全人类,实现每个人自由而全面发展的共产主义社会。
\end{solution}
\question 判断唯物史观与唯心史观的{分水岭}的是
\par\fourch{把实践当做物质性的活动}{社会存在和社会意识何者第一}{\textcolor{red}{是否始终站在最广大人民的立场上}}{从实践出发去理解社会生活的本质}
\begin{solution}【解析】C
项,是否始终站在最广大人民的立场上,是唯物史观与唯心史观的分水岭,也是判断马克思主义政党的试金石。
\end{solution}
\question 马克思主义哲学与唯心主义哲学、旧唯物主义哲学的根本区别在于
\par\twoch{坚持人的主体地位}{坚持用辩证发展的观点去认识世界}{坚持物质第一性、意识第二性}{\textcolor{red}{坚持从客观的物质实践活动去理解现实世界}}
\begin{solution}(1)本题考查马克思主义哲学是科学的世界观和方法论中马克思主义哲学的基本特征的理解。
(2)马克思主义哲学即辩证唯物主义和历史唯物主义的创立是哲学发展史上的伟大变革。马克思主义哲学的创始人马克思、恩格斯对实践概念的科学规定和实践观点的确立是实现哲学上伟大变革的关键。马克思指出:``从前的一切唯物主义(包括费尔巴哈的唯物主义)的主要缺点是:对象、现实、感性,只是从客体的或者直观的形式去理解,而不是把它们当作感性的人的活动,当作实践去理解,不是从主体方面去理解。''``和唯物主义相反,唯心主义却发展了能动的方面,但只是抽象地发展了,因为唯心主义当然是不知道现实的感性的活动本身的。''``哲学家们只是用不同的方式解释世界,问题在于改造世界。''实践的观点是马克思主义哲学的根本特征,是马克思主义哲学同以往哲学包括旧唯物主义和唯心主义哲学的根本区别,实践范畴是马克思主义哲学体系的中心范畴。
(3)D选项是符合试题要求的正确观点。A选项是人本主义的观点,B选项是唯物辩证法和唯心辩证法的共同观点,C选项是一切唯物主义的共同观点。故无论采用正选法还是排谬法,正确选项只能是D。
\end{solution}

\subsection{105-马克思主义基本原理}
\question 19
世纪中叶,马克思恩格斯把社会主义由空想变为科学,奠定这一飞跃的理论基石是
\par\twoch{阶级斗争学说}{劳动价值论}{\textcolor{red}{唯物史观}}{\textcolor{red}{剩余价值理论}}
\begin{solution}CD
两项,科学社会主义在唯物史观和剩余价值论两大发现的基础上,阐明了由资本主义社会转变为社会主义、共产主义社会的客观规律,阐明了无产阶级获得彻底解放的历史条件和无产阶级的历史使命,使社会主义由空想成为科学。
\end{solution}

\subsection{106-自觉学习和运用马克思主义}
\question 判断唯物史观与唯心史观的{分水岭}的是
\par\fourch{把实践当做物质性的活动}{社会存在和社会意识何者第一}{\textcolor{red}{是否始终站在最广大人民的立场上}}{从实践出发去理解社会生活的本质}
\begin{solution}【解析】C
项,是否始终站在最广大人民的立场上,是唯物史观与唯心史观的分水岭,也是判断马克思主义政党的试金石。
\end{solution}

\subsection{107-唯物主义和唯心主义}
\question 形而上学唯物主义物质的缺陷在于( )
\par\fourch{\textcolor{red}{把质上无限多样的物质世界归结为粒子在量上的不同}}{\textcolor{red}{把某种特殊的物质形态误认为物质的一般特征}}{\textcolor{red}{不了解人类对物质的认识是一个永无止境的发展过程}}{\textcolor{red}{割裂了自然界与人类社会的物质统一性}}
\begin{solution}此题考查的知识点是形而上学唯物主义的缺陷。具体表现在:机械性、形而上学性、不彻底性。
它没有把唯物主义贯彻到社会历史领域,即自然观上是唯物主义的,在历史观上是唯心主义的。形而上学
唯物主义的根本缺陷就在于不了解人的实践活动的物质性,不知道实践在人与世界关系中的意义和作用,
抹杀了人及其意识的能动性。ABCD都是正确选项。
\end{solution}

\subsection{108-全面依法治国的基本格局}
\question ``科学立法、严格执法、公正司法、全民守法''是全面依法治国的基本格局。其中,加快建设法治政府是
\par\twoch{科学立法的目标}{\textcolor{red}{严格执法的目标}}{公正司法的目标}{全民守法的目标}
\begin{solution}【简析】全面依法治国的基本格局是``科学立法、严格执法、公正司法、全民守法''。
科学立法以完善以宪法为核心的中国特色社会主义法律体系,加强宪法实施为目标;;严格执法以深入推进依法行政,加快建设法治政府为目标;全民守法以增强全民法治观念,推进法治社会建设为目标。B正确,A、C、D不符题意。
\end{solution}

\subsection{109-唯物辩证法与“四个全面”战略思想}
\question 习近平总书记在上海考察调研时表示,``谁牵住了科技创新这个牛鼻子,谁走好了科技创新这步
先手棋,谁就能占领先机、贏得优势''。这里的``牛鼻子''指的是
\par\fourch{事物矛盾的特殊性}{\textcolor{red}{事物发展中的主要矛盾}}{认识指导下的实践}{实践基础上的理论}
\begin{solution}曾有俗语``牵牛要牵牛鼻子''就是强调善于抓住事物的主要矛盾。本题习近平指出``牵住科技创新这个牛鼻子''就是强调把科技创新作为经济发展的主要矛盾。故选B。
\end{solution}
\question 习近平总书记在湖南考察时强调,``我国经济发展要突破瓶颈、解决深层次矛盾和问题,根本出路在于创新,关键是要靠科技力量''。这说明依靠科技力量的创新,是解决我国经济发展瓶颈的
\par\twoch{\textcolor{red}{主要矛盾}}{矛盾的主要方面}{一点论}{主流}
\begin{solution}本题考查对马克思主义辩证法的理解应用。材料中的关键信息在于``关键要靠科技力量'',这里的``关键''就是抓主要矛盾。故选A。
\end{solution}
\question ``先试点后推广''是我国推进改革的一个成功做法。一项改革特别是重大改革,先在局部试点探索,取得经验、达成共识后,再把试点的经验和做法推广开来,这样的改革比较稳当。``先试点后推广''的辩证法依据是
\par\fourch{\textcolor{red}{矛盾的个性与共性在一定条件下能够相互转化,矛盾的共性寓于个性之中}}{必然性通过偶然性表现}{矛盾的个性表现共性并优于共性}{矛盾的个性在事物发展中起决定作用}
\begin{solution}【答案】A
【解析】本题考查矛盾的普遍性和特殊性的关系。C选项本身说法错误,矛盾的普遍性(共性)和特殊性(个性)各有特点,不能说谁优于谁,排除。D选项说法本身也错误,主要矛盾在事物发展中起决定作用而不是矛盾的个性,排除。B选项本身说法没有错,但是题干所给材料与``必然性和偶然性''没有关系,应该排除。题干材料``先试点后推广''体现了由``特殊到普遍再
到特殊''的工作方法,体现了矛盾的普遍性(共性)寓于特殊性(个性)之中,并通过特殊性表现
出来,矛盾的普遍性(共性)和特殊性(个性)在一定条件下能够相互转化,本题答案是A。
\end{solution}

\subsection{110-法律权利}
\question 邓小平指出:``人们支持人权,但不要忘记还有一个国权。''如果失去了国家主权、民主独立和国家尊严,也就失去了人民民主,并且从根本上失去了人权。这说明(
)
\par\twoch{人权高于主权}{人权无国界}{\textcolor{red}{只有人民掌握政权,人民才会拥有民主}}{人权是具体的、相对的}
\begin{solution}人权是具体的、相对的,不是抽象的、绝对的。这是指人权与国家的政治状
况、经济发展、文化结构和整个社会的发展水平有很大关系。公民权利的实现和发
展,都是通过国家政权,依赖国家政权。只有人民掌握政权,巩固和发展政权,人民
才会拥有属于自己的民主、自由和权利。故选项C正确。
\end{solution}

\subsection{111-法律权利与法律义务的关系}
\question 我国宪法法律规定的公民权利与义务具有广泛性、公平性和真实性。其公平性主要体现在
\par\fourch{权利与义务的主体为全体公民,权利范围涵盖社会政治经济文化生活的各个方面}{\textcolor{red}{权利与义务为全社会的公民平等地享有或履行}}{国家从制度上、法律上、物质上保障公民权利与义务的实现}{每个人既是享受各种法律权利的主体,又是承担各种法律义务的主体}
\begin{solution}此题考查的知识点是我国宪法法律规定的权利与义务,是一道理解性试题,难度适中。B选项正确。A选项讲述的是广泛性,C选项讲述的是真实性,D选项讲述的是法律权利和法律义务二者的统一
性。故都不选。
\end{solution}

\subsection{113-中国特色社会主义法律体系的意义和内容}
\question 党的十八届四中全会提出了建设中国特色社会主义法治体系的总目标。建设中国特色社会主义法治体
系的内容包括形成完备的法律规范体系、高效的法治实施体系和
\par\twoch{\textcolor{red}{严密的法治监督体系}}{完善的社会保障体系}{\textcolor{red}{有力的法治保障体系}}{\textcolor{red}{完善的党内法规体系}}
\begin{solution}此题考查的知识点是建设中国特色社会主义法治体系的内容,是一道识记为主的试题,难度适中。ACD选项正确。B选项,完善的社会保障体系是建设和谐社会的要求。
\end{solution}

\subsection{114-我国的实体法律部门}
\question 人民法院判处犯罪分子和犯罪的单位,向国家缴纳一定金钱的刑罚方法,称为(
)
\par\twoch{管制}{\textcolor{red}{罚金}}{罚款}{没收财产}
\begin{solution}罚款和罚金都是国家机关强制违法行为者在一定期限内向国家缴纳一定数量现金的处罚方法,但两者存在区别:罚金是由人民法院判处犯罪分子或犯罪单位向国家缴纳一定数额金钱的刑罚方法。罚款分为两种:一种是由人民法院依据民法或诉讼法作出的,罚款的对象是有妨碍民事诉讼行为的人;另一种是由公安机关或者其他行政机关依照行政法规作出的,罚款的对象是违反治安管理或者是违反海关、工商、税收等这样一些行政法规的人。所以,罚款的适用对象比罚金要宽得多。没收财产是将犯罪分子个人所有财产的一部分或者全部强制无偿地收归国有的刑罚方法。因此,B正确。
\end{solution}

\subsection{115-增强辩证思维能力}
\question 老子曾说:``天下皆知美之为美,斯恶已。皆知善之为善,斯不善已。故有无相生,难易
相成,长短相形,高下相倾,音声相和,前后相随。''这段话包含的哲学观点是()
\par\fourch{矛盾的对立面可以相互转化}{\textcolor{red}{矛盾的对立面之间相辅相成}}{矛盾的斗争性是绝对的}{\textcolor{red}{任何事物都有其对立的一面}}
\begin{solution}本题考查矛盾的观点。题干表明,事物都有其对立面,事物因其对立面而生
成,因此表现出一种相辅相成的态势,例如善与不善、有与无、长与短、高与低、
前与后等。这种相辅相成正是推动事物变化发展的力量所在。故选项BD正确。
\end{solution}
\question ``我们的事业越前进、越发展,新情况新问题就会越多,面临的风险和挑战就会越多,面对的不可预料的事情就会越多。我们必须增强忧患意识,做到居安思危。这要求我们
\par\fourch{\textcolor{red}{要善于看到矛盾的普遍性,勇于承认矛盾、揭露矛盾、分析矛盾、解决矛盾}}{\textcolor{red}{要认识矛盾的对立统一,全面的、一分为二的看问题}}{\textcolor{red}{要明白意识的能动性表现在具有髙度的创造性}}{要善于把握质量互变规律}
\begin{solution}本题是对马克思主义唯物论与辩证法的综合性考查。``问题越多、风险和挑战越多、不可预料的事情越多''体现了矛盾的普遍性存在,并要求全面地看问题。``忧患意识和居安思危''强调了人的意识的能动性。材料并没有体现出质量互变规律,故选ABC。
\end{solution}

\subsection{116-认识世界和改造世界必需勇于创新}
\question 马克思主义认识论认为,认识的辩证过程是(  )
\par\fourch{\textcolor{red}{从相对真理到绝对真理的发展}}{从间接经验到直接经验的转化}{\textcolor{red}{实践——认识——实践的无限循环}}{从抽象到具体再到抽象的上升运动}
\begin{solution}【解析】认识的过程是从实践到认识再到实践的循环过程,也是感性认识上升到理性认识的过程,是相对真理向绝对真理的转变过程。
\end{solution}

\subsection{117-物质生产方式是社会发展的基础}
\question 生产方式集中体现了人类社会的物质性。生产方式中的生产力体现着人们改造自然的现实的物质力量,生产关系是人们在物质生产中发生的``物质的社会关系'',生产
力和生产关系的统一所构成的生产方式使自然界的一部分转化为社会物质生活条件,
使生物的人上升为社会的人。生产方式是社会历史发展的决定力量。这种决定力量体现为()
\par\fourch{\textcolor{red}{物质生产方式是人类其他一切活动的首要前提}}{\textcolor{red}{物质生产方式决定着社会的结构和性质,制约着全部社会生活}}{\textcolor{red}{物质生产方式决定着社会形态从低级到高级的发展}}{物质生产方式是劳动者和劳动资料结合的特殊方式}
\begin{solution}本题考查物质生产方式是社会发展的基础。生产方式是社会历史发展的决定力量。首先,物质生产活动以及生产方式是人类社会赖以存在和发展的基础,是人
类其他一切活动的首要前提。其次,物质生产活动及生产方式决定着社会的结构、
性质和面貌,制约着人们的经济生活、政治生活和精神生活等全部社会生活。最后,
物质生产活动及生产方式的变化发展决定着整个社会历史的变化发展,决定着社会
形态从低级向高级的更替和发展。故选项ABC正确。选项D不是物质生产方式决定
作用的表现,不符合题意,故不选。
\end{solution}

\subsection{118-金融垄断资本的发展}
\question 当代资本主义国际垄断同盟的高级形式是( )
\par\twoch{国际卡特尔}{混合联合企业}{跨国公司}{\textcolor{red}{国际联盟}}
\begin{solution}此题考查的知识点是当代资本主义国际垄断同盟的高级形式。当代国际垄断同盟的形式是以跨国公司和国家垄断资本主义的国际联盟为主。其中跨国公司是国际垄断同盟的重要形式,国家垄断资本主义的国际联盟是国际垄断同盟的高级形式。故D是正确选项。
\end{solution}
\question 金融寡头实现其统治的主要途径是( )
\par\twoch{\textcolor{red}{参与制}}{\textcolor{red}{个人联合}}{企业联合}{经理负责制}
\begin{solution}此题考查的知识点是金融寡头统治地位的实现途径。金融寡头实现其经济上的统治地位是通过参与制;实现其在政治上的统治,主要是通过个人联合。故AB是正确选项。
\end{solution}

\subsection{119-科学社会主义基本原则的主要内容}
\question 马克思恩格斯在新的历史条件下创立了唯物史观和剩余价值学说,揭示了人类历史发展的奥秘和资本主义剥削的秘密,论证了无产阶级的历史使命,把争取无产阶级和全
人类解放的斗争建立在社会发展的客观规律之上,从而超越了空想社会主义,创立了
科学社会主义。1848年2月,《共产党宣言》的发表,标志着科学社会主义的诞生。
科学社会主义的核心内容是( )
\par\twoch{暴力革命}{生产资料公有制}{无产阶级的国际联合}{\textcolor{red}{无产阶级专政和社会主义民主}}
\begin{solution}本题考查科学社会主义的核心内容。无产阶级专政是建立和发展社会主义的
政治保证。建设高度的社会主义民主,是工人阶级执政党为之奋斗的崇高目标和根本
任务。无产阶级专政和社会主义民主是科学社会主义的核心内容。故选项D正确。
\end{solution}

\subsection{120-实践的本质、基本特征和基本形式}
\question 社会生活的实践性体现在( )
\par\twoch{\textcolor{red}{实践是社会关系形成的基础}}{\textcolor{red}{实践形成了社会生活的基本领域}}{\textcolor{red}{实践构成了社会发展的动力}}{认识在实践中形成}
\begin{solution}此题考查的知识点是社会生活的实践性。社会生活本质是实践的,这体现在:(1)实践是社会关系形成的基础;(2)实践形成了社会生活的基本领域;(3)实践构成了社会发展的动力。选项ABC符合题意,而选项D说明的是认识与实践的关系,不合题意。
\end{solution}

\subsection{122-完善生态文明制度体系}
\question 习近平指出:``只有实行最严格的制度、最严密的法治,才能为生态文明建设提供可靠保障。''建设生态文明,必须建立系统完整的制度体系,用制度保障生态环境、推进生态文明建设。建立系统完整的制度体系,最重要的是
\par\fourch{违立生态环境损害责任终身追究制}{划定生态保护红线}{健全法律法规,完善生态环境保护管理制度}{\textcolor{red}{把体现生态文明逑设状况的指标纳人经济社会发展评价体系}}
\begin{solution}【简析】建设生态文明是一场涉及生产方式、生活方式、思维方式和价值观念的革命性变革。实现这样的根本变革,必须依靠制度和法治。对此,习近平指出:``只有实行最严格的制度、最严密的法治,才能为生态文明建设提供可靠保障。''建设生态文明,必须建立系统完整的制度体系,用制度保障生态环境、推进生态文明建设。建立系统完整的生态文明制度体系,最重要的是要把资源消耗、
环境损害、生态效益等体现生态文明建设状况的指标纳入经济社会发展评价体系,
使之成为推进生态文明建设的承要导向和约束。D正确,A、B、C错误。
\end{solution}

\subsection{123-共产主义的基本特征}
\question 共产主义社会是人的自由而全面的发展的社会,这里的全面发展指的是
\par\twoch{\textcolor{red}{人的体力和智力得到发展}}{\textcolor{red}{人的才能和工作能力得到发展}}{\textcolor{red}{人的社会联系和社会交往得到发展}}{人无所不能}
\begin{solution}ABC
三项,共产主义社会人的发展是全面的发展,不仅体力和智力得到发展,各方面的才能和工作能力得到发展,而且人的社会联系和社会交往也得到发展。D
项,全面发展不是指人无所不能。
\end{solution}
\question 马克思指出:``在共产主义社会高级阶段,在迫使个人奴隶般地服从分工的情形已经消失,从而脑力劳动和体力劳动的对立也随之消失后;在劳动已经不仅仅是谋生的手段,而且本身成了生活的第一需要之后;随着个人的全面发展,它们的生产力也增长起来,而集体财富的一切源泉都充分涌流之后,------只有在那个时候,才能完全超出资产阶级权利的狭隘眼界,社会才能在自己的旗帜上写上:各尽所能,按需分配!''下列选项对``各尽所能,按需分配''的理解正确的是(
)
\par\fourch{各尽所能,按需分配第一次以人的劳动而不是特权或资本作为分配的标准}{各尽所能,按需分配所体现的平等权利还是被限制在资产阶级的框框里}{\textcolor{red}{各尽所能,按需分配最终实现人类分配上的真正平等}}{各尽所能,按需分配不可能真正实现}
\begin{solution}各尽所能,按需分配实现人类分配上的真正平等。
\end{solution}

\subsection{124-真理问题讨论与十一届三中全会}
\question 从1979年11月起,在邓小平主持下,中共中央着手起草《关于建国以来党的若干历史问题的
决议》。在1981年6月27日到29日召开的中共十一届六中全会上,《决议》获得一致通过。
下列关于《决议》表述正确的是
\par\fourch{\textcolor{red}{《决议》科学地评价了毛泽东的历史地位,充分论述了作为党的指导思想的伟大意义}}{\textcolor{red}{《决议》肯定了中共十一届三中全会以来逐步确立的适合中国情况的建设社会主义现代化
强国的道路,进一步指明了中国社会主义事业和党的工作继续前进的方向}}{\textcolor{red}{历史决议的通过,标志着党和国家在指导思想上拨乱反正的胜利完成}}{历史决议的通过,标志着党和国家在指导思想上拨乱反正的开始}
\begin{solution}D项错误,因为标志着拨乱反正开始的是党的十一届三中全会。
\end{solution}
\question 关于真理标准问题的大讨论的重要意义有( ~)
\par\fourch{\textcolor{red}{是继延安整风之后又一场马克思主义思想解放运动}}{\textcolor{red}{成为拨乱反正和改革开放的思想先导}}{\textcolor{red}{为党重新确立实事求是的思想路线,实现历史性的转折作了思想理论准备}}{是新中国成立以来党的历史上具有深远意义的伟大转折}
\begin{solution}从1978年5月开始的关于真理标准问题的大讨论,强调实践是检验真理的唯一标准。这场讨论,是继延安整风之后又一场马克思主义思想解放运动,成为拨乱反正和改革开放的思想先导,为党重新确立实事求是的思想路线,纠正长期以来的``左''倾错误,实现历史性的转折作了思想理论准备。十一届三中全会是新中国成立以来中国共产党的历史上具有深远意义的伟大转折。
\end{solution}

\subsection{125-农村改革、四项基本原则和科学评价毛泽东思想}
\question 邓小平提出坚持四项基本原则是在( )
\par\twoch{中共十一届三中全会}{中共十一届六中全会}{\textcolor{red}{1979年理论工作务虚会}}{1978年中央工作会议}
\begin{solution}邓小平在理论工作务虚会上发表的讲话中指出:坚持社会主义道路,坚持人民民主专政,坚持共产党的领导,坚持马克思列宁主义、毛泽东思想这四项基本原则,``是实现四个现代化的根本前提''。
\end{solution}
\question 十一届三中全会以后,关于是否在农村推行家庭联产承包责任制问题上存在很大争论,对这一问题,邓小平一锤定音:``有的同志担心,这样搞会不会影响集体经济。我看这种担心是不必要的。''促使邓小平下定决心实行家庭联产承包责任制的根本原因是(
)
\par\fourch{小农经济的落后性阻碍了生产的发展}{农民被束缚在土地上}{\textcolor{red}{人民公社与分配中的平均主义阻碍了生产的发展}}{促进市场经济的发展}
\begin{solution}家庭联产承包责任制属于生产关系的一种,所以从根本上实行这种制度肯定是由于以前的制度阻碍了生产的发展。
\end{solution}
\question 1981年6月,中共十一届六中全会通过《关于建国以来党的若干历史问题的决议》,标志着指导思想上拨乱反正的胜利完成,其核心问题是(
)
\par\twoch{坚持改革开放}{平反一些重大的冤、假、错案}{解决一些“文化大革命”遗留的和历史遗留的问题}{\textcolor{red}{科学地评价毛泽东的历史功过、确立毛泽东思想的历史地位}}
\begin{solution}思想上拨乱反正的核心是针对毛泽东思想这点事没有什么异议的,属于大纲解析原文。
\end{solution}

\subsection{126-中共十二大、十三大、十四大与南方谈话、十五大}
\question 邓小平明确提出``建设有中国特色社会主义''是在( )
\par\twoch{十一届三中全会}{\textcolor{red}{十二大}}{十二届三中全会}{十三大}
\begin{solution}邓小平在中共十二大的开幕词中提出,``把马克思主义的普遍真理同我国的具体实际结合起来,走自己的道路,建设有中国特色的社会主义''。
\end{solution}
\question 提出``把马克思主义的普遍真理同我国的具体实际结合起来,走自己的道路,建设有中国特色的社会主义''的会议是(
)
\par\twoch{中共十一届三中全会}{中共十一届六中全会}{\textcolor{red}{中共十二大}}{中共十三}
\begin{solution}十二大提出``建设由中国特色的社会主义''。``把马克思主义的普遍真理同我国的具体实际结合起来,走自己的道路,建设有中国特色的社会主义''。
\end{solution}

\subsection{127-商品经济}
\question 商品是( ~)
\par\fourch{\textcolor{red}{用来交换的劳动产品}}{\textcolor{red}{使用价值和价值的统一}}{满足生产者自己需要的劳动产品}{\textcolor{red}{一定生产关系的体现}}
\begin{solution}C选项明显错误,商品是要交换的,留着自己用的不叫商品。
\end{solution}
\question 商品经济的发展经历了简单商品经济和发达商品经济两个阶段。简单商品经济又称``小商品经济'',以生产资料个体所有制和个体劳动为基础的商品经济。简单商品经济包含着一系列内在矛盾,其中最基本的矛盾是(
~)
\par\fourch{\textcolor{red}{私人劳动和社会劳动的矛盾}}{具体劳动和抽象劳动的矛盾}{生产社会化和生产资料私人占有的矛盾}{个别劳动时间和社会必要劳动时间的矛盾}
\begin{solution}私人劳动和社会劳动的矛盾是商品经济的基本矛盾,最后演变为资本主义经济的基本矛盾。
\end{solution}
\question 马克思指出,在商品经济中,价值规律是``作为起调节作用的自然规律强制地为自己开辟道路,就像房屋倒在人的头上时重力定律强制为自己开辟道路一样''。这段话表明(
~)
\par\fourch{\textcolor{red}{价值规律是商品经济的一般规律或基本规律}}{\textcolor{red}{价值规律和自然规律一样具有客观性}}{\textcolor{red}{价值规律具有自发性}}{价值规律排斥人的主观能动性}
\begin{solution}D观点错误。
\end{solution}
\question 构成社会财富的物质内容是( ~)
\par\twoch{价值}{交换价值}{\textcolor{red}{使用价值}}{价格}
\begin{solution}使用价值是商品能够满足人们需要的属性,反映了物品的自然属性,它是一切物品包括劳动产品的共性,构成社会财富的物质内容。价值是商品特有的社会属性,交换价值和价格都是价值的表现形式。
\end{solution}
\question 简单商品经济的基本矛盾是私人劳动和社会劳动的矛盾。这是因为( )
\par\twoch{\textcolor{red}{它是商品各种内在矛盾的根源}}{\textcolor{red}{它决定着商品生产者的命运}}{它是决定和影响价格的重要因素}{\textcolor{red}{它贯穿私有制商品经济产生和发展的全过程}}
\begin{solution}私人劳动与社会劳动的矛盾是简单商品生产的基本矛盾,理由在于:其一,它是商品和商品生产一切矛盾如价值与使用价值,具体劳动和抽象劳动,个别劳动时间和社会必要劳动时间的根源;其二,它贯穿商品生产产生、发展的全过程;其三,它决定着以私有制为基础的商品生产者的命运。
\end{solution}

\subsection{128-商品经济产生的条件}
\question 马克思认为,``商品形式的奥秘不过在于:商品形式在人们面前把人们本身劳动的社
会性质反映成劳动产品本身的物的性质,反映成这些物的天然的社会属性,从而把
生产者同总劳动的社会关系反映成存在于生产者之外的物与物之间的社会关系。由
于这种转换,劳动产品成了商品,成了可感觉而又超感觉的物或社会的物。''这表明
( )
\par\fourch
{\textcolor{red}{商品本质上体现的是人与人之间的关系}}
{\textcolor{red}{商品把人与人之间的关系物化了}}
{\textcolor{red}{商品的”天然的社会属性”就在于人们本身劳动的社会性质}}
{商品之所以成为商品是因为它是劳动产品}
\begin{solution}本题考查考生对商品属性的理解。劳动产品在交换中取得了商品的形式,商品
交换的过程不但是人们交换使用价值的过程,也是人们相互比较自身劳动(商品价值)
的过程。这一过程表面上是物物交换(使用价值的交换),实际上则反映了商品生产者
之间的社会关系。因此,可以说商品把人与人之间的关系物化了,商品本质上体现的是
人与人之间的关系。普通商品具有自然的和社会的双重属性,自然属性在于其使用价
值,社会属性在于其所包含的无差别的一般人类劳动,也就是C项所说的人们本身劳
动的社会性质。选项D的说法是错误的,劳动产品天然并非是商品,只有用于交换的
劳动产品才是商品,因此商品之所以成为商品的根本原因在于:它是用于交换的劳动产
品,它反映了人们本身劳动的社会性质。因此,正确答案为ABC。
\end{solution}
\question 使用价值不同的商品之所以能按一定比例相交换,是因为它们都有价值,而价值可以互相比较是因为
\par\fourch{价值是一切劳动产品所共有的属性}{\textcolor{red}{价值在质的规定性上是相同的}}{价值是具体劳动创造的}{价值和使用价值具有同一性}
\begin{solution}使用价值反映的是人与自然之间的物质关系,是一切劳动产品所共有的属性,但价值并不是一切劳动产品所共有的属性,只有用来交换的劳动产品(即商品)才具有价值,A错误。使用价值不同的商品之所以能按一定比例相交换.就是因为它们都有价值。商品价值在质的规定性上是相同的,因而彼此可以比较。B正确。具体劳动形成商品的价值实体。C错误。D不符合题意。
\end{solution}
\question 商品的价值不仅有质的规定性,而且还有量的规定性。商品价值的计量尺度是
\par\twoch{\textcolor{red}{简单劳动}}{个别劳动}{劳动强度}{具体劳动}
\begin{solution}【答案】A
【简析】商品价值是以简单劳动为尺度度计量的.复杂劳动等于自乘的或多倍的简单劳动。A正确,
\end{solution}
\question 马克思指出:``我们在这里最初看到的利润,和剩余价值是一回事,不过它具有一个神秘的形式,而这个神秘化的形式必然会从资本主义生产方式中产生出来。''恩格斯也指出:``马克思一
有机会就提醒读者注意,绝不要把他所说的剩余价值同利润或资本盈利相混淆。''对这两段话理解错误的是(
)
\par\fourch{利润是剩余价值的一种具体形式}{剩余价值是利润的本质内容}{\textcolor{red}{剩余价值是资本的盈利}}{利润常常只是剩余价值的一部分}
\begin{solution}本题考查利润的本质。剩余价值转化为利润,是与生产成本这个概念紧密联系的。在资本主义的生产过程中,不仅耗费的资本在资本主义的生产过程中发挥着作用,预付资本中暂
时没有消耗掉、还没有转移到新产品的那部分不变资本也同样参与了商品的生产过程,同样是
剩余价值生产的不可缺少的物质要素,这样剩余价值就进一步表现为全部预付资本的增加额。
当不把剩余价值看做是雇佣工人剩余劳动的产物,而是把它看做是全部预付资本的产物或增
加额时,剩余价值就转化为利润Q可见,利润和剩余价值本是同一个东西,所不同的是,剩余价值是对可变资本而言的,而利润是对全部预付资本而言的。因此,剩余价值是利润的本质,利润是剩余价值的转化形式,即从现象的表现形式来看,利润常常只是剩余价值的一部分。因此
ABD选项观点都是正确的,不符合题意。C选项观点错误,符合题意。
\end{solution}
\question 关于经济学上的财富,有各种各样的名称,有人称之为物质财富,有人称之为自然财富,有人称之为人为财富,也有人称之为财富;明确地将经济学上的财富称为社会财富的是马克思,在马克思看来,在一切社会里,社会财富的物质内容都是由(
)
\par\twoch{金银构成的}{价值构成的}{货币构成的}{\textcolor{red}{使用价值构成的}}
\begin{solution}社会财富的物质内容即有使用价值构成的。
\end{solution}
\question 在人类历史上,自从出现商品交换以来,商品的价值形式已经历了四个发展阶段,有四种不同的表现形式,依次是(
~)
\par\fourch{\textcolor{red}{简单的或偶然的价值形式、总和的或扩大的价值形式、一般价值形式、货币形式}}{简单的或偶然的价值形式、一般价值形式、总和的或扩大的价值形式、货币形式}{简单的或偶然的价值形式、一般价值形式、货币形式、总和的或扩大的价值形式}{简单的或偶然的价值形式、货币形式、一般价值形式、总和的或扩大的价值形式}
\begin{solution}商品的价值形式有四种不同的表现形式,简单的或偶然的价值形式、总和的或扩大的价值形式、一般价值形式、货币形式。
\end{solution}

\subsection{129-资本主义生产关系的产生}
\question 资本主义制度下的社会财富表现为一种惊人的庞大的商品堆积,单个的商品表现为它的元素形式。
以下说法正确的是
\par\fourch{\textcolor{red}{不论财富的社会的形式如何,使用价值总是构成财富的物质的内容}}{劳动是使用价值的唯一源泉}{商品不一定是劳动产品}{具体劳动形成商品的价值实体}
\begin{solution}【简析】马克思指出:``不论财富的社会的形式如何,使用价值总是构成财富的物质的内容A正确。劳动是使用价值的源泉之一,不是唯一源泉,B错误。
商品一定是劳动产品,C错误。具体劳动形成商品的使用价值,抽象劳动形成商品的价值实体,D错误。
\end{solution}
\question 资本主义制度下的工资之所以掩盖了资本主义剥削关系,是因为
\par\twoch{\textcolor{red}{工资表现为“劳动的价格”}}{工资表现为劳动的价值}{工资模糊了个别劳动时间和社会必要劳动时间界限}{工资的价值只计算了转移到新产品中去得那一部分}
\begin{solution}A
(注:此题考的是为什么掩盖了,不是考工资本身是什么,正是因为工资表现为``劳动的价格''或工人全部劳动的报酬,模糊了工人必要劳动和剩余劳动的界限,才掩盖了资本主义剥削关系。)
【简析】在资本主义制度下,工人工资是劳动力的价值或价格,这是资本主义工资的本质。资本家购买工人的劳动力是以货币工资形式支付的,工资表现为``劳动的价格''或工人全部劳动的报酬,这就模糊了工人必要劳动和剩余劳动的界限,掩盖了资本主义剥削关系。A正确。
\end{solution}

\subsection{130-资本主义的生产过程}
\question 在资本主义的商品生产过程中,土地、设备和原材料等生产资料的价值是借助于生产者的(
)
\par\twoch{具体劳动而实现了价值增殖}{\textcolor{red}{具体劳动转移到新产品中}}{抽象劳动而实现了价值增殖}{抽象劳动转移到新产品中}
\begin{solution}在资本主义的商品生产过程中,土地、设备和原材料等生产资料的价值是借助于生产者的具体劳动转移到新产品中的,本身不产生剩余价值。A,D观点错误,C与题干无关。
\end{solution}

\subsection{131-其他理论成果}
\question 以毛泽东为核心的党的第一代中央领导集体领导人民探索社会主义建设道路,经历艰辛和曲折,在理论和时间上取得了一系列重要成果。这一探索
\par\fourch{是马克思主义与中国实际的“第二次结合”的成功典范}{开辟了中国特色的社会主义现代化建设道路}{\textcolor{red}{启示我们必须从中国实际出发进行社会主义建设,不能急于求成}}{\textcolor{red}{巩固和发展了我国的社会主义制度}}
\begin{solution}A项错误,毛泽东没有实现马克思主义和中国实际的``第二次结合''。B项错误,也没有开辟正确道路,社会主义建设道路是以邓小平为核心的党的第二代中央领导集体开创的。
\end{solution}

\subsection{132-中国特色社会主义总布局历史回顾}
\question 按劳分配是社会主义的分配原则,也是出于主体地位的分配原则,之所以实行按劳分配,其前提条件是(
)
\par\twoch{\textcolor{red}{公有制}}{社会生产力的发展水平}{社会主义初级阶段的基本经济制度}{人们的劳动积极性}
\begin{solution}之所以在我国实行按劳分配是由公有制为主体和社会生产力发展水平决定的,其中,公有制为主体是实行按劳分配的前提条件和所有制基础,生产力发展水平是实行按劳分配的物质基础。
\end{solution}
\question 劳动、资本、技术、管理等生产要素是社会生产不可或缺的因素。在我国社会主义初级阶段,实行按生产要素分配的必要性和根据是
\par\twoch{生产要素可以转化为生产力}{\textcolor{red}{我国社会存在着生产要素的多种所有制}}{按生产要素分配是按劳分配的补充}{生产要素是价值的源泉}
\begin{solution}在我国社会主义初级阶段,实行按生产要素分配的必要性和根据是我国社会存在着生产要素的多种所有制。因此,本题正确答案是选项B。
\end{solution}
\question ``股份制是现代企业的一种资本组织形式,不能笼统地说股份制是公有还是私有'',这一观点表明(
)
\par\fourch{由法人股东而不是个人股东构成的股份制是公有制}{\textcolor{red}{公有制与私有制都可以通过股份制这一形式来实现}}{有公有制经济参股的就是公有制}{\textcolor{red}{股份制本身不具有公有还是私有的性质}}
\begin{solution}本题考查的是公有制经济的实现形式和公有制经济控股的重要意义两方面的内容。党的十五大指出:``股份制是现代企业的一种资本组织形式,有的所有权和经营权分离,有的提高企业和资本的动作效率,资本主义可以用,社会主义也可以用(B项正确)。不能笼统地说股份制是公有还是私有(D项正确),关键看控股权掌握在谁手中。国家和集体控股,具有明显的公有性(E项正确),有利于扩大公有资本的支配范围,增强公有制的主体作用。股份制不是公有制(不管是法人股东构成的股份制、还是有公有制参股的股份制都是如此),股份制中的国有成分、集体成分才是公有制。''答案为BDE三项。股份制不是所有制,不存在公有、私有之分,只有股份制经济中的国有和集体成分才是公有制。AC两项的说法错误。
\end{solution}
\question 人民政协以政治协商和民主监督作为自己的主要职能。民主监督主要是指( )
\par\fourch{\textcolor{red}{对国家的宪法和法律法规的实施情况进行监督}}{对国家工作人员提出质询、弹劾}{\textcolor{red}{对重大方针政策的贯彻执行情况进行监督}}{对人民代表大会进行工作监督}
\begin{solution}人民政协作为共产党领导的多党合作和政治协商的政治组织,始终以政治协商和民主监督为自己的主要职能。民主监督的主要内容包括:对国家的宪法和法律实施情况进行监督;对重大方针、政策的贯彻执行情况进行监督;对国家机关及其工作人员履行职责的情况进行监督。A、C、E项正确。
B、D两项错误,人民政协不同于人民代表大会,它不具有国家权力机关那种监督、检察、质询、弹劾等权力,也不能对国家权力机关人民代表大会进行工作监督。
\end{solution}
\question 在社会主义条件下,中国共产党与各民主党派长期共存,是因为( )
\par\fourch{无产阶级政党可以同资产阶级结成统一战线}{\textcolor{red}{双方有长期团结合作的历史}}{\textcolor{red}{各民主党派已经成为致力于社会主义事业的党派}}{\textcolor{red}{各民主党派在政治上接受了共产党领导}}
\begin{solution}本题考查的是中国共产党与各民主党派``长期共存、互相监督''的方针。1957年2月他又在《关于正确处理人民内部矛盾的问题》一文中阐述了实行``长期共存、互相监督''方针的依据:``为什么要让资产阶级和小资产阶级的民主党派同工人阶级政党长期共存呢?这是因为凡属一切确实致力于团结人民从事社会主义事业的、得到人民信任的党派,我们没有理由不对它们采取长期共存的方针(C项正确)。''各民主党派早在建国前夕就正式宣布接受中国共产党的领导(D项正确),中国共产党的领导是我国的多党合作制度的政治基础;新民主主义革命时期,各民主党派同中国共产党亲密合作,为民族的独立、人民解放做出了重大贡献,双方有长期团结合作的历史(B项正确);各民主党派可以发挥对共产党的监督作用,发挥民主党派的民主监督作用,能对加强和改善党的领导(E项正确)。答案为BCDE四项。社会主义改造完成后,资产阶级已经消亡,各民主党派不是资产阶级政党。A项的说法错误。
\end{solution}
\question 深化文化体制改革,要坚持的目标是( )
\par\twoch{以发展为目标}{以体制机制创新为目标}{以增强全民族文化创造活力为目标}{\textcolor{red}{以创造生产更多更好适应人民群众需求的精神文化产品为目标}}
\begin{solution}深化文化体制改革,要坚持以发展为主题,以改革为动力,以体制机制创新为重点,以创造生产更多更好适应人民群众需求的精神文化产品为目标,促进文化事业全面繁荣和文化产业快速发展。D项正确。
A项错误,发展是文化体制改革的主题;B项错误,体制机制创新是文化体制改革的重点;C项错误,增强全民族文化创造活力是建设社会主义文化强国的关键。
\end{solution}
\question 党的十六届六中全会提出的``建设社会主义核心价值体系''与``文化多样性''``坚持先进文化的前进方向''的内在联系是(
)
\par\fourch{\textcolor{red}{社会主义核心价值体系与文化多样性统一于社会主义文化建设中}}{\textcolor{red}{建设社会主义核心价值体系有利于坚持先进文化的前进方向}}{\textcolor{red}{尊重文化多样性不能违背社会主义核心价值体系}}{把握先进文化的前进方向关键在于尊重文化的多样性}
\begin{solution}本题以建设社会主义核心价值体系作为切入点,考查文化的多样性和文化前进方向。把握先进文化的前进方向关键在于坚持社会主义思想道德建设,D说法错误。
\end{solution}
\question 依法治国,建设社会主义法治国家是建设中国特色社会主义的重要目标。依法治国是( )
\par\fourch{\textcolor{red}{社会文明进步的显著标志,是国家长治久安的重要保障}}{发展社会主义民主,实现人民当家作主的根本保证}{\textcolor{red}{有利于社会主义市场经济体制的完善和发展,为扩大对外开放保驾护航}}{决定当代中国命运的关键抉择}
\begin{solution}此题考查的知识点是依法治国的地位和作用,属识记与记忆相结合的试题,难度适中。AC是正确选项。B项错误,因为中国共产党的领导是实现人民当家做主的根本保证。D项错误在于改革开放是决定当代中国命运的关键抉择。
\end{solution}
\question 在社会主义初级阶段,多种分配方式并存是收人分配制度的一大特点。按劳分配以
外的多种分配方式,其实质就是按对生产要素的占有状况进行分配。生产要素归纳
起来可以分为两类,一是物质生产条件,二是人的劳动,包括人们在生产过程中提
供的活劳动、技术、信息等。按生产要素分配有多种不同的分配形式,就其内容不
同可以分为( )
\par\fourch{\textcolor{red}{专利等管理和知识产权类的生产要素参与分配}}{\textcolor{red}{以劳动作为生产要素参与分配}}{以生产资料所有者的身份参与分配}{\textcolor{red}{以利息、租金等劳动以外的生产要素所有者参与分配}}
\begin{solution}本题考查多种分配方式并存。选项ABD均正确。选项C错误。按生产要素
分配不包括公有制中的按劳分配,因为按劳分配得到收人的劳动者不是凭借作为独
立的生产要素的所有者参与分配,而是以生产资料所有者的身份凭自己提供的劳动
来参与分配的。
\end{solution}
\question 李某是国内某家国有企业的技术员工,他的收入包括三部分。第一部分是企业每个月发给他的工资,第二部分是他的技术供企业采用所得的收人。同时,他家的土地被一家工厂占用,也有一部分收入。李某的收人参与的分配方式有
\par\twoch{\textcolor{red}{按劳分配}}{\textcolor{red}{劳动以外的生产要素参与分配}}{\textcolor{red}{管理和知识产权类的生产要素参与分配}}{以劳动作为生产要素参与分配}
\begin{solution}李某的第一部分收入属于按劳分配,第二部分收入属于管理和知识产权类的生产要素参与分配,第三部分属于劳动以外的生产要素参与分配。
\end{solution}
\question ``罗马城不是一天建起来的''。公平正义是一个逐步实现的过程,将随着社会经济发展不断螺旋式上升。这表明
\par\fourch{\textcolor{red}{维护和促进社会公正是一个渐进的过程,而不可能一蹴而就}}{\textcolor{red}{在新的起点上推进中国特色社会主义,就应把公平正义放到更加突出的位置,使追求公平正义体现到社会生活的方方面面,更好地促进社会和谐稳定}}{\textcolor{red}{要加紧建设对社会公平正义具有重大作用的制度,逐步建立以权利公平、机会公平、规则公平为主要内容的社会公平保障体系}}{公平正义是中国特色社会主义的本质属性}
\begin{solution}D项错误,中国特色社会主义的本质属性是社会和谐。
\end{solution}
\question 判断一个国家的政党制度究竟好不好,要从它的基本国情出发来认识,要从它的实践效果来分析:实践证明,中国共产党领导的多党合作和政治协商制度能够在中国特色社会主义共同目标下,把中国共产党领导和多党派合作有机结合起来,实现
\par\twoch{选举民主和协商民主的统一}{\textcolor{red}{广泛参与和集中领导的统一}}{\textcolor{red}{社会进步和国家稳定的统一}}{\textcolor{red}{充满活力和富有效率的统一}}
\begin{solution}新中国成立以来,中国共产党领导的多党合作与政治协商制度的重要性不断增强。实践证明,这一制度能够在中国特色社会主义共同目标下把中国共产党领导和多党派合作有机结合起来,实现广泛参与与集中领导的统一、社会进步与国家稳定、充满活力与富有效率的统一。B、C、D正确。中国共产党领导的多党合作和政治协商制度体现的是协商民主而不是选举民主,人民政协是协商民主的重要渠道和专门协商机构,A错误。
\end{solution}
\question 中国处理民族问题的根本原则也是中国民族政策的核心内容是
\par\twoch{维护祖国统一,反对民族分裂}{坚持民族平等}{\textcolor{red}{坚持民族团结}}{坚持各民族共同繁荣}
\begin{solution}【解析】社会主义时期处理民族问题的基本原则是:维护祖国统一,反对民族分裂坚持民族平等、民族团结、各民族共同繁荣。其中民族平等是民族团结、各民族共同繁荣的政治前提和基础是中国民族政策的基石。民族团结是维护国家统一、实现各民族共同发展的根本保证,是中国处理民族问题的根本原则,也是中国民族政策的核心内容。各民族的共同繁荣是解决民族问题的根本出发点和归宿。根据题意应选C。
\end{solution}
\question 我国社会主义民主政治的特有形式和独特优势,党的群众路线在政治领域的重要体现是
\par\twoch{民主集中制}{\textcolor{red}{协商民主}}{人民代表大会制度}{基层民主制度}
\begin{solution}【解析】党的十八届三中全会指出:要推进协商民主广泛多层制度化发展。协商民主是我国社会主义民主政治的特有形式和独特优势,是党的群众路线在政治领域的重要体现。在党的领导下,以经济社会发展重大问题和涉及群众切身利益的实际问题为内容,在全社会开展广泛协商,坚持协商于决策之前和决策实施之中。
\end{solution}
\question 实行民族区域自治,是党根据我国的历史发展、文化特点、民族关系和民族分布等具体情况作出的制度安排,符合各民族人民的共同利益和发展要求。具体而言(
)
\par\fourch{\textcolor{red}{统一的多民族国家的长期存在和发展,是我国实行民族区域自治的历史依据}}{中华民族多元一体、万流归宗的文化传统和文化结构,是我国实行民族区域自治的思想前提}{\textcolor{red}{近代以来在反抗外来侵略斗争中形成的爱国主义精神,是实行民族区域自治的政治基础}}{\textcolor{red}{各民族大杂居、小聚居的人口分布格局,各地区资源条件和发展的差异,是实行民族区域自治的现实条件}}
\begin{solution}ACD 为正确选项。其中的每一句,均可单独命制单选题。B是杜撰的干扰项。
\end{solution}
\question 加大个人收入分配调节力度,合理调整收入分配格局。为了实现这一目标,除了要着力提高低收入者收入、努力扩大中等收入者比重、切实对过高收入进行有效调节之外,还需要(
)
\par\twoch{鼓励自主创业、自谋职业}{\textcolor{red}{取缔非法收入}}{规范和协调劳动关系}{\textcolor{red}{规范垄断行业的收入}}
\begin{solution}加大个人收入分配调节力度,合理调整收入分配格局。一要着力提高低收入者收入。要强化支农惠农政策,促进农民持续增收,建立企业职工工资正常增长机制和支付保障机制,逐步提高扶贫标准和最低工资标准,使城乡居民特别是低收入者收入随着经济发展逐步较多地増加。二要努力扩大中等收入者比重。要通过采取多种措施,创造条件让更多群众拥有财产性收入,使更多低收入者进入中等收入者行列。三要切实对过高收入进行有效调节。要正确运用税收手段,使过高收入者的一部分收入通过税收等形式由国家集中用于再分配。四要取缔非法收入。要严格执法,对偷税漏税、侵吞公有财产、权钱交易等各种非法收入依法取缔。五要规范垄断行业的收入。AC是扩大就业的举措。
\end{solution}



\section{[史纲]中国近现代史纲要}


\subsection{133-帝国主义的侵略}
\question 香港、澳门问题是历史上殖民主义侵略遗留下来的问题。香港是被英国殖民主义者通过向中国发动侵略战争,强迫清政府先后签订哪些不平等条约强占的
\par\twoch{\textcolor{red}{《南京条约》}}{《虎门条约》}{\textcolor{red}{《北京条约》}}{\textcolor{red}{《展拓香港界址专条》}}
\begin{solution}【答案】ACD
【解析】港澳台关系今年是个热点问题,《史纲》可能从历史角度去命题。同学们需要了解一下相关的背景知识。香港是被英国殖民主义者通过向中国发动侵略战争,强迫清政府先后签订
《南京条约》《北京条约》《展拓香港界址专条》等不平等条约而强占的。新中国成立后,中国共产党和中国政府根据当时的国际国内形势,对香港、澳门这两个地区采取了如下立场:香
港、澳门地区是中国的领土,不承认外国强加在中国人民头上的不平等条约;对于这一历史遗
留下来的问题,将在适当时机通过谈判予以解决;未解决之前暂时维持现状。1997年,香港回归祖国的怀抱。
\end{solution}
\question 帝国主义侵略中国的最终目的,是要瓜分中国、灭亡中国。1895年中国在甲午战争中战败后,列强掀起了瓜分中国的狂潮,这集中表现在(
)。
\par\fourch{从侵占中国周边邻国发展到蚕食中国边疆地区}{设立完全由外国人直接控制和统治的租界}{外国资本在中国近代工业中争夺垄断地位}{\textcolor{red}{竞相租借港湾和划分势力范围}}
\begin{solution}帝国主义列强对中国的争夺和瓜分的图谋,在1894年中日甲午战争爆发后达到高潮。《中日马关条约》的签订,更大大刺激了帝国主义列强瓜分中国领土的野心,并激化了列强争夺中国的矛盾。俄国法国德国三国干涉还辽,要求租借中国港湾作为报酬。由此,德、俄、英、法、日等国于1898年至1899年竞相租借港湾和划分势力范围,掀起了瓜分中国才狂潮。
\end{solution}
\question 标志着中国完全沦为半殖民地半封建社会的不平等条约是( ~)
\par\twoch{《南京条约》}{《北京条约》}{《马关条约》}{\textcolor{red}{《辛丑条约》}}
\begin{solution}通过《辛丑条约》,帝国主义列强强迫清政府做出永远禁止中国人成立或加入任何反对它们的组织的承诺,并规定清政府各级官员如对人民反抗斗争``弹压惩办''不力,``即行革职,永不叙用''。中外反动势力彻底勾结在一起,中国完全沦为半殖民地半封建社会。
\end{solution}

\subsection{134-近代社会的性质}
\question 近代以来,中华民族面临着两个历史任务:一是求得民族独立和人民解放,二是实现国家的繁荣富强和人民的共同富裕,两者的相互关系是(
)
\par\fourch{\textcolor{red}{既相互区别又相互紧密联系}}{\textcolor{red}{前一个任务为后一个任务扫除障碍,创造必要的前提}}{\textcolor{red}{后一个任务是前一个任务的最终目的和必然要求}}{前一个任务是要发展社会生产力,实现中国的现代化,后一个任务是要改变落后 的生产关系和上层建筑}
\begin{solution}两个历史任务既相互区别又相互紧密联系。两大历史任务的主题、内容和实现
方式都不一样,前一个任务是从根本上推翻半殖民地半封建的统治秩序,改变落后的生
产关系和上层建筑,后一个任务是要改变近代中国经济、文化落后的地位和状况,发展
社会生产力,实现中国的现代化;前一个任务为后一个任务扫除障碍,创造必要的前
提;后一个任务是前一个任务的最终目的和必然要求。故选项ABC正确。
\end{solution}

\subsection{135-反抗外族侵略的斗争}
\question 近代中国人包括统治阶级中的爱国人物在反侵略斗争中表现出来的爱国主义精神,铸成了中华民族的民族魂,他们乃是中华民族的脊梁。在抗击外国侵略的战争中,许多爱国官兵英勇献身以身殉国,其代表人物有
\par\fourch{\textcolor{red}{鸦片战争期间,广东水师提督关天培、江南提督陈化成、副都统海龄}}{中法战争期间,督办台湾事务大臣刘铭传、老将冯子材}{\textcolor{red}{第二次鸦片战争中,提督史荣椿、乐善}}{\textcolor{red}{中日甲午战争时,致远舰管带(舰长)邓世昌、经远舰管带林永升}}
\begin{solution}【解析】1859年6月,英法联军大举进攻大沽炮台,守军沉着应战,击沉、击伤敌舰多艘。中法战争期间,1884年8月,法舰进犯台湾基隆,同年10月,又进犯淡水,都被督办台湾事务大臣刘铭传指挥守军击退。1885年初,法舰炮轰浙江镇海炮台,也被守军击退。3月,在中越边境镇南关(今友谊关),年近70的老将冯子材身先士卒,率部勇猛冲杀,大败法军,取得镇南关大捷。所以B项为干扰项。在抗击外国侵略的战争中,许多爱国官兵英勇献身。如:鸦片战争期间,广东水师提督关天培、江南提督陈化成、副都统海龄(满族);第二次鸦片战争中,提督史荣椿、乐善《蒙古族);中日甲午战争时,致远舰管带(舰长)邓世昌、经远舰管带林永升等,都以身殉国。
\end{solution}
\question 近代中国新产生的新的被压迫阶级是工人阶级,``中国工人阶级比中国资产阶级资格要老。因而它的社会力量和社会基础也更广大些'',这是因为(
~)
\par\fourch{它伴随着中国民族资产阶级的发生、发展而来}{\textcolor{red}{它伴随着外国资本在中国的直接经营的企业而来}}{它伴随着洋务派创办的洋务企业而来}{它伴随着买办、官僚地主投资兴办的企业而来}
\begin{solution}帝国主义通过与清政府签订条约获得了在中国投资办厂的权利,最早的工人就是给外企打工的员工。所以中国工人阶级比资产阶级先产生。
\end{solution}
\question 义和团运动的领导者宣称:``若辈洋人,借通商与传教以掠夺国人之土地、粮食与衣服,不仅污蔑我们的圣教,尚以鸦片毒害我们,以淫邪污辱我们。自道光以来,夺取我们的土地,骗取我们的金钱;蚕食我们的子女如食物,筑我们的债台如高山;焚烧我们的宫殿,消灭我们的属国;占据上海,蹂躏台湾,强迫开放胶州,而现在又想来瓜分中国。''上述材料说明(
~)
\par\fourch{义和团运动具有宗教战争的性质}{\textcolor{red}{义和团运动是民族意识觉醒的结果}}{义和团运动盲目排外}{\textcolor{red}{义和团运动具有反帝爱国性质}}
\begin{solution}义和团运动的性质首先是一场反帝爱国主义,反对西方侵略,又发生在甲午中日战争后,所以BD正确。外国资本主义入侵,同时宗教文化也进来,但是义和团没有以宗教的名义发动战争,不具有宗教战争的性质。材料中也没有提到义和团盲目排外。
\end{solution}

\subsection{136-反侵略战争的失败和民族意识的觉醒}
\question 近代以来中华民族面临的两大历史任务,就是争取民族独立、人民解放和实现国家富强、人民富裕。关于两大历史任务关系,下列说法错误的是
\par\fourch{两大历史任务领导阶级都一样}{前一个任务是从根本上推翻半殖民地半封建的统治秩序,改变落后的生产关系和上层建筑;后一个任务是要改变近代中国经济、文化落后的地位和状况,发展社会生产力,实现中国的现代化}{前一个任务为后一个任务扫除障碍,创造必要的前提;后一个任务是前一个任务的最终目的和必然要求}{\textcolor{red}{相互区别又紧密联系,两大历史任务的主题、内容与实现方式都一样}}
\begin{solution}【解析】两个历史任务都是由无产阶级领导。第一项任务是1949年新中国成立,标志着取得民族独立、人民解放的任务的完成。1949年以后至目前,我们正在进行的是第二项历史任务。两个历史任务领导阶级都是无产阶级------通过其先锋队中国共产党领导的。因此,A选项正确。近代以来中华民族面临的两大历史任务,就是争取民族独立、人民解放和实现国家富强、人民富裕。它们是相互区别又紧密联系的。两大历史任务的主题、内容与实现方式都不一样,因此D选项符合题意。前一个任务是从根本上推翻半殖民地半封建的统治秩序,改变落后的生产关系和上层建筑;后一个任务是要改变近代中国经济、文化落后的地位和状况,发展社会生产
力,实现中国的现代化。前一个任务为后一个任务扫除障碍,创造必要的前提;后一个任务是前一个任务的最终目的和必然要求。综上,ABC选项观点正确。
\end{solution}

\subsection{137-洋务运动}
\question 洋务派兴办民用企业的主要方式有( ~)
\par\twoch{\textcolor{red}{官办}}{\textcolor{red}{官督商办}}{\textcolor{red}{官商合办}}{商办}
\begin{solution}洋务派举办的民用企业的资金全部或大部由政府筹集,也吸收一部分商股,主要由政府派官员管理,有官办、官督商办、官商合办几种形式。其中,多数都采取官督商办的方式。因其资金主要来源于政府,因此不是商办的形式。
\end{solution}
\question 李鸿章说:``溯自各国通商以来,进口洋货日增月盛\ldots{}\ldots{}出口土货,年减一年,往往不能相敌。推原其故,由于各国制造均用机器。\ldots{}\ldots{}臣拟遴派绅商,在上海购买机器,设局仿造布匹,冀稍分洋商之利。''李鸿章的这段言论表明洋务派开始(
~)
\par\twoch{\textcolor{red}{兴办民用企业}}{\textcolor{red}{采用机器制造布匹,降低生产成本}}{发展民族资本主义}{\textcolor{red}{稍分洋商之利}}
\begin{solution}注意洋务运动是为了维护封建统治而发动的一场爱国救亡运动,从来没有主张变革社会制度或者发展民族资本主义。
\end{solution}

\subsection{138-维新运动}
\question 戊戌维新前维新派与守旧派的论战,实质上是( ~)
\par\fourch{\textcolor{red}{资产阶级思想与封建主义思想在中国的第一次正面交锋}}{守旧派与洋务派之间的一次论战}{保皇派与革命派的一次论战}{帝党与后党的权力之争}
\begin{solution}戊戌维新前封建守旧派和反对改变封建政治制度的洋务派,都利用自己的地位和权力,对维新思想发动攻击。维新派与守旧派的论战,实质上是资产阶级思想与封建主义思想在中国的第一次正面交锋。BCD表述错误。
\end{solution}

\subsection{139-辛亥革命的性质与早期工作}
\question 在资产阶级民主革命思潮广泛传播、革命形势日益成熟的时候,康有为、梁启超等人坚持走改良道路,反对用革命手段推翻清朝统治。1905年至1907年间,围绕中国究竟是采用革命手段
还是改良方式这个问题,革命派与改良派分别以《民报》《新民丛报》为主要舆论阵地,展开了一场大论战。双方论战涉及的核心问题主要有三个,其中双方论战的焦点是(
)
\par\fourch{要不要推翻帝制,实行共和}{\textcolor{red}{要不要以革命手段推翻清王朝}}{要不要实行君主立宪}{要不要社会革命}
\begin{solution}【解析】要不要以革命手段推翻清王朝,这是双方论战的焦点。改良派说,革命会引起下层社会
暴乱,招致外国的干涉、瓜分,使中国``流血成河''``亡国灭种'',所以要爱国就不能革命,只能改良、立宪。革命派针锋相对地指出,清政府是帝国主义的``鹰犬'',因此爱国必须革命。只有通
过革命,才能``免瓜分之祸'',获得民族独立和社会进步。A和D选项内容也属于这次论战三个
内容的其他两个。C选项``要不要实行君主立宪''是维新派和保守派论战的内容之一。
\end{solution}
\question 中国同盟会成立的革命宗旨是( ~)
\par\fourch{反对君主立宪派}{实行民族主义、民权主义、民生主义}{驱除鞑虏,恢复中华,创立合众政府}{\textcolor{red}{驱除鞑虏,恢复中华,创立民国,平均地权}}
\begin{solution}考查同盟会纲领的表述。C选项是兴中会的纲领,注意区分兴中会与同盟会。B选项是三民主义的表述,是同盟会纲领的概括。
\end{solution}
\question 1905年8月20日,中国同盟会成立。它的纲领是( ~)
\par\fourch{驱除鞑虏,恢复中华}{\textcolor{red}{驱除鞑虏,恢复中华,创立民国,平均地权}}{驱除鞑虏,恢复中华,创立合众政府}{驱除鞑虏,恢复中华,创立民国}
\begin{solution}同盟会的纲领是民族、民权、民生三大主义。民族主义包括``驱除鞑虏,恢复中华''两项内容;民权主义的内容是``创立民国'',民生主义的内容为``平均地权''。
\end{solution}

\subsection{140-理论工作:三民主义和革命改良的辩论}
\question 1905年11月,在同盟会机关报《民报》发刊词中,孙中山先生将同盟会的纲领概括为三大主义。后被称为三民主义。三民主义的核心是(
~)
\par\twoch{民族主义}{\textcolor{red}{民权主义}}{民生主义}{暴力革命}
\begin{solution}本题考查三民主义知识点。1905年,孙中山同黄兴、宋教仁等在日本东京组
建了中国同盟会。提出了纲领``驱除鞑虏,恢复中华,创立民国,平均地权''。后将这个纲领阐发为三民主义。民族主义是三民主义的前提,基本内容就是``驱除鞑虏,
恢复中华''。以``平均地权''为核心内容的民生主义,是孙中山三民主义中最具特色的部分。以``创立民国''为内涵的民权主义是三民主义的核心思想。
\end{solution}
\question 孙中山先生是伟大的民族英雄,伟大的爱国主义者,中国民主革命的伟大先躯,一生以革命为已任,立场救国救民,为中华民族作出了彪炳史册的贡献。孙中山先生的伟大表现在(
)。
\par\fourch{\textcolor{red}{坚定维护民主共和国制度和国家完整统一}}{发动了推翻北洋军阀统治为目标的北伐战争}{\textcolor{red}{重新解释三民主义并提出了联俄、联共、扶助农工三大政策}}{\textcolor{red}{领导了辛亥革命}}
\begin{solution}本题考查的是纪念孙中山诞辰150周年。习近平主席提到孙中山是伟大的爱国主义者,创立兴中会、同盟会,提出民族、民权、民生的三民主义,积极传播革命思想,广泛联合革命力量,连续发动武装起义,为推进民主革命四处奔走、大声疾呼。中国共产党成立后,孙中山先生同中国共产党人真诚合作,在中国共产党帮助下,把旧三民主义发展为新三民主义,实行联俄、联共、扶助农工三大政策。北伐战争是1926年-1927年,而孙中山先生1925年3月12日就逝世了,所以不能选B。
\end{solution}
\question 同盟会的政治纲领是``驱除鞑虏,恢复中华,创立民国,平均地权'',孙中山后来将其概括为三大主义,即民族主义、民权主义、民生主义,其中民族主义的含义是(
~)
\par\fourch{\textcolor{red}{反满,推翻清政府}}{排斥不同种族的人,建立汉族政权}{\textcolor{red}{建立民族独立的国家}}{\textcolor{red}{民族平等}}
\begin{solution}民族主义最大的缺陷就是没有反帝,并且反封主要指的是推翻满清贵族的统治。建立民族平等和独立的国家。
\end{solution}
\question 在孙中山的三民主义中,民生主义指的就是``平均地权'',也就是孙中山所说的社会革命,孙中山的民生主义(
~)
\par\fourch{\textcolor{red}{“平均地权”主张并非将土地所有权分给农民}}{\textcolor{red}{没有正面触及封建土地所有制}}{没有从正面鲜明的提出反对帝国主义的纲领}{\textcolor{red}{在革命中难以成为发动广大工农群众的理论武器}}
\begin{solution}民生即平均地权主要是指城市用地和城郊用地而不是耕地,所以没有主张讲土地所有权分给农民,没有触及封建土地所有制,也难以成为发动广大工农群众的理论武器。
\end{solution}
\question 1905年至1907年间,围绕中国究竟是采用革命手段还是改良方式这个问题,革命派与改良派展开了一场大论战。这场论战的主要内容包括(
~)
\par\fourch{\textcolor{red}{要不要以革命手段推翻清王朝}}{要不要实行政治革命}{\textcolor{red}{要不要推翻帝制,实行共和}}{\textcolor{red}{要不要社会革命}}
\begin{solution}考查的是辛亥革命中革命派与改良派的论战。这次论战的主要内容是:要不要以革命手段推翻清王朝,要不要推翻帝制,实行共和,要不要社会革命。
\end{solution}

\subsection{141-武昌起义与封建帝制的覆灭}
\question 毛泽东指出,辛亥革命``有它胜利的地方,也有它失败的地方''。其``失败的地方''表现在
\par\fourch{\textcolor{red}{中国反帝反封建的任务没有完成}}{\textcolor{red}{中国半殖民地半封建社会的性质没有改变}}{中国的封建君主专制制度没有改变}{\textcolor{red}{近代中国社会的主要矛盾没有解决}}
\begin{solution}常识题目,辛亥革命推翻了封建君主专制,所以C错误。
\end{solution}

\subsection{142-土地改革与第二条战线}
\question 在解放战争胜利发展的同时,解放区开展了轰轰烈烈的土地改革运动。土地制度的改革
\par\fourch{\textcolor{red}{是从根本上摧毁中国封建制度根基的社会大变革}}{\textcolor{red}{消灭了封建的生产关系}}{\textcolor{red}{使得农村生产力得到解放,工农联盟进一步巩固和加强}}{标志着社会主义改造在农村已经开始}
\begin{solution}D项错误,因为社会主义改造在农村开始是三大改造。
\end{solution}
\question 在中国共产党七届二中全会上,毛泽东提出了``两个务必''的思想,即务必使同志们继续地保持谦虚、谨慎、不骄、不躁的作风,务必使同志们继续保持艰苦奋斗的作风。其原因主要是
\par\fourch{全国大陆即将解放}{\textcolor{red}{中国共产党即将成为执政党}}{党的工作方式发生了变化}{中国将由新民主义社会转变为社会主义社会}
\begin{solution}更多免费押题卷,请下载口袋题库app
题干强调的``两个务必''是从党的作风的高度提出来的,B是最符合题意的正确选项。因为执政对党尤其是对党的作风是严峻的考验。A、D没有直接指明党即将成为执政党,因而与党的作风联系不直接;C仅仅从党的工作方式发生变化的角度看提出``两个务必''的原因也不准确。
\end{solution}
\question 明确规定``废除封建性及半封建性剥削的土地制度,实现耕者有其田的土地制度'',``乡村中一切地主的土地及公地,由乡村农会接收'',分配给无地或少地的农民的是(
~)
\par\fourch{\textcolor{red}{《中国土地法大纲》}}{《关于清算、减租及土地问题的指示》}{《兴国土地法》}{《井冈山土地法》}
\begin{solution}《中国土地法大纲》
~明确规定``废除封建性及半封建性剥削的土地制度,实现耕者有其田的土地制度'',``乡村中一切地主的土地及公地,由乡村农会接收'',分配给无地或少地的农民。注意与五四指示相区别。
\end{solution}

\subsection{143-新民主主义革命胜利的原因和基本经验}
\question 在中国共产党自身建设的问题上,提出了``两个务必''的要求的是在( ~)
\par\twoch{中共七大}{新政治协商会议}{\textcolor{red}{七届二中全会}}{十二月会议}
\begin{solution}1949年3月,中国共产党在西柏坡村召开了七届二中全会。在中国共产党自身建设的问题上,提出了``两个务必''的要求。
\end{solution}

\subsection{144-中共八大}
\question 1956年9月15日至27日,中国共产党第八次全国代表大会在北京举行。大会正确分析了社会主义改造完成后中国社会的主要矛盾和主要任务,制定了经济建设、政治建设、执政党建设的方针政策,指出
\par\fourch{\textcolor{red}{国内主要矛盾是人民对于经济文化迅速发展的需要同当前经济文化不能满足人民需要的状况之间的矛盾}}{\textcolor{red}{根本任务是在新的生产关系下保护和发展生产力}}{经济建设的方针是“三个主体、三个补充”}{\textcolor{red}{执政党建设上强调健全党内民主集中制,坚持集体领导制度,反对个人崇拜}}
\begin{solution}【简析】中共八大提出:社会主义制度在我国己经基本上建立起来;我们还必须为解放台湾、为彻底完成社会主义改造、最后消灭剥削制度和继续肃清反革命残余势力而斗争,但是国内主要矛盾己经不再是工人阶级和资严阶级的矛盾,而是人民对于经济文化迅速发展的需要同当前经济文化不能满足人民需要的状况之间的矛盾;全国人民的主要任务是集中力量发展社会生产力,实现国家工业化,逐步满足人民日益增长的物质和文化需要,虽然还有阶级斗争,还要加强人民民主专政,但其根本任务己经是在新的生产关系下保护和发展生产力。在执政党建设上,强调要提高全党的马克思列宁主义思想水平,健全党内民主集中制,坚持集体领导制度,反对个人崇拜,发展党内民主和人民民主,加强党和群众的
联系。A、B、D正确。在经济建设上,人会坚持既反保守又反冒进即在综合平衡中稳步前进的方针。陈云在人会发言中,提出``三个主体、三个补充''的思想,即:国家经营和集体经营是主体,一定数量的个体经营为补充;计划生产是主体,一定范围的自由生产为补充;国家市场是主体,一定范围的自由市场为补充。这个思想为大会所采纳,并写入决议,成为突破传统观念、探索适合中国特点的经
济体制的重要步骤。C 不符合题意。
\end{solution}
\question 中共八大指出,党和全国人民当前的主要任务是( ~)
\par\fourch{正确处理人民内部矛盾}{实现社会主义四个现代化}{把我国推进到社会主义社会}{\textcolor{red}{把我国从落后的农业国变为先进的工业国}}
\begin{solution}正确处理人民内部矛盾是中共八大以后,毛泽东在《关于正确处理人民内部矛盾的问题》中提出的;实现社会主义四个现代化是在十三届人大上提出的;三大改造完成以后,我国已经进入社会主义社会。
\end{solution}
\question 1956年11月召开的中共八届二中全会上正式提出整风,其要反对的错误倾向有(
~)
\par\twoch{\textcolor{red}{主观主义}}{\textcolor{red}{宗派主义}}{教条主义}{\textcolor{red}{官僚主义}}
\begin{solution}1957年4月27日,中共中央下发《关于整风运动的指示》,指出:由于党在全国范围内处于执政地位,有必要在全党进行一次反对官僚主义、宗派主义和主观主义的整风运动。教条主义是延安整风运动时主观主义的表现之一。
\end{solution}

\subsection{145-《论十大关系》和《关于正确处理人民内部矛盾的问题》}
\question 1957年2月,毛泽东在扩大的最高国务会议上发表《关于正确处理人民内部矛盾的问题》的讲话,阐明社会主义社会的重大理论问题,其中为社会主义制度的完善和发展奠定了理论基石的是
\par\fourch{\textcolor{red}{社会主义社会基本矛盾是非对抗性的}}{区分社会主义社会两类不同性质的社会矛盾}{团结全国各族人民进行一场新的战争一向自然界开战}{在社会主义制度下,人民的根本利益是一致的}
\begin{solution}1957年2月,毛泽东在扩大的最高国务会议上发表《关于正确处理人民内部矛盾的问题》的讲话,阐明社会主义社会的重大理论问题。主要内容有:第一,关于社会主义社会两类不同性质的社会矛盾。毛泽东指出:在社会主义制度下,人民的根本利益是一致的,但还存在着敌我矛盾和人民内部矛盾。必须区分社会主义社会两类不同性质的社会矛盾,把正确处理人民内部矛盾作为国家政治生活的主题。毛泽东提出正确处理人民内部矛盾的问题,有一个重要的指导思想,这就是:``团结全国各族人民进行一场新的战争一向自然界开战,发展我们的经济,发展我们的文化,使全体人民比较顺利地走过目前的过渡时期,巩固我们的新制度,建设我们的新国家''。第二,关于社会主义社会的基本矛盾。毛泽东首次提出社会主义社会基本矛盾的概念,并对社会主义社会的基本矛盾作了科学分析。他指出:矛盾是普遍存在的。社会主义社会充满着矛盾,正是这些矛盾推动着社会主义社会不断向前发展。在社会主义社会中,基本的矛盾仍然是生产关系和生产力之间的矛盾、上层建筑和经济基础之间的矛盾。社会主义社会基本矛盾是非对抗性的,可以通过社会主义制度本身的自我调整和自我完善不断地得到解决。这实际上为社会主义制度的完善和发展奠定了理论基石。故本
题选A。
\end{solution}
\question 以毛泽东为主要代表的中国共产党人开始探索中国自己的社会主义建设道路的标志是(
~)
\par\fourch{《关于正确处理人民内部矛盾的问题》}{\textcolor{red}{《论十大关系》}}{中共八大的召开}{“双百”方针的形成}
\begin{solution}《论十大关系》在新的历史条件下从经济方面(这是主要的)和政治方面提出了新的指导方针,为中共八大的召开作了理论准备。它是以毛泽东为主要代表的中国共产党人开始探索中国自己的社会主义建设道路的标志。《关于正确处理人民内部矛盾的问题》是一篇重要的马克思主义文献。它运用马克思主义对立统一规律,创造性地阐述了社会主义社会矛盾学说,是对科学社会主义理论的重要发展,进一步丰富和发展了中共八大路线,对中国社会主义事业具有长远的指导意义;中共八大正确地分析了国内的主要矛盾和主要任务,对于社会主义建设事业和党的事业的发展有着长远的指导意义;``双百''方针是在讨论十大关系时形成的。
\end{solution}
\question 1957年2月,毛泽东在扩大的最高国务会议上发表《关于正确处理人民内部矛盾的问题》的讲话,指出社会主义改造基本完成后,正确处理人民内部矛盾的方针有(
~)
\par\fourch{\textcolor{red}{“统筹兼顾,适当安排”}}{\textcolor{red}{“团结——批评——团结”}}{\textcolor{red}{“百花齐放,百家争鸣”}}{\textcolor{red}{“长期共存、互相监督”}}
\begin{solution}大纲解析原文表述,属于识记型题目。
\end{solution}
\question 毛泽东回顾说:``前八年照抄外国的经验。但从1956年提出十大关系起,开始找到自己的一条适合中国的路线。''毛泽东高度评价《论十大关系》是因为它(
~)
\par\fourch{正确分析了社会主义改造完成后中国社会的主要矛盾和主要任务}{\textcolor{red}{是中国共产党人开始探索中国自己的社会主义建设道路的标志}}{\textcolor{red}{总结经济建设的初步经验,借鉴苏联建设的经验教训,概括提出了十大关系}}{提出了“三个主体,三个补充”的思想}
\begin{solution}正确分析社会主义改造完成后中国社会主要矛盾和主要任务的是八大。``三个主体,三个补充''也是八大的内容。
\end{solution}

\subsection{146-整风与反右斗争}
\question 整风运动的主要内容是反对主观主义以整顿学风,反对宗派主义以整顿党风,反对党八股以整顿文风,其意义有
\par\fourch{\textcolor{red}{它是一场全党范围的集中的普遍的马克思主义思想教育运动}}{\textcolor{red}{它也是一场破除长期以来在中国共产党内存在的教条主义错误倾向的伟大的思想解放运动}}{\textcolor{red}{实事求是的马克思主义思想路线,在全党范围确立起来了}}{是中国共产党探索社会主义建设道路的新成果,增强了运用马克思主义基本原理解决中国革命实际问题的自觉性和能力}
\begin{solution}D项错在``探索社会主义建设道路的新成果'',后面那句是对的。
\end{solution}

\subsection{147-独立的,比较完整的工业体系和国民经济体系的建立}
\question 以毛泽东为主要代表的中国共产党人在创建新中国和探索适合中国情况的社会主义建设道路过程中,创造了一系列重要的理论。在社会主义民主政治建设方面的成果主要有
\par\fourch{\textcolor{red}{要把“造成一个又有集中又有民主,又有纪律又有自由,又有统一意志、又有个人心情舒畅、生动活泼,那样一种政治局面”作为努力的目标}}{\textcolor{red}{把正确处理人民内部矛盾作为国家政治生活的主题,坚持人民民主,尽可能团结一切可以团结的力量}}{\textcolor{red}{处理好中国共产党同各民主党派的关系,坚持长期共存、互相监督的方针,巩固和扩大爱国统一战线}}{坚持民主集中制和集体领导原则,反对任何形式的个人崇拜,保证中国共产党决策的科学化、民主化}
\begin{solution}【解析】以毛泽东为主要代表的中国共产党人在创建新中国和探索适合中国情况的社会主义建设道路过程中,创造了一系列重要的理论。在社会主义民主政治建设方面的成果有要把``造成一个又有集中又有民主,又有纪律又有自由,又有统一意志、又有个人心情舒畅、生动活泼,那样一种政治局面''作为努力的目标;把正确处理人民内部矛盾作为国家政治生活的主题,坚持人民民主,尽可能团结一切可以团结的力量;处理好中国共产党同各民主党派的关系,坚持长期共存、互相监督的方针,巩固和扩大爱国统一战线;切实保障人民当家作主的各项权利,尤其是人民参与国家和社会事务管理的权利;社会主义法制要保护劳动人民利益,保护社会主义经济基础,保护社会生产力。D项``必须坚持民主集中制和集体领导原则,反对任何形式的个人崇拜,保证中国共产党决策的科学化、民主化''是文化大革命留给共产党的深刻教训之一。
\end{solution}
\question 毛泽东提出要造就``一个又有集中又有民主,又有纪律又有自由,又有统一意志、又有个人心情舒畅、生动活泼,那样一种政治局面''(通称``六又''政治局面)的思想是在(
~)
\par\fourch{《论十大关系》}{《关于正确处理人们内部矛盾的问题》}{中共八大}{\textcolor{red}{《一九五七年夏季的形势》}}
\begin{solution}``六又''的政治局面思想是在《一九五七年夏季的形势》中提出来的,大纲解析原文。
\end{solution}



\section{[思修法基]思想道德修养与法律基础}


\subsection{148-树立理想信念:信仰与理想}
\question 如果说社会是大海,人生是小舟,那么理想信念就是引航的灯塔和推进的风帆。没有科学的理想信念的人生,就像失去了方向和动力的小船,在生活的波浪中随处漂泊,甚至会沉没于急流之中。这说明理想信念能够(
~)
\par\fourch{保证人生的追求成功}{\textcolor{red}{指引人生的奋斗目标}}{\textcolor{red}{提供人生的前进动力}}{\textcolor{red}{提高人生的精神境界}}
\begin{solution}本题难度不大,理想信念的作用表现在三个方面,可以直接选出来BCD选项。
\end{solution}
\question 马克思曾经说过:``作为确定的人,现实的人,你就有规定,就有使命,就有任务,至于你是否意识到这一点,那都是无所谓的。这个任务是由于你的需要及其与现存世界的联系而产生的。''当代大学生承担的历史使命是(
~)
\par\fourch{实现社会主义的现代化}{\textcolor{red}{建设中国特色社会主义}}{\textcolor{red}{实现中华民族伟大复兴}}{实现全面建设小康社会}
\begin{solution}当代大学生承担的历史使命表述为建设中国特色社会主义和实现中华民族的伟大复兴。AD选项属于建设中国特色社会主义的内容。
\end{solution}
\question 文王拘而演《周易》;仲尼厄而作《春秋》;屈原放逐,乃赋《离骚》;左丘失明,厥有《国语》;孙子膑脚,兵法修列;不韦迁蜀,世传《吕览》;韩非囚秦,《说难》、《孤愤》;《诗》三百篇,大抵圣贤发愤之所为作也。这对我们正确对待实现理想过程中的逆境的启示是(
~)
\par\fourch{人们在逆境中更容易接近和实现目标}{\textcolor{red}{人们在逆境中可以磨练意志、陶冶品格}}{\textcolor{red}{逆境没有消解实现理想目标的可能性}}{\textcolor{red}{逆境增大了人们向理想目标前进的难度}}
\begin{solution}文王拘而演《周易》;仲尼厄而作《春秋》;屈原放逐,乃赋《离骚》;左丘失明,厥有《国语》;孙子膑脚,兵法修列;不韦迁蜀,世传《吕览》;韩非囚秦,《说难》、《孤愤》;《诗》三百篇,都是说明要实现理想新信念,要在逆境中奋斗,付出更大的努力和更多的艰辛才能成功。逆境只是增大了人们实现目标的难度,所以A排除。
\end{solution}

\subsection{149-在实践中化理想为现实}
\question ``艰苦奋斗始终是激励我们为实现国家富强、民族振兴和人民幸福而共同奋斗的强大精神力
量''。这说明``艰苦奋斗''是
\par\fourch{通往理想彼岸的桥梁}{\textcolor{red}{实现理想的重要条件}}{立党立国的根本指导思想}{大学生成长成才的必由之路}
\begin{solution}本题是思想道德修养与法律基础第一章的调整考点,``艰苦奋斗''是实现理想的重要条件。
\end{solution}
\question 邓小平说:``美好的前景如果没有切实的措施和工作去实现它,就有成为空话的危险。''这说明(
~)
\par\fourch{\textcolor{red}{社会实践是联系理想和现实的桥梁}}{\textcolor{red}{有了理想并不意味着成功,更不意味着已经成功}}{\textcolor{red}{把理想转变为现实需要艰苦奋斗、勇于实践}}{只要付诸行动,人们对于美好未来的向往和追求都能成为现实}
\begin{solution}邓小平的这句话主要在说明理想如何转化为现实的问题。``切实的措施和工作''就已经说明实践是联系理想和现实的桥梁,只有实践才可以把理想转化为现实。因此,ABC正确。
\end{solution}
\question 一个小孩正在作画,先用寥寥几笔起了个草稿,未成形且没颜色,正当小孩对画作继续加工时,围观的人七嘴八舌地嘲笑小孩,画的是鸡?是鸟?是怪物?但小孩并不气馁,最后大功告成,一只美丽的凤凰跃然纸上。小孩最终能够画出令人惊艳的美丽凤凰的原因在于(
~)
\par\twoch{\textcolor{red}{坚定的信念}}{精湛的画艺}{\textcolor{red}{持之以恒的精神}}{\textcolor{red}{不畏挫折的勇气}}
\begin{solution}题干中体现的是不畏挫折,坚持到底的精神。围观的人的嘲笑说明表明画艺并不精湛,所以本题排除B选项。
\end{solution}

\subsection{150-爱国主义的要求与传统}
\question 在五千多年的发展史中,中华民族形成了以爱国主义为核心的团结统一、爱好和平、勤劳勇敢、自强不息的伟大民族精神。其中,中华民族创造一个又一个人间奇迹的重要精神动力是(
~)
\par\twoch{团结统一}{爱好和平}{\textcolor{red}{勤劳勇敢}}{自强不息}
\begin{solution}本题基本没有难度,属于概念记忆的题型。
\end{solution}
\question 中华民族的爱国主义优良传统源远流长。许多为国家和民族作出杰出贡献的仁人志士,其英雄业绩为历史和人民所铭记。下列选项反映了中华民族爱国主义优良传统的有(
~)
\par\fourch{\textcolor{red}{“人生自古谁无死,留取丹心照汗青”}}{\textcolor{red}{“怒发冲冠”、“还我河山”}}{\textcolor{red}{“王师北定中原日,家祭无忘告乃翁”}}{\textcolor{red}{“一年三百六十日,多是横戈马上行”}}
\begin{solution}选项中的诗句都很好理解,都体现了热爱祖国,矢志不渝;天下兴亡,匹夫有责;维护统一,反对分裂和同仇敌忾,抗御外辱的爱国主义精神。
\end{solution}
\question 法国杰出的科学家、微生物学的奠基人巴斯德曾经说过:``科学是没有国界的,因为它是属于全人类的财富,是照亮世界的火把,但学者是属于祖国的。''这说明(
~)
\par\fourch{\textcolor{red}{科学具有超越性和国际性}}{\textcolor{red}{科学是人类智慧的结晶,理应为全人类服务}}{科学所具有的民族性决定了学者是属于祖国的}{\textcolor{red}{科学事业的发展和科学家的命运都与自己的祖国有着密切的关系}}
\begin{solution}C科学没有民族性,是全人类的财富,故C错误。
\end{solution}
\question 显克微支说:``不忠不义,不爱祖国,无有良知,无有韧力,而此德行的沉沦,是任何国家任何民族最不齿的。''做忠诚的爱国者,自觉维护国家利益,要求我们(
~)
\par\fourch{\textcolor{red}{承担对国家应尽的义务}}{\textcolor{red}{维护改革发展稳定的大局}}{\textcolor{red}{树立民族自尊心和自豪感}}{行使国家赋予的权利}
\begin{solution}做忠诚的爱国者,自觉维护国家利益,这是我们应履行的责任,而非权利,故D错误。
\end{solution}

\subsection{151-新时期的爱国主义与爱国主义的时代价值}
\question 邓小平指出:``港澳、台湾、海外的爱国同胞,不能要求他们都拥护社会主义,但是至少也不能反对社会主义的新中国,否则怎么叫爱祖国呢?''这句话表明(
)
\par\fourch{爱国主义与爱社会主义具有一致性}{\textcolor{red}{港澳、台湾、海外同胞,只要承认世界上只有一个中国,就是爱国}}{\textcolor{red}{爱国主义与拥护祖国统一的一致性,是对港澳台、海外同胞的基本要求}}{爱国主义与经济全球化具有一致性}
\begin{solution}从论述可知爱国主义与爱社会主义不一致。故选项A错误。选项D本身错
误。选项BC正确。
\end{solution}

\subsection{152-民族精神与时代精神}
\question 鲁迅曾说:``惟有民魂是值得宝贵的,惟有他发扬起來,中国才有真进步。''实现中国梦必须弘扬中国精神。中国精神是兴国强国之魂,是
\par\fourch{\textcolor{red}{实现民族复兴的精神引领}}{\textcolor{red}{凝聚中国力量的精神纽带}}{\textcolor{red}{提升综合国力的重要保证}}{政治文明建设的里要内容}
\begin{solution}实现中国梦必须弘扬中国精神。
这就是以爱国主义为核心的民族精神,以改革创新为核心的时代精神。中国精神作为兴国强国之魂,是实现中华民族伟大复兴不可或缺的精神支撑和精神动力。第一,是实现民族复兴的精神引领。第二,是凝聚中国力量的精神纽带。第三,是提升综合国力的重要保证。A
、B、C正确,中国精神属于精神文明的范畴, D不符合题意。
\end{solution}

\subsection{153-人生观、人生目的、人生态度与人生价值}
\question 促进个人与他人的和谐要坚持四个原则,其中作为保证的是
\par\twoch{平等原则}{宽容原则}{\textcolor{red}{诚信原则}}{互助原则}
\begin{solution}诚信是促进个人与他人和谐的保证。诚信包含着诚实和守信两方面的意思,诚是信的内在思想基础,信是诚的外在表现。诚信历来被视为处理个人与他人关系的基本准则。C正确。平等待人是促进个人与他人和谐的前提;宽容是促进个人与他人和谐必不可少的条件;互助是促进个人与他人和谐的必然要求,A、B、D不符合题意。
\end{solution}
\question 在人类历史长河中涌现过形形色色的人生观,只有以为人民服务为核心内容的人生观,才是科学高尚的人生观,才值得终生尊奉和践行。树立为人民服务的人生观,要坚决抵制各种错误思想的影响。由于受国内外各种错误思潮、腐朽观念等各种因素的影响,现实中还存在拜金主义、享乐主义和极端个人主义等对人生目的的错误看法。这些错误的思想观念容易侵蚀
大学生的纯洁心灵,不利于大学生树立科学高尚的人生观和价值观。上述种种错误的思想和观念,尽管在形式上五花八门,内容不尽一致,但它们却有着共同的特征,它们(
~)
\par\fourch{\textcolor{red}{都表达了剥削阶级的人生观对人生目的的主张}}{\textcolor{red}{都没有把握个人与社会的正确关系}}{都是生产资料私有制的产物,是资产阶级世界观的核心}{\textcolor{red}{对人的需要的理解是片面的,夸大了人生的某方面需要,而无视人的全面性和人生的整体需要}}
\begin{solution}【解析】拜金主义、享乐主义和极端个人主义等错误的人生观,尽管在形式上五花八门,内容不
尽一致,但它们却有着共同的特征。其一,它们都表达了剥削阶级的人生观对人生目的的主
张,反映的都是剥削阶级的腐朽观念,不可能具有劳动人民的宽广胸怀和远大志向,更不能代
表人民群众的利益。其二,它们都没有把握个人与社会的正确关系,忽视或否认社会性是人
的存在和活动的本质属性,它们讨论人生问题的出发点和落脚点都是褊狭的一己之私利。其
三,它们对人的需要的理解是片面的,夸大了人生的某方面需要,而无视人的全面性和人生的
整体需要。据此,本题选ABD。C选项不能人选。个人主义是生产资料私有制的产物,是资
产阶级世界观的核心,不能笼统说三者都是。
\end{solution}
\question 人生观的核心是( ~)
\par\twoch{人生价值}{\textcolor{red}{人生目的}}{人生态度}{人生信仰}
\begin{solution}人生目的决定着人生价值的大小、类型,是人生观的核心。教材仅论及``人生目的、人生态度、人生价值''三个概念。因此,B正确。
\end{solution}
\question 成就何种人生,是事业有成,还是庸碌无为;是崇高善良,还是卑鄙邪恶;是彪炳史册,还是遗臭万年,除了客观历史条件和机遇等因素的影响外,在很大程度上,将取决于人们有什么样的人生观、价值观。作为人生观核心的是(
~)
\par\twoch{\textcolor{red}{人生目的}}{人生态度}{人生价值}{人生意义}
\begin{solution}概念记忆题。
\end{solution}
\question 世界观是指人们对世界的总的根本的看法。人生观是指对人生的看法,也就是对于人类生存的目的、价值和意义的看法。世界观和人生观是紧密联系在一起的。这主要表现在(
~)
\par\fourch{\textcolor{red}{世界观决定人生观}}{人生观决定世界观}{\textcolor{red}{人生观对世界观的巩固、发展和变化起重要作用}}{世界观对人生观的巩固、发展和变化起重要作用}
\begin{solution}本题难度很小,书面知识记忆。
\end{solution}

\subsection{154-人生价值及其实现}
\question 华罗庚生前曾说:``我们最好把自己的生命看作是前人生命的延续,是现在人类共同的生命的一部分,同时也是后人生命的开端。如此延续下去,科学就会一天比一天更灿烂,社会就会一天比一天更美好。''这段话对我们如何实现人的个人价值的教益是(
~)
\par\fourch{\textcolor{red}{个人价值的实现与社会价值的实现是统一的}}{\textcolor{red}{个人价值的实现是一个历史过程}}{个人价值的实现是社会价值的实现的归宿}{个人价值的实现和个人生命的长短相一致}
\begin{solution}本题干扰项也是较为容易排除的。C选项,明显表述``反了'',应该是社会价值的实现是个人价值的实现的归宿。D选型,明显荒谬,如果D选项是对的,那么雷锋同志的人生价值就很少了。所以选出AB选项。
\end{solution}
\question 人生价值具有的特点有( ~)
\par\twoch{\textcolor{red}{客观性}}{\textcolor{red}{社会性}}{\textcolor{red}{差异性}}{\textcolor{red}{创造性}}
\begin{solution}人生价值是一种特殊的价值,它是人的生活实践对于社会和个人所具有的作用和意义。它是在社会的创造性的实践中实现的,它必须符合社会的客观规律,而且每个人由于具有个体的差异性,他们的人生价值也有所不同。C最易被漏选。
\end{solution}
\question 所谓人生的社会价值,就是个体的人生对于社会和他人的意义。人生社会价值的基本标志有(
~)
\par\twoch{\textcolor{red}{劳动}}{\textcolor{red}{贡献}}{索取}{实践}
\begin{solution}人生价值评价的根本尺度,是看一个人的人生活动是否符合社会发展的客观规律,是否通过实践促进了历史的进步。劳动以及通过劳动对社会和他人做出的贡献,是社会评价一个人的人生价值的普遍标准。
\end{solution}
\question 马克思指出:``人是最名副其实的政治动物,不仅是一种合群的动物,而且是只有在社会中才能独立的动物。''这表明(
~)
\par\fourch{\textcolor{red}{社会是个人生存和发展的基础}}{\textcolor{red}{个人是构成社会的前提}}{\textcolor{red}{个人与社会不可分离}}{\textcolor{red}{个人与社会是对立统一的}}
\begin{solution}题干中讲人与社会的对立统一的关系。社会是个人生存发展的基础,社会是由个人构成的。
\end{solution}

\subsection{155-科学对待人生环境}
\question 人们通过生活实践所形成的对人生问题的一种稳定的心理倾向和基本意图是( ~)
\par\twoch{人生观}{人生价值}{\textcolor{red}{人生态度}}{人生目的}
\begin{solution}本题考查人生观与人生价值、人生态度、人生目的之间的区别。人生观是世界观的重要组成部分,是人们在实践中形成的对于人生目的和意义的根本看法。人生观主要是通过人生目的、人生态度和人生价值三个方面体现出来的。人生目的,回答``人为什么活着'';人生态度,表明人``应当怎样对待生活,究竟应该怎样活着''也即人们通过生活实践形成的对人生问题的一种稳定的心理倾向和基本意愿,极易与``人生目的''混淆。人生价值,判别``什么样的人生才有意义,什么样的人生目的最值得追求''。因此,C正确。
\end{solution}

\subsection{156-道德及其作用}
\question 道徳的功能集中表现为,它是处理个人与他人、个人与社会之间关系的行为规范及实现自我完善的一种重要精神力量。在道德的功能系统中,主要的功能是
\par\twoch{\textcolor{red}{认识功能}}{\textcolor{red}{规范功能}}{\textcolor{red}{调节功能}}{表达功能}
\begin{solution}【简析】道徳的功能集中表现为,它是处理个人与他人、个人与社会之间关系的行为规范及实现自我完善的一种重要精神力量。在道德的功能系统中,主要的功能是认识功能、规范功能和调节功能。A、B、C
正确,D错误。
\end{solution}
\question 道德与法律都属于社会规范的范畴,道德与法律既有区别又有联系。下列有关法律与道德的几种表述中,错误的是(
~)
\par\fourch{法律具有既重权利又重义务的“两面性”,道德具有只重义务的“一面性”}{道德的强制是一种精神上的强制}{道德是法律的补充}{\textcolor{red}{法律是道德的补充}}
\begin{solution}道德规范的范围更广,有些领域用法律去规范是不合适的,例如说情侣之间的关系,所以只能用道德去规范,道德是法律的补充而非法律是道德的补充。
\end{solution}

\subsection{157-中华民族传统道德}
\question 中华传统美德和中国革命道德是一脉相承的。其一脉相承性主要体现在
\par\fourch{\textcolor{red}{中华传统美德是中国革命道德的渊源之一}}{\textcolor{red}{它们形成的基础都是中华传统道德的精华}}{\textcolor{red}{它们都是对中国优良道德传统的延续和发展}}{中国革命道德超越了中华传统美德的时代局限}
\begin{solution}此题考查的知识点是对中国革命道德的理解,是一道理解性试题,难度较大。中国革命道德是指中国共产党人、人民军队、一切先进分子和人民群众在中国新民主主义革命和社会主义革命、建设与改革
中所形成的优良道德。它是马克思主义与中国革命、建设与改革的伟大实践相结合的产物,是对中国优良
道德传统的继承和发展,是中华传统美德新的升华和质的飞跃。中华传统美德是中国革命道德的渊源之
一,从一定意义上来说,没有中华传统美德的长期发展和丰厚积淀,就不可能有中国革命道德的形成和发
展。中国革命道德继承了中国传统道德的精华,摒弃了传统道德的糟粕,是中国优良传统道德的延续和发展,是超越了中华传统美德的时代局限而形成的一种崭新的道德。ABC选项正确。D选项不选的原因在于:它是讲中国革命道德与中华传统美德的不同。
\end{solution}
\question 下列名言与``诚者天之道也,思诚者人之道也。至诚而不动者,未之有也;不诚,未有能动者也''
意思表述一致的是
\par\fourch{守法和有良心的人,即使有迫切的需要也不会偷窃,可是,即使把百万金元给了盗贼,也没
法儿指望他从此不偷不盗}{没有伟大的品格,就没有伟大的人,甚至也没有伟大的艺术家,伟大的行动者}{不自见,故明;不自是,故彰;不自伐,故有功;不自矜,故长}{\textcolor{red}{小信成则大信立,故明主积于信。赏罚不信,则禁令不行}}
\begin{solution}本题考查思想道德对古文诗词的理解应用。抓住古诗文当中``诚''这一关
键字,对应D项中的``诚信''。
\end{solution}
\question 关于对待传统道德态度的表述,错误的有( ~)
\par\fourch{\textcolor{red}{坚持文化复古主义,中国的落后就是因为儒家文化的失落}}{古为今用,洋为中用}{\textcolor{red}{实行历史虚无主义,即中国要全盘西化}}{坚持以我为主,为我所用的基本原则}
\begin{solution}在对待传统道德的问题上,文化复古主义和历史虚无主义都是应该予以否定的错误思潮,我们应该坚持的是用马克思主义的立场、观点、看法,坚持以我为主、为我所用的原则,既反对全盘西化、机械照搬,又反对全盘否定,盲目排外,在批判的基础上加以借鉴、吸收,剔除其带有阶级和时代局限性的糟粕,吸收其带有普遍性和一般性,对今天有积极意义的精华。因此,AC正确。
\end{solution}
\question 中国古代思想家强调在``义''和``利''发生矛盾时,应当``义以为上''、``先义后利''、``见利思义'',主张``义然后取'',反对``重利轻义''和``见利忘义''。这种思想反映了中华民族优良道德传统的(
~)
\par\fourch{\textcolor{red}{注重整体利益、国家利益和民族利益}}{讲求谦敬礼让,强调克骄防矜}{倡导言行一致,强调恪守诚信}{\textcolor{red}{强调对社会、民族、国家的责任意识和奉献精神}}
\begin{solution}``义以为上,先义后利'',``重利轻义,见利忘义''强调的是注重整体利益,国家民族利益,强调对社会、国家民族的奉献精神。
\end{solution}


\section{形势与政策以及当代世界经济与政治}

\subsection{158-世界总体格局的演变}
\question 20世纪70年代,毛泽东审时度势,果断地决定打开中美关系的大门,提出的外交战略是
\par\fourch{“另起炉灶”“打扫干净屋子再请客”“一边倒”}{和平共处五项原则}{“真正的不结盟”}{\textcolor{red}{“一条线”}}
\begin{solution}【答案】D
【解析】新中国成立初期,毛泽东提出``另起炉灶''``打扫干净屋子再请客''
``一边倒''三大外交方针。1953年12月,周恩来在会见印度政府代表团时,首次系统地提出了和平共处五项原则。20世纪70年代,毛泽东审时度势,果断地决定打开了中美关系的大门,提出了``一条线''
的外交战略。20世纪80年代以来,邓小平根据形势的变化,果断、及时地指导我们党改变了以往的外交战略,确定了``真正的不结盟''战略。
\end{solution}

\subsection{159-加强和规范党内政治生活加强和规范党内政治生活(新增)}
\question 全面从严治党,必须狠抓组织、纪律和作风建设。我们党之所以强大,之所以成为得到人民衷心拥护的坚强马克思主义政党,靠的就是党组织的先进性、铁的纪律和优良的作风。党要管
党、从严治党,必须首先落实在党的组织建设方面。组织建设的重点是( ) '
\par\fourch{\textcolor{red}{造就高素质的党员、干部队伍}}{解决为官不正、为官不为、为官乱为等问题}{把党的纪律和规矩立起来、严起来,执行到位}{促使干部自觉践行“三严三实”要求}
\begin{solution}党要管党、从严治党,必须首先落实在党的组织建设方面。组织建设的重点是造就高
素质的党员、干部队伍。本题属于简单记忆型考点。
\end{solution}

\subsection{161-十月国内时政}
\question 党的十九大报告指出,中国特色社会主义进入新时代,我国社会主要矛盾已经转化为人民日益增长的美好生活需要和不平衡不充分的发展之间的矛盾。我国社会主要矛盾的变化
\par\fourch{\textcolor{red}{没有改变我们对我国社会主义所处历史阶段的判断}}{\textcolor{red}{没有改变我国仍她于并将长期处于社会主义初级阶段的基本国情}}{\textcolor{red}{没有改变我国是世界最大发展中国家的国际地位}}{没有改变此前我们对我国发展所处历史方位的判断}
\begin{solution}
\end{solution}
\question 党的十九大,是在全面建成小康社会决胜阶段、中国特色社会主义发展关键时期召开的一次十分重要的大会
。 我们党要明确宣示
\par\fourch{\textcolor{red}{举什么旗、走什么路}}{\textcolor{red}{以什么样的精神状态}}{\textcolor{red}{担负什么样的历史使命}}{\textcolor{red}{实现什么样的奋斗目标}}
\begin{solution}
\end{solution}

\subsection{162-2018年1月国际时政}
\question {{每年一度最大规模的国际消费电子展在(}
{)举办,出席的中国参展商有}1300{多家,占全部数量的}{30\%}{;美国参展商}{1600}{多家,占}{38\%}{。}}
\par\fourch{\textcolor{red}{拉斯维加斯}}{香港 }{深圳}{米兰 }
\begin{solution}时事题
\end{solution}
\question {1{月}{22}{日,美国彭博社发布}{2018}{年世界创新指数,中国位居(
),比前一年提升两位。}}
\par\fourch{第十五位}{第十七位}{\textcolor{red}{第十九位}}{第二十一位}
\begin{solution}C
\end{solution}
\question {{北美自由贸易协定(}
{)谈判}1{月}{29}{日在加拿大蒙特利尔结束。尽管谈判``向前迈出了一步'',与会的加拿大、美国和墨西哥开始触及某些核心问题,但``进展极为缓慢'',仍存在巨大分歧。}}
\par\fourch{第二轮}{第四轮}{\textcolor{red}{第六轮}}{第八轮}
\begin{solution}C
\end{solution}
\question {{银联国际与南非(}
{)}2{月}{5}{日在约翰内斯堡达成合作协议,银联将借助(
)非洲市场网络渠道,深入推广银联卡支付服务,为中非经贸往来提供更大便利。(
)也由此成为非洲首家获发行银联卡资质的金融机构。}}
\par\fourch{\textcolor{red}{标准银行}}{国际银行}{国家银行}{人民银行}
\begin{solution}A\\
\end{solution}

\subsection{163-2016年张修齐政治真题}
\question 2015年召开的中央统战工作会议强调,我们党所处的历史方位、所面临的内外形势、所肩负的使命任务发生了重大变化。越是变化大,越是要把统一战线发展好、把统战工作开展好。统一战线作为党的一项长期方针,决不能动摇。中国共产党之所以高度重视统战工作,因为统一战线是
\par\fourch{\textcolor{red}{夺取革命、建设和改革事业胜利的重要法宝}}{\textcolor{red}{实现中华民族伟大复兴中国梦的重要法宝}}{\textcolor{red}{中国共产党的一大政治优势}}{人民当家作主的根本保证}
\begin{solution}答案:ABC\\
修齐解析(原文复现类)大纲知识点回顾``统一战线作为中国共产党不断夺取革命、建设和改革事业胜利的重要法宝,作为实现祖国统一和中华民族伟大复兴的重要法宝,决不能丢掉;作为党的一个政治优势,决不能削弱;作为党的一项长期方针,决不能动摇。''所以选ABC。对于选项D,``中国共产党的领导是人民当家作主和依法治国的根本保证。''\\
本题难度系数0.645.
\end{solution}

\subsection{164-2013年张修齐政治真题}
\question 党的十八大把科学发展观同马克思列宁主义、毛泽东思想、邓小平理论、``三个代表''重要思想一道确立为党必须长期坚持的指导思想。科学发展观是~~~
\par\fourch{中国革命、建设、改革经验的科学总结}{\textcolor{red}{中国特色社会主义理论体系的最新成果}}{\textcolor{red}{指导党和国家全部工作的强大思想武器}}{\textcolor{red}{中国共产党集体智慧的结晶}}
\begin{solution}答案: BCD\\
修齐解析(时效热点类)十八大报告指出``科学发展观是中国特色社会主义理论体系的最新成果,中国共产党集体智慧的结晶,是指导党和国家全部工作的强大思想武器。''选BCD。另外对于A选项,科学发展观并不是中国革命经验的科学总结。\\
本题难度系数0.331.主要干扰项ABCD-0.528.
\end{solution}

\subsection{165-资本主义的发展及趋势}
\question 第二次世界大战结束以来,在国家垄断资本主义获得充分发展的同时,国家通过宏观调节和微观规制对生产、流通、分配、消费各个环节的干预也更加深人。其中,微观规制的类型主要有
\par\twoch{\textcolor{red}{社会经济规制}}{\textcolor{red}{公共事业规制}}{财政政策、货币政策等经济手段}{\textcolor{red}{反托拉斯法}}
\begin{solution}【简析】微观规制主要是国家运用法律手段规范市场秩序,限制垄断,保护竞争,维护社会公众的合法权益。微观规制主要有三种类型:其一,反托拉斯法;其二,公共事业规制;其三,社会经济规制。A、B、D正确。国家运用财政政策、货币政策等经济手段,对社会总供求进行调节,属于宏观调节的主要手段,不是微观规制的内容。C错误。
\end{solution}
\question 全球最大的手机芯片厂商之一美国髙通公司2015年2月10日宣布,将向中国官方支付60.88亿元人民币(约合9.75亿美元)的反垄断罚款,了结为期14个月的反垄断调査,这是中国反垄断开出的最大一张罚单。2015年11月中旬,高通又被韩国反垄断机构认定其专利授权模式违反该国反垄断法,导致该公司股票在11月18日暴跌10\%。这是高通在该月遭遇的第二次监管挫折。高通之前还披露,虽然已经在中国和解了反垄断诉讼,但该公司在与部分中国手机制造商洽谈新的授权协议时仍然面临一些问题。这两起事件共导致高通股票市值蒸发200亿美元。垄断资本向世界扩展的经济动因包括
\par\fourch{\textcolor{red}{争夺商品销售市场}}{\textcolor{red}{将国内过剩的资本输出,以在别国谋求髙额利润}}{将部分要害核心技术转移到国外,以取得在别国的垄断优势}{\textcolor{red}{确保原材料和能源的可靠来源}}
\begin{solution}【简析】垄断资本在1肉建立了垄断统治后,必然要把其统治势力扩展到国外,建立国际垄断统治。垄断资本向世界范围扩展的经济动因:一是将国内过剩的资本输出,以在别国谋求高额利润;二是将部分非要害技术转移到国外,以取得在别国的垄断优势;三是争夺商品销售市场;四是确保原材料和能源的可靠来源。这呰经济上的动因与垄断资本主义政治上、文化上、外交上的利益交织一起,共同促进了垄断资本主义向世界范闹的扩展。正确,C错误。注意:C错在``要害核心技术'',正确说法应该是``非要害技术''。
\end{solution}
\question 各资本主义国家的垄断组织,通过订立协议建立起国际垄断资本的联盟,即国际垄断同盟,以便在世界范闹形成垄断,并在经济上瓜分世界。以下说法正确的是
\par\fourch{\textcolor{red}{国际垄断同盟在经济上瓜分世界是通过垄断组织间的协议实现的}}{早期的国际垄断同盟主要是跨国公司}{当代国际垄断同盟的形式以国际卡特尔和国家垄断资本主义的国际联盟为主}{\textcolor{red}{国家垄断资本主义的国际联盟,是国际垄断同盟的高级形式}}
\begin{solution}【简析】A、D正确。早期的国际垄断同盟主要是国际卡特尔,即若干垄断资本主义国家的生产或经荐某种产品的垄断组织,通过订立国际卡特尔协议,垄断和瓜分这种产品的世界市场,规定垄断价格,谋求垄断利润。当代国际垄断同盟的形式以跨国公司和国家垄断资本主义的国际联盟为主D
B、C错误。
\end{solution}
\question 随着资本输出的不断增加和垄断资本势力范ffl的迅速扩大,各资本主义国家的垄断组织,通过订立协议违立起围际垄断资本的联盟,即诚际垄断同盟,同时,还建立起国际经济调节机制,以加强国际协调\ldots{}.第二次世界大战后,从事国际经济协调、维护国际经济秩序的闽际性协调组织主要有
\par\twoch{联合国}{\textcolor{red}{国际货币基金组织}}{\textcolor{red}{世界银行}}{\textcolor{red}{世界贸易组织}}
\begin{solution}【简析】随着资本输出的不断增加和垄断资本势力范围的迅速扩大,各国之间的经济联系曰益密切,同时,彼此间的竞争更为激烈,矛盾和冲突也更为突出。在这个背景下,各资本主义国家的垄断组织,通过订立协议建立起国际垄断资本的联盟,即国际垄断同盟,以便在世界范围形成垄断,并在经济上瓜分世界。国际垄断资本还建立起国际经济调节机制,以加强国际协调。第二次世界大战后,从事国际经济协调、维护国际经济秩序的国际性协调组织主要有三个:国际货币基金组织、世界银行和世界贸易组织B、C、D正确,A错误。
\end{solution}
\question 经济全球化是指在生产不断发展、科技加速进步、社会分工和国际分工不断深化、生产的社会化和了际化程度不断提高的情况下,世界各国、各地区的经济活动越来越超出一国和地区的范围而相互联系、相互依赖的一体化过程。经济全球化的表现包括
\par\fourch{国际垂直分工成为居主导地位的分工形式}{\textcolor{red}{贸易的全球化}}{\textcolor{red}{金融的全球化}}{\textcolor{red}{企业生产经营的全球化}}
\begin{solution}【简析】经济全球化的表现包括:一是国际分工进一步深化。在经济全球化过程中,国际水平分工逐渐取代国际垂直分工成为居主导地位的分工形式。A错误。二是贸易的全球化。贸易全球化是指商品和劳务在全球范闹内自由流动。三是金融的全球化。金融全球化是指世界各国、各地区在金融业务、金融政策等方面相互协调、相互渗透、相互竞争不断加强,使全球金融市场更加开放、金融体系更加融合、金融交易更加自由的过程。四是企业生产经营的全球化。企业生产经营全球化指跨国公司在全球范围内建立分支机构,借助母公司与分支机构之间各种形式的联系,实行跨国投资和生产的过程。B、C、D正确。
\end{solution}
\question 导致经济全球化迅猛发展的因素主要有
\par\fourch{\textcolor{red}{科学技术的进步和生产力的发展}}{\textcolor{red}{跨国公司的发展}}{\textcolor{red}{各国经济体制的变革}}{发展中国家的斗争}
\begin{solution}【简析】略
\end{solution}
\question 当代资本主义国家经济出现的新变化主要表现在
\par\fourch{\textcolor{red}{生产资料所有制的变化}}{\textcolor{red}{劳资关系和分配关系的变化}}{改良主义政党在政治舞台上的影响日益扩大}{\textcolor{red}{经济调节机制和经济危机形态的变化}}
\begin{solution}【简析】改良主义政党在政治舞合上的影响日益扩大属于政治制度的变化,与题干所问的经济方面的变化不符,不选。
\end{solution}
\question 对资本主义国家的国家资本所有制认识正确的是
\par\fourch{\textcolor{red}{国家作为出资人,拥有国有企业的所有权和控制权}}{\textcolor{red}{固有企业的重要职能是推行政府的社会政策和经济政策,为私人垄断资本的发展提供服务和保障}}{是一种基于资本雇佣劳动的垄断资本集体所有制}{\textcolor{red}{所有制的性质仍然是资本主义性质,体现着总资本家剥削雇佣劳动者的关系}}
\begin{solution}【简析】A、B是资本主义国家所有制的主要特点,D是其地位与性质,都是正确选项。法人资本所有制是一种基于资本雇佣劳动的垄断资本集体所有制;国家资本所有制是生产资料由国家占有并服务于垄断资本的所有制形式,体现着总资本家剥削劳动者的关系,C错误。
\end{solution}
\question 当代资本主义国家在劳资关系和分配关系上出现的新变化布
\par\twoch{\textcolor{red}{职工参与决策}}{\textcolor{red}{终身雇佣、职工持股}}{\textcolor{red}{建立并实施了普及化、全民化的社会福利制度}}{建立职工选举管理者制度}
\begin{solution}【简析】资本主义国家没有也不可能建立职工选举管理者制度,D错误。
\end{solution}
\question 在当代资本主义生产关系中,阶层、阶级结构发生了新的变化,主要有
\par\fourch{\textcolor{red}{髙级职业经理成为大公司经营活动的实际控制者}}{\textcolor{red}{知识型和服务型劳动者的数坫不断增加}}{中产阶级迅速崛起,占有社会财富的绝大部分}{\textcolor{red}{资本家的地位和作用已经发生很大变化}}
\begin{solution}【简析】在当代资本主义生产关系中,阶层、阶级结构发生了新的变化。一趄资本家的地位和作用发生了很大的变化。资本所有权和经营权发生分离,拥有所有权的资本家一般不再a接经营和管理企业,而是靠拥有的企业股票等有价诬券的利息收人为生。二是高级职业经理成为大公司经营活动的实际控制者。三是知识型和服务型劳动者的数®不断增加,劳动方式发生了新变化。A、B、D正确。资本主义国家的少数富人占有社会财富的大部分,C不符合事实,不选。
\end{solution}
\question 与第二次世界大战前的资本生义相比,当代资本主义在许多方而已经并正在发生变化。其中在政治制度上的变化有
\par\fourch{\textcolor{red}{政治制度出现多元化的趋势,公民权利有所扩大}}{民粹主义影响上升}{\textcolor{red}{法制建设得到重视和加强,以协调社会各阶级、阶层之间的利益}}{\textcolor{red}{改良主义政党在政治舞台上的影响日益扩大}}
\begin{solution}【简析】当代资本主义在政治制度上的变化有:首先,政治制度出现多元化的趋势,公民权利有所扩大。其次,法制建设得到重视和加强,以协调社会各阶级、阶层之间的利益。最后,改良主义政党在政治舞台上的影响日益扩大。A、C、D正确,B错误。
\end{solution}
\question 当代资本主义出现的新变化原因是多方面的,主要有
\par\fourch{\textcolor{red}{科学技术革命和生产力的发展,是资本主义变化的根本推动力量}}{\textcolor{red}{工人阶级争取自身权力斗争的作用,是推动资本主义变化的重要力量}}{\textcolor{red}{社会主义制度初步显示的优越性对资本主义产生了一定影响}}{主张改良主义的政党对资本主义制度的改革触动资本主义统治的根基}
\begin{solution}【简析】当代资本主义发生新变化的原因主要有:首先,科学技术革命和也产力的发展,是资本主义变化的根本推动力诳;其次,工人阶级争取自身权刺斗争的作用,是推动资本主义变化的重要力量;再次,社会主义制度初步显示的优越性对资本主义产生了一定影响;最后,主张改良主义的政党对资本主义制度的改难,也对资本主义的变化发挥了重耍作用。A、B、C正确。当代资本主义的新变化是深刻的,其意义也是深远的,但是,这些变化并没布触动资本主义统治的根基,并没有改变资本主义制度的性质。D错误。
\end{solution}

\subsection{166-社会主义社会的发展及其规律}
\question 2016年是英国人托马斯莫尔发表《乌托邦》一书500周年。《乌托邦》的发表标志着空想社会主义的诞生。空想社会主义虽然不是科学的思想体系,但它对未来新制度的描绘,闪烁着诸多天才的火花。莫尔在《乌托邦》一书中描绘了一个美好的社会,在那个社会里
\par\fourch{\textcolor{red}{没有私有财产和剥削现象}}{\textcolor{red}{人们有计划地从事生产,不需要商品货币和市场}}{\textcolor{red}{城乡之间没有对立}}{实现按劳分配}
\begin{solution}【简析】莫尔1516年发表的《乌托邦》一书描绘了一个美好的社会:在那里,没有私有财产和剥削现象,人们有计划地从事生产,城乡之间没有对立,不需要商品货币和市场,实现按需分配。A、B、C正确,D错误。
\end{solution}

\subsection{167-新民主主义革命理论}
\question 毛泽东第一次提出新民主主义革命的科学概念和总路线的著作是
\par\fourch{\textcolor{red}{《中国革命和中国共产党》}}{《新民主主义论》}{《在晋绥干部会议上的讲话》}{《论人民民主专政》}
\begin{solution}【简析】A正确。毛泽东在《中国革命和中国共产党》一文中,首次明确提出了``新民主主义革命''这个概念,并对新民主主义革命的总路线即对象、任务、动力、性质和前途等问题作了深刻的论述。
\end{solution}
\question 新民主主义革命的最基本的动力是
\par\twoch{贫农}{知识分子}{\textcolor{red}{无产阶级}}{工人和农民}
\begin{solution}【简析】C正确。无产阶级是新民主主义革命的最基本的动力,贫农是无产阶级最可靠的同盟军,知识分子是新民主主义革命的动力之一。
\end{solution}
\question 新民主主义革命的主要内容是
\par\fourch{没收封建地主阶级的土地归新民主主义国家所有}{没收官僚资产阶级的垄断资本归新民主主义国家所有}{\textcolor{red}{没收封建地主阶级的土地归农民所有}}{保护民族工商业}
\begin{solution}【简析】土地革命即没收封建地主阶级的土地归农民所有,是新民主主义革命的主要内容,也是新民主主义的经济纲领的内容之一。C是正确项。B、D是三大经济纲领中的两项内容,但不是新民主主义革命的主要内容。A项内容错误。新民主主义革命解决土地问题的目标,是消灭封建的土地制度,实现``耕者有其田''。这里要注意新民主主义革命的主要内容与中国革命的中心问题的区别,中心问题是领导权问题。
\end{solution}
\question 抗日战争时期,中国共产党在敌后抗日根据地实行的土地政策是
\par\fourch{没收地主土地分配给农民}{保持原有的土地状态}{没收一切土地平均分配}{\textcolor{red}{减租减息}}
\begin{solution}【简析】略
\end{solution}
\question 毛泽东系统阐述中国革命三大法宝的文章是
\par\fourch{\textcolor{red}{《〈共产党人〉发刊词》}}{《中国革命和中国共产党》}{《新民主主义论》}{《论联合政府》}
\begin{solution}【简析】略
\end{solution}
\question 毛泽东提出``须知政权是由枪杆子中取得的''著名论断的会议是
\par\twoch{中共四大}{\textcolor{red}{八七会议	}}{古田会议	}{遵义会议}
\begin{solution}【简析】略
\end{solution}

\subsection{168-社会主义改造理论}
\question 土地改革完成后,毛泽东分析我国农民的两大积极性是指
\par\fourch{\textcolor{red}{个体经济积极性}}{\textcolor{red}{互助合作积极性}}{集体经济积极性	}{个体劳动积极性}
\begin{solution}【简析】土地改革完成后,我国广大农民从封建剥削制度下解放出来,生产积极性大大提高。这种积极性表现在两个方面:一是个体经济的积极性,二是互助合作的积极性。A、B正确。
\end{solution}
\question 社会主义基本制度的确立是中国历史上最深刻最伟大的社会变革,为当代中国一切发展进步奠定了制度基础,也为中国特色社会主义制度的创新和发展提供了重要条件。社会主义基本制度确立的重大意义在于
\par\fourch{\textcolor{red}{极大地提髙了工人阶级和广大劳动人民的积极性和创造性,极大地促进了我国社会生产力的发展}}{形成了社会主义制度的中国模式,为其他国家确立社会主义提供了标准}{\textcolor{red}{使广大劳动人民真正成为国家的主人}}{\textcolor{red}{是世界社会主义运动史上又一个历史性的伟大胜利,进一步改变了世界政治经济格局}}
\begin{solution}【简析】社会主义没有统一的标准和模式,B错误。
\end{solution}

\subsection{169-社会主义本质和建设中国特色社会主义总任务}
\question 全面依法治国是``四个全面''战略布局的重要一环。全面建成小康社会'实现``两个一百年''的奋斗目标和中华民族伟大复兴的中国梦,全面深化改革、完善和发展中国特色社会主义制度,全面从严治党、保持长治久安,都必须在全面依法治国上作出总体部署。全面依法治国
\par\fourch{\textcolor{red}{为全面深化改革提供制度规则}}{\textcolor{red}{既是全面建成小康社会的内在要求,又是基本保障}}{\textcolor{red}{与全面从严治党本质一致、辩证统一}}{是实现中华民族伟大复兴中国梦的“关键一步”}
\begin{solution}【简析】全面依法治国是``四个全面''战略布局的重要一环。全面建成小康社会、实现``两个一百年''的奋斗目标和中华民族伟大复兴的中国梦,全面深化改革、完善和发展中il特色社会主义制度,全面从严治党、保持长治久安,都必须在全面依法治国上作出总体部著。第一,全面依法治国为全面深化改革提供制度规则。重大改革需要于法有据。第二,全面依法治国既是全面建成小康社会的内在要求,又是基本保障。全面深化改革、全面依法治国如鸟之两翼、车之双轮,推动全面建成小康社会的目标如期实现。第三,全面依法治国与全面从严治党本质一致、辩证统一。要跳出历史周期律、走好中国道路,必须落实好全面依法治国的战略部署。第四,全社会都能够尊法学法守法用法,法治才能成为中华民族伟大复兴中国梦的坚强保障。A、B、C正确。全面建成小康社会是阶段目标,是实现中华民族伟大复兴中国梦的``关键一步''。D不符合题意。
\end{solution}

\subsection{170-社会主义改革开放理论}
\question 党的十一届三中全会以后,邓小平在总结历史经验教训的基础上,对社会主义社会的基本矛盾,特别是社会主义初级阶段的主要矛盾进行了深人的思考,在新的实践中丰富和发展了这一理论。其主要内容有
\par\fourch{\textcolor{red}{判断一种生产关系和生产力是否相适应,主要希它是否适应当时当地生产力的耍求,能否推动生产力发展}}{\textcolor{red}{提出在社会主义社会依然有解放生产力的问题}}{\textcolor{red}{社会主义社会基本矛盾、主要矛盾和根本任务是统一的,它们要求必须把经济建设作为党和国家的工作重心,不断解放和发展生产力}}{\textcolor{red}{改革是社会主义制度下解放和发展生产力的必由之路}}
\begin{solution}【简析】A、C、D是明显的正确选项。过去只讲在社会主义条件下发展生产力,没有讲还要通过改革解放生产力,不完全,邓小平在社会主义本质理论中首先就提到解放生产力,B也正确。
\end{solution}
\question 中共十八届五中全会通过的《中共中央关于全面深化改革若干重大问题的决定》提出,全面深化改革的总目标是
\par\fourch{完善和发展社会主义市场经济体制	}{坚持和发展社会主义基本经济制度}{\textcolor{red}{推进国家治理体系和治理能力现代化}}{\textcolor{red}{完善和发展中国特色社会主义制度}}
\begin{solution}【简析】全面深化改革的总目标是完善和发展中国特色社会主义制度,推进国家治理体系和治理能力现代化。C、D正确。
\end{solution}
\question 习近平总书〖己在系列重要讲话中提出,全面深化改革要把握和处理好的一些重大关系,包括处理好解放思想和实事求是的关系、整体推进和重点突破的关系、全局和局部的关系、顶层设计和摸着石头过河的关系、胆子要大和步子耍稳的关系、改革发展和稳定的关系,等等。改革、发展、稳定的统一是我国社会主义现代化建设的三个重要支点。正确把握和处理好三者的关系,必须
\par\fourch{\textcolor{red}{把改革力度、发展速度和社会可承受程度统一起来}}{\textcolor{red}{把改善人民生活作为正确处理改革、发展、稳定关系的重要结合点}}{把保持社会稳定作为推进改革和发展的根本出发点和落脚点}{\textcolor{red}{在保持社会稳定中推进改革和发展,通过改革发展促进社会稳定}}
\begin{solution}【简析】改革、发展、稳定的统一是我国社会主义现代化建设的H个重要支点,必须正确处理好3者之间的关系。改革是经济社会发展的强大动力,发展是解决一切绍济社会问题的关键,稳定是改革发展的前提。十多年来,我国改革开放的实践充分证明,只有社会稳定,改革发展才能不断推进;只有改革、发展不断推进,社会稳足才能其有坚实基础。要坚持改革、发展、稳定的统一,把改革力度、发展速度和社会可承受程度统一起来,把改善人民生活作为正确处理改革、发展、稳定关系的重要结合点,在保持社会稳定中推进改革和发展,通过改革发展促进社会稳定。A、B、D正确,C错误。
\end{solution}

\subsection{171-建设中国特色社会主义总布局}
\question 中共十八届三中全会通过的《中共中央关于全面深化改革若干重大问题的决定》提出,坚持和

完善公有制为主体、多种所有制经济共同发展的基本经济制度。以下选项内容正确的是
\par\fourch{\textcolor{red}{公有制经济和非公有制经济都是我国经济社会发展的重要基础}}{公有制经济和非公有制经济都是我国基本经济制度的重要实现形式}{\textcolor{red}{必须不断增强国有经济活力、控制力、影响力}}{\textcolor{red}{必须激发非公有制经济活力和创造力}}
\begin{solution}【简析】公有制为主体、多种所有制经济共同发展的基本经济制度,是中国特色社会主义制度的重要支柱,也是社会主义市场经济体制的根基。公有制经济和非公有制经济都是社会主义市场经济的重要组成部分,都是我国经济社会发展的重要基础。必须毫不动摇巩固和发展公宥制经济,坚持公宥制主体地位,发挥国有经济主导作用,不断增强国有经济活力、控制力、影响力;必须毫不动摇鼓励、支持、引导非公有制经济发展,激发非公有制经济活力和创造力,A、C、D正确。公有制经济和非公有制经济是我国社会主义初级阶段基本经济制度的组成部分,但不是基本经济制度的实现形式。因为所有制和所有制的实现形式是两个不同层次的问题。公有制经济和非公有制经济讲的是所有权的归厲,而所有制的实现形式要解决的是发展生产力的组织形式和经营方式问题,两者不能混淆,B错误。
\end{solution}

\subsection{172-中国特色社会主义外交和国际战略}
\question 推动述立以合作共赢为核心的新型国际关系,是我们党立足时代发展潮流和我国根本利益做出的战略选择,反映了中国人民和世界人民的共同心愿。新型国际关系,新在
\par\twoch{相互尊重}{\textcolor{red}{合作共赢}}{不冲突}{不对抗}
\begin{solution}【简析】新型国际关系,新在合作共记主赢。强调把本国利益同各国共同利益结合起来,努力扩大各方共同利益的汇合点;要积极树立双赢、多赢、共嬴的新理念,摒弃你输我赢、赢者通吃的旧思维。B正确,A、C、D错误
\end{solution}

\subsection{173-建设中国特色社会主义的根本目的和依靠力量理论}
\question 2014年年底,习近平总书记在江苏调研时将``全面从严治党''与全面建成小康社会、全面深化改革、个而推进依法治国一道作为``四个全面''战略布局的基本内容。之所以强调坚持党要管党、从严治党,是因为坚持党要管党、从严治党
\par\fourch{\textcolor{red}{是我们党的一个重要经验}}{\textcolor{red}{是我们党应对国际国内风险考验、完成党的执政使命的客观需要}}{\textcolor{red}{是保持党的先进性纯洁性、巩固党的执政地位的必然要求}}{是确立我们党在国际共产主义运动中的领导权威的重要条件}
\begin{solution}【简析】坚持党要管党、从严治党是我们党的一个重嬰经验;是我们党应对国际国内风险考验、完成党的执政使命的客观需要;是保持党的先进性纯洁性、巩固党的执政地位的必然要求。``全面从严治党''作为``四个全而''战略布局的基本内容提出后,对党要管党、从严治党提出了更高、更全面的要求。A、B、C正确,D错误。
\end{solution}

\subsection{174-2018年1月国内时事}
\question {~新年前夕,习近平通发表了二○一八年新年贺词。(
)激荡光荣与梦想,充满信心与斗志,见证情怀。2018贺词最能体现抓落实的是:}
\par\fourch{撸起袖子加油干}{“以造福人民为最大政绩”}{“幸福都是奋斗出来的”}{\textcolor{red}{“不驰于空想、不骛于虚声”}}
\begin{solution}D
\end{solution}
\question {根据国办有关通知,我国于2017年12月31日前全面停止商业性(
)加工销售活动。}
\par\fourch{牛角}{象牙}{\textcolor{red}{红珊瑚}}{狐皮}
\begin{solution}C
\end{solution}
\question {在河长制先行地(
)流域,河长上岗,机制到位,在流域经济总量增长5倍、人口增加1100多万的背景下,(
)水实现了稳中向好,(
)流域走出一条经济发达、人口稠密地区的人水和谐共生新路。}
\par\fourch{洞庭湖}{鄱阳湖}{青海湖}{\textcolor{red}{太湖}}
\begin{solution}D
\end{solution}
\question {1月3日,中央军委隆重举行2018年开训动员大会,习近平向全军发布(
),号召全军贯彻落实党的十九大精神和新时代党的强军思想,全面加强实战化军事训练,全面提高打赢能力。}
\par\fourch{政令}{\textcolor{red}{训令}}{宪令}{作战指令}
\begin{solution}B
\end{solution}
\question 习近平1月11日在中纪委第二次全体会议上讲话强调,(
),是引领伟大斗争、伟大事业、最终实现伟大梦想的根本保证。
\par\fourch{新组织人事建设}{\textcolor{red}{新党建工程}}{新意识形态建设}{新统一战线建设}
\begin{solution}B
\end{solution}

\subsection{175-2018年2月国际时事}
\question {{巴基斯坦瓜达尔港自由区开园仪式日前落下帷,这标志着瓜达尔港建设进入新阶段,中巴经济走廊通往(}
{)的门户已经开启。}}
\par\fourch{地中海}{波斯湾}{\textcolor{red}{印度洋}}{大西洋}
\begin{solution}C
\end{solution}
\question {``(
)''。{2}{月}{9}{日}{20}{时,第二十三届冬季奥林匹克运动会在韩国平昌开幕。习近平主席特别代表、中共中央政治局常委韩正应邀出席开幕式。}}
\par\fourch{\textcolor{red}{激情平昌,和谐世界}}{热情平昌,欢迎世界}{激情平昌,点燃世界}{热情平昌,和谐世界}
\begin{solution}A
\end{solution}
\question {{(}
{)政府日前宣布,将拨款约}1270{万美元全球种子库进行修缮升级。该种子库位于(
)北部斯瓦尔巴群岛,目前储存着来自世界各地近}{90}{万份植物种子,作为``备份''以防人类赖以生存的农作物因灾难而绝种。}}
\par\fourch{瑞典 }{芬兰}{冰岛}{\textcolor{red}{挪威}}
\begin{solution}D
\end{solution}

\subsection{176-2004年考研政治真题解析}
\question {{贯彻}}{``}{{三个代表}}{''}{{重要思想,关键在}}\\
\par\fourch{坚持党的先进性}{坚持执政为民}{坚持党的阶级性}{\textcolor{red}{坚持与时俱进}}
\begin{solution}{{本题考点:贯彻}``三个代表''重要思想的关键。}{}

{{江泽民同志在党的十六大报告中指出:贯彻}``三个代表''重要思想,关键在坚持与时俱进,核心在坚持党的先进性,本质在坚持执政为民。因此选项{D}{正确{。}}}\\
\end{solution}

\subsection{177-毛泽东思想的形成与发展}
\question 在中国共产党的历史上,对毛泽东思想做出系统概括和阐述的党的文献有
\par\fourch{《关于若干历史问题的决议》}{\textcolor{red}{刘少奇在七大上所作的《关于修改党的章程的报告》}}{邓小平在八大上所作的《关于修改党的章程》的报告}{\textcolor{red}{《关于建国以来党的若干历史问题的决议》}}
\begin{solution}本题考关于毛泽东思想的科学含义问题,在中国共产党的历史上对毛泽东思想有两次概括:第一次是1945年中共七大的决议和刘少奇作的《关于修改党的章程》的报告,第二次是1981年十一届六中全会通过的《关于建国以来党的若干历史问题的决议》。两者既有一脉相承的继承关系,又有在新时期根据新情况做出的新认识,新判断。至于A、C是干扰项。A项是1945年中共六届七中全会通过的决议,总结了党自1921年产生以来领导中国革命的经验,尤其是1931年党的六届四中全会以来的路线是非,肯定了毛泽东同志是马克思列宁主义的普遍真理和中国革命的具体实践相结合的代表,为中共七大奠定了思想基础和组织基础。1956年召开的中共八大,邓小平在八大所作的《关于修改党的章程的报告》中,未提``毛泽东思想''这个概念,故AC应排除。本题历史性较强,考生需了解有关历史背景知识。
\end{solution}

\subsection{178-2018年4月国内时事}
\question 4月17日,联合国教科文组织正式批准中国提交申报的四川光雾山---诺水河地质公园、(
)成为联合国教科文组织世界地质公园。这也是我国第三十六、三十七个世界地质公园。
\par\fourch{\textcolor{red}{湖北黄冈大别山地质公园}}{湖南张家界世界地质公园}{安徽黄山世界地质公园}{江西庐山世界地质公园}
\begin{solution}{参考答案】:A ~ }
\end{solution}
\question {4月18号,中国科学院海洋研究所``科学三号''调查船从青岛起航,参加``()''工程2018年度渤海、南黄海标准断面调查项目综合考察。}
\par\fourch{美丽海洋}{环保海洋}{\textcolor{red}{透明海洋}}{魅力海洋}
\begin{solution}{ 【参考答案】:C ~}
\end{solution}
\question {4月18日电,围绕()产业的发展,首届全国光伏牡丹产业高峰论坛日前在安徽省宿州市举行。}
\par\fourch{光伏牡丹}{光伏}{牡丹}{\textcolor{red}{光伏油用牡丹}}
\begin{solution}{【参考答案】:D ~ }
\end{solution}
\question {5月9日2时28分,我国在(
)卫星发射中心用长征四号丙运载火箭成功发射高分五号卫星。}
\par\fourch{罗布泊}{四子王旗}{\textcolor{red}{太原 }}{秦山}
\begin{solution}{【参考答案】:C }
\end{solution}
\question {
5月9日,国务院总理李克强在东京迎宾馆同日本首相安倍晋三、韩国总统文在寅共同出席(
)中日韩领导人会议,就中日韩合作以及地区和国际问题交换看法。}
\par\fourch{第五次  }{第一次 }{第三次}{\textcolor{red}{第七次}}
\begin{solution}{【参考答案】:D ~}
\end{solution}
\question {6月7日,中国铁路总公司在京沈高铁启动高速动车组(
)系统现场试验,这标志着中国铁路在智能高铁关键核心技术自主创新上取得重要阶段成果。}
\par\twoch{\textcolor{red}{“自动驾驶”}}{“无人调度” }{“超长编组” }{“无人检票”  }
\begin{solution}{【参考答案】:A}
\end{solution}
\question {6月8日,全国``扫黄打非''办公室约谈网易云音乐、百度网盘、B站、猫耳FM、蜻蜓FM等多家网站负责人,要求各平台大力清理涉色情低俗问题的(
)内容,加强对相关内容的监管和审核。}
\par\fourch{HTML }{\textcolor{red}{ASMR }}{CPPA}{PTTP }
\begin{solution}{【参考答案】:B}
\end{solution}
\question {上合组织成员国元首理事会第十八次会议6月10日在青岛举行,习近平指出,进一步弘扬(
),提倡创新、协调、绿色、开放、共享的发展观,践行共同、综合、合作、可持续的安全观,秉持开放、融通、互利、共赢的合作观,树立平等、互鉴、对话、包容的文明观,坚持共商共建共享的全球治理观。}
\par\fourch{\textcolor{red}{“上海精神” }}{“海丝共识” }{“亚信宣言” }{“巴黎协定” }
\begin{solution}{【参考答案】:A }
\end{solution}
\question {截至6月30日,全军10.6万个有偿服务项目中,应停的(
)项目已全部按期停止,停偿工作取得决定性成果,实现既定阶段目标。}
\par\fourch{5万个}{7万个}{8万个}{\textcolor{red}{10万个 }}
\begin{solution}{ 【参考答案】:D ~ }
\end{solution}

\subsection{179-2018年4月国际时事}
\question ~4月4日,土耳其、俄罗斯和伊朗三国领导人在土耳其首都(
)举行会晤,就政治解决叙利亚问题达成多项共识。
\par\twoch{伊斯坦布尔}{索契}{\textcolor{red}{安卡拉	}}{加的夫}
\begin{solution}【参考答案】:C
\end{solution}
\question (
)东盟峰会及系列会议4月25日在新加坡拉开帷幕,东盟10国领导人和代表将讨论如何推动创新发展、建立智慧城市网络以及加强网络安全合作等议题。
\par\twoch{第三十届}{第三十一届}{\textcolor{red}{第三十二届}}{第三十三届}
\begin{solution}【参考答案】:C
\end{solution}
\question ~外交部发言人华春莹26日表示,(
)和巴基斯坦加入上海合作组织后,上合组织已成为人口最多、地域最广、潜力巨大的综合性区域组织,将在地区和国际事务中发挥更加积极的作用。
\par\twoch{\textcolor{red}{印度}}{印度尼西亚}{俄罗斯}{哈沙克斯坦}
\begin{solution}【参考答案】:A
\end{solution}
\question ~当地时间5月9日夜,以色列出动28架次战机,对叙利亚境内的(
)实施了大规模轰炸,共发射约60枚炮弹。
\par\fourch{“俄罗斯军事目标”  }{“黎巴嫩军事目标”}{\textcolor{red}{“伊朗军事目标” }}{“伊拉克军事目标”}
\begin{solution}~【参考答案】:C~~
\end{solution}
\question 美国总统特朗普5月24日说,因近期朝鲜表示出的``公开敌意'',他决定取消原定于6月中旬与朝鲜最高领导人金正恩在(
)的会晤。
\par\twoch{丰溪里}{板门店  }{\textcolor{red}{新加坡}}{伊斯坦布尔 }
\begin{solution}【参考答案】:C
\end{solution}
\question 5月27日,2017-2018赛季欧冠联赛决赛在基辅举行,(
)3-1击败利物浦,完成了欧冠联赛三连冠的伟业。
\par\twoch{巴塞罗那}{尤文图斯}{拜仁慕尼黑 }{\textcolor{red}{皇家马德里 }}
\begin{solution}【参考答案】:D~~
\end{solution}
\question 6月28日,中国企业在(
)为埃塞俄比亚产出了第一桶原油,标志着该国油气产业发展进入新阶段。
\par\fourch{6月28日,中国企业在( )为埃塞俄比亚产出了第一桶原油,标志着该国油气产业发展进入新阶段。}{\textcolor{red}{欧加登盆地}}{戈兰高地}{东非大裂谷  }
\begin{solution}【参考答案】:B~
\end{solution}
\question 6月6日在法国巴黎联合国教科文组织总部举行的《保护非物质文化遗产公约》缔约国大会(
)会议上,中国以123票高票当选保护非物质文化遗产政府间委员会委员国,本届新当选委员国的任期从2018年到2022年。
\par\twoch{第三届}{第五届}{\textcolor{red}{第七届}}{第九届}
\begin{solution}【参考答案】:C~ ~
\end{solution}

\subsection{180-2018年5月国际时事}
\question ~4月4日,土耳其、俄罗斯和伊朗三国领导人在土耳其首都(
)举行会晤,就政治解决叙利亚问题达成多项共识。
\par\twoch{伊斯坦布尔}{索契}{\textcolor{red}{安卡拉	}}{加的夫}
\begin{solution}【参考答案】:C
\end{solution}
\question (
)东盟峰会及系列会议4月25日在新加坡拉开帷幕,东盟10国领导人和代表将讨论如何推动创新发展、建立智慧城市网络以及加强网络安全合作等议题。
\par\twoch{第三十届}{第三十一届}{\textcolor{red}{第三十二届}}{第三十三届}
\begin{solution}【参考答案】:C
\end{solution}
\question ~外交部发言人华春莹26日表示,(
)和巴基斯坦加入上海合作组织后,上合组织已成为人口最多、地域最广、潜力巨大的综合性区域组织,将在地区和国际事务中发挥更加积极的作用。
\par\twoch{\textcolor{red}{印度}}{印度尼西亚}{俄罗斯}{哈沙克斯坦}
\begin{solution}【参考答案】:A
\end{solution}
\question ~当地时间5月9日夜,以色列出动28架次战机,对叙利亚境内的(
)实施了大规模轰炸,共发射约60枚炮弹。
\par\fourch{“俄罗斯军事目标”  }{“黎巴嫩军事目标”}{\textcolor{red}{“伊朗军事目标” }}{“伊拉克军事目标”}
\begin{solution}~【参考答案】:C~~
\end{solution}
\question 美国总统特朗普5月24日说,因近期朝鲜表示出的``公开敌意'',他决定取消原定于6月中旬与朝鲜最高领导人金正恩在(
)的会晤。
\par\twoch{丰溪里}{板门店  }{\textcolor{red}{新加坡}}{伊斯坦布尔 }
\begin{solution}【参考答案】:C
\end{solution}
\question 5月27日,2017-2018赛季欧冠联赛决赛在基辅举行,(
)3-1击败利物浦,完成了欧冠联赛三连冠的伟业。
\par\twoch{巴塞罗那}{尤文图斯}{拜仁慕尼黑 }{\textcolor{red}{皇家马德里 }}
\begin{solution}【参考答案】:D~~
\end{solution}
\question 6月28日,中国企业在(
)为埃塞俄比亚产出了第一桶原油,标志着该国油气产业发展进入新阶段。
\par\fourch{6月28日,中国企业在( )为埃塞俄比亚产出了第一桶原油,标志着该国油气产业发展进入新阶段。}{\textcolor{red}{欧加登盆地}}{戈兰高地}{东非大裂谷  }
\begin{solution}【参考答案】:B~
\end{solution}
\question 6月6日在法国巴黎联合国教科文组织总部举行的《保护非物质文化遗产公约》缔约国大会(
)会议上,中国以123票高票当选保护非物质文化遗产政府间委员会委员国,本届新当选委员国的任期从2018年到2022年。
\par\twoch{第三届}{第五届}{\textcolor{red}{第七届}}{第九届}
\begin{solution}【参考答案】:C~ ~
\end{solution}

\subsection{181-2018年6月国际时事}
\question ~4月4日,土耳其、俄罗斯和伊朗三国领导人在土耳其首都(
)举行会晤,就政治解决叙利亚问题达成多项共识。
\par\twoch{伊斯坦布尔}{索契}{\textcolor{red}{安卡拉	}}{加的夫}
\begin{solution}【参考答案】:C
\end{solution}
\question (
)东盟峰会及系列会议4月25日在新加坡拉开帷幕,东盟10国领导人和代表将讨论如何推动创新发展、建立智慧城市网络以及加强网络安全合作等议题。
\par\twoch{第三十届}{第三十一届}{\textcolor{red}{第三十二届}}{第三十三届}
\begin{solution}【参考答案】:C
\end{solution}
\question ~外交部发言人华春莹26日表示,(
)和巴基斯坦加入上海合作组织后,上合组织已成为人口最多、地域最广、潜力巨大的综合性区域组织,将在地区和国际事务中发挥更加积极的作用。
\par\twoch{\textcolor{red}{印度}}{印度尼西亚}{俄罗斯}{哈沙克斯坦}
\begin{solution}【参考答案】:A
\end{solution}
\question ~当地时间5月9日夜,以色列出动28架次战机,对叙利亚境内的(
)实施了大规模轰炸,共发射约60枚炮弹。
\par\fourch{“俄罗斯军事目标”  }{“黎巴嫩军事目标”}{\textcolor{red}{“伊朗军事目标” }}{“伊拉克军事目标”}
\begin{solution}~【参考答案】:C~~
\end{solution}
\question 美国总统特朗普5月24日说,因近期朝鲜表示出的``公开敌意'',他决定取消原定于6月中旬与朝鲜最高领导人金正恩在(
)的会晤。
\par\twoch{丰溪里}{板门店  }{\textcolor{red}{新加坡}}{伊斯坦布尔 }
\begin{solution}【参考答案】:C
\end{solution}
\question 5月27日,2017-2018赛季欧冠联赛决赛在基辅举行,(
)3-1击败利物浦,完成了欧冠联赛三连冠的伟业。
\par\twoch{巴塞罗那}{尤文图斯}{拜仁慕尼黑 }{\textcolor{red}{皇家马德里 }}
\begin{solution}【参考答案】:D~~
\end{solution}
\question 6月28日,中国企业在(
)为埃塞俄比亚产出了第一桶原油,标志着该国油气产业发展进入新阶段。
\par\fourch{6月28日,中国企业在( )为埃塞俄比亚产出了第一桶原油,标志着该国油气产业发展进入新阶段。}{\textcolor{red}{欧加登盆地}}{戈兰高地}{东非大裂谷  }
\begin{solution}【参考答案】:B~
\end{solution}
\question 6月6日在法国巴黎联合国教科文组织总部举行的《保护非物质文化遗产公约》缔约国大会(
)会议上,中国以123票高票当选保护非物质文化遗产政府间委员会委员国,本届新当选委员国的任期从2018年到2022年。
\par\twoch{第三届}{第五届}{\textcolor{red}{第七届}}{第九届}
\begin{solution}【参考答案】:C~ ~
\end{solution}

\subsection{182-2018年7月国内时事}
\question {15年前,时任浙江省委书记的习近平同志全面系统阐释浙江的八个优势,提出改革发展的八项举措。(
)其精神实质对全国各地区的改革发展也有着重大指导意义。}
\par\fourch{“十五战略”  }{\textcolor{red}{“八八战略” }}{“三去一降一补战略” }{“猎鹰战略” }
\begin{solution}{【参考答案】:B }
\end{solution}
\question {习近平对吉林长春长生生物疫苗案件作出重要指示指出,以(
)的决心,完善我国疫苗管理体制,坚决守住安全底线,全力保障群众切身利益和社会安全稳定大局。}
\par\fourch{久久为功 }{功成不必在我 }{\textcolor{red}{猛药去疴、刮骨疗毒}}{不积跬步无以至千里}
\begin{solution}{【参考答案】:C }
\end{solution}
\question { 8月20日中午12时,(
)北京段通车。这标志着,在北京市域内,全面消除了国高网断头路。}
\par\fourch{沈海高速公路 }{\textcolor{red}{京秦高速公路 }}{京藏高速公路}{京港高速公路}
\begin{solution}{【参考答案】:B}
\end{solution}
\question {习近平强调,全党全社会要弘扬尊师重教的社会风尚,努力提高教师(
),让广大教师享有应有的社会声望,在教书育人岗位上为党和人民事业作出新的更大的贡献。}
\par\fourch{政治地位 }{社会地位}{职业地位 }{\textcolor{red}{政治地位、社会地位、职业地位 }}
\begin{solution}{【参考答案】:D}
\end{solution}

\subsection{183-2018年7月国际时事}
\question 习近平7月23日在(
)同卢旺达总统卡加梅举行会谈,两国元首一致同意共同推动双方互利合作结出更加丰硕成果,为中卢人民、中非人民带来更多福祉。
\par\fourch{雅加达 }{约翰内斯堡}{\textcolor{red}{基加利}}{内罗毕}
\begin{solution}【参考答案】:C
\end{solution}
\question 习近平7月24日同南非总统拉马福萨举行会谈,就推进新时期中南(
)达成重要共识,让两国人民更多享受中南合作成果。
\par\twoch{合作伙伴关系}{战略伙伴关系}{\textcolor{red}{全面战略伙伴关系}}{全天候合作伙伴关系}
\begin{solution}【参考答案】:C
\end{solution}
\question 联合国秘书长古特雷斯7月12日表示,希望联合国成员国通过谈判解决彼此间的贸易争端,他同时肯定(
)在解决贸易争端方面的作用。
\par\fourch{北大西洋公约组织}{亚太经合组织}{\textcolor{red}{世界贸易组织 }}{国际货币基金组织 }
\begin{solution}【参考答案】:C
\end{solution}
\question 德国总理默克尔7月20日说,德国愿意在(
)遭遇巨大压力时仍与美国合作,但重申欧洲不能再依靠美国。
\par\fourch{跨太平洋关系  }{跨印度洋关系}{\textcolor{red}{跨大西洋关系}}{跨亚欧大陆关系}
\begin{solution}【参考答案】:C
\end{solution}
\question ~欧盟外交和安全政策高级代表莫盖里尼8月6日与(
)三国外长发表联合声明,对美国即将重启对伊朗制裁深表遗憾。
\par\fourch{法意奥  }{荷法西 }{\textcolor{red}{英法德}}{葡西奥 }
\begin{solution}~【参考答案】:C~~
\end{solution}
\question 8月5日,刚果(金)卫生部发布公报说,该国暴发新一轮(
)出血热疫情,已经造成33人死亡。这距离世界卫生组织宣布该国结束上一轮疫情仅有一周时间。
\par\twoch{疟疾}{\textcolor{red}{埃博拉 }}{非洲猪瘟 }{霍乱}
\begin{solution}~【参考答案】:B
\end{solution}
\question 8月12日,哈萨克斯坦、俄罗斯、阿塞拜疆、伊朗、土库曼斯坦五国元首出席会议,并共同签署了《(
)法律地位公约》,为该水域的资源开发以及相关合作奠定了法律基础。
\par\fourch{黑海  }{\textcolor{red}{里海}}{波罗的海}{巴伦支海}
\begin{solution}【参考答案】:B~
\end{solution}
\question 第十八届亚运会正在(
)如火如荼地举行,来自亚洲45个国家和地区约13万名运动员在雅加达和巨港等地展开角逐,亚运会不仅吸引了大量资金流和人员流,更有望带来新的发展机遇。
\par\fourch{菲律宾  }{马来西亚}{越南}{\textcolor{red}{印度尼西亚 }}
\begin{solution}【参考答案】:D~~
\end{solution}
\question 我国与毛里求斯于9月2日结束中毛自由贸易协定谈判,这一协定是我国与非洲国家商签的(
)自贸协定,下一步双方将为最终签署协定做好准备。
\par\twoch{\textcolor{red}{第一个}}{第二个}{第三个}{第四个}
\begin{solution}【参考答案】:A~
\end{solution}
\question ~第四届(
)全会9月12日在符拉迪沃斯托克举行,习近平强调中方愿同地区国家一道,维护地区和平安宁,实现各国互利共赢,巩固人民传统友谊,实现综合协调发展,促进本地区和平稳定和发展繁荣。
\par\fourch{达沃斯经济论坛}{\textcolor{red}{东方经济论坛 }}{亚布力经济论坛 }{博鳌经济论坛 }
\begin{solution}~【参考答案】:B
\end{solution}
\question 9月12日晚,(
)冲突各方在埃塞俄比亚首都亚的斯亚贝巴签署最终和平协议,标志着(
)自2013年底爆发的内战宣告结束。
\par\twoch{苏丹}{\textcolor{red}{南苏丹}}{叙利亚}{阿富汗}
\begin{solution}【参考答案】:B~
\end{solution}
\question 9月16日,厄立特里亚总统伊萨亚斯与埃塞俄比亚总理阿比在沙特阿拉伯国王萨勒曼主持下,在沙特海滨城市(
)签署了和平协议。联合国官网评论称,该协议结束了双方持续数十年的冲突,是一份``历史性的和平协议''。
\par\fourch{\textcolor{red}{吉达 }}{利雅得}{胡贝尔}{达曼}
\begin{solution}【参考答案】:A~~
\end{solution}
\question 联合国秘书长古特雷斯9月24日说,从全球范围看,实现(
)可持续发展议程每年需要投资5万亿至7万亿美元。
\par\fourch{2020年  }{2025年  }{\textcolor{red}{2030年}}{2050年}
\begin{solution}~【参考答案】:C~~
\end{solution}

\subsection{184-2018年8月国内时事}
\question {15年前,时任浙江省委书记的习近平同志全面系统阐释浙江的八个优势,提出改革发展的八项举措。(
)其精神实质对全国各地区的改革发展也有着重大指导意义。}
\par\fourch{“十五战略”  }{\textcolor{red}{“八八战略” }}{“三去一降一补战略” }{“猎鹰战略” }
\begin{solution}{【参考答案】:B }
\end{solution}
\question {习近平对吉林长春长生生物疫苗案件作出重要指示指出,以(
)的决心,完善我国疫苗管理体制,坚决守住安全底线,全力保障群众切身利益和社会安全稳定大局。}
\par\fourch{久久为功 }{功成不必在我 }{\textcolor{red}{猛药去疴、刮骨疗毒}}{不积跬步无以至千里}
\begin{solution}{【参考答案】:C }
\end{solution}
\question { 8月20日中午12时,(
)北京段通车。这标志着,在北京市域内,全面消除了国高网断头路。}
\par\fourch{沈海高速公路 }{\textcolor{red}{京秦高速公路 }}{京藏高速公路}{京港高速公路}
\begin{solution}{【参考答案】:B}
\end{solution}
\question {习近平强调,全党全社会要弘扬尊师重教的社会风尚,努力提高教师(
),让广大教师享有应有的社会声望,在教书育人岗位上为党和人民事业作出新的更大的贡献。}
\par\fourch{政治地位 }{社会地位}{职业地位 }{\textcolor{red}{政治地位、社会地位、职业地位 }}
\begin{solution}{【参考答案】:D}
\end{solution}

\subsection{185-2018年8月国际时事}
\question 习近平7月23日在(
)同卢旺达总统卡加梅举行会谈,两国元首一致同意共同推动双方互利合作结出更加丰硕成果,为中卢人民、中非人民带来更多福祉。
\par\fourch{雅加达 }{约翰内斯堡}{\textcolor{red}{基加利}}{内罗毕}
\begin{solution}【参考答案】:C
\end{solution}
\question 习近平7月24日同南非总统拉马福萨举行会谈,就推进新时期中南(
)达成重要共识,让两国人民更多享受中南合作成果。
\par\twoch{合作伙伴关系}{战略伙伴关系}{\textcolor{red}{全面战略伙伴关系}}{全天候合作伙伴关系}
\begin{solution}【参考答案】:C
\end{solution}
\question 联合国秘书长古特雷斯7月12日表示,希望联合国成员国通过谈判解决彼此间的贸易争端,他同时肯定(
)在解决贸易争端方面的作用。
\par\fourch{北大西洋公约组织}{亚太经合组织}{\textcolor{red}{世界贸易组织 }}{国际货币基金组织 }
\begin{solution}【参考答案】:C
\end{solution}

\subsection{186-2018年9月国内时事}
\question {15年前,时任浙江省委书记的习近平同志全面系统阐释浙江的八个优势,提出改革发展的八项举措。(
)其精神实质对全国各地区的改革发展也有着重大指导意义。}
\par\fourch{“十五战略”  }{\textcolor{red}{“八八战略” }}{“三去一降一补战略” }{“猎鹰战略” }
\begin{solution}{【参考答案】:B }
\end{solution}
\question {习近平对吉林长春长生生物疫苗案件作出重要指示指出,以(
)的决心,完善我国疫苗管理体制,坚决守住安全底线,全力保障群众切身利益和社会安全稳定大局。}
\par\fourch{久久为功 }{功成不必在我 }{\textcolor{red}{猛药去疴、刮骨疗毒}}{不积跬步无以至千里}
\begin{solution}{【参考答案】:C }
\end{solution}
\question { 8月20日中午12时,(
)北京段通车。这标志着,在北京市域内,全面消除了国高网断头路。}
\par\fourch{沈海高速公路 }{\textcolor{red}{京秦高速公路 }}{京藏高速公路}{京港高速公路}
\begin{solution}{【参考答案】:B}
\end{solution}
\question {习近平强调,全党全社会要弘扬尊师重教的社会风尚,努力提高教师(
),让广大教师享有应有的社会声望,在教书育人岗位上为党和人民事业作出新的更大的贡献。}
\par\fourch{政治地位 }{社会地位}{职业地位 }{\textcolor{red}{政治地位、社会地位、职业地位 }}
\begin{solution}{【参考答案】:D}
\end{solution}

\subsection{187-2018年9月国际时事}
\question 习近平7月23日在(
)同卢旺达总统卡加梅举行会谈,两国元首一致同意共同推动双方互利合作结出更加丰硕成果,为中卢人民、中非人民带来更多福祉。
\par\fourch{雅加达 }{约翰内斯堡}{\textcolor{red}{基加利}}{内罗毕}
\begin{solution}【参考答案】:C
\end{solution}
\question 习近平7月24日同南非总统拉马福萨举行会谈,就推进新时期中南(
)达成重要共识,让两国人民更多享受中南合作成果。
\par\twoch{合作伙伴关系}{战略伙伴关系}{\textcolor{red}{全面战略伙伴关系}}{全天候合作伙伴关系}
\begin{solution}【参考答案】:C
\end{solution}
\question 联合国秘书长古特雷斯7月12日表示,希望联合国成员国通过谈判解决彼此间的贸易争端,他同时肯定(
)在解决贸易争端方面的作用。
\par\fourch{北大西洋公约组织}{亚太经合组织}{\textcolor{red}{世界贸易组织 }}{国际货币基金组织 }
\begin{solution}【参考答案】:C
\end{solution}

